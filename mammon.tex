\section*{\center{THE PARABLE\\
of the\\
WICKED MAMMON}}
\addcontentsline{toc}{section}{The Parable of the Wicked Mammon}

\begin{abstract}
That faith, the mother of all good works, justifieth us, before we can bring 
forth any good work: as the husband marrieth his wife before he can have 
any lawful children by her. Furthermore, as the husband marrieth not 
his wife that she should continue unfaithful as before, and as she was in 
the state of virginity, (wherein it was impossible for her to bear fruit) but 
contrariwise to make her fruitful; even so faith justifieth us not, that is to 
say, marrieth us not to God, that we should continue unfruitful as before, 
but that he should put the seed of his Holy Spirit in us, (as St John in his 
first Epistle calleth it) and to make us fruitful. For, saith Paul, (Eph. ii.) 
``By grace are ye made safe through faith, and that not of yourselves: for 
it is the gift of God, and cometh not of the works, lest any man should 
boast himself. For we are his workmanship, created in Christ Jesus unto 
good works, which God hath ordained that we should walk in them." Amen. 
\end{abstract}


\subsection*{Preface}
\begin{center}
WILLIAM TYNDALE,\\
OTHERWISE CALLED HITCHINS,\\
TO THE READER. 
\end{center}



GRACE and peace, with all manner [of]
spiritual feeling and living, worthy of the
kindness of Christ, be with the reader, and with 
all that trust the will of God. Amen. 

The cause why I set my name before this
little Treatise, and have not rather done it in
the New Testament, is, that then I followed
the counsel of Christ, which exhorteth men
(Matt. vi.) to do their good deeds secretly, and
to be content with the conscience of well-doing; 
and that God seeth us; and patiently to abide 
the reward of the last day, which Christ hath 
purchased for us: and now would I fain have 
done likewise, but am I compelled otherwise 
to do. 

While I abode, a faithful companion, which 
now hath taken another voyage upon him, to 
preach Christ where, I suppose, he was never 
yet preached, (God, which put in his heart thither 
to go, send his Spirit with him, comfort him, 
and bring his purpose to good effect,) one 
William Roye, a man somewhat crafty, when he
cometh unto new acquaintance, and before he
be thorough known, and namely, when all is 
spent, came unto me and offered his help. As 
long as he had no money somewhat I could 
rule him; but as soon as he had gotten him 
money, he became like himself again.
Nevertheless, I suffered all things till that was ended, 
which I could not do alone without one, both to 
write, and to help me to compare the texts 
together. When that was ended, I took my 
leave, and bade him farewell for our two lives; 
and, as men say, a day longer. After we were 
departed, he went and gat him new friends, 
which thing to do he passeth all that ever I yet 
knew. And there, when he had stored him of 
money, he gat him to Argentine, where he
professeth wonderful faculties, and maketh boast of 
no small things. A year after that, and now 
twelve months before the printing of this work, 
came one Jerome, a brother of Greenwich also, 
through Worms to Argentine, saying that he
intended to be Christ's disciple another while, and 
to keep (as nigh as God would give him grace,) 
the profession of his baptism, and to get his 
living with his hands, and to live no longer idly, 
and of the sweat and labour of those captives, 
which they had taught not to believe in Christ, 
but in cut shoes, and russet coats. Which 
Jerome, with all diligence, I warned of Roye's 
boldness, and exhorted him to beware of him, 
and to walk quietly, and with all patience and 
long-suffering, according as we have Christ and 
his apostles for an ensample, which thing he also 
promised me. 

Nevertheless, when he was come to
Argentine, William Roye (whose tongue is able 
not only to make fools stark mad, but also to 
deceive the wisest, that is at the first sight and 
acquaintance,) gat him to him, and set him a 
work to make rhymes, while he himself translated 
a dialogue out of Latin into English, in whose 
Prologue he promiseth more a great deal than I 
fear me he will ever pay. Paul saith, (2 Tim. ii.)
``The servant of the Lord must not strive, but 
be peaceable unto all men, and ready to teach, 
and one that can suffer the evil with meekness, and 
that can inform them that resist; if God at any 
time will give them repentance for to know the 
truth." It becometh not then the Lord's servant 
to use railing rhymes, but God's word, which is 
the right weapon to slay sin, vice and all iniquity. 
The Scripture of God is good to teach and to 
improve. (2 Tim. iii. and 2 Thess. ii.) Paul 
speaking of Antichrist, saith, ``Whom the Lord 
shall destroy with the Spirit, or breath of his 
mouth;" that is, with the word of God, And 
(2 Cor. x.) ``The weapons of our war are not
carnal things, (saith he) but mighty in God to 
cast down strong holds" and so forth; that is, 
to destroy high buildings of false doctrine. The 
word of God is that day whereof Paul speaketh, 
(1 Cor. iii.) which shall declare all things, and
that fire which shall try every man's work, and 
consume false doctrine; with that sword ought 
men sharply to fight, and not to rail with foolish 
rhymes. Let it not offend thee that some walk 
inordinately; let not the wickedness of Judas 
cause thee to despise the doctrine of his fellows. 
No man ought to think that Stephen was a false 
preacher because that Nicholas, which was 
chosen fellow with him (Acts vi.) to minister 
unto the widows, fell after into great heresies, as 
histories make mention. Good and evil go 
always together, one cannot be known without 
the other. 

Mark this also above all things, — that
Antichrist is not an outward thing, that is to say, a 
man that should suddenly appear with wonders, 
as our fathers talked of him. No, verily; for 
Antichrist is a spiritual thing. And is as much 
to say as against Christ; that is, one that 
preacheth false doctrine, contrary to Christ. 
Antichrist was in the Old Testament, and fought 
with the prophets; he was also in the time of 
Christ and the apostles, as thou readest in the 
Epistles of John, and of Paul to the Corinthians 
and Galatians, and other Epistles, Antichrist is 
now, and shall, (I doubt not) endure till the 
world's end. But his nature is (when he is 
uttered and overcome with the word of God) to 
go out of the play for a season, and to disguise 
himself, and then to come in again with a new 
name and new raiment As thou seest how 
Christ rebuketh the Scribes and the Pharisees 
in the gospel, (which were very Antichrists,)
saying, ``Woe be to you, Pharisees, for ye rob 
widows' houses; ye pray long prayers under a 
colour; ye shut up the kingdom of heaven, and 
suffer them not that would to enter in; ye have 
taken away the key of knowledge; ye make men 
break God's commandments with your traditions; 
ye beguile the people with hypocrisy and such 
like; Which things all our prelates do, but have 
yet gotten them new names, and other garments, 
and are otherwise disguised. There is difference 
in the names between a pope, a cardinal, a bishop, 
and so forth, and to say a scribe, a pharisee, 
a senior, and so forth; but the thing is all one. 
Even so now, when we have uttered him, he will 
change himself once more, and turn himself 
into angel of light. (2 Cor. xi.) Read the place, 
I exhort thee, whatsoever thou art that readest 
this, and note it well. The Jews look for Christ, 
and he is come fifteen hundred years ago, and 
they not aware: we also have looked for
Antichrist, and he hath reigned as long, and we not 
aware; and that because either of us looked 
carnally for him, and not in the places where we 
ought to have sought. The Jews had found 
Christ verily if they had sought him in the law 
and the prophets, whither Christ sendeth them 
to seek. (John v.) We also had spied out
Antichrist long ago if we had looked in the doctrine 
of Christ and his apostles, where, because the 
beast seeth himself now to be sought for, he 
roareth, and seeketh new holes to hide himself 
in, and changeth himself into a thousand fashions, 
with all manner [of] wiliness, falsehood, subtilty, 
and craft. Because that his excommunications are 
come to light, he maketh it treason unto the 
king to be acquainted with Christ. If Christ and 
they may not reign together, one hope we have 
that Christ shall live ever. The old Antichrists 
brought Christ unto Pilate, saying, By our law 
he ought to die; and when Pilate bade them 
judge him after their law, they answered. It is 
not lawful for us to kill any man; which they did 
to the intent that they which regarded not the 
shame of their false excommunications, should 
yet fear to confess Christ, because that the
temporal sword had condemned him. They do all 
things of a good zeal, they say; they love you 
so well, that they had rather burn you, than that 
you should have fellowship with Christ. They 
are jealous over you amiss, (as saith St. Paul 
Gal. iv.) They would divide you from Christ and 
his holy Testament, and join you to the Pope, 
to believe in his testament and promises. Some 
man will ask, peradventure, Why I take the 
labour to make this work, inasmuch as they will 
burn it, seeing they burnt the gospel? I answer, 
In burning the New Testament they did none 
other thing than that I looked for, no more shall 
they do if they burn me also, if it be God's will 
it shall so be. 

Nevertheless, in translating the New Testament 
I did my duty, and so do I now, and will do as 
much more as God hath ordained me to do. And 
as I offered that to all men to correct it,
whosoever could, even so I do this. Whosoever, 
therefore, readeth this, compare it unto the 
Scripture. If God's word bear record unto it, 
and thou also feelest in thine heart that it is so, 
be of good comfort, and give God thanks. If 
God's word condemn it, then hold it accursed, 
and so do all other doctrines: as Paul
counselleth his Galatians:— Believe not every spirit 
suddenly, but judge them by the word of God, 
which is the trial of all doctrine, and lasteth for 
ever. Amen. 

\subsection*{}
\begin{center}
THE PARABLE 

OF THE 

WICKED MAMMON. 
\end{center}

\begin{verse}
``There was a certain rich man which had a steward, that was accused
unto him that he had wasted his goods; and called him, and said unto him. How 
is it that I hear this of thee? Give account of thy stewardship; for thou 
mayest be no longer my steward. The steward said within himself, What shall 
I do, for my master will take away from me my stewardship? I cannot dig, and 
to beg I am ashamed. I wot what to do, that when I am put out of my
stewardship, they may receive me into their houses. Then called he all his master's 
debtors, and said unto the first. How much owest thou unto my master? And 
he said. An hundred tons of oil. And he said to him. Take thy bill, and sit 
down quickly, and write fifty. Then said he to another, What owest thou? 
Aad he said, An hundred quarters of wheat. He said to him. Take thy bill, and 
write fourscore. And the lord commended the unjust steward, because he had 
done wisely. For the children of this world are in their kind wiser than the 
children of light. And I say also unto you, make you friends of the wicked 
mammon, that when ye shall have need, they may receive you into everlasting 
habitations." (Luke xvith chapter.) 
\end{verse}

\subsection*{Introduction}
FORASMUCH as with this, and divers such other 
texts, many have enforced to draw the people from the 
true faith, and from putting their trust in the truth of 
God's promises, and in the merits and deserving of his 
Christ our Lord; and have also brought it to pass, (for 
many false prophets shall arise and deceive many, and 
much wickedness must also be, saith Christ; (Matt. xxiv.) 
and Paul saith, (2 Tim. iii.) Evil men and deceivers 
shall prevail in evil, while they deceive, and are deceived 
themselves;) and have taught them to put their trust in 
their own merits; and brought them in belief that they 
shall be justified in the sight of God by the goodness of 
their own works, and have corrupted the pure word of 
God, to confirm their Aristotle withal. For though that 
the philosophers, and worldly wise men, were enemies 
above all enemies to the gospel of God; and though the 
worldly wisdom cannot comprehend the wisdom of God, 
as thou mayest see 1 Cor. i. and ii. And though worldly 
righteousness cannot be obedient unto the righteousness of 
God, (Rom. x.) yet whatsoever they read in Aristotle, 
that must be first true. And to maintain that, they rend 
and tear the Scriptures with their distinctions, and
expound them violently, contrary to the meaning of the text, 
and to the circumstances that go before and after, and to 
a thousand clear and evident texts. Wherefore I have 
taken in hand to expound this gospel, and certain other 
places of the New Testament; and, (as far forth as God 
shall lend me grace,) to bring the Scripture unto the 
right sense, and to dig again the wells of Abraham, and 
to purge and cleanse them of the earth of worldy wisdom 
wherewith these Philistines have stopped them. Which 
grace, grant me, God, for the love that he hath unto 
his Son, Jesus our Lord, unto the glory of his name. 
Amen. 

\subsection*{}
That faith only before all works and without all merits, 
but Christ's only, justifieth and setteth us at peace with 
God, is proved by Paul in the first chapter to the Romans. 
I am not ashamed (saith he) of the gospel, that is to say, 
of the glad tidings and promises which God hath made, 
and sworn to us in Christ. For it (that is to say the gospel) 
is the power of God unto salvation to all that believe. 
And it followeth in the foresaid chapter, the just or
righteous must live by faith.

For in the faith which we have in Christ, and in God's
promises find we mercy, life, favour and peace. In the
law we find death, damnation, and wrath: moreover, the
curse and vengeance of God upon us. And it (that is to 
say the law) is called of Paul (2 Cor. iii.) the ministration 
of death and damnation. In the law we are proved to be 
the enemies of God, and that we hate him. For how can 
we be at peace with God and love him, seeing we are
conceived and born under the power of the devil, and are his 
possession and kingdom, his captives and bondmen, and 
led at his will, and he holdeth our hearts, so that it is
impossible for us to consent to the will of God, much more 
is it impossible for a man to fulfil the law of his own 
strength and power, seeing that we are by birth and of 
nature, the heirs of eternal damnation. As saith Paul, 
Eph. ii. We (saith he) are by nature the children of
wrath, which thing the law doth but utter only, and helpeth 
us not, yea requireth impossible things of us. The law 
when it commandeth that thou shalt not lust, giveth thee 
not power so to do, but damneth thee, because thou canst 
not so do. 

If thou wilt therefore be at peace with God, and love 
him, thou must turn to the promises of God, and to the 
gospel, which is called of Paul in the place before rehearsed
to the Corinthians, the ministration of righteousness, and
of the Spirit. For faith bringeth pardon, and forgiveness
freely purchased by Christ's blood, and bringeth also the
Spirit, the Spirit looseth the bonds of the devil, and setteth 
us at liberty. For where the Spirit of the Lord is, there 
is liberty, saith Paul in the same place to the Corinthians, 
that is to say, there the heart is free, and hath power to love 
the will of God, and there the heart mourneth that he 
cannot love enough. Now is that consent of the heart 
unto the law of God eternal life, yea, though there be no 
power yet in the members to fulfil it. Let every man 
therefore (according to Paul's counsel in the vith chapter to 
the Ephesians,) arm himself with the armour of God; 
that is to understand, with God's promises. And above all 
things (saith he) take unto you the shield of faith,
wherewith ye may be able to quench all the fiery darts of the 
wicked, that ye may be able to resist in the evil day of 
temptation, and namely at the hour of death. 

See therefore thou have God's promises in thine heart, 
and that thou believe them without wavering; and when 
temptation ariseth, and the devil layeth the law and thy 
deeds against thee, answer him with the promises, and turn 
to God, and confess thyself to him, and say, it is even so 
or else how could he be merciful? but remember that he 
is the God of mercy and of truth, and cannot but fulfil 
his promises. Also remember, that his son's blood is 
stronger than all the sins and wickedness of the whole 
world, and therewith quiet thyself, and thereunto commit 
thyself and bless thyself in all temptation, (namely at the 
hour of death) with that holy candle. Or else perishest 
thou, though thou hast a thousand holy candles about thee, 
a hundred ton of holy water, a ship full of pardons, a 
cloth-sack full of friar's coats, and all the ceremonies in 
the world, and all the good works, deservings, and merits 
of all the men in the world, be they, or were they, never so 
holy, God's word only lasteth for ever, and that which he 
hath sworn doth abide, when all other things perish. So 
long as thou findest any consent in thine heart unto the law 
of God, that it is righteous and good, and also displeasure 
that thou canst not fulfil it, despair not, neither doubt but 
that God's Spirit is in thee, and that thou art chosen for 
Christ's sake to the inheritance of eternal life. 

And again, (Rom. iii.) We suppose that a man is justified 
through faith, without the deeds of the law, And likewise 
(Rom. iv.) we say that faith was reckoned to Abraham for 
righteousness. Also (Rom. v.) seeing that we are justified 
through faith, we are at peace with God, Also (Rom x.) 
with the heart doth a man believe to be made righteous. 
Also (Gal. iii.) received ye the Spirit by the deeds of the 
law, or by hearing of the faith? Doth he which
ministereth the Spirit unto you, and worketh miracles among 
you, do it of the deeds of the law, or by hearing of faith? 
Even as Abraham believed God, and it was reckoned [to] 
him for righteousness. Understand therefore (saith he) 
that the children of faith are the children of Abraham. 
For the Scripture saw before that God would justify the 
heathen or gentiles by faith, and shewed before glad tidings 
unto Abraham, In thy seed shall all nations be blessed. 
Wherefore they which are of faith are blessed, that is to 
wit, made righteous with righteous Abraham. For as 
many as are of the deeds of the law, are under curse. For 
it is written (saith he) Cursed is every man that continueth 
not in all things which are written in the book of the law, 
to fulfil them. 

Also, (Gal. ii.) where he resisted Peter in the face, he 
saith, We which are Jews by nation, and not sinners of the 
Gentiles, know that a man is not justified by the deeds of 
the law, but by the faith of Jesus Christ, and have therefore 
believed on Jesus Christ, that we might be justified by the 
faith of Christ, and not by the deeds of the law, for by the 
deeds of the law shall no flesh be justified. Item, in the 
same place he saith, Touching that I now live, I live in the 
faith of the son of God, which loved me, and gave himself 
for me; I despise not the grace of God, for if
righteousness come by the law, then Christ is dead in vain. And 
of such like ensamples are all the Epistles of Paul full.
Mark how Paul laboureth with himself to express the
exceeding mysteries of faith, in the Epistle to the Ephesians, 
and in the Epistle to the Colossians. Of these and many 
such like texts, are we sure that the forgiveness of sins 
and justifying [are] appropriate unto faith only, without the 
adding to of works. 

Take forth also the similitude that Christ maketh, (Mat. 
vii.) A good tree bringeth forth good fruit, and a bad tree 
bringeth forth bad fruit. There seest thou, that the fruit 
maketh not the tree good, but the tree the fruit; and that 
the tree must aforehand be good, or be made good, ere it 
can bring forth good fruit. As he also saith, (Matt. xii.) 
Either make the tree good and his fruit good also, either 
make the tree bad and his fruit bad also. How can ye 
speak well while ye yourselves are evil? So likewise is this 
true, and nothing more true, that a man before all good 
works must first be good, and that it is impossible that 
works should make him good, if he were not good before, 
ere he did good works. For this is Christ's principle and 
(as we say) a general rule. How can ye speak well, 
while ye are evil? so Likewise how can ye do good, while 
ye are evil? 

This is therefore a plain, and a sure conclusion not to 
be doubted of, that there must be first in the heart of a 
man before he do any good works, a greater and a more 
precious thing than all the good works in the world, to
reconcile him to God, to bring the love and favour of God to 
him, to make him love God again, to make him righteous 
and good in the sight of God, to do away his sin, to
deliver him and loose him out of that captivity wherein he 
was conceived and born, in which he could neither love 
God, neither the will of God. Or else how can he work 
any good work that should please God, if there were not 
some supernatural goodness in him given of God freely, 
whereof that good work must spring? even as a sick man 
must first be healed or made whole, ere he can do the deeds 
of an whole man; and as the blind man must first have sight 
given him ere he can see; and he that hath his feet in 
fetters, gives, or stocks, must first be loosed, ere he can go, 
walk or run, and even as they which thou readest of in 
the gospel, that they were possessed of the devils, could 
not laud God till the devils were cast out. 

That precious thing which must be in the heart, ere a 
man can work any good work, is the word of God, which 
in the gospel preacheth, profereth, and bringeth unto all 
that repent and believe, the favour of God in Christ.
Whosoever heareth the word and believeth it, the same is thereby 
righteous, and thereby is given him the Spirit of God, 
which leadeth him unto all that is the will of God, and is 
loosed from the captivity and bondage of the devil, and 
his heart is free to love God, and hath lust to do the will 
of God. Therefore it is called the word of life, the word 
of grace, the word of health, the word of redemption, 
the word of forgiveness, and the word of peace; he that 
heareth it not, or believeth it not, can by no means be made 
righteous before God. This confirmeth Peter in the xvth
of the Acts, saying that God through faith doth purify the 
hearts. For of what nature soever the word of God is, 
of the same nature must the hearts be which believe 
thereon, and cleave thereunto. Now is the word living, 
pure, righteous and true, and even so maketh it the hearts 
of them that believe thereon. 


If it be said that Paul (when he saith in the iiird to the 
Romans, No flesh shall be, or can be justified by the 
deeds of the law) meaneth it of the ceremonies or
sacrifices, it is a lie, verily. For it followeth immediately, — 
by the law cometh the knowledge of sin. Now are 
they not the ceremonies that utter sin, but the law of
commandments. In the ivth he saith The law causeth wrath, 
which cannot be understood of the ceremonies, for they 
were given to reconcile the people to God again after they 
had sinned. If, as they say, the ceremonies which were 
given to purge sin and to reconcile, justify not, neither
bless but temporally, much more the law of
commandments justifieth not. For that which proveth a man sick, 
healeth him not, neither doth the cause of wrath bring to 
favour, neither can that which damneth save a man. When 
the mother commandeth her child but even to rock the 
cradle, it grudgeth, the commandment doth but utter the 
poison that lay hid, and setteth him at bate [contention] 
with his mother, and maketh him believe she loveth him not. 

These commandments also, Thou shalt not covet thy 
neighbour's house, thou shall not lust, desire, or wish after 
thy neighbour's wife, servant, maid, ox, or ass, or whatsoever 
pertaineth unto thy neighbour, give me not power so to do, 
but utter the poison that is in me and damn me because I 
cannot so do, and prove that God is wrath with me, seeing 
that his will and mine are so contrary. Therefore saith 
Paul (Gal. iii.) If there had been given such a law that 
could have given life, then no doubt righteousness had 
come by the law, but the Scripture concludeth all under sin 
(saith he) that the promise might be given unto them that 
believe through the faith that is in Jesus Christ.

The promises, when they are believed, are they that
justify, for they bring the Spirit which looseth the heart, 
giveth lust to the law, and certifieth us unto the good-will 
of God unto usward. if we submit ourselves unto God 
and desire him to heal us, he will do it, and will in the 
mean time (because of the consent of the heart unto the 
law) count us for full whole, and will no more hate us, but 
pity us, cherish us, be tender hearted to us, and love us as 
he doth Christ himself. Christ is our Redeemer, Saviour, 
peace, atonement and satisfaction, and hath made amends 
or satisfaction to Godward for all the sin which they that 
repent (consenting to die law and believing the promises) 
do, have done, or shall do. So that if through fragility 
we fall a thousand times in a day, yet if we do repent 
again, we have alway mercy laid up for us in store in Jesus 
Christ our Lord. 


What shall we say then to those Scriptures which go so 
sore upon good works? As we read Matt. xxv. I was 
an hungred, and ye gave me meat, \&c. and such like. 
Which all sound as though we should be justified, and
accepted unto the favour of God in Christ through good 
works. Thiswise answer I, Many there are, which when 
they hear or read of faith, at once they consent thereunto, 
and have a certain imagination or opinion of faith, as when 
a man telleth a story or a thing done in a strange land, that 
pertaineth not to them at all. Which yet they believe, and 
tell as a true thing. And this imagination or opinion they 
call faith. They think no farther than that faith is a thing 
which standeth in their own power to have, as to do other
natural works which men work; but they feel no manner 
[of] working of the Spirit, neither the terrible sentence 
of the law, the fearful judgments of God, the horrible 
damnation and captivity under Satan. Therfore as soon 
as they have this opinion, or imagination in their hearts, 
that saith, Verely this doctrine seemeth true, I believe it is 
even so. Then they think that the right faith is there. 
But afterward when they feel in themselves, and also see in 
other, that there is none alteration, and that the works 
follow not, but that they are altogether even as before, and 
abide in their old estate; then think they that faith is not 
sufficient, but that it must be some greater thing than faith 
that should justify a man. 

So fall they away from faith again, and cry, saying, Faith 
only justifieth not a man, and maketh him acceptable to 
God. If thou ask them, Wherefore? they answer, See 
how many there are that believe, and yet do no more than 
they did before. These are they which Jude in his
epistle calleth dreamers which deceive themselves with their 
own fantasies. For what other thing is their imagination 
which they call faith, than a dreaming of faith, and an 
opinion of their own imagination wrought without the grace 
of God? These must needs be worse at the latter end 
than at the beginning. These are the old vessels that rent 
when new wine is poured into them; (Mat. ix.) that is, they 
hear God's word, but hold it not, and therefore wax 
worse than they were before. But the right [faith] springeth 
not of man's fantasy, neither is it in any man's power to 
obtain it, but is altogether the pure gift of God poured 
into us freely, without all manner [of] doing of us, without 
deserving and merits, yea aud without seeking for of us. 
And is (as saith Paul in the second to the Ephesians) even 
God's gift and grace purchased through Christ.
Therefore is it mighty in operation, full of virtue, and ever 
working, which also reneweth a man, and begetteth him 
afresh, altereth him, changeth him, and turneth him
altogether into a new nature and conversation, so that a man 
feeleth his heart altogether altered and changed, and far
otherwise disposed than before, and hath power to love that 
which before he could not but hate, and delighteth in that 
which before he abhorred, and hateth that which before he 
could not but love. And it setteth the soul at liberty, and 
maketh her free to follow the will of God, and doth to the 
soul even as health doth unto the body; after that a man is 
pined and wasted away with a long soaking disease, the legs 
cannot bear him, he cannot lift up his hands to help himself, 
his taste is corrupt, sugar is bitter in his mouth, his stomach 
abhorreth [meat,] longing after slibbersause and swash, at 
which a whole stomach is ready to cast his gorge. When 
health cometh, she changeth and altereth him clean, giveth 
him strength in all his members, and lust to do of his own 
accord that which before he could not do, neither could 
suffer that any man exhorted him to do, and hath now 
lust in other things, and his members are free and at 
liberty, and have power to do of their own accord all 
things, which belong to an whole man to do, which afore 
they had no power to do, but were in captivity and 
bondage. So likewise in all things doth right faith to the 
soul. 

The Spint of God accompanieth faith, and bringeth 
with her light, wherewith a man beholdeth himself in the 
law of God, and seeth his miserable bondage and
captivity, and humbleth himself, and abhorreth himself; she 
bringeth God's promises of all good things in Christ. God 
worketh with his word, and in his word. And as his word 
is preached, faith rooteth herself in the hearts of the elect, 
and as faith entereth, and the word of God is believed, the 
power of God looseth the heart from the captivity and 
bondage under sin, and knitteth and coupleth him to God, 
and to the will of God; altereth him, changeth him clean, 
fashioneth, and forgeth him anew, giveth him power to 
love, and to do that which before was impossible for him 
either to love or do, and turneth him unto a new nature, 
so that he loveth that which he before hated, and hateth that 
which he before loved; and is clean altered, and changed, 
and contrary disposed; and is knit and coupled fast to 
God's will, and naturally bringeth forth good works, that 
is to say, that which God commandeth to do, and not 
things of his own imagination. And that doth he of his 
own accord, as a tree bringeth forth fruit of her own accord.
And as thou needest not to bid a tree to bring forth fruit, 
so is there no law put unto him that believeth, and is
justified through faith (as saith Paul in the first Epistle to 
Timothy, the first chapter). Neither is it needful, for 
the law of God is written and graved in his heart, and his 
pleasure is therein. And as without commandment, but 
even of his own nature, he eateth, drinketh, seeth, heareth, 
talketh, and goeth, even so of his own nature, without
coaction or compulsion of the law, bringeth he forth good 
works. And as a whole man, when he is athirst, tarrieth 
but for drink, and when he hungreth abideth but for meat, 
and then drinketh and eateth naturally; even so is the 
faithful ever athirst, and an hungred after the will of God, 
and tarrieth but for occasion. And whensoever an
occasion is given, he worketh naturally the will of God: for 
this blessing is given to all them that trust in Christ's 
blood, that they thirst and hunger to do God's will He 
that hath not this faith, is but an unprofitable babler of 
faith and works, and wotteth neither what he bableth, nor 
what he meaneth, or whereunto his words pertain: for he 
feeleth not the power of faith, nor the working of the 
Spirit in his heart, but interpreteth the Scriptures, which 
speak of faith and works, after his own blind reason and 
foolish fantasies, and not of any feeling that he hath in his 
heart; — as a man rehearseth a tale of another man's 
mouth, and wotteth not whether it be so or no, as he saith,
nor hath any experience of the thing itself. Now doth the 
Scripture ascribe both faith and works, not to us, but 
to God only, to whom they belong only, and to whom 
they are appropriate, whose gift they are, and the proper 
work of his Spirit. 

Is it not a froward and perverse blindness, to teach how 
a man can do nothing of his own self, and yet
presumptuously take upon them the greatest and highest work of 
God, even to make faith in themselves of their own power, 
and of their own false imagination and thoughts?
Therefore, I say, we must despair of ourselves, and pray God 
(as Christ's apostles did) to give us faith, and to encrease 
our faith. When we have that, we need no other thing 
more. For she bringeth the Spirit with her, and he not 
only teacheth us all things, but worketh them also mightily 
in us, and carrieth us through adversity, persecution, death, 
and hell, unto heaven and everlasting life. 


Mark diligently, therefore, seeing we are come to
answer. The Scripture, (because of such dreams and feigned 
faith's sake) useth such manner of speakings of works, 
not that a man should thereby be made good to God-ward, 
or Justified; but to declare unto other, and to take of 
other the difference between false feigned faith, and right 
faith. For where right faith is, there bringeth she forth 
good works; if there follow not good works, it is (no 
doubt) but a dream and an opinion or feigned faith. 

Wherefore look, as the fruit maketh not the tree good, 
but declareth and testifieth outwardly that the tree is good, 
(as Christ saith) Every tree is known by his fruit; even 
so shall ye know the right faith by her fruit. 

Take for an ensample Mary that anointed Christ's feet. 
(Luke vii.) When Simon which had Christ to his house 
had condemned her, Christ defended her, and justified her, 
saying, Simon, I have a certain thing to say unto thee, and 
he said, Master, say on, There was a certain lender which 
had two debtors, the one owed five hundred pence, and 
the other fifty. When they had nothing to pay, he forgave 
both. Which of them, tell me, will love him most?
Simon answered and said, I suppose that he to whom he
forgave most. And he said to him. Thou hast truly judged. 
And he turned him to the woman, and said unto Simon, 
Seest thou this woman? I entered into thine house, and 
thou gavest me no water to my feet; but she hath washed 
my feet with tears, and wiped them with the hairs of her 
head. Thou gavest me no kiss, but she, since the time I 
came in, hath not ceased to kiss my feet. My head with
oil thou hast not anointed. And she hath anointed my feet
with costly and precious ointment. Wherefore I say unto 
thee, many sins are forgiven her, for she loveth much. To 
whom less is forgiven, the same doth love less, \&c. 
Hereby, see we, that deeds and works are but outward 
signs of the inward grace of the bounteous and plenteous 
mercy of God, freely received without all merits of deeds, 
yea, and before all deeds. Christ teacheth to know the 
inward faith and love, by the outward deeds. Deeds are the 
fruits of love, and love is the fruit of faith. Love, and 
also the deeds, are great or small, according to the
proportion of faith. Where faith is mighty and strong, there 
is love fervent, and deeds plenteous, and done with
exceeding meekness: where faith is weak, there is love cold, 
and the deeds few, and seldom bear flowers and blossoms 
in winter. 

Simon believed, and had faith yet but weakly, and
accordlag to the proportion of his faith loved coldly, and had deeds 
thereafter: he had Christ unto a simple and bare feast only, 
and received him not with any great humanity. But Mary 
had a strong faith, and therefore burning love, and
honourable deeds, done with exceeding profound and deep
meekness. On the one side she saw herself clearly in the law, 
both in what danger she was in, and cruel bondage
under sin, her horrible damnation, and also the fearful
sentence and judgment of God upon sinners. On the other
side she heard the gospel of Christ preached, and in the
promises she saw with eagles' eyes the exceeding abundant 
mercy of God that passeth all utterance of speech, which 
is set forth in Christ for all meek sinners which knowledge 
their sins; and she believed the word of God mightily, and 
glorified God over his mercy and truth; and being
overcome and overwhelmed with the unspeakable, yea, and
incomprehensible abundant riches of the kindness of God, 
did inflame and burn in love; yea, was so swollen in love, 
that she could not abide, nor hold, but must break out; and 
was so drunk in love that she regarded nothing, but even to 
utter the fervent and burning love of her heart only; she 
had no respect to herself, though she was never so great and 
notable a sinner; neither to the curious hypocrisy of the 
Pharisees, which ever disdain weak sinners; neither the 
costliness of her ointment; but with all humbleness did run
unto his feet; washed them with the tears of her eyes, and wiped 
them with the hairs of her head, and anointed them with 
her precious ointment; yea, and would no doubt have run 
into the ground under his feet, to have uttered her love
toward him; yea would have descended down into hell, if it 
had been possible. Even as Paul in the ixth chapter of 
his Epistle to the Romans was drunk in love, and over- 
whelmed with the plenteousness of the infinite mercy of 
God, (which he had received in Christ unsought for) wished 
himself banished from Christ and damned, to save the Jews, 
if it might have been. For as a man feeleth God in himself, 
so is he to his neighbour. 

Mark another thing also. We, for the most part, because 
of our grossness in all our knowledge, proceed from that which 
is last and hindmost, unto that which is first; beginning at 
the latter end, disputing and making our arguments back- 
ward. We begin at the effect, and work and proceed unto 
the natural cause. As for an ensample: we first see the moon 
dark, and then search the cause, and find that the putting of 
the earth between the sun and the moon is the natural cause 
of the darkness, and that the earth stoppeth the light. Then 
dispute we backward, saying, the moon is darkened, there- 
fore is the earth directly between the sun and moon. Now 
yet is not the darkness of the moon the natural cause that 
the earth is between the sun and the moon, but the effect 
thereof, and cause declarative, declaring and leading us 
unto the knowledge, how that the earth is between the sun 
and the moon directly, and causeth the darkness, stopping 
the light of the sun from the moon. And contrarywise, the 
being of the earth directly between the sun and the moon 
is the natural cause of the darkness. Likewise he hath a son, 
therefore is he a father, and yet the son is not cause of the 
father, but contrarywise. Notwithstanding, the son is the 
cause declarative, whereby we know that the other is a 
father. After the same manner here, many sins are for- 
given her, for she loveth much, thou mayest not understand 
by the word for, that love is the natural cause of the for- 
giving of sins, but declareth it only; and contrariwise, the 
forgiveness of sins is the natural cause of love. 

The works declare love. And love declareth that there is 
some benefit and kindness shewed, or else would there be 
no love. Why worketh one and another not? or one more 
than another? because that one loveth and the other not, or 
that the one loveth more than the other. Why loveth one 
and another not, or one more than another? because that 
one feeleth the exceeding love of God in his heart and another
not, or that one feeleth it more than another. Scripture speak-
eth after the most gross manner. Be diligent therefore
that thou be not deceived, with curiousness, for men of no 
small reputation have been deceived with their own sophistry. 


Hereby now seest thou, that there is great difference 
between being righteous and good in a man's self, and de- 
claring and uttering righteousness and goodness. The 
faith only maketh a man safe, good, righteous, and the 
friend of God, yea, and the son and the heir of God, and of 
all his goodness, and possesseth us with the Spirit of God.
The work declareth the self faith and goodness. Now
useth the Scripture the common manner of speaking,
and the very same that is among the people. As when a fa- 
ther saith to his child, Go, and be loving, merciful, and good 
to such or such a poor man, he biddeth him not therewith 
to be made merciful, kind, and good, but to testify and 
declare the goodness that is in him already, with the out- 
ward deed, that it may break out to ihe profit of other, and 
that other may feel it which have need thereof. 

After the same manner shalt thou interpret the Scriptures 
which make mention of works: that God thereby will that 
we show forth that goodness which we have received by 
faith, and let it break forth and come to the profit of other, 
that the false faith may be known and weeded out by the 
roots. For God giveth no man his grace that he should let 
it lay still and do no good withal, but that he should en- 
crease it and multiply it with lending it to others, and with 
open declaring of it with the outward works, provoke and 
draw others to God. As Christ saith in Matthew the vth. 
chapter, Let your light so shine in the sight of men, that they 
may see your good works, and glorify your Father which is 
in heaven. Or else where it is a treasure digged in the 
ground, and hid wisdom, in which what profit is there?

Moreover therewith the goodness, favour, and gifts of 
God which are in thee, not only shall be known unto other, 
but also unto thine own self, and thou shalt be sure that thy 
faith is right, and that the true Spirit of God is in thee, and 
that thou art called and chosen of God unto eternal life, 
and loosed from the bonds of Satan, whose captive thou 
wast; as Peter exhorteth in the First of his Second Epistle, 
through good works to make our calling and election 
(wherewith we are called and chosen of God) sure. For 
how dare a man presume to think that his faith is right, 
and that God's favour is on him, and that God's Spirit is 
in him, when he feeleth not the working of the Spirit, nei- 
ther himself disposed to any godly thing? Thou canst ne- 
ver know or be sure of thy faith but by the works, if works 
follow not, yea, and that of love, without loooking after any 
reward, thou mayest be sure that thy faith is but a dream, 
and not right, and even the same that James called in his 
Epistle, the second chapter, dead faith and not justifying. 

Abraham through works, (Genesis xxiind) was sure of his
faith to be right, and that the true fear of God was in him, 
when he had offered his son, as the Scripture saith, Now 
know I that thou fearest God, that is to say, now is it 
open and manifest that thou fearest God, inasmuch as thou 
hast not spared thy only son for my sake. 


So now by this abide sure and fast, that a man inwardly 
in the heart and before God, is righteous and good through 
faith only, before all works: notwithstanding, yet out- 
wardly and openly before the people, yea, and before him- 
self, is he righteous through the work, that is, he know- 
eth and is sure through the outward work that he is a 
true believer, and in the favour of God, and righteous and 
good through the mercy of God: that thou mayest call the 
one an open and an outward righteousness, and the other, 
an inward righteousness of the heart; so yet, that thou un- 
derstand by the outward righteousness, no other thing save 
the fruit that followeth, and a declaring of the inward 
justifying and righteousness of the heart, and not that it 
maketh a man righteous before God, but that he must be first 
righteous before him in the heart; even as thou mayest call the 
fruit of the tree the outward goodness of the tree which fol- 
loweth and uttereth the inward natural goodness of the tree. 

This meaneth James in his Epistle, where he saith. Faith 
without works is dead, that is, if works follow not, it is a 
sure and an evident sign that there is no faith in the heart, but 
a dead imagination and dream, which they falsely call faith 

Of the same wise is this saying of Christ to be under- 
stood: Make you friends of the unrighteous mammon, that is, 
shew your faith openly and what ye are within in the heart, 
with outward giving and bestowing your goods on the poor, 
that ye may obtain friends; that is, that the poor on whom 
thou hast showed mercy may at the day of judgment, tes- 
tify and witness of thy good works. That thy faith and what 
thou wast within thy heart before God, may there appear 
by thy fruits openly to all men. For unto the right be- 
lieving shalt all things be comfortable, and unto consolation, 
at that terrible day: and contrariwise unto the unbelieving, 
all things shall be unto desperation and confusion, and 
every man shall be judged openly and outwardly, in the 
presence of all men, according to their deeds and works. 
So that not without a cause thou mayest call them thy 
friends which testify at that day of thee, that thou livedst as 
a true and a right christian man, and followedst the steps 
of Christ in shewing mercy, as no doubt he doth which 
feeleth God merciful in his heart. And by the works is the 
faith known, that it was right and perfect. For the outward 
works can never please God, nor make friend, except they 
spring of faith. Forasmuch as Christ himself (Matt. vi. and 
vii.) disalloweth and casteth away the works of the Pharisees, 
yea, prophesying and working of miracles and casting out 
of devils, which we count and esteem for very excellent 
virtues, yet make they no friends with their works, while 
their hearts are false and impure, and their eye double. 
Now without faith is no heart true or eye single, so that 
we are compelled to confess that the works make not a man 
righteous or good, but that the heart must first be righteous 
and good, ere any good work proceed thence. 


Secondarily, all good works must be done free with a 
single eye, without respect of any thing, and that no profit 
be sought thereby. 

That commandeth Christ, where he saith, (Mat. x.) Freely 
have ye received, freely give again. For look, as Christ 
with all his works did not deserve heaven, for that was his 
already, but did us service therewith, and neither looked, 
nor sought his own profit, but our profit, and the honour 
of God the Father only; even so we, with all our works, 
may not seek our own profit neither in this world nor in 
heaven, but must, and ought, freely to work to honour God 
withal, and without all manner [of] respect, seek our 
neighbour's profit, and do him service. That meaneth 
Paul (Phil ii.) saying, Be minded as Christ was, which 
being in the shape of God, equal unto God, and even very 
God, laid that apart, that is to say hid it, and took on him 
the form and fashion of a servant. That is, as concerning 
himself he had enough, that he was full and had all plen- 
teousness of the Godhead, and in all his works sought our 
profit, and became our servant. 

The cause is: forasmuch as faith justifieth and putteth 
away sin in the sight of God, bringeth life, health, and the 
favour of God, maketh us the heirs of God, poureth the 
Spirit of God into our souls, and filleth us with all godly 
fulness in Christ; it were too great a shame, rebuke and 
wrong unto the faith, yea to Christ's blood, if a man would 
work any thing to purchase that wherewith faith hath en- 
dued him already, and God hath given him freely. Even 
as Christ had done rebuke and shame unto himself, if he 
would have done good works, and wrought to have been 
made thereby God's son and heir over all, which thing he 
was already. Now doth faith make us the sons or children 
of God. (John i.) He gave them might or power to be 
the sons of God, in that they believed on his name. If 
we be sons, so are we also heirs. (Rom. viii. and Gal iv.) 
How can or ought we then to work for to purchase that 
inheritance withal, whereof we are heirs already by faith? 

What shall we say then to those Scriptures, which sound 
as though a man should do good works, and live well for 
heaven's sake or eternal reward? As these are, Make 
you friends of the unrighteous mammon. And (Mat. vii.) 
Gather you treasures together in heaven. Also (Mat. 
xix.) If thou wilt enter into life, keep the commandments: 
and such like. This say I, that they which understand 
not, neither feel in their hearts what faith meaneth, talk 
and think of the reward, even as they do of the work; 
neither suppose they that a man ought to work, but in a 
respect to the reward. For they imagine, that it is in the 
kingdom of Christ, as it is in the world among men, that 
they must deserve heaven with their good works. Howbeit 
their thoughts are but dreams and false imaginations. Of 
these men speaketh Malachi (chap. i.) Who is it among 
you that shutteth a door for my pleasure for nought, that is 
without respect of reward? These are servants that 
seek gains and vantage, hirelings and day labourers, which 
here on earth receive their rewards, as the Pharisees with 
their prayers and fastings. (Mat. v.) 

But on this wise goeth it with heaven, with everlasting 
life and eternal reward: likewise as good works naturally 
follow faith (as it is above rehearsed) so that thou needest 
not to command a true believer to work, or to compel him 
with any law, for it is unpossible that he should not work; 
he tarrieth but for an occasion; he is ever disposed of him- 
self, thou needest but to put him in remembrance, and 
that to know the false faith from the true. Even so naturally 
doth eternal life follow faith and good living, without seeking 
for, and is impossible that it should not come, though no 
man thought thereon. Yet is it rehearsed in the Scripture, 
alleged and promised to know the difference between a 
false believer and a true believer, and that every man may 
know what followeth good living naturally and of itself, 
without taking thought for it. 

Take a gross ensample: hell, that is, everlasting death, is 
threatened unto sinners, and yet followeth it sin naturally 
without seeking for. For no man doth evil to be damned 
therefore, but had rather avoid it. Yet there the one fol- 
loweth the other naturally, and though no man told or 
warned him of it, yet should the sinner find it and feel it. 
Nevertheless it is therefore threatened, that men may know 
what followeth evil living. Now then as after evil living 
followeth his reward unsought for, even so after good living 
followeth his reward naturally unsought for, or unthought 
upon. Even as when thou drinkest wine, be it good or 
bad, the taste followeth of itself, though thou therefore 
drink it not. Yet testifieth the Scripture, and it is true, 
that we are by inheritance heirs of damnation; and that 
ere we be born, we are vessels of the wrath of God, and 
full of that poison whence naturally all sins spring; and 
wherewith we cannot but sin, which thing the deeds that 
follow (when we behold ourselves in the glass of the law of 
God) do declare and utter, kill our consciences, and show 
us what we were and wist not of it, and certifieth us that 
we are heirs of damnation. For if we were of God we 
should cleave to God, and lust after the will of God. But 
now our deeds compared to the law declare the contrary, 
and by our deeds we see ourselves, both what we be and 
what our end shall be. 

So now thou seest that life eternal and all good things 
are promised unto faith and belief; so that he that be- 
lieveth on Christ shall be safe. Christ's blood hath pur- 
chased life for us, and hath made us the heirs of God; so 
that heaven cometh by Christ's blood. If thou wouldst 
obtain heaven with the merits and deservings of thine own 
works, so didst thou wrong, yea, and shamedst the blood 
of Christ, and unto thee were Christ dead in vain. Now 
is the true believer heir of God by Christ's deservings, yea, 
and in Christ was predestinate and ordained unto eternal 
life before the world began. And when the gospel is 
preached unto us, we believe the mercy of God, and in be- 
lieving we receive the Spirit of God, which is the earnest 
of eternal life, and we are in eternal life already, and feel 
already in our hearts the sweetness thereof, and are over- 
come with the kindness of God and Christ, and therefore 
love the will of God, and of love are ready to work freely, 
and not to obtain that which is given us freely, and whereof 
we are heirs already. 

Now when Christ saith, Make you friends of unrighteous 
Mammon: Gather you treasure together in heaven, and 
such like: thou seest that the meaning and intent is no 
other but that thou shouldst do good, and so will it follow 
of itself naturally, without seeking and taking of thought, 
that thou shalt find friends and treasure in heaven, and 
receive a reward. So let thine eye be single, and look 
unto good living only, and take no thought for the reward, 
but be content. Forasmuch as thou knowest and art sure 
that the reward and all things contained in God's promises 
follow good living naturally; and thy good works do but 
testify only, and certify thee that the Spirit of God is in 
thee, whom thou hast received in earnest of God's truth; 
and that thou art heir of all the goodness of God, and that 
all good things are thine already, purchased by Christ's blood, 
and laid up in store against that day, when every man shall 
receive according to his deeds, that is according as his 
deeds declare and testify, what he is or was. For they 
that look unto the reward, are slow, false, subtle and crafty 
workers, and love the reward more than the work, yea, 
hate the labour, yea, hate God which commandeth the 
labour, and are weary both of the commandment, and 
also of the Commander, and work with tediousness. But 
he that worketh of pure love, without seeking of reward, 
worketh truly. 

Thirdly, that not the saints, but God only receiveth us 
into eternal tabernacles, is so plain and evident, that it 
needeth not to declare or prove it. How shall the saints 
receive us into heaven, when every man hath need for him- 
self that God only receive him to heaven, and every man 
hath scarce for himself? As it appeareth by the five wise 
virgins, (Mat. xxv.) which would not give of their oil unto 
the unwise virgins. And Peter saith in the ivth of his 
first Epistle, that the righteous is with difficulty saved. 
So seest thou the saying of Christ, Make you friends, and 
so forth, that they may receive you into everlasting taber- 
nacles, pertaineth not unto the saints which are in heaven, 
but is spoken of the poor and needy which are here present 
with us on earth; as though he should say. What, buildest 
thou churches, foundest abbeys, chauntries and colleges, 
in the honour of saints, to my Mother, St. Peter, Paul, and 
saints that be dead, to make of them thy friends? They 
need it not, yea, they are not thy friends, but theirs which 
lived then when they did, of whom they were holpen. 
Thy friends are the poor, which are now in thy time, and 
live with thee; thy poor neighbours which need thy help 
and succour. Them make thy friends with thy unrighteous
mammon, that they may testify of thy faith, and thou mayest
know and feel that thy faith is right and not feigned.

Unto the second, such receiving into everlasting habi- 
tations is not to be understood that men shall do it. For 
many, to whom we shew mercy and do good, shall not 
come there; neither skilleth it so we meekly and lovingly 
do our duty, yea, it is a sign of strong faith and fer- 
vent love, if we do well to the evil, and study to draw them 
to Christ in all that lieth in us. But the poor give us an 
occasion to exercise our faith, and the deeds make us feel 
our faith, and certify us and make us sure that we are safe, 
and are escaped and translated from death unto life, and 
that we are delivered and redeemed from the captivity and 
bondage of Satan, and brought into the liberty of the sons 
of God, in that we feel lust and strength in our heart to 
work the will of God. And at that day shall our deeds ap- 
pear and comfort our hearts, witness our faith and trust, 
which we now have in Christ, which faith shall then keep us 
from shame, as it is written, None that believeth in him shall 
be ashamed, (Rom. ix.) So that good works help our faith, 
and make us sure in our consciences, and make us feel the 
mercy of God. Notwithstanding, heaven, everlasting life, 
joy eternal, faith, the favour of God, the Spirit of God, lust 
and strength unto the will of God, are given us freely of the 
bounteous and plenteous riches of God, purchased by 
Christ, without our deservings, that no man should rejoice 
but in the Lord only. 

For a farther understanding of this gospel, here may be 
made three questions, What mammon is, Why it is called un- 
righteous, and after what manner Christ biddeth us coun- 
terfeit and follow the unjust and wicked steward, which with 
his Lord's damage provided for his own profit and vantage, 
which thing no doubt is unrighteous and sin? 

First, mammon is an Hebrew word, and signifies riches 
or temporal goods, and namely, all superfluity, and all that 
is above necessity, and that which is required unto our ne- 
cessary uses, wherewith a man may help another without 
undoing or hurting himself; for Hamon, in the Hebrew 
speech, signifies a multitude or abundance, or many, and 
there hence cometh mahamon, or mammon, abundance or 
plenteousness of good or riches. 

Secondarily, it is called unrighteous mammon, not be- 
cause it is got unrighteously, or with usury, for of unrighteous 
gotten goods can no man do good works, but ought to re- 
store them home again. As it is said (Isaiah lxi.) I am a 
God that hateth offering that cometh of robbery; and Solo- 
mon (Prov. iii.) saith, Honour the Lord of thine own good. 
But therefore it is called unrighteous, because it is in un- 
righteous use. As Paul speaketh unto the Ephesians vth 
how that The days are evil though that God hath made 
them, and they are a good work of God's making. How- 
beit they are yet called evil, because that evil men use them 
amiss, and much sin, occasions of evil, peril of souls are 
wrought in them. Even so are riches called evil because 
that evil men bestow them amiss and misuse them. For 
where riches are there goeth it after the common proverb, 
He that hath money hath what him listeth. And they cause 
flghting, stealing, laying await, lying, flattering, and all un- 
happiness against a man's neighbour. For all men hold on 
riches' part. 

But singularly before God is it called unrighteous mam- 
mon, because it is not bestowed and ministered unto our 
neighbour's need. For if my neighbour need and I give him 
not, neither depart liberally with him of that which I have, 
then withhold I from him unrighteously that which is his 
own. For as much as I am bounden to help him by the law 
of nature, which is Whatsoever thou wouldest that another 
did to thee, that do thou also to him; and Christ (Matt. v.) 
Give to every man that desireth thee; and John in his first 
Epistle, If a man have this world's good and see his brother 
need, how is the love of God in him? And this unrighteous- 
ness in our mammon see very few men: because it is spi- 
ritual, and in those goods which are gotten most truly and 
justly, which beguile men. For they suppose they do no man 
wrong in keeping them, in that they got them not with 
stealing, robbing, oppression, and usury, neither hurt any 
man now with them. 

Thirdly, many have busied themselves in studying what, 
or who, this unrighteous steward is, because that Christ so 
praiseth him. But shortly and plainly this is the answer. That 
Christ praiseth not the unrighteous steward, neither setteth 
him forth to us to counterfeit because of his unrighteousness, 
but because of his wisdom only, in that he, with unright, so 
wisely provided for himself. As if I would provoke another 
to pray or study, say, The thieves watch all night to rob 
and steal, why canst not thou watch to pray and to study? 
here praise not I the thief and murderer for their evil doing, 
but for their wisdom, that they so wisely and diligently wait 
on their unrighteousness. Likewise when I say miss women 
tire themselves with gold and silk to please their lovers: 
what wilt not thou garnish thy soul with faith to please 
Christ? here praise I not whoredom, but the diligence 
which the whore misuseth. 

On this wise Paul also (Rom. v.) likeneth Adam and 
Christ together, saying that Adam was a figure of Christ. 
And yet of Adam have we but pure sin, and of Christ grace 
only, which are out of measure contrary. But the similitude 
or likeness standeth in the original birth, and not in the vir- 
tue and vice of the birth. So that as Adam is father of all 
sin, so is Christ father of all righteousness: and as all sinners 
spring of Adam, even so all righteous men and women 
spring of Christ. After the same manner is here the un- 
righteous steward an ensample unto us, in his wisdom and 
diligence only, in that he provided so wisely for himself, 
that we with righteousness should be as diligent to provide 
for our souls as he with unrighteousness provided for his 
body. 

Likewise mayest thou solve all other texts which sound 
as though it were between us and God, as it is in the world, 
where the reward is more looked upon than the labour; 
yea, where men hate the labour, and work falsely with the 
body and not with the heart, and no longer than they are 
looked upon, that the labour may appear outward only. 


When Christ saith (Matt. v.) Blessed are ye when 
they rail on you, and persecute you, and say all manner [of] 
evil sayings against you, and yet lie, and that for my sake, 
rejoice and be glad, for your reward is great in heaven. 
Thou mayest not imagine that our deeds deserve the joy 
and glory that shall be given unto us, for then Paul saith 
(Rom, xi.) Favour were not favour, I cannot receive it of 
favour and of the bounties of God freely, and by deserving 
of deeds also. But believe as the gospel, glad tidings and 
promises of God say unto thee, that far Christ's blood sake 
only, through faith, God is at one with thee, and thou re- 
ceived to mercy, and art become the son of God and heir 
annexed with Christ, of all the goodness of God, the earnest 
whereof is the Spirit of God poured into our hearts. Of 
which things the deeds are witnesses, and certify our con- 
sciences that our faith is unfeigned, and that the right 
Spirit of God is in us. For if I patiently suffer adversity and 
tribulation for conscience of God only, that is to say, be- 
cause I know God and testify the truth, then am I sure 
that God hath chosen me in Christ and for Christ's sake, 
and hath put in me his Spirit as an earnest of his promises, 
whose working I feel in mine heart, the deeds bearing wit- 
ness unto the same. Now is it Christ's blood only that de- 
served all the promises of God, and that which I suffer and 
do, is partly the curing, healing, and mortifying of my 
members, and killing of that original poison, wherewith I 
was conceived and born, that I might be altogether like 
Christ, and partly the doing of my duty to my neighbour, 
whose debtor I am of all that I have received of God; to 
draw him to Christ with all suffering, with all patience, and 
even with shedding my blood for him, not as an offering 
or merit for his sins, but as an ensample to provoke him. 
Christ's blood only puttelh away all the sin that ever 
was, is, or shall be, from them that are elect and repent, 
believing the gospel, that is to say, God's promises in
Christ. 


Again in the same vth chapter, Love your enemies, 
bless them that curse you, do well to them that hate you 
and persecute you, that ye may be sons of your father 
which is in heaven: for he maketh his sun shine upon 
evil, and on good, and sendeth his rain upon just 
and unjust. Not that our works make us the sons 
of God, but testify only, and certify our consciences, 
that we are the sons of God, and that God hath chosen 
us, and washed us in Christ's blood, and hath put his 
Spirit in us. And it followeth, If ye love them that love 
you, what reward have ye? do not the Publicans even the 
same? and if ye shall have favour to your friends only, 
what singular thing do ye? do not the Publicans even the 
same? Ye shall be perfect therefore, as your Father 
which is in heaven is perfect. That is to say, if that ye do 
nothing but that the world doth, and they which have the 
spirit of the world, whereby shall ye know that ye are the 
sons of God, and beloved of God more than the world? 
But, and if ye counterfeit, and follow God in well 
doing, then no doubt it is a sign that the Spirit of God 
is in you, and also the favour of God, which is not in 
the world, and that ye are inheritors of all the promises 
of God, and elect unto the fellowship of the blood of 
Christ. 

Also (Matt. vi.) Take heed to your alms, that ye do
it not in the sight of men, to the intent that ye would be 
seen of them, or else have ye no reward with your Father 
which is in heaven. Neither cause a trumpet to be blown 
afore thee when thou doest thine alms, as the hypocrites 
do in the synagogues, and in the streets, to be glorified of 
the world. But when thou doest thine alms, let not thy left 
hand know what thy right hand doth; that thy alms may 
be in secret, and thy Father which seeth in secret shall 
reward thee openly. This putteth us in remembrance of 
our duty, and sheweth what followeth good works; not 
that works deserve it, but that the reward is laid up for us 
in store, and we thereunto elect through Christ's blood, 
which the works testify: for, if we be worldly minded, 
and do our works as the world doth, how shall we know 
that God hath chosen us out of the world? But and if 
we work freely, without all manner [of] worldly respect, 
to shew mercy, and to do our duty to our neighbour, and 
to be unto him as God is to us, then are we sure that the 
favour and mercy of God is upon us, and that we shall 
enjoy all the good promises of God through Christ, which 
hath made us heirs thereof. 


Also, in the same chapter it followeth, When thou prayest, 
be not as the hypocrites, which love to stand and pray in the 
synagogues, and in the corners of the streets, for to be seen 
of men. But when thou prayest, enter into thy chamber, 
and shut thy door to, and pray to thy Father which is in 
secret, and thy Father which seeth in secret, shall reward 
thee openly. And likewise, when we fast (teacheth Christ 
in the same place) that we should behave ourselves that it 
appear not unto men how that we fast, but unto our Father 
which is in secret, and our Father which seeth in secret, 
shall reward us openly. These two texts do but declare 
what followeth good works, for eternal life cometh not by 
the deserving of works, but is, (saith Paul, in the vith to 
the Romans) the gift of God through Jesus Christ. Nei- 
ther do our works justify us. For except we were justified 
by faith which is our righteousness, and had the Spirit 
of God in us, to teach us, we could do no good work 
freely, without respect of some profit, either in this world, 
or in the world to come; neither could we have spiritual 
joy in our hearts in time of affliction, and mortifying of 
the flesh. 

Good works are called the fruits of the Spirit, (Gal v.) 
for the Spirit worketh them in us, and sometime fruits of 
righteousness, as in the second Epistle to the Corinthians 
and ixth chapter. Before all works therefore, we must have 
a righteousness within in the heart, the mother of all works, 
and from whence they spring. The righteousness of the 
Scribes and Pharisees, and of them that have the spirit of 
this world, is the glorious shew and outward shining of 
works. But Christ saith to us (Mat. v.) Except your 
righteousness exceed the righteousness of the Scribes and 
Pharisees ye cannot enter into the kingdom of heaven. It 
is righteousness in the world if a man kill not. But a chris- 
tian perceiveth righteousness if he love his enemy, even 
when he suffereth persecution and torment of him, and the 
pains of death, and mourneth more for his adversary's 
blindness than for his own pain, and prayeth God to open 
his eyes and to forgive him his sins, as did Stephen in the 
Acts of the Apostles the viith chapter, and Christ, Luke 
xxiii.

A Christian considereth himself in the law of God, 
and there putteth off him all manner [of] righteousness. 
For the law suffereth no merits, no deservings, no righte- 
ousness, neither any man to be justified in the sight of 
God. The law is spiritual and requireth the heart and 
commandments to be fulfilled with such love and obedience 
as was in Christ. If any fulfil all that is the will of God 
with such love and obedience, the same may be bold to 
sell pardons of his merits, and else not. 

A Christian therefore when he beholdeth himself in the 
law, putteth off all manner [of] righteousness, deservings 
and merits, and meekly and unfeignedly knowledgeth his 
sin and misery, his captivity and bondage in the flesh, his 
trespass and guilt, and is thereby blessed with the poor in 
spirit, (Mat. chap. v.) Then he mourneth in his heart, 
because he is in such bondage that he cannot do the will of 
God, and is an hungred and athirst after righteousness. 
For righteousness (I mean) which springeth out of Christ's 
blood, for strength to do the will of God. And turneth 
himself to the promises of God, and desireth him for his 
great mercy and truth, and for the blood of his son Christ 
to fulfil his promises and to give him strength. And thus 
his Spirit ever prayeth within him. He fasteth also not one 
day for a week, or a lent for an whole year, but professeth 
in his heart a perpetual soberness, to tame the flesh, and to 
subdue the body to the Spirit, until he wax strong in the 
Spirit, and grow ripe into a full righteousness after the 
fulness of Christ. And because this fulness happeneth 
not till the body be slain by death, a christian is ever a 
sinner in the law, and therefore fasteth and prayeth to God 
in the Spirit, the world seeing it not. Yet in the promises 
he is ever righteous through faith in Christ, and is sure that 
he is heir of all God's promises, the Spirit which he hath 
received in earnest, bearing him witness, his heart also, and 
his deeds testifying the same. 

Mark this then: To see inwardly that the law of God is 
so spiritual, that no flesh can fulfil it. And then for to 
mourn and sorrow and to desire, yea to hunger and thirst 
after strength to do the will of God from the ground of 
the heart, and, (notwithstanding all the subtilty of the devil, 
weakness and feebleness of the flesh, and wondering of 
the world,) to cleave yet to the promises of God, and to 
believe that for Christ's blood sake thou art received to the 
inheritance of eternal life, is a wonderful thing, and a 
thing that the world knoweth not of; but whosoever feeleth 
that, though he fall a thousand times in a day, doth yet rise 
again a thousand times, and is sure that the mercy of God 
is upon him. 


If ye forgive other men their trespasses, your heavenly 
Father shall forgive you yours. (Mat. chap. vi.) If I 
forgive, God shall forgive me, not for my deeds sake, but 
for his promises' sake, for his mercy and truth, and for the 
blood of his Son, Christ our Lord. And my forgiving
certifieth my spirit that God shall forgive me, yea that he
hath forgiven me already. For if I consent to the will of
God in my heart, though through infirmity and weakness 
I cannot do the will of God at all times; moreover though 
I cannot do the will of God so purely as the law requireth 
it of me, yet if I see my fault and meekly knowledge my 
sin, weeping in mine heart, because I cannot do the will of 
God I and thirst after strength, I am sure that the Spirit of 
God is in me, and his favour upon me. For the world 
lusteth not to do the will of God, neither sorroweth because 
he cannot, though he sorrow some time for fear of the pain 
that he believeth shall follow. He that hath the spirit of 
this world cannot forgive without amends making, or a 
greater vantage. If I forgive now how cometh it? Verily 
because I feel the mercy of God in me. For as a man 
feeleth God to himself, so is he to his neighbour. I know 
by mine own experience, that all flesh is in bondage under 
sin, and cannot but sin, therefore am I merciful, and desire 
God to loose the bonds of sin even in mine enemy. 


Gather not treasure together in earth, \&c. (Mat. vi.)
but gather you treasure in heaven, \&c. Let not your 
hearts be glued to worldly things, study not to heap treasure 
upon treasure, and riches upon riches, but study to bestow 
well that which is gotten already, and let your abundance 
succour the lack and need of the poor which have not. 
Have an eye to good works, to which if ye have lust 
and also power to do them, then are ye sure that the Spirit 
of God is in you, and ye in Christ elect to the reward of 
eternal life which followeth good works. But look that 
thine eye be single and rob not Christ of his honour, 
ascribe not that to the deserving of thy works, which is 
given thee freely by the merits of his blood. In Christ we 
are sons; in Christ we are heirs; in Christ God chose us 
and elected us before the beginning of the world, created 
us anew by the word of the gospel, and put his Spirit in 
us for because we should do good works. A Christian 
man worketh, because it is the will of his Father only. 
If we do no good work, nor be merciful, how is our lust 
therein? If we have no lust to do good works, how is 
God's Spirit in us? If the Spirit of God be not in us, 
how are we his sons? How are we his heirs, and heirs 
annexed with Christ of the eternal life, which is promised 
to all them that believe in him? Now do our works tes- 
tify and witness what we are, and what treasure is laid up 
for us in heaven, so that our eye be single, and look upon 
the commandment without respect of anything save because 
it is God's will, and that God desireth it of us, and Christ 
hath deserved that we do it. 

Not all they that say unto me, Lord, Lord, shall enter in- 
to the kingdom of heaven, but he that doth the will of 
my Father which is in heaven. (Mat. vii.) Though thou 
canst laud God with thy lips, and call Christ Lord, and 
canst babble and talk of the Scripture, and knowest all the 
stories of the Bible, yet shalt thou thereby never know 
thine election, or whether thy faith be right. But if thou 
feel lust in thine heart to the will of God, and bringest 
forth the fruits thereof, then hast thou confidence and hope; 
and thy deeds, and also the Spirit whence thy deeds spring 
certify thine heart that thou shalt enter, yea, art already 
entered into the kingdom of heaven. For it followeth, 
He that heareth the word and doth it buildeth his house 
upon a rock, and no tempest of temptations can over- 
throw it. For the Spirit of God is in his heart and com- 
forteth him, and holdeth him fast to the rock of the 
merits of Christ's blood, in whom he is elect. Nothing 
is able to pluck him out of the hands of God, God is 
stronger than all things. And contrariwise, he that 
heareth the word, and doth it not, buildeth on the sand of 
his own imagination, and every tempest overthroweth his 
building. The cause is, he hath not God's Spirit in him, 
and therefore understandeth it not aright, neither worketh 
aright. For no man knoweth the things of God (saith 
Paul in the 1st Epistle to the Corinthians in the iind chap-
ter) save the Spirit of God, as no man knoweth what is in 
a man but a mans spirit which is in him. So then if the 
Spirit be not in a man he worketh not the will of God, 
neither understandeth it, though he babble never so much 
of the Scriptures. Nevertheless such a man may work after 
his own imagination, but God's will can he not work, he 
may offer sacrifice, but to do mercy knoweth he not. It is 
easy to say unto Christ, Lord, Lord, but thereby shalt thou 
never feel or be sure of the kingdom of heaven. But and if 
thou do the will of God, then art thou sure that Christ is 
thy Lord indeed and that thou in him art also a lord, in 
that thou feelest thyself loosed and freed from the bondage 
of sin, and lusty and of power to do the will of God. 

Where the Spirit is there is feeling; for the Spirit maketh 
us feel all things. Where the Spirit is not there is no feeling,
but a vain opinion or imagination. A physician serveth
but for sick men, and that for such sick men as feel their
sicknesses, and mourn therefore and long for health.
Christ likewise serveth but for sinners only as feel their sin, 
and that for such sinners that sorrow and mourn in their 
hearts for health. Health is power or strength to fulfil the 
law, or to keep the commandments. Now he that longeth 
for that health, that is to say, for to do the law of God, is 
blessed in Christ, and hath a promise that his lust shall be 
fulfilled, and that he shall be made whole. (Matt. v.) 
Blessed are they which hunger and thirst for righteousness' 
sake, (that is, to fulfil the law,) for their lust shall be fulfilled. 
This longing and consent of the heart unto the law of God, 
is the working of the Spirit which God hath poured into 
thine heart, in earnest that thou mightest be sure that God 
will fulfil all his promises that he hath made thee. It is 
also the seal and mark which God putteth on all men 
that he choseth unto everlasting life. So long as thou 
seest thy sin and mournest and consentest to the law, and 
longest (though thou be never so weak) yet the Spirit shall 
keep thee in all temptations from desperation, and certify 
thine heart that God for his truth shall deliver thee and save 
thee, yea, and by thy good deeds shalt thou be saved, not 
which thou hast done, but which Christ has done for thee, 
For Christ is thine and all his deeds are thy deeds. Christ 
is in thee and thou in him, knit together inseparably. Neither 
canst thou be damned except Christ be damned with thee: 
neither can Christ be saved except thou be saved with him. 
Moreover thy heart is good, right, holy and just, for thy 
heart is no enemy to the law but a friend and a lover. The 
law and thy heart are agreed and at one, and therefore is God 
at one with thee. The consent of the heart unto the law, 
is unity and peace between God and man. For he is not 
mine enemy which would fain do me pleasure, and mourneth 
because he hath not wherewith. Now he that opened thy 
disease unto thee and made thee long for health, shall, as 
he hath promised, heal thee, and he that hath loosed thy 
heart, shall at his godly leisure, loose thy members. He 
that hath not the Spirit hath no feelings neither lusteth or 
longeth after power to fulfil the law, neither abhorreth the 
pleasures of sin, neither hath any more certainty of the pro- 
mises of God, than I have of a tale of Robin Hood, or 
of some jest that a man telleth me was done at Rome. 
Another man may lightly make me doubt or believe the 
contrary, seeing I have no experience thereof myself; so is it 
of them that feel not the working of the Spirit, and therefore 
in time of temptation the buildings of their imaginations fall. 

He that receiveth a prophet in the name of a prophet, that 
is, because he is a prophet, shall receive the reward of a 
prophet; and He that giveth one of these little ones a cup 
of cold water to drink in the name of a disciple, shall not 
loose his reward, (Matt. x.) Note this, that a prophet sig- 
nifieth as well him that interpreteth the hard places of Scrip- 
ture as him that prophesieth things to come. Now he that 
receiveth a prophet, a just man, or a disciple, shall have the 
same or like reward, that is to say, shall have the same 
eternal life which is appointed for them in Christ's blood 
and merits. For except thou were elect to the same eternal 
life and hadst the same faith and trust in God, and the same 
Spirit, thou couldst never consent to their deeds and help 
them. But thy deeds testify what thou art, and certify thy 
conscience that thou art received to mercy, and sanctified 
in Christ's passions and sufferings, and shalt hereafter, with 
all them that follow God, receive the reward of eternal life. 

Of thy words thou shalt be justified, and of thy words 
thou shalt be condemned; (Matt. xii.) That is, thy words 
as well as other deeds shall testify with thee or against thee 
at the day of judgment. Many there are which abstain 
from the outward deeds of fornication and adultery, never- 
theless rejoice to talk thereof and laugh; their words and 
laughter testify against them that their heart is impure, and 
they adulterers and fornicators in the sight of God. The 
tongue and other signs ofttimes utter the malice of the 
heart though a man for many causes abstain his hand from 
the outward deed or act. 

If thou wilt enter into life, keep the commandments; 
(Matt. xix.) First, remember that when God commandeth 
us to do any thing, he doth it not therefore because that we 
of ourselves are able to do that he commandeth; but that 
by the law we might see and know our horrible damnation 
and captivity under sin, and [therefore] should repent and 
come to Christ, and receive mercy and the Spirit of God to 
loose us, strengthen us, and to make us able to do God's 
will, which is the law. Now when he saith, If thou will en- 
ter into life keep the commandments, is as much [as] to say, 
as he that keepeth the commandments is entered into life: 
for except a man have first the Spirit of life in him by 
Christ's purchasing, it is impossible for him to keep the 
commandments, or that his heart should be loose or at liberty 
to lust after them, for of nature we are enemies to the law 
of God. 

As touching that, Christ saith afterward. If thou will be 
perfect go, and sell thy substance and give it to the poor; 
he saith it not as who should say that there were any greater 
perfection than to keep the law of God, (for that is all per- 
fection,) but to shew the other his blindness, which saw not 
that the law is spiritual, and requireth the heart. But 
because he was not knowing that he had hurt any man with 
the outward deed, he supposed that he loved his neighbour 
as himself. But when he was bid to shew the deeds of 
love, and give of his abundance to them that needed, he de- 
parted mourning. Which is an evident token that he 
loved not his neighbour as well as himself. For if he had 
need himself, it would not have grieved him to have re- 
ceived succour of another man. Moreover, he saw not that it 
was murder and theft, that a man should have abundance 
of riches lying by him, and not to shew mercy therewith, 
and kindly to succour his neighbour's need. God hath 
given one man riches to help another at need. If thy 
neighbour need, and thou help him not, being able, thou 
withholdest his duty from him, and art a thief before God. 

That also, that Christ saith, how that it is harder for a 
rich man (who loveth his riches so that he cannot find in 
his heart liberally and freely to help the poor and needy) 
to enter into the kingdom of heaven, than a camel to go 
through the eye of a needle, declareth that he was not en- 
tered into the kingdom of heaven, that is to say, eternal 
life. But he that keepeth the commandments is entered 
into life, yea, hath life and the Spirit of life in him. 


This kind of devils goeth not out but by prayer and 
fasting, (Mat. xxvii.) Not that the devil is cast out by 
merits of fasting or praying. For he saith before, that for 
their unbelief's sake, they could not cast him out. It is 
faith no doubt that casteth out the devils, and faith it is 
that fasteth and prayeth. Faith hath the promises of God 
whereunto she cleaveth, and in all things thirsteth [for] the 
honour of God. She fasteth to subdue the body unto the 
spirit that the prayer be not let, and that the spirit may 
quietly talk with God: she also, whensoever opportunity is 
given, prayeth God to fulfil his promises unto his praise 
aod glory. And God, which is merciful in promising, and 
true to fulfil them, casteth out the devils, and doth all that 
faith desireth, and satisfieth her thirst. 


Come, ye blessed of my Father, inherit the kingdom 
prepared for you from the beginning of the world; for I 
was athirst, and ye gave me drink, \&c. (Mat. xxv.) Not
that a man with works deserveth eternal life as a workman 
or labourer his hire or wages. Thou readest in the text, 
that the kingdom was prepared for us from the beginning of 
the world. And we are blessed and sanctified. In Christ's 
blood are we blessed from that bitter curse and damnable 
captivity under sin, wherein we were born and conceived. 
And Christ's Spirit is poured into us, to bring forth good 
works, and our works are the fruits of the Spirit, and the 
kingdom is the deserving of Christ's blood, and so is faith 
and the Spirit and good works also. Notwithstanding the 
kingdom followeth good works, and good works testify that 
we are heirs thereof, and at the day of judgment shall they 
testify for the elect unto their comfort and glory: and to 
the confusion of the ungodly, unbelieving and faithless 
sinners, which had not trust in the word of God's pro- 
mises, nor lust to the will of God; but were carried of 
the spirit of their father the devil unto all abomination, to 
work wickedness with all lust, delectation, and greediness. 

Many sins are forgiven her, for she loveth much; 
(Luke vii.) Not that love was cause of forgiveness of sins,
but contrariwise the forgiveness of sins caused love, as it 
followeth, to whom less was forgiven that same loveth less. 
And afore he commended the judgment of Simon, which 
answered that he loveth most to whom most was forgiven: 
and also said at the last, Thy faith hath saved thee (or 
made thee safe) go in peace. We cannot love except we 
see some benefit and kindness. As long as we look on
the law of God only, where we see but sin and damnation
and the wrath of God upon us, yea where we were damned 
afore we were born we cannot love God. No, we cannot 
but hate him as a tyrant, unrighteous, unjust, and 
flee from him as did Cain. But when the gospel, that [those] 
glad tidings and joyful promises are preached, how that in 
Christ, God loveth us first, forgiveth us, and hath mercy on 
us, then love we again, and the deeds of our love declare 
our faith. This is the manner of speaking: as we say, 
Summer is nigh, for the trees blossom. Now is the blos- 
soming of the trees not the cause that summer draweth 
nigh; but the drawing nigh of summer is the cause of 
the blossoms, I and the blossoms put us in remembrance 
that summer is at hand. So Christ here teacheth Simon 
by the ferventness of love in the outward deeds, to see a 
strong faith within whence so great love springeth. As 
the manner is to say, Do your charity, shew your charity, 
do a deed of charity, shew your mercy, do a deed of mercy, 
meaning thereby that our deeds declare how we love our 
neighbours, and how much we have compassion on them 
at their need. Moreover it is not possible to love except 
we see a cause. Except we see in our hearts the love and 
kindness of God to usward in Christ our Lord, it is not 
possible to love God aright. 

We say also, He that lovetb not my dog loveth not me. 
Not that a man should love my dog first, but if a man 
loved me, the love wherewith he loved me would compel 
him to love my dog, though the dog deserved it not, yea, 
though the dog had done him a displeasure, yet if he loved 
me, the same love would refrain him from revenging himself, 
and cause him to refer the vengeance unto me. Such 
speakings find we in Scripture; John in the ivth of his 
first Epistle saith, He that saith I love God, and yet hateth 
his brother, is a liar; For how can he that loveth not his 
brother whom he seeth, love God whom he seeth not? 
This is not spoken that a man should first love his brother 
and then God, but as it followeth: For this commandment 
have we of him, that he which loveth God should love his 
brother also. To love my neighbour is the commandment; 
which commandment he that loveth not, loveth not God. 
The keeping of the commandment declareth what love I
have to God. If I loved God purely, nothing that my
neighbour could do were able to make me either to hate
him, either to take vengeance on him myself, seeing that
God hath commanded me to love him, and to remit all 
vengeance unto him. Mark now, how much I love the com- 
mandment, so much I love God; how much I love God, 
so much believe I that he is merciful, kind and good, yea, 
and a father unto me for Christ's sake. How much I believe 
that God is merciful unto me, and that he will for 
Christ's sake fulfil all his promises unto me; so much I 
see my sins, so much do my sins grieve me, so much do 
I repent and sorrow that I sin, so much displeaseth me 
that poison that moveth me to sin, and so greatly desire I 
to be healed. So now by the natural order, first I see my 
sin; then I repent and sorrow; then believe I God's pro- 
mises, that he is merciful unto me, and forgiveth me, and 
will heal me at the last: then love I, and then I prepare 
myself to the commandment. 


This do and thou shalt live. (Luke x.) That is to say,
Love thy Lord God with all thy heart, with all thy soul, and
with all thy strength, and with all thy mind, and thy neigh-
bour as thyself. As who should say, if thou do this, or
though thou canst not do it, yet if thou feelest lust there-
unto, and thy spirit sigheth, mourneth, and longeth after 
strength to do it, take a sign and evident token thereby, 
that the Spirit of life is in thee, and that thou art elect to 
life everlasting by Christ's blood, whose gift and purchase 
is thy faith, and that Spirit that worketh the will of God 
in thee, whose gift also are thy deeds, or rather the deeds 
of the Spirit of Christ, and not thine, and whose gift is 
the reward of eternal life, which followeth good works. 

It followeth also in the same place of Luke, When he 
should depart he plucked out twopence and gave them to 
the host and said unto him, Take the charge or cure of him, 
and whatsoever thou spendest more I will recompesne it 
thee at my coming again. Remember, this is a parable, and 
a parable may not be expounded word for word; but the ta- 
lent of the similitude must be sought but only in the whole pa- 
rable. The intent of the similitude is to shew to whom a man 
is a neighbour or who is a man's neighbour, which is both 
one, and what is to love a man's neighbour as himself. 

The Samaritan helped him and shewed mercy as long 
as he was present, and when he could be no longer 
present, he left his money behind him, And if that were not 
sufficient, he left his credence to make good the rest, and 
forsook him not as long as the other had need. Then said 
Christ, Go thou and do likewise; that is, without difference 
or respection of persons: whosoever needeth thy help, him 
count thy neighbour, and his neighbour be thou, and shew 
mercy on him as long as he needeth thy succour, and that 
is to love a man's neighbour as himself. Neighbour is a 
word of love, and signifieth that a man should be ever nigh 
and at hand, and ready to help in time of need. 

They that will interpret parables word by word, fall into 
straights ofttimes, whence they cannot rid themselves; and 
preach lies instead of the truth. As do they which interpret 
by the twopence, the Old Testament and the New, and by 
that which is bestowed Opera supererogationis. Howbeit 
Superarrogantia were a meeter term. That is to say, deeds 
which are more than the law requireth, deeds of perfection 
and of liberality which a man is not bound to do but of his 
free will: and for them he shall have an higher place in hea- 
ven and may give to other of his merits; or of which the 
Pope after his death may give pardons from the pains 
of purgatory. 

Against which exposition I answer; first, a greater per- 
fection than the law is there not. A greater perfection 
than to love God and his will, which is the commandments, 
with all thine heart, with all thy soul, with all thy strength, 
with all thy mind, is there none; and to love a man's 
neighbour as himself is like the same. It is a wonderful 
love wherewith a man loveth himself. As glad as I would 
be to receive pardon of mine own life, (if I had deserved 
death) so glad ought I to be to defend my neighbour's life 
without respect of my life, or of my good. A man ought 
neither to spare his goods nor yet himself for his brother's 
sake, after the ensample of Christ (1 John iii.) Herein 
saith he, perceive we love, in that he, that is to say Christ, 
gave his life for us, we ought therefore to bestow our lives 
for the brethren. Now saith Christ, (John xv.) There is 
no greater love than that a man bestow his life for his friend. 

Moreover no man can fulfil the law, for John saith (1st 
chapter of the said Epistle,) If we say we have no sin, we 
deceive ourselves and truth is not in us, if we knowledge 
our sins, he is faithful and righteous to forgive us our sins 
and to purge us from all iniquity. And in the Paternoster 
also we say, Father, forgive us our sins. Now if we be all 
sinners, none fulfilleth the law: for he that fulfilleth the law 
is no sinner. In the law: may neither Peter nor Paul nor 
any other creature save Christ only rejoice. In the blood 
of Christ, which fulfilled the law for us, may every person 
that repenteth, believeth, loveth the law, and mourneth for 
strength to fulfil it, rejoice, be he never so weak a sinner. 
The twopence therefore and the credence that he left be- 
hind him to bestow more, if need were, signifieth that he 
was everywhere merciful, both present and absent, without 
feigning, cloaking, complaining, or excusing, and forsake not 
his neighbour as long as he has need. Which example I 
pray God men may follow; and let opera supererogationis 
alone. 


Mary hath chosen a good part which shall not be taken 
from her, (Luke x.) She was first chosen of God and called 
by grace, both to know her sin and also to hear the word 
of faith, health, and glad tidings of mercy in Christ; and 
faith was given her to believe, and the Spirit of God loosed 
her heart from the bondage of sin: then consented she to 
the will of God again, and above all things had delectation 
to hear the word wherein she had obtained everlasting health, 
and namely, of his own mouth, which had purchased so 
great mercy for her. God chooseth us first and loveth us 
first, and openeth our eyes to see his exceeding abundant 
love to us in Christ, and then love we again, and accept his 
will above all things, and serve him in that office whereunto 
he hath chosen us. 

Sell that ye have, and give alms, and make you bags 
which wax not old, and treasure which faileth not in heaven. 
(Luke xii.) This and such like are not spoken that we should 
work as hirelings in respect of reward, and as though we 
should obtain heaven with merit: for he saith a little afore, 
Fear not, little flock, for it is your Father's pleasure to give 
you a kingdom. The kingdom cometh then of the good will 
of Almighty God Uirough Christ, and such things are spoken 
partly to put us in remembrance of our duty to be kind 
again. As is that saying, Let your light so shine before 
men that they, seeing your good works, may glorify your 
Father which is in heaven: as who should say, if God 
hath given you so great gifts see ye be not unthankful, but 
bestow them unto his praise. Some things are spoken to 
move us to put our trust in God, as are these; Behold the 
lilies of the field; Behold the birds of the air: If your 
children ask you bread will ye proffer them a stone? and many 
such like. Some are spoken to put us in remembrance to 
be sober, to watch and pray, and to prepare ourselves 
against temptations, and that we should understand and 
know how that temptations and occasion of evil come then 
most, when they are least looked for; lest we should be 
careless and sure of ourselves, negligent and unprepared. 
Some things are spoken that we should fear the wonder- 
ful and incomprehensible judgments of God lest we should 
presume. Some to comfort us that we despair not. And for 
like causes are all the ensamples of the Old Testament. 
In conclusion, the Scripture speaketh many things as the 
world speaketh, but they may not be worldly understood, 
but ghostly and spiritually, yea, the Spirit of God only un- 
derstandeth them, and where he is not there is not the un- 
derstanding of the Scripture; but unfruitful disputing and 
brawling about words. 

The Scripture saith, God seeth, God heareth, God smel- 
leth, God walketh, God is with them, God is not with 
them, God is angry, God is pleased, God sendeth his Spirit, 
God taketh his Spirit away, and a thousand such like: and 
yet is none of them true after the worldly manner, and as the 
words sound. Read the iind chapter of Paul to the Cor- 
rinthians: The natural man understandeth not the things 
of God, but the Spirit of God only. And we, saith he, have 
received the Spirit which is of God to understand the things 
which are given us of God: or without the Spirit it is im- 
possible to understand them. Read also the viiith to the 
Romans: They that are led with the Spirit of God, are 
the sons of God: now the son knoweth his father's will 
and the servant that hath not the Spirit of Christ, (saith 
Paul) is none of his: likewise he that hath not the Spirit 
of God, is none of God's, for it is both one Spirit, as thou 
mayest see in the same place. 

Now he that is of God heareth the word of God, (John 
viii.) And who is of God but he that hath the Spirit of God? 
furthermore, saith he, Ye hear it not, because ye are not 
of God; that is, ye have no lust in the word of God, for ye 
understand it not, and that because his Spirit is not in you. 

Forasmuch then as the Scripture is nothing else but that 
which the Spirit of God hath spoken by the prophets and 
apostles, and cannot be understood but of the same Spirit, 
Let every man pray to God to send him his Spirit to loose 
him from his natural blindness and ignorance, and to give 
him understanding, and feeling of the things of God, and of 
the speaking of the Spirit of God. And mark this process: 
first, we are damned of nature, so conceived and born, as 
a serpent is a serpent, and a toad a toad, and a snake a 
snake by nature. And as thou seest a young child which 
hath pleasure in many things wherein is present death, as in 
fire, water, and so forth, would slay himself with a thousand 
deaths if he were not waited upon and kept therefrom,
even so we, if we should live these thousand years, could in
all that time delight in no other thing, nor yet seek any
other thing but that wherein is death of the soul. 

Secondarily, of the whole multitude of the nature of man, 
whom God hath elect and chosen, and to whom he hath 
appointed mercy and grace in Christ, to them sendeth he his 
Spirit, which openeth their eyes, sheweth them their misery, 
and bringeth them unto the knowledge of themselves, so 
that they hate and abhor themselves, are astonished and 
amazed, and at their wit's end, neither wot what to do, or 
where to seek health. Then lest they should flee from God 
by desperation, he comforteth them again with his sweet 
promises in Christ, and certifieth their hearts that for Christ's 
sake they are received to mercy, and their sins forgiven, 
and they elect and made the sons of God, and heirs with 
Christ of eternal life: and thus, through faith, are they set 
at peace with God. 

Now may not we ask why God chooseth one and not 
another; either think that God is unjust to damn us afore 
we do any actual deed; seeing that God hath power overall 
his creatures of right to do with them what he list, or to 
make of every one of them as be listeth. Our darkness 
cannot perceive his light. God will be feared, and not have 
his secret judgments known. Moreover we by the light of 
faith see a thousand things which are impossible to an in- 
fidel to see: so likewise no doubt, in the light of the clear 
vision of God, we shall see things which now God will not 
have known. For pride ever accompanieth high knowledge, 
but grace accompanieth meekness. Let us therefore give 
diligence rather to do the will of God, than to search his 
secrets which are not profitable for us to know. 

When we are thus reconciled to God, made the friends 
of God and heirs of eternal life, the Spirit that God hath 
poured into us testifieth that we may not live after our old 
deeds of ignorance: for how is it possible that we should 
repent and abhor them, and yet have lust to live in them? 
We are sure therefore that God hath created and made us 
new in Christ and put his Spirit in us that we should live 
a new life which is the life of good works. 

That thou mayest know what are good works, and the 
intent of good works, or wherefore good works serve, mark 
this that followeth. 

The life of a Christian man is inward between him and 
God, and properly is the consent of the Spirit to the will 
of God and to the honour of God. And God's honour is the 
final end of all good works. 

Good works are all things that are done within the laws 
of God, in which God is honoured, and for which thanks 
are given to God. 

Fasting is to abstain from surfeiting, or overmuch eating, 
from drunkenness, and care of the world (as thou mayest 
read Luke xxi.) and the end of fasting is to tame the body 
that the Spirit may have a free course to God, and may 
quietly talk with God. For overmuch eating and drinking, 
and care of worldly business, press down the spirit, choke 
her and tangle her that she cannot lift up herself to God. 
Now he that fasteth for any other intent than to subdue 
the body that the Spirit may wait on God, and freely ex- 
cercise herself in the things of God; the same is blind, 
and wotteth not what he doth, erreth and shooteth at a 
wrong mark, and his intent and imagination is abominable 
in the sight of God. When thou fastest from meat and 
drinkest all day, is that a Christian fast? either to eat at 
one meal that were sufficient for four? A man at four 
times may bear that he cannot at once. Some fast from 
meat and drink, and yet so tangle themselves in worldly 
business that they cannot once think on God. Some 
abstain from butter, some from eggs, some from all manner 
[of] white meat, some this day, some that day, some in the 
honour of this saint, some of that, and every man for a 
sundry purpose. Some for the tooth ache, some for the 
head ache, for fevers, pestilence, for sudden death, for 
hanging, drowning, and to be delivered from the pains of 
hell. Some are so mad, that they fast one of the Thursdays 
between the two St. Mary days, in the worship of that 
saint, whose day is hallowed between Christmas and Can- 
dlemas; and that to be delivered from the pestilence. All 
those men fast without conscience of God, and without 
knowledge of the true intent of fasting, and do no other 
than honour saints, as the Gentiles and heathen worshipped 
their idols, and are drowned in blindness, and know not 
of the Testament, that God hath made to man ward in 
Christ's bloods In God have they neither hope nor con- 
fidence, neither believe his promises, neither know his will, 
but are yet in captivity under the prince of darkness. 


Watch, is not only to abstain from sleep, but also to 
be circumspect and to cast all perils; as a man should 
watch a tower or a castle. We must remember that the 
snares of the devil are infinite and innumerable, and that 
every moment arise new temptations, and that in all places 
meet us fresh occasions; against which we must prepare 
ourselves and turn to God and complain to him, and make 
our moan, and desire him of his mercy to be our shield, 
our tower, our castle, and defence from all evil, to put his 
strength in us, for without him we can do nought, and 
above all things we must call to mind what promises God 
hath made and what he hath sworn that he will do to us 
for Christ's sake, and with strong faith cleave unto him and 
desire him of his mercy and for the love that he hath to 
Christ, and for his truth's sake, to fulfil his promises. If 
we thus cleave to God with strong faith and believe his 
words, then as saith Paul, (ist Cor. x.) God is faithful that 
he will not suffer us to be tempted above that we are able, 
or above our might, that is to say, if we cleave to his pro- 
mises and not to our own fantasies and imaginations, he 
will put might and power into us that shall be stronger than 
all the temptation which he shall suffer to be against us. 


Prayer is a mourning, a longing, and a desire of the
man mourneth and sorroweth in his heart, longing for 
health. Faith ever prayeth. For after that by faith we 
are reconciled to God, and have received mercy and for- 
giveness of God, the spirit longeth and thirsteth for 
strength to do the will of God, and that God may be 
honoured, his name hallowed, and his pleasure and will 
fulfilled. The spirit waiteth and watcheth on the will of 
God, and ever hath her own fragility and weakness before 
her eyes; and when she seeth temptation and peril draw 
nigh, she turneth to God, and to the Testament that God 
hath made to all that believe and trust in Christ's blood, 
and desireth God for his mercy, and truth, and for the love he 
hath to Christ, that he will fulfil his promise, that he will 
succour, and help, and give us strength, and that he will sanc- 
tify his name in us, and fulfil his godly will in us, and that 
he will not look on our sin and iniquity, but on his mercy, 
on his truth, and on the love that he oweth to his Son 
Christ, and for his sake to keep us from temptation, that 
we be not overcome, and that he deliver us from evil, and 
whatsoever moveth us contrary to his godly will. 

Moreover, of his own experience he feeleth other men's
need, and no less commendeth to God the infirmities of 
other than his own, knowing that there is no strength, no 
help, no succour, but of God only. And as merciful as 
he feeleth God in his heart to himselfward, so merciful is 
he to other; and as greatly as he feeleth his own misery, 
so great compassion hath he on other. His neighbour is 
no less care to him than himself: he feeleth his neighbour's 
grief no less than his own. And whensoever he seeth oc- 
casion, he cannot but pray for his neighbour as well as 
for himself: his nature is to seek the honour of God in 
all men, and to draw (as much as in him is) all men unto 
God. This is the law of love, which springeth out of 
Christ's blood into the hearts of all them that have their 
trust in him. No man needeth to bid a Christian man to pray, 
if he see his neighbour's need: if he see it not, put him in 
remembrance only, and then he cannot but do his duty. 

Now, as touching we desire one another to pray for us, 
that do we to put our neighbour in remembrance of his 
duty, and not that we trust in his holiness. Our trust is 
in God, in Christ, and in the truth of God's promises; 
we have also a promise, that when two or three, or more, 
agree together in any thing, according to the will of God, 
God heareth us. Notwithstanding, as God heareth many, 
so heareth he few, and so heareth he one, if he pray after 
the will of God, and desire the honour of God. He that 
desireth mercy, the same feeleth his own misery and sin, 
and mourneth in his heart for to be delivered, that he 
might honour God; and God for his truth must hear him, 
which saith by the mouth of Christ, (Matt. v.) Blessed are 
they that hunger and thirst after righteousness, for they 
shall be filled. God, for his truth's sake, must put the 
righteousness of Christ in him, and wash his unrighteous- 
ness away in the blood of Christ. And be the sinner never 
so weak, never so feeble and frail, sin he never so oft and 
so grievous, yet so long as this lust, desire, and mourning 
to he delivered remaineth in him, God seeth not his sins, 
reckoneth them not, for his truth's sake, and love to 
Christ. He is not a sinner in the sight of God that would 
be no sinner. He that would be delivered hath his heart 
loose already. His heart sinneth not, but mourneth, re- 
penteth, and consenteth unto the law and will of God, and 
justifieth God; that is, beareth record that God which 
made the law is righteous and just. And such an heart, 
trusting in Christ's blood, is accepted for full righteous. 
And his weakness, infirmity, and frailty is pardoned, and 
his sins not looked upon: until God put more strength in 
him, and fulfil his lust. 

When the weak in the faith, and unexpert in the mys- 
teries of Christ, desire us to pray for them, then ought we 
to lead them to the truth and promises of God, and teach 
them to put their trust in the promises of God, in love 
that God hath to Christ and to us for his sake, and to 
strengthen their weak consciences, shewing and proving by 
the Scripture, that as long as they follow the Spirit and 
resist sin, it is impossible they should fall so deep that 
God shall not pull them up again, if they hold fast by the 
anchor of faith, having trust and confidence in Christ. 
The love that God hath to Christ is infinite; and Christ 
did and suffered all things not for himself, to obtain favour 
or aught else; for he had ever the full favour of God, and 
was ever Lord over all things; but to reconcile us to God, 
and to make us heirs with him of his Father's kingdom. 
And God hath promised, that whosoever calleth on his 
name shall never be confounded or ashamed. (Rom. ix.) 
If the righteous fall (saith the Scripture) he shall not be 
bruised; the Lord shall put his hand under him. Who is 
righteous but he that trusteth in Christ's blood, be he 
never so weak? Christ is our righteousness; and in him 
ought we to teach all men to trust, and to expound unto 
all men the Testament, that God hath made to us sinners 
in Christ's blood. This ought we to do, and not make a 
prey of them to lead them captive, to sit in their consciences, 
and to teach them to trust in our holiness, good deeds and 
prayers, to the intent that we would feed our idle and slow 
bellies of their great labour and sweat, and so to make 
ourselves Christs and Saviours. For if I take on me to 
save other by my merits, make I not myself a Christ 
and a Saviour, and am indeed a false prophet, and a true 
Antichrist, and exalt myself and sit in the temple of God; 
that is to wit, the consciences of men? 

Among Christian men, love maketh all things common; 
every man is other's debtor, and every man is bound 
to minister to his neighbour, and to supply his neigh- 
bour's lack of that wherewith God hath endowed 
him. As thou seest in the world, how the lords and 
officers minister peace in the commonwealth, punish 
murderers, thieves, and evil doers, and to maintain their 
order and estate, do the commons minister to them again 
rent, tribute, toll, and custom: so in the gospel, the curates 
which in every parish preach the gospel, ought of duty to 
receive an honest living for them and their households; and 
even so ought the other officers, which are necessarily re- 
quired in the commonwealth of Christ. We need not to 
use filthy lucre in the gospel, to chop and change, and to 
play the taverners, altering the word of God as they do 
their wines, to their most advantage, and to fashion God's 
word after every man's mouth; or to abuse the name of 
Christ, to obtain thereby authority and power to feed our 
slow bellies. Now seest thou what prayer is, the end 
thereof, and wherefore it serveth. 

If thou give me a thousand pounds to pray for thee, I 
am no more bound than I was before. Man's imagination 
can make the commandment of God neither greater nor 
smaller, neither can to the law of God either add or 
minish. God's commandment is as great as himself. I am 
bound to love the Turk with all my might and power; yea, 
and above my power, even from the ground of my heart, after 
the ensample that Christ loved me, -— neither to spare goods, 
body, or life, to win him to Christ. And what can I do 
more for thee if thou gavest me all the world? Where I see 
need there can I not but pray, if God's Spirit be in me. 

Alms is a Greek word, and signifieth mercy. One 
Christian is debtor to another at his need, of all that he is 
able to do for him, until his need be sufficed. Every 
Christian man ought to have Christ always before his eyes, 
as an ensample to counterfeit and follow, and to do to his 
neighbour as Christ hath done to him, as Paul teacheth in 
all his epistles, and Peter in his first, and John in his first 
also. This order useth Paul in all his epistles: first, he 
preacheth the law, and proveth that the whole nature of 
man is damned, in that the heart lusteth contrary to the 
will of God. For if we were of God, no doubt we 
should have lust in his will. Then preacheth he Christ, 
the gospel, the promises, and the mercy that God hath set 
forth to all men in Christ's blood: which they that believe, 
and take it for an earnest thing, turn themselves to God, 
begin to love God again, and to prepare themselves to his 
will by the working of the Spirit of God in them. Last of all, 
exhorteth he to unity, peace, and soberness; to avoid brawl- 
ings, sects, opinions, disputing and arguing about words, 
and to walk in the plain and single faith and feeling of the 
Spirit, and to Love one another after the ensample of 
Christ, even as Christ loved us; and to be thankful, and 
to walk worthy of the gospel, and as it becometh Christ, 
and with the ensample of pure living to draw all to Christ. 

Christ is lord over all; and every Christian is heir an- 
nexed with Christ, and therefore Lord of all; and every 
one Lord of whatsoever another hath. If thy brother or 
neighbour therefore need, and thou have to help him, 
and yet shewest not mercy, but withdrawest thy hands 
from him, then robbest thou him of his own, and art a 
thief. A Christian man hath Christ's spirit. Now is Christ 
a merciful thing: if, therefore, thou be not merciful, after 
the ensample of Christ, then hast thou not his Spirit. If 
thou have not Christ's Spirit, then art thou none of his, 
(Rom. viii.) nor hast any part with him. Moreover, 
though thou shew mercy unto thy neighbour, yet if thou do 
it not with such burning love as Christ did unto thee, so 
must thou knowledge thy sin, and desire mercy in Christ. 
A Christian man hath nought to rejoice in, as concerning his 
deeds. His rejoicing is that Christ died for him, and that 
he is washed in Christ's blood. Of his deeds rejoiceth he 
not, neither counteth his merits, neither giveth pardons of 
them, neither seeketh an higher place in heaven of them, 
neither maketh himself a saviour of other men through his 
good works: but giveth all honour to God, and in his 
greatest deeds of mercy, knowledgeth himself a sinner un- 
feignedly, and is abundantly content with that place that is 
prepared for him of Christ; and his good deeds are to him 
a sign only that Christ's Spirit is in him, and he in Christ, 
and, through Christ, elect to eternal life. 

The order of love or charity which some dream, the
gospel of Christ knoweth not of, that a man should begin
at himself, and serve himself first, and then descend, I wot 
not by what steps. Love seeketh not her own profit, (1 Cor. 
xiii.) but maketh a man to forget himself, and to turn his 
profit to another man, as Christ sought not himself, or his 
own profit, but ours. This term, myself, is not in the 
gospel; neither yet father, mother, sister, brother, kins- 
man, that one should be preferred in love above another. 
But Christ is all in all things. Every Christian man to 
another is Christ himself; and thy neighbour's need hath 
as good right in thy goods as hath Christ himself, which is 
heir and Lord over all. And look, what thou owest to 
Christ, that thou owest to thy neighbour's need: to thy 
neighbour owest thou thine heart, thyself, and all that thou 
hast and canst do. The love that springeth out of Christ 
excludeth no man, neither putteth difference between one 
and another. In Christ we are all of one degree, without 
respect of persons. Notwithstanding, though a Christian 
man's heart be open to all men, and receiveth all men, yet, 
because that his ability of goods extendeth not so far, this 
provision is made, — that every man shall care for his own 
household, as father and mother, and thine elders that have 
holpen thee, wife, children, and servants. If thou shouldest 
not care and provide for thine household, then were thou 
an infidel, seeing thou hast taken on thee so to do, 
and forasmuch as that is thy part committed to thee of the 
congregation. When thou hast done thy duty to thine 
household, and yet hast further abundance of the blessing 
of God, that owest thou to the poor that cannot labour, 
or would labour and can get no work, and are destitute of 
friends; to the poor, I mean, which thou knowest, to them 
of thine own parish. For that provision ought to be had 
in the congregation, that every parish care for their poor. 
If thy neighbours which thou knowest be served, and thou 
yet have superfluity, and hearest necessity to be among the 
brethren a thousand miles of, to them art thou debtor. 
Yea, to the very infidels we be debtors, if they need, as 
far forth as we maintain them not against Christ, or to 
blaspheme Christ. Thus is every man that needeth thy 
help, thy father, mother, sister and brother in Chist; even 
as every man that doth the will of the father, is father, 
mother, sister and brother unto Christ. 

Moreover if any be an infidel and a false christian, and 
forsake his household, his wife, children, and such as can- 
not help themselves, then art thou bound and thou have 
wherewith even as much as to thine own household. And 
they have as good right in thy goods as thou thyself: 
and if thou withdraw mercy from them, and hast wherewith 
to help them, then art thou a thief. If thou shew mercy, 
so doest thou thy duty, and art a faithful minister in the 
household of Christ, and of Christ shalt thou have thy re- 
ward and thanks. If the whole world were thine, yet hath 
every brother his right in thy goods, and is heir with thee, 
as we are all heirs with Christ. Moreover the rich and they 
that have wisdom with them must see the poor set a work, 
that as many as are able may feed themselves with the la- 
bour of their own hands, according to the Scripture and 
commandment of God. 

Now seest thou what alms-deeds meaneth, and wherefore 
it serveth. He that seeketh with his alms more than to be mer- 
ciful to a neighbour, to succour his brother's need, to do his 
duty to his brother, to give his brother that he oweth him, 
the same is blind and seeth not what it is to be a christian 
man, and to have fellowship in Christ's blood. 

As pertaining to good works, understand that all works 
are good which are done within the law of God, in faith 
and with thanksgiving to God, and understand that thou in 
doing them pleasest God, whatsoever thou doest within the 
law of God, as when thou makest water. And trust me, if 
either wind or water were stopped, thou shouldest feel what 
a precious thing it were to do either of both, and what 
thanks ought to be given God therefore. Moreover put 
no difference between works, but whatsoever cometh into 
thy hands that do, as time, place, and occasion giveth, and 
as God hath put thee in degree high or low. For as 
touching to please God, there is no work better than 
another. God looketh not first on thy work as the world 
doth, as though the beautifutness of the work pleased him 
as it doth the world, or as though he had need of them: 
but God looketh first on thy heart, what faith thou hast to 
his words, how thou believest him, trustest him, and how 
thou lovest him for his mercy that he hath showed thee; he 
looketh with what heart thou workest, and not what thou 
workest, how thou acceptest the degree that he hath put 
thee in, and not of what degree thou art, whether thou be an 
apostle or a shoemaker. Set this ensample before thine eyes. 
Thou art a kitchen page, and washest thy master's dishes, 
another is an apostle, and preacheth the word of God. Of 
this apostle hark what Paul saith in the iind Cor. ix. If 
I preach, saith he, I have nought to rejoice in, for necessity 
is put unto me; as who should say, God hath made me so. 
Woe is unto me if I preach not. If I do it willingly, saith 
he, then have I my reward, that is, then am I sure that 
God's Spirit is in me and that I am elect to eternal life. 
If I do it against my will, an office is committed unto me; 
that is, if I do it not of love to God, but to get a living 
thereby, and for a worldly purpose, and had rather other- 
wise live, then do I that office which God hath put me in, 
and yet please not God myself. Note now, if this apostle 
preach not, as many do not, which not only make them- 
selves apostles, but also compel men to take them for greater 
than apostles, yea, for greater than Christ himself: then 
woe is unto him, that is, his damnation is just. If he preach 
and his heart not right, yet ministereth he the office that God 
hath put him in, and they that have the Spirit of God, hear 
the voice of God, yea, though he speak in an ass. More- 
over howsoever he preacheth he hath not to rejoice in that 
he preacheth. But and if he preach willingly, with a true 
heart, and of concience to God, then hath he his reward, 
that is, then feeleth he the earnest of eternal life, and the 
working of the Spirit of God in him. And as he feeleth 
God's goodness and mercy, so be thou sure he feeleth his 
own infirmity, weakness, and unworthiness, and mourneth 
and knowledgeth his sin, in that the heart will not arise to 
work with that full lust and love that is in Christ our 
Lord: And nevertheless is yet at peace with God, through 
faith and trust in Christ Jesu. For the earnest of the Spi- 
rit that worketh in him, testifieth and beareth witness unto 
his heart that God hath chosen him, and that his grace shall 
suffice him, which grace is now not idle in him. In his 
works putteth he his trust. 

Now thou that ministerest in the kitchen, and art but a 
kitchen page, receivest all things of the hand of God, 
knowest that God hath put thee in that office, submittest thy- 
self to his will and servest thy master not as a man, but as 
Christ himself with a pure heart, according as Paul 
teacheth us, puttest thy trust in God, and with him seekest 
thy reward. Moreover there is not a good deed done, but 
thy heart rejoiceth therein, yea, when thou hearest that the 
word of God is preached by this apostle and seest the 
people turn to God, thou consentest unto the deed; thine 
heart breaketh out in joy, springeth and leapeth in thy breast, 
that God is honoured: and in thine heart doest the same 
that that apostle doth, and haply with greater delectation, 
and a more fervent spirit. Now he that receiveth a prophet 
in the name of a prophet shall receive the reward of a 
prophet; (Matt. x.) that is, he hath consenteth to the deed 
of a prophet, and maintaineth it, the same hath the same 
Spirit and earnest of everlasting life, which the prophet 
hath, and is elect as the prophet is. 

Now if thou compare deed to deed, there is difference 
betwixt washing of dishes, and preaching of the word of 
God; but as touching to please God none at all: for 
neither that nor this pleaseth, but as far forth as God hath 
chosen a man, hath put his Spirit in him, and purified his 
heart by faith and trust in Christ. 

Let every man therefore wait on the office wherein Christ 
hath put him, and therein sene his brethren. If he be of 
low degree let him patiently therein abide, till God pro- 
mote him, and exalt him higher. Let kings and head officers 
seek Christ in their offices, and minister peace and quiet- 
ness unto the brethren; punish sin, and that with mercy, 
even with the same sorrow and grief of mind as they would 
cut off a finger or joint, a leg or arm, of their own body, 
if there were such disease in them that either they must be 
cut off, or else all the body must perish. 

Let every man of whatsoever craft or occupation he be 
of, whether brewer, baker, tailor, victualler, merchant, or 
husbandman, refer his craft and occupation unto the com- 
monwealth, and serve his brethren as he would do Christ 
himself. Let him buy and sell truly, and not set dice on 
his brethren; and so sheweth he mercy, and his occupation 
pleaseth God. And when thou receivest money for thy la- 
bour or ware thou receivest thy duty. For whereinsoever 
thou minister to thy brethren, thy brethren are debtors to 
give thee wherewith to maintain thyself and household. 
And let your superfluities succour the poor, of which sort 
shall ever be some in all towns, and cities, and villages, and 
that I suppose the greatest number. Remember that we 
are members of one body, and ought to minister one to 
another mercifully: and remember that whatsoever we have 
it is given us of God, to bestow it on our brethren. Let 
him that eateth, eat and give God thanks, only let not thy 
meat pull thine heart from God; and let him that drinketh 
do likewise. Let him that hath a wife, give God thanks for 
his liberty, only let not thy wife withdraw thine heart from 
God, and then pleasest thou God, and hast the word of God 
for thee. And in all things look on the word of God, and 
therein put thy trust, and not in a visor, in a disguised gar- 
ment, and a cut shoe. 

Seek the word of God in all things, and without the 
word of God do nothing, though it appear never so glorious. 
Whatsoever is done without the word of God that count 
idolatry. The kingdom of heaven is within us. (Luke xvii.) 
Wonder therefore at no monstrous shape, nor at any out- 
ward thing without the word: for the world was never drawn 
from God but with an outward shew and glorious appearance 
and shining of hypocrisy, and of feigned and visored fasting, 
praying, watching, singing, offering, sacrificing, hallowing 
of superstitious ceremonies, and monstrous disguising. 

Take this for an ensample: John Baptist which had tes- 
timony of Christ and of the gospel that there never rose a 
greater among womens' children, with his fasting, watching, 
praying, rayment, and strait living, deceived the Jews, 
and brought them in doubt whether John were very Christ 
or not, and yet no Scripture or miracle testifying it, so 
greatly the blind nature of man looketh on the outward 
shining of works and regardeth not the inward word which 
speaketh to the heart. When they sent to John asking him 
whether he were Christ, he denied it. When they asked him 
what he was, and what he said of himself? he answered not, 
I am he that watcheth, prayeth, drinketh no wine nor 
strong drink, eateth neither fish nor flesh, but live with wild 
honey and grasshoppers, and wear a coat of camel's hair and 
a girdle of a skin; but said, I am a voice of a crier. My 
voice only pertaineth to you. Those outward things ye wonder 
at, pertain to myself, only unto the taming of my body. 
To you am I a voice only, and that which I preach. My 
preaching (if it be received into a penitent or repenting heart) 
shall teach you how to live and please God, according as 
God shall shed out his grace on every man. John preached 
repentance, saying, Prepare the Lord's way and make his 
paths straight. The Lord's way is repentance, and not 
hypocrisy of man's imagination, and invention. It is not 
possible that the Lord Christ should come to a man, except 
he know himself and his sin, and truly repent: make his 
paths straight:— the paths are thelaw, if you understand it 
aright as God hath given it. Christ saith in the xviith of 
Matt. Elias shall first come, that is, shall come before 
Christ, and restore all things, meaning of John Baptist. 
John Baptist did restore the law and the Scripture unto the 
right sense and understanding, which the Pharisees partly 
had darkened and made of none effect through their own 
traditions; (Matt. xv.) Where Christ rebuketh them saying 
Why transgress ye the commandments of God through 
your traditions? and partly had corrupted it with glosses 
and false interpretations, that no man could understand it. 
Wherefore Christ rebuketh them, (Matt. xxiii.) saying, Woe 
be to you pharisees, hypocrites, which shut up the kingdom 
of heaven before men: ye enter not yourselves, neither suffer 
them that come to enter in: and partly did beguile the people 
and blind their eyes in disguising themselves, as thou readest 
in the same xxiiird chapter, how they made broad and large 
phylacteries, and did all their works to be seen of men, that 
the people should wonder at their disguisings and visoring 
themselves otherwise than God hath made them: and partly 
mocked them with hypocrisy of false holiness, in fasting, 
praying, and alms-giving; (Matt. vi.) And this did they for 
lucre, to be in authority, to sit in the consciences of people, 
and to be counted as God himself, that people should trust 
in their holiness, and not in God, as thou readest in the 
place above rehearsed; (Matt. xxiii.) Woe be to you pha- 
risees, hypocrites, which devour widows' houses under a 
colour of a long prayer. Counterfeit therefore nothing with- 
out the word of God, when thou understandest that it shall 
teach thee all things, how to apply outward things and 
whereunto to refer them. Beware of thy good intent, good 
mind, good affection, or zeal, as they call it. Peter of a good 
mind and of a good affection or zeal, chid Christ, (Matt. xvi.) 
because that he said he must go to Jerusalem, and there be 
slain; but Christ called him Satan for his labour, a name 
that belongeth to the devil, and said, That he perceived not 
godly things but worldly. Of a good intent, and of a fervent 
affection to Christ, the sons of Zebedee would have had 
fire to come down from heaven to consume the Samaritans, 
(Luke ix.) but Christ rebuked them, saying that they wist 
not of what Spirit they were: that is, that they understood 
not how that they were altogether worldly and fleshly minded. 
Peter smote Malchus of a good zeal, but Christ con- 
demned his deed. The very Jews of a good intent and of 
a good zeal slew Christ and persecuted the apostles as 
Paul beareth them record; (Rom. x.) I bear them record 
(saith he) that they have a fervent mind to Godward, but 
not according to knowledge. It is another thing then, to 
do of a good mind, and to do of knowledge. Labour 
for knowledge that thou mayest know God's will, and 
what he would have thee to do. Our mind, intent and 
affection or zeal, are blind, and all that we do of them is 
damned of God, and for that cause hath God made a 
testament between him and us, wherein is contained both 
what he would have us to do, and what he would have us 
to ask of him. See therefore that thou do nothing to 
please God withal but that he commandeth, neither ask 
any thing of him, but that he hath promised thee. The 
Jews also as it appeareth (Acts vii.) slew Stephen of a 
good zeal; because he proved, by the Scripture, that God 
dwelleth not in churches or temples made with hands. 
The churches at the beginning were ordained, that the 
people should thither resort to hear the word of God there 
preached only, and not for the use wherein they now are. 
The temple wherein God will be worshipped, is the heart 
of man. For God is a Spirit (saith Christ, John iv.) and 
will be worshipped in the Spirit and in truth: that is, 
when a penitent heart consenteth unto the law of God, 
and with a strong faith longeth for the promises of God. 
So is God honoured on all sides, in that we count him 
righteous in all his laws and ordinances, and also trust in 
all his promises. Other worshipping of God is there 
none, except we make an idol of him. 


It shall be recompensed thee, at the rising again of 
the righteous. (Luke xiv.) Read the text before, and 
thou shalt perceive that Christ doth here that same that he 
doth Mat. v. that is, he putteth us in remembrance of our 
duty, that we be to the poor as Christ is to us, and also 
teacheth us, how that we can never know whether our 
love be right, and whether it spring of Christ or no, as long 
as we are but kind to them only which do as much for us 
again. But and we be merciful to the poor, for conscience 
to God, and of compassion and hearty love, which com- 
passion and love spring of the love we have to God in 
Christ, for the pure mercy and love that he hath shewed 
on us: then have we a sure token that we are beloved of 
God, and washed in Christ'a blood, and elect by Christ's 
deserving unto eternal life. 

The Scripture speaketh as a father doth to his young 
son, Do this or that, and then will I love thee; yet the 
father loveth his son first, and studieth with all his power 
and wit to overcome his child with love, and with kindness 
to make him do that which is comely, honest, and good for 
itself. A kind father and mother love their children even 
when they are evil, that they would shed their blood to 
make them better, and to bring them into the right way. 
And a natural child studieth not to obtain his father's love 
with works, but considereth with what love his father 
loveth him withal, and therefore loveth again, is glad to do 
his father's will, and studieth to be thankful. 

The spirit of the world understandeth not the speaking 
of God, neither the spirit of the wise of this world, 
neither the spirit of philosophers, neither the spirit of 
Socrates, of Plato, or of Aristotle's Ethics, as thou mayest 
see in the first and second chapter of the first to the Corin- 
thians. Though that many are not ashamed to rail and 
blaspheme, saying, How should he understand the Scripture 
seeing he is no philosopher, neither hath seen his meta- 
physic? Moreover they blaspheme, saying, How can he 
be a divine, and wotteth not what is subjectum in theo- 
logia? Nevertheless as a man, without the spirit of Aris- 
totle or philosophy, may by the Spirit of God understand 
Scripture: even so by the Spirit of God understandeth he 
that God is to be sought in all the Scripture, and in all 
things, and yet wotteth not what meaneth subjectum in 
theologia, because it is a term of their own making. If 
thou shouldest say to him that hath the Spirit of God, the 
love of God is the keeping of the commandments, and to 
love a man's neighbour is to shew mercy, he would, without 
arguing or disputing, understand, how that of the love of 
God springeth the keeping of his commandments, and of 
the love to thy neighbour springeth mercy. Now would 
Aristotle deny such speaking, and a Dun's man would 
make twenty distinctions. If thou shouldest say, (as saith 
John in the ivth of his Epistle) How can he that loveth 
not his neighbour whom he seeth, love God whom he seeth 
not? Aristotle would say, Lo, a man must first love his 
neighbour and then God, and out of the love to thy neigh- 
bour springeth the love to God. But he that feeleth the 
working of the Spirit of God, and also from what ven- 
geance the blood of Christ hath delivered him, understandeth 
how that it is impossible to love either father or mother, 
sister, brother, neighbour, or his own self aright, except it 
spring out of the love to God, and perceiveth that the love 
to a man's neighbour is a sign of the love to God, as good 
fruit declareth a good tree, and that the love to a man's 
neighbour accompanieth and followeth the love of God, as 
heat accompanieth and followeth fire. 

Likewise when the Scripture saith, Christ shall reward 
every man at the resurrection, or uprising again, according 
to his deeds, the spirit of Aristotle's Ethics would say,
Lo, with the multitude of good works mayest thou, and
must thou, obtain everlasting life. And also a place in 
heaven high or low, according as thou hast many or few 
good works: and yet wotteth not what a good work 
meaneth, as Christ speaketh of good works, as he that 
seeth not the heart, but outward things only. But he 
that hath God's Spirit understandeth it. He feeleth that 
good works are nothing but fruits of love, compassion, 
mercifulness, and of a tenderness of heart which a 
Christian hath to his neighbour, and that love springeth 
of that love which he hath to God, to his will and com- 
mandments, and understandeth also, that the love which 
man hath to God springeth of the infinite love and bot- 
tomless mercy which God in Christ shewed first to us 
as saith John in the Epistle and Chapter above rehearsed. 
In this (saith he) appeareth the love of God to usward, 
because that God sent his only begotten Son into the 
world that we might live through him. Herein is love, 
not that we loved God, but that he loved us, and sent his 
Son to make agreement for our sins. In conclusion, a 
Christian man feeleth that that unspeakable love and mercy 
which God hath to us, and that Spirit which worketh all 
things that are wrought according to the will of God, 
and that love wherewith we love God, and that love 
which we have to our neighbour, and that mercy and com- 
passion which we shew on him, and also that eternal life 
which is laid up in store for us in Christ, are altogether 
the gift of God, through Christ's purchasing. 

If the Scripture said always, Christ shall reward thee ac- 
cording to thy faith, or according to thy hope and trust thou 
hast in God, or according to the love thou hast to God and thy 
neighbour, so were it true also as thou seest, 1 Pet i. Receiv- 
ing the end or reward of our faith, the health or salvation of 
your souls. But the spiritual things could not be known save 
by their works, as a tree cannot be known but by her 
fruit. How could I know that I loved my neighbour, if 
never occasion were given me to shew mercy unto him? 
how should I know that I loved God, if I never suffered 
for his sake? how should I know that God loved me, if 
there were no infirmity, temptation, peril aud jeopardy 
whence God should deliver me? 


There is no man that forsaketh house, either father, or 
mother, either brethren or sisters, wife or children for the 
kingdom of heavens' sake, which shall not receive much 
more in this world, and in the world to come everlasting 
life. (Luke xviii.) 

Here seest thou that a Christian man in all his works 
hath respect to nothing, but unto the glory of God only, 
and to the maintaining of the truth of God, and doth, 
and leaveth undone all things of love, to the glory and 
honour of God only, as Christ teacheth in the Pater- 
noster. 

Moreover when he saith, He shall receive much more 
in this world, of a truth, yea he hath received much more 
already. For except he had felt the infinite mercy, 
goodness, love, and kindness of God, and the fellowship 
of the blood of Christ, and the comfort of the Spirit of 
Christ in his heart, he could never have forsaken any thing 
for God's sake. Notwithstanding (as saith Mark x.) 
Whosoever for Christ's sake and the gospel's, forsaketh 
house, brethren or sisters \&c. he shall receive an hun- 
dred fold houses, brethren \&c. that is spiritually. For 
Christ shall be all things unto thee. The angels, all 
Christians, and whosoever doth the will of the father, 
shall be father, mother, sister and brother unto thee, and 
all theirs shall be thine. And God shall take the care of 
thee, and minister all things unto thee, as long as thou 
seekest but his honour only. Moreover if thou wert 
Lord over all the world, yea, of ten worlds, before thou 
knewest God; yet was not thine appetite quenched, thou 
thirstedst for more. But if thou seek his honour only, 
then shall he slake thy thirst, and thou shalt have all that 
thou desirest, and shall be content, yea if thou dwell 
among infidels, and among the most cruel nations of 
the world; yet shall he be a father unto thee, and shall 
defend thee as he did Abraham, Isaac and Jacob, and all 
saints whose lives thou readest in the Scripture. For all 
that are past and gone before are but ensamples to strengthen 
our faith and trust in the word of God. It is the same 
God, and hath sworn to us all that he sware unto them, 
and is as true as ever he was, and therefore cannot but ful- 
fil his promises to us, as well as he did to them, if we 
believe as they did. 

The hour shall come when all they that are in the graves 
shall hear his voice, that is to say, Christ's voice, and shall 
come forth; they that have done good into the resurrection 
of life, and they that have done evil into the resurrection of 
damnation. (John v.) This, and all like texts, declare 
what followeth good works, and that our deeds shall testify 
with us, or against us at that day; and putteth us in re- 
membrance to be diligent, and fervent in doing good. 
Hereby mayest thou not understand that we obtain the 
favour of God, and the inheritance of life, through the 
merits of good works, as hirelings do their wages. For 
then shouldest thou rob Christ, of whose fulness we have 
received favour for favour; (John i.) that is, God's favour 
was so full in Christ, that for his sake he giveth us his 
favour, as affirmeth also Paul, (Eph. i.) He loved us in 
his beloved, by whom we have (saith Paul) redemption 
through his blood, and forgiveness of sins. The forgive- 
ness of sins, then, is our redemption in Christ, and not 
the reward of works. In whom (saith he in the same 
place) he chose us before the making of the world, that is 
long before we did good works. Through faith in Christ 
are we also the sons of God, as thou readest (John i.) In 
that they believed on his name, he gave them power to 
be the sons of God. God, with all his fulness aud 
riches, dwelleth in Christ, and out of Christ must we 
fetch all things. Thou readest also (John iii.) He that be- 
lieveth on the Son hath eternal life: and he that believeth 
not shall see no life, but the wrath of God abideth upon 
him. Here seest thou that the wrath and vengeance of 
God possesseth every man till faith come. Faith and 
trust in Christ expelleth the wrath of God, and bringeth 
favour, the Spirit, power to do good, and everlasting life. 
Moreover, until Christ hath given thee light, thou knowest 
not wherein standeth the goodness of thy works; and 
until his Spirit hath loosed thine heart, thou canst not 
consent unto good works. All that is good in us, both 
will and works, cometh of the favour of God, through 
Christ, to whom be all the laud. Amen. 


If any man will do his will (he meaneth the will of the 
Father,) he shall know of the doctrine whether it he of 
God, or whether I speak of myself. (John vii.) This 
text meaneth not that any man of his own strength, power, 
and free will, (as they call it,) can do the will of God 
before he hath received the Spirit and strength of 
Christ, through faith. But here is meant that which is 
spoken in the iiird of John, when Nicodemus marvelled 
how it were possible that a man should be born again: 
Christ answered, That which is born of the flesh is flesh, 
and that which is born of the Spirit is Spirit; as who 
should say, He that hath the Spirit through faith, and is 
born again, and made anew in Christ, understandeth the 
things of the Spirit, and what he that is spiritual meaneth. 
But he that is flesh, and as Paul saith, (1 Cor. ii.) a natu- 
ral man, and led of his blind reason only, can never 
ascend to the capacity of the Spirit. And he giveth an 
ensample, saying, The wind bloweth where it listeth, and 
thou hearest his voice, and wottest not whence he cometh, 
nor whither he will: so is every man that is born of the 
Spirit: he that speaketh of the Spirit can never be un- 
derstood of the natural man, which is but flesh, and 
savoureth no more than things of the flesh. So here 
meaneth Christ, If any man have the Spirit, and con- 
senteth unto the will of God, this same at once wotteth 
what I mean. 


If ye understand these things, happy are ye if ye do 
them. (John xiii.) A Christian man's heart is with the
will of God, with the law and commandments of God, 
and hungereth and thirsteth after strength to fulfil them, 
and mourneth day and night, desiring God, according to 
his promises, for to give him power to fulfil the will of 
God with love and lust: then testifieth his deed that he is 
blessed, and that the Spirit which blesseth us in Christ is 
in him, and ministereth such strength. The outward deed 
testifieth what is within us, as thou readest, (John v.) The
deeds which I do testify of me, saith Christ. And 
(John xiii.) Hereby shall all men know that ye are my 
disciples, if ye love one another. And (John xiv.) He that 
hath my commandments, and keepeth them, the same it 
is that loveth me. And again: He that loveth me keepeth 
my commandments; and he that loveth me not keepeth 
not my commandments: the outward deed testifieth of 
the inward heart. And (John xv.) If ye shall keep my 
commandments ye shall continue in my love, as I keep 
my Father's commandment, and continue in his love. 
That is, As ye see the love that I have to my Father, in 
that I keep his commandments, so shall ye see the love 
that ye have to me, in that ye keep my commandments. 

Thou mayest not think that our deeds bless us first, and 
that we prevent God and his grace in Christ, as though 
we, in our natural gifts, and being as we were born in 
Adam, looked on the law of God, and of our own strength 
fulfilled it, and so became righteous, and then, with that 
righteousness, obtained the favour of God. As philo- 
sophers write of righteousness, and as the righteousness of 
temporal law is, where the law is satisfied vnth the hypo- 
crisy of the outward deed. For contrary to that readest 
thou (John xv.) Ye have not chosen me (saith Christ,) but 
I have chosen you, that ye go and bring forth fruit, and 
that your fruit remain. And in the same chapter: I am 
a vine, and ye the branches; and without me can ye do 
nothing. With us, therefore, so goeth it. In Adam are 
we all, as it were, wild crab-trees, of which God chooseth 
whom he will, and plucketh them out of Adam, and 
planteth them in the garden of his mercy, and stocketh 
them, and grafteth the Spirit of Christ in them, which 
bringeth forth the fruit of the will of God; which fruit 
testifieth that God hath blessed us in Christ. Note this 
also; that as long as we live we are yet partly carnal and 
fleshly, (notwithstanding that we are in Christ, and though 
it be not imputed unto us for Christ's sake,) for there abideth 
and remaineth in us yet of the old Adam, as it were of the 
stock of the crab-tree; and ever among, when occasion is 
given him, shooteth forth his branches and leaves, bud, 
blossom, and fruit: against whom we must fight and subdue
him, and change all his nature by a little and a little, with
prayer, fasting, and watching; with virtuous meditation and
holy works, until we be altogether spirit. The kingdom of
heaven, saith Christ, (Matt. xiii.) is like leaven, which a
woman taketh and hideth in three pecks of meal till all 
be leavened. The leaven is the Spirit, and we the meal, 
which must be seasoned with the Spirit a little and a lit- 
tle, till we be throughout spiritual. 

Which shall reward every man according to his deed; 
(Rom. ii.) That is, according as the deeds are so shall every
man's reward be: the deeds declare what we are, as the 
fruit the tree; according to the fruit shall the tree be 
praised. The reward is given of the mercy and truth of 
God, and by the deserving and merits of Christ. Whoso- 
ever repenteth, believeth the gospel, and putteth his trust 
in Christ's merits, the same is heir with Christ of eternal 
life; for assurance whereof, the Spirit of God is poured 
into his heart as an earnest, which looseth him from the 
bonds of Satan, and giveth him lust and strength every 
day more and more, according as he is diligent to ask of 
God for Christ's sake: and eternal life followeth good 
living. I suppose, (saith St. Paul in the same epistle, the 
viiith chapter,) that the afflictions of this world are not
worthy of the glory which shall be shewed on us; that is 
to say, that which we here suffer can never deserve that 
reward which there shall be given us. 

Moreover, if the reward should depend and hang of the 
works, no man should be saved: forasmuch as our best
deeds, compared to the law, are damnable sin. By the
deeds of the law is no flesh justified, as it is written in the
third chapter to the Romans. The law justifieth not, but
uttereth the sin only, and compelleth and driveth the
penitent, or repenting sinner, to flee unto the sanctuary of 
mercy in the blood of Christ. Also repent we never so 
much, be we never so well willing unto the law of God, 
yet are we so weak, and the snares and occasions so in- 
numerable, that we fall daily and hourly: so that we could 
not but despair if the reward banged of the work. Who- 
soever ascribeth eternal life unto the deserving and merit 
of works, must fall in one of two inconveniences; either 
must he be a blind Pharisee, not seeing that the law 
is spiritual and he carnal, and look and rejoice in the out- 
ward shining of his deeds, despising the weak, and, in 
respect of them, justify himself; or else (if he see how 
that the law is spiritual, and he never able to ascend unto 
that which the law requireth,) he must needs despair. 
Let every Christian man, therefore, rejoice in Christ our 
hope, trust, and righteousness, in whom we are loved, 
chosen, and accepted unto the inheritance of eternal life; 
neither presuming in our perfectness, neither despairing 
in our weakness. The perfecter a man is the clearer is 
his sight, and seeth a thousand things which displease 
him, and also perfectness that cannot be obtained in this 
life; and therefore desireth to be with Christ, where 
is no more sin. Let him that is weak and cannot do that 
he would fain do, not despair, but turn to Him that is 
strong, and hath promised to give strength to all that ask 
of him in Christ's name; and complain to God, and desire 
him to fulfil his promises, and to God commit himself; 
and he shall of his mercy and truth strengthen him, and 
make him feel with what love he is beloved for Christ's 
sake, though he be never so weak. 


They are not righteous before God which hear the
law; but they which do the law shall be justified. (Rom. ii.)
This test is plainer than that it needeth to be expounded.
In the chapter before, Paul proveth that the law natural holp 
not the Gentiles, (as appeareth by the law, statutes, and 
ordinances which they made in their cities,) yet kept they 
them not. The great keep the small under, for their own
profit, with the violence of the law. Every man praiseth
the law as far forth as it is profitable and pleasant unto 
himself. But when his own appetites should be re- 
frained, then grudgeth he against the law. Moreover, he 
proveth that no knowledge holp the Gentiles. For though 
the learned men (as the philosophers,) came to the know- 
ledge of God by the creatures of the world, yet had they 
no power to worship God. In this second chapter proveth 
he that the Jews, (though they had the law written,) yet it 
holp them not: they could not keep it, but were idolaters, 
and were also murderers, adulterers, and whatsoever the 
law forbad. He concludeth, therefore, that the Jew is as 
well damned as the Gentile. If hearing of the law only 
might have justified, then had the Jews been righteous. 
But it requireth that a man do the law if he will be 
righteous; which, because the Jew did not, he is no less 
damned than the Gentile. The publishing and declaring 
of the law doth but utter a man's sin, and giveth neither 
strength nor help to fulfil the law. The law killeth thy 
conscience, and giveth thee no lust to fulfil the law. Faith 
in Christ giveth lust and power to do the law. Now, is it 
true, that he which doth the law is righteous, but that doth 
no man save he that believeth and putteth his trust in 
Christ. 


If any man's work that he hath built upon abide, he 
shall receive a reward. (1 Cor. ii.) The circumstance of 
the same chapter, that is to wit, that which goeth before 
and that which followeth, declareth plainly what is meant. 
Paul talketh of learning, doctrine, or preaching: he saith 
that he himself hath laid the foundation, which is Jesus 
Christ, and that no man can lay any other. He ex- 
horteth, therefore, every man to take heed what he buildeth 
upon; and borroweth a similitude of the goldsmith which 
trieth his metals with fire, saying that the fire, that is, 
the judgment of the Scripture, shall try every man's work, 
that is, every man's preaching and doctrine. If any 
build upon the foundation laid of Paul, I mean Jesus 
Christ, gold, silver, or precious stone, which are all one 
thing, and signify true doctrine, which, when it is exa- 
mined, the Scripture alloweth; then shall he have his re- 
ward, that is, he shall be sure that his learning is of God, 
and that God's Spirit is in him, and that he shall have the 
reward that Christ hath purchased for him. On the other 
side, if any man build thereon timber, hay, or stubble, 
which are all one, and signify doctrine of man's imagina- 
tion, traditions, and fantasies, which stand not with Christ 
when they are judged and examined by the Scripture, he 
shall suffer damage, but shall be saved himself, yet as it 
were through fire; that is, it shall be painful unto him 
that he hath lost his labour, and to see his building perish; 
notwithstanding, if he repent, and embrace the truth in 
Christ, he shall obtain mercy and be saved. But if Paul 
were now alive, and would defend his own learning, he 
should be tried through fire; not through fire of the 
judgment of Scripture, (for that light men now utterly 
refuse,) but by the Pope's law, and with fire of fagots. 


We must all appear before the judgment-seat of Christ, 
for to receive every man according to the deeds of his 
body: (2 Cor. v.) As thy deeds testify of thee so shall thy 
reward be. Thy deeds be evil, then is the wrath of God 
upon thee, and thine heart is evil; and so shall thy reward 
be, if thou repent not. Fear, therefore, and cry to God 
for grace, that thou mayest love his laws. And when 
thou lovest them, cease not till thou have obtained power 
of God to fulfil them; so shalt thou be sure that a good 
reward shall follow. Which reward, not thy deeds, but 
Christ's have purchased for thee; whose purchasing also 
is that lust which thou hast to God's law, and that might 
wherewith thou fulfillest them. Remember also, that a 
reward is rather called that which is given freely, than that 
which is deserved. That which is deserved is called (if 
thou wilt give him his right name,) hire or wages. A 
reward is given freely, to provoke unto love and to make 
friends. 

Remember, that whatsoever good thing any man doth, 
that shall he receive of the Lord: (Eph. vi.) remembering 
that ye shall receive of the Lord the reward of inheritance. 
(Col. iii.) These two texts are exceeding plain. Paul 
meaneth, as Peter doth, (1 Pet. ii.) that servants should 
obey their masters with all their hearts, and with good 
will, though they were never so evil. Yea, he will 
that all who are under power obey, even of heart, and of 
conscience to God, because God will have it so, be the 
rulers never so wicked. The children must obey father 
and mother, be they never so cruel or unkind; likewise 
the wife her husband, the servant his master, the subjects 
and commons their lord or king. Why? For ye serve the 
Lord, saith he in Coloss. iii. We are Christ's, and Christ 
has bought us, as thou readest Rom. xiv. 1 Cor. vi. 
1 Pet. i. Christ is our Lord, and we his possession, and 
his also is the commandment. Now, ought not the cruel- 
ness and churlishness of father and mother, of husband, 
master, lord, or king, cause us to hate the commandment 
of our so kind a Lord Christ; which spared not his blood 
for our sakes; which also hath purchased for us with his 
blood that reward of eternal life, which life shall follow 
the patience of good living, and whereunto our good deeds 
testify that we are chosen. Furthermore, we are so car- 
nal, that if the rulers be good, we cannot know whether 
we keep the commandment for the love that we have to 
Christ, and to God through him, or no. But and if thou 
canst find in thine heart, to do good unto him that rewardeth 
thee evil again, then art thou sure that the same Spirit is 
in thee that is in Christ. And it followeth in the same 
chapter to the Colossians, He that doth wrong shall re- 
ceive for the wrong that he hath done. That is, God 
shall avenge thee abundantly, which seeth what wrong is 
done unto thee, and yet suffereth it for a time, that thou 
mightest feel thy patience and the working of his Spirit in 
thee, and be made perfect. Therefore, see that thou 
not once desire vengeance, but remit all vengeance unto 
God, as Christ did, which saith Peter, (1 Pet. ii.) when 
he was reviled, reviled not again, neither threatened when 
he suffered. Unto such obedience, unto such patience, 
unto such a poor heart, and unto such feeling, is Paul's 
meaning to bring all men, and not unto the vain disputing 
of them that ascribe so high a place in heaven unto their 
pilled merits; which, as they feel not the working of 
God's Spirit, so obey they no man. If the king do unto 
them but right, they will interdict the whole realm, curse, 
excommunicate, and send them down far beneath the 
bottom of hell, as they have brought the people out of 
their wits, and made them mad to believe. 


Thy prayers and alms are come up into remembrance 
in the presence of God; (in the Acts x.) that is, God for- 
getteth thee not, though he cometh not at the first calling, 
he looketh on and beholdeth thy prayers and alms. 
Prayer cometh from the heart. God looketh first on the 
heart and then on the deed, as thou readest (Gen. iv.) 
God beheld or looked first on Abel, and then on his of- 
fering. If the heart be unpure the deed verily pleaseth not, 
as thou seest in Cain. Mark the order, in the beginning of 
the chapter thou readest, There was a certain man named 
Cornelius which feared God, gave much alms, and prayed 
God alway. He feared God, that is, he trembled and 
quaked to break the commandments of God. Then prayed 
he alway. Prayer is the fruit, effect, deed or act of faith, and 
is nothing but the longing of the heart for those things which 
a man lacketh, and which God hath promised to give him. 
He doth also alms, alms is the fruit, effect, or deed of com- 
passion and pity which we have to our neighbour. O a glori- 
ous faith and a right which so trusteth God, and believeth 
his promises, that she feareth to break his commandments, 
and is also merciful unto her neighbour! This is that faith 
whereof thou readest, namely m Peter, Paul, and John, that 
we are thereby both justified and saved, and whosoever ima- 
gineth any other faith deceiveth himself and is a vain disputer, 
and a brawler about words, and hath no feeling in his heart. 

Though thou consent to the law, that it is good, 
righteous, and holy, sorrowest and repentest, because thou 
hast broken it, mournest because thou hast no strength to 
fulfil it, yet art not thou thereby at one with God: yea, 
thou shouldest shortly despair and blaspheme God, if the 
promises of forgiveness and of help were not thereby, and 
faith in thine heart to believe them; faith therefore set- 
teth thee at one with God. 

Faith prayeth always. For she hath always her infirmities 
and weaknesses before her eyes, and also God's promises, 
for which she always longeth, and in all places. But blind 
unbelief prayeth not alway, nor in all places, but in the 
church only, and that in such a church where it is not law- 
ful to preach God's promises, neither to teach men to trust 
therein. Faith, when she prayeth, setteth not her good 
deeds before her, saying, Lord, for my good deeds do this 
or that; nor bargaineth with God, saying, Lord, grant me 
this, or do this or that, and I will do this or that for thee; 
as mumble so much daily, go so far, or fast this or that fast, 
enter this religion or that, with such other points of infi- 
delity, yea, rather idolatry; but she setteth her infirmities 
and her lack before her face, and God's promises, saying, 
Lord, for thy mercy and truth which thou hast sworn, be 
merciful unto me, and pluck me out of this prison and out of 
this hell, and loose the bonds of Satan, and give me power 
to glorify thy name: faith therefore justifieth in the heart 
and before God, and the deeds justify outwardly before 
the world, that is, testify only before men, what we are in- 
wardly before God. 

Whosoever looketh in the perfect law of liberty and con- 
tinueth therein, (if he be not a forgetful hearer, but a doer 
of the work) he shall be happy in his deed: (James i.) The 
law of liberty, that is, which required a free heart, or if 
thou fulfil it, declareth a free heart, loosed from the bonds 
of Satan. The preaching of the law maketh no man 
free, but bindeth,for it is the key that bindeth all conciences 
unto eternal damnation, when it is preached; as the pro- 
mises or gospel is the key that looseth all consiences that 
repent, when they are bound through preaching of the 
law. He shall be happy in his deed, that is, by his deed shall 
he know that he is happy and blessed of God, which hath 
given him a good heart, and power to fulfil the law; by 
hearing the law thou shall not know that thou art blessed, but 
if thou do it, it declareth that thou art happy and blessed. 

Was not Abraham justified of his deeds when he offered 
his son Isaac upon the altar? (James iii.) His deed justified 
him before the world, that is, it declared and uttered the 
faith which both justified him before God, and wrought that 
wonderful work, as James also affirmeth. 

Was not Rahab the harlot justified when she received the 
messengers, and sent them out another way? (James iii.) 
That is likewise outwardly, but before God she was justified 
by faith which wrought that outward deed, as thou mayest 
see, Josh. ii. She had heard what God had done in 
Egypt, in the red sea, in the desert, and unto the two 
kings of the Amoreans, Silion and Og. And she confessed 
saying, Your Lord God, he is God in heaven above and in 
earth beneath. She also believed that God, as he had 
promised the children of Israel, would give them the land 
wherein she dwelt; and consented thereunto, submitted her- 
self unto the will of God, and holp God, (as much as in her 
was) and saved his spies and messengers. The other feared 
that which she believed, but resisted God with all their 
might, and had no power to submit themselves unto the will 
of God. And therefore perished they, and she was saved 
and that through faith: as we read Heb. xi. where thou 
mayest see how the holy fathers were saved through faith, and 
how faith wrought in them. Faith is the goodness of all the 
deeds that are done within the law of God, and maketh 
them good and glorious, seem they never so vile; and un- 
belief maketh them damnable, seem they never so glorious. 


As pertaining to that which James in this iiird chapter 
saith, What availeth though a man say that he hath faith 
if he have no deeds? can faith save him? and again, faith 
without deeds is dead in itself; and the devils believe and 
tremble: and as the body without the spirit is dead, even 
so faith without deeds is dead; it is manifest and clear, 
that he meaneth not of the faith whereof Peter and Paul 
speak in their epistles. John in his gospel and first epistle, 
and Christ in the gospel, when he saith, Thy faith hath 
made thee safe, be it to ihee according to thy faith, or 
great is thy faith, and so forth; and of which James him- 
self speaketh in the first chapter, saying, Of his own will 
begat he us with the word of life, that is, in believing the 
the promises wherein is life, are we made the sons of 
God. 

Which thing I also thiswise prove: Paul saith how 
shall or can they believe without a preacher? how should 
they preach except they were sent? Now I pray you when 
was it heard that God sent any man to preach unto the 
devils, or that he made them any good promise? He threat- 
eneth them oft, but never sent any ambassadors to preach 
any atonement between him and them. Take an ensample
that thou mayest understand: let there be two poor men
both destitute of rayment in a cold winter, the one strong 
that he feeleth no grief, the other grievously mourning for 
pain of the cold. I then come by, and, moved with pity and 
compassion, say unto him that feeleth his disease, Come to 
such a place and I will give thee raiment sufficient. He 
believeth, cometh and obtaineth that which I have promised. 
That other seeth all this and knoweth it, but is partaker of 
nought, for he hath no faith, and that is because there is 
no promise made him. So is it of the devils, the devils have 
no faith, for faith is but earnest believing of God's 
promises. Now are there no promises made unto the de- 
vils, but sore threatenings. The old philosophers knew that 
there was one God, but yet had no faith, for they had
no power to seek his will, neither to worship him. The
Turks and the Saracens know that there is one God, but
yet have no faith, for they have no power to worship God 
in spirit, to seek his pleasure, and to submit them unto his 
will. They made an idol of God, (as we do for the most 
part) and worshipped him every man after his own ima- 
gination, and for a sundry purpose. What we will have 
done, that must God do, and to do our will worship we 
hymn and pray unto him; but what God will have done, 
that will neither Turk nor Saracen, nor the most part of us 
do. Whatsoever we imagine righteous, that must God ad- 
mit; but God's righteousness will not our hearts admit. 
Take another ensample: let there be two such as I spake 
of before, and I promise both, and the one because he 
feeleth not his disease cometh not: so is it of God's 
promises: no man is holpen by them but sinners that feel 
their sins, mourn and sorrow for them, and repent with 
all their hearts. For John Baptist went before Christ and 
preached repentance, that is, he preached the law of God 
right, and brought the people into knowledge of them- 
selves, and unto the fear of God, and then sent them unto 
Christ to be healed. For in Christ and for his sake only, 
hath God promised to receive us unto mercy, to forgive us, 
and to give us power to resist sin. How shall God save 
thee, when thou knowest not thy damnation? how shall 
Christ deliver thee from sin, when thou wilt not know- 
ledge thy sin? Now I pray thee how many thousands are 
there of them that say, I believe that Christ was born of 
a virgin, that he died, that he rose again, and so forth, and 
thou canst not bring them in belief that they have any sin 
at all! How many are there of the same sort, which thou 
canst not make believe that a thousand things are sin which 
God damneth for sin all the Scripture throughout! as to 
buy as good cheap as he can, and to sell as dear as he can, 
to raise the market of corn and victuals for his own vantage, 
without respect of his neighbour, or of the poor of the com- 
mon wealth, and such like. Moreover how many hundred 
thousand are there, which when they have sinned, and 
knowledge their sins; yet trust in a bald ceremony, or in 
a lousy friar's coat and merits, or in the prayers of them
that devour widows' houses, and eateth the poor out of
house and harbour, in a thing of his own imagination, in 
a foolish dream, and a false vision; and not in Christ's 
blood, and in the truth that God hath sworn! All these 
are faithless, for they follow their own righteousness, and 
are disobedient unto all manner [of] righteousness of 
God: both unto the righteousness of God's law, where- 
with he damneth all our deeds, (for though some of them 
see their sins for fear of pain, yet had they rather that 
such deeds were no sin,) and also unto the righteousness 
of the truth of God in his promises, whereby he saveth 
all that repent and believe them. For though they believe 
that Christ died, yet believe they not that he died for their 
sins, and that his death is a sufficient satisfaction for 
their sins, and that God for his sake will be a father unto 
them, and give them power to resist sin. 

Paul saith to the Romans in the xth chapter, If thou
confess with thy mouth that Jesus is the Lord, and 
believe with thine heart that God raised him up from 
death, thou shalt be safe. That is, if thou believe he 
raised him up again for thy salvation. Many believe 
that God is rich and almighty, but not unto themselves, 
and that he will be good unto them, and defend them, 
and be their God. 

Pharaoh for pain of the plague, was compelled to 
confess his sins, but had yet no power to submit himself
unto the will of God, and to let the children of Israel
go, and to lose so great profit for God's pleasure, As 
our prelates confess their sins, saying, Though we be never 
so evil, yet have we the power. And again, the scribes 
and pharisees (say they) sat in Moses's seat, do as they 
teach, but not as they do; thus confess they that they are 
abominable. But to the second I answer, if they sat on 
Christ's seat they would preach Christ's doctrine, now 
preach they their own traditions, and therefore not to be 
heard. If they preached Christ, we ought to hear them 
though they were never so abominable, as they of them- 
selves confess, and have yet no power to amend, neither 
to let loose Christ's flock to serve God in the Spirit, 
which they hold captive, compelling them to serve their 
false lies. The devils felt the power of Christ, and were 
compelled against their wills to confess that he was the 
Son of God, but had no power to be content therewith, 
neither to consent unto the ordinance and eternal counsel 
of the everlasting God; as our prelates feel the power 
of God against them, but yet have no grace to give room 
unto Christ, because that they (as the devil's nature is) 
will themselves sit in his holy temple, that is to wit, the 
consciences of men. 

Simon Magus believed, (Acts. iii.) with such a faith 
as the devils confessed Christ, but had no right faith, as 
thou seest in the said chapter. For he repented not, 
consenting unto the law of God. Neither believed the 
promises or longed for them, but wondered only at the mi- 
racles which Philip wrought, and because that he himself 
in Philip's presence had no power to use his witchcraft, 
sorcery and art magic, wherewith he mocked and deluded 
the wits of the people. He would have bought the gift 
of God, to have sold it much dearer, as his successors 
now do, and not the successors of Simon Peter. For 
were they Simon Peter's successors, they would preach 
Christ as he did; but they are Simon Magus's successors, 
of which Simon Peter well prophesied in the second chapter 
of his second Epistle, saying, There were false prophets 
among the people (meaning of the Jews) even as there 
shall be false teachers or doctors among you, which privily 
shall bring in sects damnable, (sects is part-taking, as one 
holdeth of Francis, another of Dominic, which thing also 
Paul rebuketh, I Cor i. and iii.) even denying the Lord 
that bought them (for they will not be saved by Christ, 
neither suffer any man to preach him to other.) And 
many shall follow their damnable ways, (thou wilt say, 
Shall God suffer so many to go out of the right way so 
long? I answer, many must follow their damnable ways, 
or else must Peter be a false prophet) by which the way 
of truth shall be evil spoken of, (as it is now at this present 
time, for it is heresy to preach the truth) and through 
covetousness shall they with feined words make merchan- 
dise of you. Of their merchandise and covetousness it needeth 
not to make rehearsal, for they that be blind see it evidently. 

Thus seest thou that James, when he saith, Faith without 
deeds is dead, and as the body without the spirit is dead,
so is faith without deeds, and the devils believe; that he
meaneth not of the faith and trust that we have in the
truth of God's promises, and his holy Testament, made
unto us in Christ's blood; which faith followeth repentance, 
and the consent of the heart unto the law of God, and 
maketh a man safe, and setteth him at peace with God. 
But speaketh of that false opinion and imagination where- 
with some say, I believe that Christ was born of a virgin, 
and that he died, and so forth. That believe they verily, 
and so strongly, that they are ready to slay whosoever 
would say the contrary. But they believe not that 
Christ died for their sins, and that his death hath appeased 
the wrath of God, and hath obtained for them all that 
God hath promised in the Scripture. For how can they 
believe that Christ died for their sins, and that he is their 
only and sufficient Saviour, seeing that they seek other 
Saviours of their own imagination, and seeing that they 
feel not their sins, neither repent, except that some repent 
(as I above said) for fear of pain, but for no love, nor consent 
unto the law of God, nor longing that they have for 
those good promises which he hath made them in Christ's 
blood. If they repented and loved the law of God, and 
longed for that help which God hath promised to give to 
all that call on him for Christ's sake, then verily must God's 
truth give them power and strength to do good works, when- 
soever occasion were given, either must God be a false God. 
But let God be true, and every man a liar as Scripture 
saith. For the truth of God lasteth ever, to whom only 
be all honour and glory for ever. Amen. 

Be not offended, most dear reader, that divers things 
are overseen, through negligence in this little treatise. For 
verily the chance was such, that I marvel that it is so well 
as it is. Moreover it becometh the book even so to come 
as a mourner, and in vile apparel to wait on his master, 
which sheweth himself now again, not in honour and 
glory, as between Moses and Elias; but in rebuke and 
shame, as between two murderers, to try his true friends, 
and to prove whether there be any faith on the earth. 

