\section*{\center{THE\\
OBEDIENCE OF A CHRISTIAN MAN,\\
\small{AND\\
HOW CHRISTIAN RULERS OUGHT\\
TO GOVERN.}}}
\addcontentsline{toc}{section}{The Obedience of a Christian Man}

\begin{center}\tiny
WHEREIN ALSO (IF THOU MARK DILIGENTLY) THOU SHALT
FIND EYES TO PERCEIVE THE CRAFTY CONYEYANCE OF ALL JUGGLERS.
\end{center}

\subsection*{Preface}
\begin{center}
WILLIAM TYNDALE\\
OTHERWISE CALLED HITCHINS,\\
TO THE READER. 
\end{center}

GRACE, peace, and increase of knowledge in 
our Lord Jesus Christ be with thee, reader, 
and with all that call on the name of the Lord 
unfeignedly, and with a pure conscience. Amen. 

Let it not make thee despair, neither yet discourage
thee, O reader, that it is forbidden thee 
in pain of life and goods, or that it is made 
breaking of the king's peace, or treason unto his
highness to read the word of thy soul's health;
but much rather be bold in the Lord and comfort
thy soul, forasmuch as thou art sure, and hast an 
evident token through such persecution, that it is 
the true word of God; which word is ever hated 
of the world, neither was ever without persecution,
(as thou seest in all the stories of the 
Bible, both of the New Testament and also of the 
Old,) neither can be, no more than the sun can be
without his light. And forasmuch as contrarywise,
thou art sure that the Pope's doctrine is
not of God, which, as thou seest, is so agreeable 
unto the world, and is so received of the world, 
or which rather so receiveth the world and the 
pleasures of the world, and seeketh nothing but 
the possessions of the world, and authority in the 
world, and to bear a rule in the world; and persecuteth
the word of God, and with all wiliness 
driveth people from it, and with false and sophistical
reasons maketh them afraid of it: yea curseth 
them and excommunicateth them, and bringeth 
them in belief that they be damned if they look 
on it; and that it is but doctrine to deceive men; 
and moveth the blind powers of the world to 
slay, with fire, water, and sword, all that cleave 
unto it: for the world loveth that which is his, 
and hateth that which is chosen out of the world 
to serve God in the Spirit, as Christ saith to his 
disciples, John xv. If ye were of the world the 
world would love his own; but I have chosen you 
out of the world, and therefore the world hateth 
you. 

Another comfort hast thou, that as the weak 
powers of the world defend the doctrine of the 
world: so the mighty power of God defendeth 
the doctrine of God, which thing thou shalt 
evidently perceive if thou call to mind the wonderful
deeds which God hath ever wrought for 
his word in extreme necessity, since the world 
began, beyond all man's reason, which are written, 
as Paul saith, (Rom. xv.) for our learning and not 
for our deceiving, that we, through patience 
and comfort of the Scripture, might have 
hope. The nature of God's word is to fight 
against hypocrites. It began at Abel, and hath 
ever since continued, and shall, I doubt not, until
the last day. And the hypocrites have alway 
the world on their sides, as thou seest in the 
time of Christ; they had the elders, that is to
wit, the rulers of the Jews on their side; they 
had Pilate and the emperor's power on their side; 
they had Herod also on their side: moreover 
they brought all their worldly wisdom to pass, and 
all that they could think or imagine to serve for 
their purpose. First, to fear the people withal,
they excommunicated all that believed in him,
and put them out of the temple, as thou seest 
John ix. Secondly, they found the means to 
have him condemned by the emperor's power, and 
made it treason to Caesar to believe in him. 
Thirdly, they obtained to have him hanged as a 
thief or a murderer, which after their belly wisdom,
was a cause above all causes that no man 
should believe in him: for the Jews take it for a 
sure token of everlasting damnation, if a man be 
hanged; for it written in their law, (Deut. xxi.) 
Cursed is whosoever hangeth on [a] tree. Moses
also in the same place commandeth, If any 
man be hanged to take him down the same day 
and bury him, for fear of polluting or defiling 
the country, that is, lest they should bring the 
wrath and curse of God upon them. And therefore
the wicked Jews themselves, which with so 
venomous hate persecuted the doctrine of Christ, 
and did all the shame that they could do unto 
him, though they would fain have had Christ to 
hang still on the cross, and there to rot, as he 
should have done by the emperor's law: yet for 
fear of defiling their sabbath, and of bringing 
the wrath and curse of God upon them, begged 
of Pilate to take him down, (John xix.) which 
was against themselves. 

Finally, when they had done all they could, and 
that they thought sufficient, and when Christ was 
in the heart of the earth, and so many bills and 
poleaxes about him, to keep him down, and when 
it was past man's help, then holp God. When 
man could not bring him again, God's truth 
fetched him again. The oath that God had 
sworn to Abraham, to David, and to other holy 
fathers and prophets, raised him up again to bless 
and save all that believe in him. Thus became 
the wisdom of the hypocrites foolishness. Lo, this 
was written for thy learning and comfort.

How wonderfully were the children of Israel 
locked in Egypt! In what tribulation, cumbrance, 
and adversity were they in! The land also that 
was promised them was far off, and full of great 
cities, walled with high walls up to the sky, and 
inhabited with great giants; yet God's truth 
brought them out of Egypt, and planted them in 
the land of the giants. This is also written for 
our learning: for there is no power against God's,
neither any wisdom against God's wisdom: he is 
stronger and wiser than all his enemies. What 
holp it Pharaoh to drown the men children: so little,
(I fear not,) shall it at the last help the pope 
and his bishops, to burn our men children, 
which manfully confess that Jesus Christ is the 
Lord, and that there is no other name given unto 
men to be saved by, as Peter testifieth, Acts iv. 

Who dried up the red sea? who slew Goliath? 
who did all those wonderful deeds which thou 
readest in the Bible? who delivered the Israelites 
evermore from thraldom and bondage, as soon as 
they repented and turned to God? Faith verily, 
and God's truth, and the trust in the promises 
which he had made. Read the xith to the Hebrews
for thy consolation. 

When the children of Israel were ready to 
despair for the greatness and the multitude of the 
giants, Moses comforted them ever, saying, Remember
what your Lord God hath done for you 
in Egypt, his wonderful plagues, his miracles,
his wonders, his mighty hand, his stretched out 
arm, and what he hath done for you hitherto. He 
shall destroy them, he shall take their hearts 
from them, and make them fear and fly before 
you. He shall storm them, and stir up a tempest 
among them, and scatter them, and bring them to 
nought. He hath sworn, he is true, he will fulfil 
the promises that he hath made unto Abraham, 
Isaac, and Jacob. This is written for our learning: 
for verily he is a true God, and is our God as
well as theirs, and his promises are with us, as
well as with them; and he present with us as well 
as he was with them. If we ask, we shall obtain; 
if we knock, he will open; if we seek, we shall 
find; if we thirst, his truth shall fulfil our lust. 
Christ is with us until the world's end. (Matt. 
the last.) Let this little flock be bold therefore: 
for if God be on our side, what matter maketh 
it who be against us? be they bishops, cardinals, 
popes, or whatsoever names they will.



Mark this also, if God send thee to the sea, 
and promise to go with thee, and to bring 
thee safe to land, he will raise up a tempest 
against thee, to prove whether thou wilt abide 
by his word, and that thou mayest feel thy 
faith and perceive his goodness: for if it 
were always fair weather and thou never 
brought into such jeopardy whence his mercy 
only delivered thee; thy faith should be but 
a presumption, and thou shouldest be ever 
unthankful to God and merciless unto thy neighbour.

If God promise riches, the way thereto is poverty.
Whom he loveth, him he chasteneth;
whom he exalteth, he casteth down: whom he 
saveth, he damneth first: he bringeth no man to 
heaven, except he send him to hell first: if he 
promise life, he slayeth first: when he buildeth,
he casteth all down first: he is no patcher, 
he cannot build on another man's foundation:
he will not work until all be past remedy, and 
brought unto such a case, that men may see, how 
that his hand, his power, his mercy, his goodness, 
and truth, hath wrought altogether. He will let no 
man be partaker with him of his praise and glory:
his works are wonderful, and contrary unto man's 
works. Whoever, save he, delivered his own 
son, his only son, his dear son, unto the death, 
and that for his enemies' sake: to win his enemy, 
to overcome him with love, that he might see 
love, and love again, and of love to do likewise to 
other men, and to overcome them with well doing. 



Joseph saw the sun and the moon, and the 
twelve stars worshipping him. Nevertheless, 
ere that came to pass, God laid him where he 
could neither see sun nor moon, neither any star 
of the sky, and that many years; and also undeserved;
to nurture him, to humble, to meek, and
to teach him God's ways, and to make him apt 
and meet for the room and honour against he 
came to it, that he might perceive and feel that 
it came of God, and that he might be strong in 
the Spirit to minister it godly. 

He promised the children of Israel a land 
with rivers of milk and honey; but brought them 
for the space of forty years into a land, where 
not only rivers of milk and honey were not, but 
where so much as a drop of water was not; to 
nurture them, and to teach them, as a father 
doth his son, and to do them good at the latter 
end; and that they might be strong in their 
spirit and souls, to use his gifts and benefits 
godly, and after his will.

He promised David a kingdom, and immemediately
stirred up king Saul against him to 
persecute him; to hunt him as men do hares 
with greyhounds, and to ferret him out of every 
hole, and that for the space of many years; to 
tame him, to meek him, to kill his lusts; to make 
him feel other men's diseases; to make him merciful;
to make him understand that he was made 
king to minister and to serve his brethren, and 
that he should not think that his subjects were 
made to minister unto his lusts, and that it were 
lawful for him to take away from them life and 
goods at his pleasure.

Oh, that our kings were so nurtured now-a- 
days! which our holy bishops teach of a far 
other manner, saying, Your Grace shall take 
your pleasure; yea, take what pleasure you list, 
spare nothing; we shall dispense with you, we 
have power, we are God's vicars: and let us 
alone with the realm, we shall take pain for you, 
and see that nothing be well: your Grace shall 
but defend the faith only. 

Let us, therefore, look diligently whereunto 
we are called, that we deceive not ourselves. 
We are called, not to dispute as the Pope's disciples
do; but to die with Christ, that we may 
live with him; and to suffer with him, that we 
may reign with him. We be called unto a kingdom
that must be won by suffering only, as a 
sick man winneth health. God is he that doth all 
things for us, and fighteth for us, and we do but 
suffer only, Christ saith, (John xx.) As my Father 
sent me, so send I you; and (John xv.) If they 
persecute me, then shall they persecute you:
and Christ saith, (Matt. x.) I send you forth as 
sheep among wolves. The sheep fight not, but 
the shepherd fighteth for them, and careth for 
them. Be harmless as doves, therefore, saith 
Christ, and wise as serpents. The doves imagine
no defence, nor seek to avenge themselves.
The serpent's wisdom is, to keep his 
head, and those parts wherein his life resteth.
Christ is our head, and God's word is that 
wherein our life resteth. To cleave, therefore, 
fast unto Christ, and unto those promises which 
God hath made us for his sake, is our wisdom. 
Beware of men, (saith he,) for they shall deliver 
you up unto their councils, and shall scourge 
you; and ye shall be brought before rulers and 
kings for my sake: the brother shall betray, or 
deliver, the brother to death, and the father the 
son; and the children shall rise against father 
and mother, and put them to death. Hear what 
Christ saith more: The disciple is not greater 
than his master, neither the servant greater, or 
better than his lord. If they have called the 
good man of the house Beelzebub, how much 
rather shall they call his household servants so? 
And (Luke xivth.) saith Christ, Which of you, 
disposed to build a tower, sitteth not down first 
and counteth the cost, whether he have sufficient 
to perform it? Lest when he hath laid the foundation,
and then not able to perform it, all that 
behold it begin to mock him, saying, This man 
began to build, and was not able to make an 
end: so likewise, none of you that forsaketh not 
all that he hath, can be my disciple. Whosoever,
therefore, casteth not this aforehand, I must 
jeopard life, goods, honour, worship, (and all 
that there is, for Christ's sake,) deceiveth himself,
and maketh a mock of himself, to the godless
hypocrites and infidels. No man can serve 
two masters, God and Mammon; that is to say, 
wicked riches also. (Matt. vi.) Thou must love 
Christ above all things: but that doest thou 
not if thou be not ready to forsake all for his 
sake: if thou have forsaken all for his sake, then 
art thou sure that thou lovest him. Tribulation
is our right baptism, and is signified by plunging 
into the water. We that are baptized in the 
name of Christ (saith Paul, Rom. vi.) are baptized
to die with him. 

The Spirit, through tribulation, purgeth us, 
and killeth our fleshly wit, our worldly understanding,
and belly-wisdom, and filleth us full of 
the wisdom of God. Tribulation is a blessing 
that cometh of God, as witnesseth Christ, (Matt v.) 
Blessed are they that suffer persecution for 
righteousness' sake, for their's is the kingdom of 
heaven. Is not this a comfortable word? Who 
ought not rather to choose, and desire to be blessed 
with Christ in a little tribulation, than to be cursed 
perpetually with the world for a little pleasure? 

Prosperity is a right curse, and a thing that 
God giveth to his enemies. Woe be to you 
rich, saith Christ, (Luke vi.) lo, ye have your 
consolation: woe be to you that are full, for ye 
shall hunger: woe be to you that laugh, for ye 
shall weep: woe be to you when men praise you, 
for so did their fathers unto the false prophets: yea, 
and so have our fathers done to the false hypocrites. 
The hypocrites, with worldly preaching, have not 
gotten the praise only, but even the possessions 
also, and the dominion and rule of the whole world. 

Tribulation for righteousness is not a blessing 
only, but also a gift that God giveth unto none 
save his special friends. The apostles (Acts v.) 
rejoiced that they were counted worthy to suffer 
rebuke for Christ's sake. And Paul, in the second 
Epistle and third chapter to Timothy, saith, All 
that will live godly in Christ Jesus must suffer 
persecution: and (Phil. i.) he saith, Unto you it is 
given not only to believe in Christ, but also to suffer 
for his sake. Here seest thou that it is God's gift 
to suffer for Christ's sake. 1 Peter iv. saith, Happy 
are ye if ye suffer for the name of Christ; for the 
glorious Spirit of God resteth in you. Is it not an 
happy things to be sure, that thou art sealed with 
God's Spirit to everlasting life? And, verily, thou 
art sure thereof, if thou suffer patiently for his 
sake. By suffering art thou sure; but by persecuting
canst thou never be sure: for Paul, 
(Rom. v.) saith. Tribulation maketh feeling; that 
is, it maketh us feel the goodness of God, and 
his help, and the working of his Spirit. And the
xii. chap. of the 2 Epistle to the Corinthians the
Lord said to Paul, My grace is sufficient for thee;
for my strength is made perfect through weakness.
(2 Cor. xii.) Lo, Christ is never strong in 
us till we be weak. As our strength abateth, so 
groweth the strength of Christ in us: when we 
are clean emptied of our own strength, then are
we full of Christ's strength: and, look, how much
of our own strength remaineth in us, so much 
lacketh there of the strength of Christ. Therefore
saith Paul, in the said place in the second 
Epistle to the Corinthians, Very gladly will I 
rejoice in my weakness, that the strength of 
Christ may dwell in me. Therefore have I delectation,
saith Paul, in infirmities, in rebukes, in 
need, in persecutions, and in anguish for Christ's 
sake; for when I am weak then am I strong. 
Meaning, that the weakness of the flesh is the 
strength of the Spirit. And by flesh understand
wit, wisdom, and all that is in a man before the 
Spirit of God come; and whatsoever springeth 
not of the Spirit of God, and of God's word. And
of like testimonies is all the Scripture full.

Behold, God setteth before us a blessing and 
also a curse. A blessing, verily, and that a 
glorious and an everlasting, if we suffer tribulation
and adversity with our Lord and Saviour 
Christ. And an everlasting curse, if, for a little 
pleasure sake, we withdraw ourselves from the 
chastising and nurture of God, wherewith he 
teacheth all his sons, and fashioneth them after 
his godly will, and maketh them perfect (as he 
did Christ,) and maketh them apt and meet 
vessels to receive his grace and his Spirit, that 
they might perceive and feel the exceeding 
mercy which we have in Christ, and the innumerable
blessings, and the unspeakable inheritance,
whereto we are called and chosen, and 
sealed in our Saviour Jesus Christ, unto whom 
be praise for ever. Amen. 

Finally: whom God chooseth to reign everlastingly
with Christ, him sealeth he with his 
mighty Spirit, and poureth strength into his 
heart, to suffer afflictions also with Christ, for 
bearing witness unto the truth. And this is the 
difference between the children of God and of 
salvation, and between the children of the devil 
and of damnation: that the children of God
have power in their hearts to suffer for God's 
word, which is their life and salvation, their 
hope and trust, and whereby they live in 
the soul and Spirit before God. And the 
children of the devil, in time of adversity, 
fly from Christ, whom they followed feignedly; 
their hearts not sealed with his holy and mighty 
Spirit, and get them to the standard of their 
right father the devil, and take his wages, the
pleasures of this world, which are the earnest of
everlasting damnation: which conclusion the 
xiith chapter to the Hebrews well confirmeth, 
saying, My son, despise not thou the chastising of 
the Lord, neither faint when thou art rebuked of 
him: for whom the Lord loveth, him he chastiseth;
yea, and he scourgeth every son whom
he receiveth. Lo, persecution and adversity for
the truth's sake is God's scourge, and God's rod, 
and pertaineth unto all his children indifferently:
for when he saith he scourgeth every son, he 
maketh none exception. Moreover, saith the 
text, if ye shall endure chastising, God offereth 
himself unto you as unto sons. What son is it 
that the Father chastiseth not? If ye be not 
under correction, (whereof all are partakers,) 
then are ye bastards, and not sons. 

Forasmuch, then, as we must needs be baptized
in tribulations, and through the red sea, 
and a great and a fearful wilderness, and a land 
of cruel giants, into our natural country; yea,
and inasmuch as it is a plain earnest that there
is no other way into the kingdom of life than
through persecution, and suffering of pain, and 
of very death, after the ensample of Christ:
therefore let us arm our souls with the comfort 
of the Scriptures: how that God is ever ready 
at hand in time of need to help us; and how 
that such tyrants and persecutors are but God's 
scourge, and his rod to chastise us. And as the 
Father hath alway in time of correction the rod 
fast in his hand, so that the rod doth nothing but 
as the Father moveth it; even so hath God all 
tyrants in his hand, and letteth them not to do 
whatsoever they would, but as much only as he 
appointeth them to do, and as far forth as it is 
necessary for us. And as when the child submitteth
himself unto his father's correction and 
nurture, and humbleth himself altogether unto the 
will of his father, then the rod is taken away:
even so, when we are come unto the knowledge 
of the right way, and have forsaken our own 
will, and offer ourselves clean to the will of 
God, to walk which way soever he will have us:
then turneth he the tyrants; or else if they 
enforce to persecute us any further, he putteth
them out of the way, according unto the comfortable
ensamples of the Scripture. 

Moreover, let us arm our souls with the 
promises both of help and assistance, and also of 
the glorious reward that followeth. Great is 
your reward in heaven, saith Christ, (Matt. v.) 
And he that knowledgeth me before men, him
will I knowledge before my Father that is in 
heaven; (Matt. x.) and, Call on me in time of
tribulation, and I will deliver thee; (Psalm lxv.)
and, Behold the eyes of the Lord are over them 
that fear him, and over them that trust in his 
mercy; to deliver their souls from death, and to 
feed them in time of hunger. (Psalm xlvi.) And 
in Psalm xlvii. saith David, The Lord is nigh 
them that are troubled in their hearts, and the 
meek in spirit will he save. The tribulations of 
the righteous are many, and out of them all will 
the Lord deliver them. The Lord keepeth all 
the bones of them, so that not one of them shall be 
bruised. The Lord shall redeem the souls of 
his servants. And of such like consolations are 
all the Psalms full: would to God when ye read 
them ye understood them. And (Matt. x.) When 
they deliver you, take no thought what ye shall 
say; it shall be given you the same hour what ye 
shall say: for it is not ye that speak, but the 
Spirit of your Father which speaketh in you. The 
very hairs of your heads are numbered, saith 
Christ also, Matt. x. If God care for our hairs, 
he much more careth for our souls, which he hath 
sealed with his Holy Spirit. Therefore, saith 
Peter, (1 Pet. iv.) cast all your care upon him;
for he careth for you. And Paul (1 Cor. x.)
saith, God is true, he will not suffer you to be
tempted above your might. And Psalm lxxi.
Cast thy care upon the Lord. 

Let thy care be to prepare thyself with all thy
strength, for to walk which way he will have
thee, and to believe that he will go with thee, 
and assist thee, and strengthen thee against all 
tyrants, and deliver thee out of all tribulation. 
But what way, or by what means he will do 
it, that commit unto him, and his godly pleasure 
and wisdom, and cast that care upon him. And 
though it seem never so unlikely, or never so 
impossible unto natural reason, yet believe stedfastly
that he will do it. And then shall he (according
to his old use) change the course of the 
worlds even in the twinkling of an eye, and come 
suddenly upon our giants, as a thief in the 
night, and compass them in their wiles and 
worldly wisdom: when they cry Peace and all is 
safe; then shall their sorrows begin, as the 
pangs of a woman that travaileth with child:
and then shall he destroy them, and deliver thee, 
unto the glorious praise of his mercy and truth. 
Amen. 


And as pertaining unto them that despise 
God's word, counting it as a phantasy, or a 
dream; and to them also that for fear of a little 
persecution fall from it, set this before thine 
eyes; — how God since the beginning of the 
world, before a general plague, ever sent his 
true prophets and preachers of his word to warn 
the people, and gave them space to repent. But 
they, for the greatest part of them, hardened 
their hearts, and persecuted the word that was 
sent to save them. And then God destroyed 
them utterly, and took them clean from the earth. 
As thou seest what followed the preaching of 
Noah in the old world; what followed the 
preaching of Lot among the Sodomites; and
the preaching of Moses and Aaron among the 
Egyptians; and that suddenly against all possibility
of man's wit. Moreover, as oft as the 
children of Israel fell from God to the worshipping
of images, he sent his prophets unto them;
and they persecuted and waxed hard-hearted:
and then he sent them into all places of the world 
captive. Last of all he sent his own Son to 
them, and they waxed more hard-hearted than 
ever before: and see what a fearful example of 
his wrath and cruel vengeance he hath made of 
them to all the world, now almost fifteen 
hundred years. 

Unto the old Britons also (which dwelled 
where our nation doth now,) preached Gildas, 
and rebuked them of their wickedness, and prophesied
both to the spiritual (as they will be 
called,) and unto the lay men also, what vengeance
would follow, except they repented. But 
they waxed hard-hearted, and God sent his 
plagues and pestilences among them, and sent 
their enemies in upon them on every side, and 
destroyed them utterly. 

Mark, also, how Christ threateneth them that 
forsake him, for whatsoever cause it be; whether 
for fear, either for shame, either for loss of 
honour, friends, life, or goods. He that denieth 
me before men, him will I deny before my 
Father that is in heaven. He that loveth father 
or mother more than me, is not worthy of me. 
All this saith he Matt. x. And Mark viii. he 
saith, Whosoever is ashamed of me, or my
words, among this adulterous and sinful generation,
of him shall the Son of Man be ashamed 
when he cometh in the glory of his Father, with 
his holy angels. And Luke ix. also: None 
that layeth his hand to the plough, and looketh 
back, is meet for the kingdom of heaven.

Nevertheless, yet if any man have resisted 
ignorantly, as Paul did, let him look on the 
truth which Paul wrote after he came to knowledge.
Also, if any man clean against his heart 
(but overcome with the weakness of the flesh,) 
for fear of persecution, have denied, as Peter 
did, or have delivered his book, or put it away 
secretly; let him (if he repent,) come again, and 
take better hold, and not despair, or take it for a 
sign that God hath forsaken him; for God ofttimes
taketh his strength even from his very elect, 
when they either trust in their own strength, or 
are negligent to call to him for his strength. And 
that doth he to teach them, and to make them 
feel that in that fire of tribulation, for his word's 
sake, nothing can endure and abide save his 
word, and that strength only which he hath promised.
For the which strength he will have us to 
pray unto him night and day, with all instance. 

That thou mayest perceive how that the Scripture
ought to be in the mother tongue, and that 
the reasons which our spirits make for the contrary,
are but sophistry and false wiles to fear 
thee from the light, that thou mightest follow 
them blindfold, and be their captive to honour 
their ceremonies, and to offer to their belly. 

First, God gave the children of Israel a law 
by the hand of Moses, in their mother tongue;
and all the prophets wrote in their mother 
tongue, and all the Psalms were in the mother 
tongue. And there was Christ but figured, and described
in ceremonies, in riddles, in parables, and 
in dark prophecies. What is the cause that we 
may not have the Old Testament, with the New 
also, which is the light of the Old, and wherein 
is openly declared before thine eyes that which 
there was darkly prophesied? I can imagine no 
cause verily, except it be that we should not see 
the work of Antichrist and juggling of hypocrites. 
What should be the cause, that we, which walk 
in the broad day, should not see as well as they 
that walked in the night, or that we should not 
see as well at noon as they did in the twilight?
Came Christ to make the world more blind? By 
this means Christ is the darkness of the world, 
and not the light, as he saith himself. (John viii.)

Moreover Moses saith, (Deut. vi.) Hear Israel,
let these words which I command thee this day,
stick fast in thine heart, and whet them on thy
children, and talk of them as thou sittest in thine
house, and as thou walkest by the way, and when
thou liest down, and when thou risest up, and
bind them for a token to thine hand, and let 
them be a remembrance between thine eyes, 
and write them on the posts and gates of thine 
house. This was commanded generally unto all 
men. How cometh it that God's word pertaineth
less unto us, than unto them? yea, how cometh 
it, that our Moseses forbid us, and command us 
the contrary, and threaten us if we do, and will 
not that we once speak of God's word? How 
can we whet God's word (that is, to put it in
practice, use, and exercise) upon our children 
and household, when we are violently kept from 
it and know it not? How can we (as Peter commandeth)
give a reason of our hope, when we 
wot not what it is that God hath promised, or 
what to hope? Moses also commandeth in the 
said chapter: If the son ask what the testimonies,
laws, and observances of the Lord mean; that 
the father teach him. If our children ask what 
our ceremonies (which are more than the Jews 
were) mean: no father can tell his son. And in 
the xith chapter, he repeateth all again for fear,
of forgetting.

They will say haply, the Scripture requireth a 
pure mind, and a quiet mind. And therefore 
the lay man because he is altogether cumbred 
with worldly business cannot understand them. 
If that be the cause, then it is a plain case, that 
our prelates understand not the Scriptures themselves:
for no lay man is so tangled with worldly 
business as they are. The great things of the 
world are ministered by them; neither do the 
lay people any great thing, but at their assignment.

If the Scripture were in the mother tongue,
they will say, then would the lay people understand
it, every man after his own ways. Wherefore
serveth the curate, but to teach him the 
right way? Wherefore were the holy days 
made, but that the people should come and learn? 
Are ye not abominable school-masters, in that ye 
take so great wages if ye will not teach? If ye 
would teach, how could ye do it so well, and 
with so great profit, as when the lay people have 
the Scripture before them in their mother tongue?
for then should they see by the order of the text 
whether thou juggledst or not: and then would 
they believe it, because it is the Scripture of God, 
though thy living be never so abominable.
Where now, because your living and your 
preaching are so contrary, and because they 
grope out in every sermon your open and manifest
lies, and smell your unsatiable covetousness,
they believe you not when you preach truth. 
But alas! the curates themselves (for the most 
part) wot no more what the New or Old Testament
meaneth, than do the Turks: neither know 
they of any more than that they read at mass, 
matins, and evensong, which yet they understand
not: neither care they, but even to mumble 
up so much every day, as the pie and popinjay 
speak they wot not what to fill their bellies 
withal. If they will not let the lay man have 
the word of God in his mother tongue, yet let 
the priests have it, which for a great part of them 
do understand no Latin at all: but sing, and say, 
and patter all day with the lips only, that which 
the heart understandeth not. 


Christ commandeth to search the Scriptures.
(John. v.) Though that miracles bare record 
unto his doctrine, yet desired he no faith to be 
given either to his doctrine, or to his miracles,
without record of the Scripture. When 
Paul preached (Acts xvii.) the other searched 
the Scriptures daily, whether they were as he 
alleged them. Why shall not I likewise see, whether
it be the Scripture thou allegest? yea, why 
shall I not see the Scripture, and the circumstances, 
and what goeth before and after, that I may 
know whether thine interpretation be the right 
sense, or whether thou jugglest, and drawest the 
Scripture violently unto thy carnal and fleshly 
purpose? or whether thou be about to teach me 
or to deceive me? 

Christ saith, that there shall come false prophets
in his name, and say that they themselves 
are Christ; that is, they shall so preach Christ 
that men must believe in them, in their holiness,
and things of their imagination without God's word; 
yea, and that against Christ, or Antichrist, that 
shall come, is nothing but such false prophets,
that shall juggle with the Scripture, and 
beguile the people with false interpretations, as 
all the false prophets, scribes and pharisees did 
in the Old Testament. How shall I know whether
ye are against Christ, or false prophets, or 
no, seeing ye will not let me see how ye allege 
the Scriptures? Christ saith: By their deeds 
ye shall know them. — Now when we look on your 
deeds we see that ye are all sworn together, and 
have separated yourselves from the lay people, 
and have a several kingdom among yourselves, 
and several laws of your own making, wherewith 
ye violently bind the lay people that never consented
unto the making of them. A thousand 
things forbid ye which Christ made free, and 
dispense with them again for money: neither is 
there any exception at all, but lack of money. 
Ye have a secret council by yourselves. All 
other men's secrets and counsels know ye, and no 
man your's: ye seek but honour, riches, promotion,
authority, and to reign over all, and will obey 
no man. If the father give you ought of courtesy,
ye will compel the son to give it violently, 
whether he will or not, by craft of your own 
laws. These deeds are against Christ.

When a whole parish of us hire a schoolmaster
to teach our children, what reason is it, 
that we should be compelled to pay this schoolmaster
his wages, and he should have license to 
go where he will, and to dwell in another country, 
and to leave our children untaught? Doth not 
the Pope so? Have we not given up our tithes 
of courtesy unto one, for to teach us God's 
word? and cometh not the Pope, and compelleth 
us to pay it violently to them that never teach? 
Maketh he not one parson, which cometh never 
at us? yea one shall have five or six, or as many 
as he can get, and wotteth oftentimes where 
never one of them standeth. Another is made 
Vicar, to whom he giveth a dispensation to go 
where he will, and to set in a parish priest, which 
can but minister a sort of dumb ceremonies. 
And he, because he hath most labour and least 
profit, polleth on his part, and setteth here a mass 
penny, there a trental, yonder dirige money,
and for his bead roll, with a confession penny and 
such like. And thus are we never taught, and 
are yet nevertheless compelled: yea, compelled 
to hire many costly school-masters. These 
deeds are verily against Christ. Shall we therejore
judge you by your deeds, as Christ commandeth?
So are ye false prophets, and the 
disciples of Antichrist, or against Christ. 

The sermons which thou readest in the Acts 
of the apostles, and all that the apostles preached 
were no doubt preached in the mother tongue. 
Why then might they not be written in the 
mother tongue? As if one of us preach a good 
sermon, why may it not be written? Saint 
Jerom also translated the Bible into his mother 
tongue. Why may not we also? They will say 
it cannot be translated into our tongue it is so 
rude. It is not so rude as they are false liars.
For the Greek tongue agreeth more with the 
English than with the Latin. And the properties 
of the Hebrew tongue agreeth a thousand times 
more with the English, than with the Latin. The 
manner of speaking is both one, so that in a 
thousand places thou needest not but to translate 
it into the English, word for word, when thou 
must seek a compass in the Latin, and yet shall
have much work to translate it well favouredly, 
so that it have the same grace and sweetness,
sense and pure understanding with it in the 
Latin, and as it hath in the Hebrew. A thousand
parts better may it be translated into the 
English, than into the Latin. Yea, and except 
my memory fail me, and that I have forgotten 
what I read when I was a child; thou shalt find in 
the English chronicle, how that King Adelstone
caused the holy Scripture to be translated into
the tongue that then was in England, and how 
the prelates exhorted him thereto. 

Moreover seeing that one of you ever preacheth 
contrary to another: And when two of you 
meet, the one disputeth and brawleth with the 
other, as it were two scolds. And forasmuch 
as one holdeth this doctor, another that. One 
followeth Duns, another Saint Thomas, another
Bonaventure, Alexander de Hales, Raymond, 
Lyre, Brigot, Dorbel, Holcot, Gorram, Trumbett,
Hugy de Sancto Victore, De Monte Regio, 
De Nova Villa, De Media Villa, and such like out 
of number. So that if thou hadst but of every 
author one book, thou couldst not pile them up 
in any warehouse in London, and every author 
is one contrary unto an other. In so great diversity
of spirits, how shall I know who lieth, and 
who sayeth truth? Whereby shall I try and 
judge them? Verily by God's word which 
only is true. But how shall I that do, when thou 
wilt not let me see Scripture? 

Nay, say they, the Scripture is so hard, that 
thou couldst never understand it but by the doctors. 
That is, I must measure the meteyard by the 
cloth. Here be twenty cloths of divers lengths 
and of divers breadths; how shall I be sure of 
the length of the meteyard by them? I suppose,
rather, I must be first sure of the length of 
the meteyard, aud thereby measure and judge 
the clothes. If I must first believe the doctor,
then is the doctor first true, and the truth of the 
Scripture dependeth of his truth; and so the 
truth of God springeth of the truth of man. 
Thus Antichrist turneth the roots of the trees 
upward. What is the cause that we damn 
some of Origen's works, and allow some? How 
know we that some is heresy and some not? 
By the Scripture, I trow. How know we that 
St. Augustine (which is the best, or one of the 
best, that ever wrote upon the Scripture,) wrote 
many things amiss at the beginning, as many 
other doctors do? Verily, by the Scriptures; 
as he himself well perceived afterward, when he 
looked more diligently upon them, and revoked 
many things again. He wrote of many things 
which he understood not when he was newly 
converted, yet he had thoroughly seen the Scriptures;
and followed the opinions of Plato, and 
the common persuasions of man's wisdom that 
were then famous. 

They will say yet more shamefully, that no 
man can understand the Scriptures without philautia,
that is to say, philosophy. A man must 
first be well seen in Aristotle, ere he can understand
the Scripture, say they. Aristotle's doctrine
is, that the world was without beginning, 
and shall be without end; and that the first man 
never was, and the last shall never be. And 
that God doth all of necessity, neither careth 
what we do, neither will ask any accounts of 
that we do. Without this doctrine, how could 
we understand the Scripture, that saith, God 
created the world of nought, and God worketh 
all things of his free will, and for a secret purpose;
and that we shall all rise again, and that 
God will have accounts of all that we have done 
in this life. Aristotle saith, Give a man a law
and he hath power of himself to do or fulfil the 
law, and becometh righteous with working 
righteously. But Paul, and all the Scripture
saith, That the law doth but utter sin only, 
and helpeth not. Neither hath any man power 
to do the law till the Spirit of God be given him 
through faith in Christ. Is it not a madness then 
to say, that we could not understand the Scripture
without Aristotle? Aristotle's righteousness,
and all his virtues, spring of man's free 
will. And a Turk, and every infidel and idolater,
may be righteous, and virtuous with that 
righteousness and those virtues. Moreover, 
Aristotle's felicity and blessedness standeth in 
avoiding of all tribulations; and in riches, health, 
honour, worship, friends, and authority; which 
felicity pleaseth our spiritualty well. Now, 
without these, and a thousand such like points, 
couldst thou not understand Scripture, which
saith that righteousness cometh by Christ, and 
not of man's will; and how that virtues are the 
fruits and the gift of God's Spirit, and that Christ 
blesseth us in tribulations, persecution, and adversity.
How, I say, couldst thou understand 
the Scripture without philosophy, inasmuch as 
Paul, in the second to the Colossians, warned 
them to beware, lest any man should spoil them 
(that is to say, rob them of their faith in Christ,) 
through philosophy and deceitful vanities, and 
through the traditions of men, and ordinances 
after the world, and not after Christ. 

By this means, then, thou wilt that no man 
teach another, but that every man take the 
Scripture and learn by himself. Nay, verily, so 
say I not. Nevertheless, seeing that ye will not 
teach, if any man thirst for the truth, and read 
the Scripture by himself, desiring God to open 
the door of knowledge unto him, God, for his 
truth's sake, will and must teach him. Howbeit, 
my meaning is, that as a master teacheth his apprentice
to know all the points of the meteyard; 
first, how many inches, how many feet, and the 
half yard, the quarter, and the nail, and then 
teacheth him to mete other things thereby: even 
so will I that ye teach the people God's law, 
and what obedience God requireth of us to 
father and mother, master, lord, king, and all 
superiors, and with what friendly love he commandeth
one to love another. And teach them 
to know that natural venom, and birth poison, 
which moveth the very hearts of us to rebel 
against the ordinances and will of God, and 
prove that no man is righteous in the sight of 
God, but that we are all damned by the law. 
And then, (when thou hast meeked them and 
feared them with the law,) teach them the Testament
and promises which God hath made unto 
us in Christ, and how much he loveth us in 
Christ. And teach them the principles and the 
ground of the faith, and what the sacraments 
signify, and then shall the Spirit work with thy 
preaching, and make them feel. So would it 
come to pass, that as we know by natural wit 
what followeth of a true principle of natural 
reason; even so, by the principles of the faith, 
and by the plain Scriptures, and by the circumstances
of the text, should we judge all men's 
exposition, and all men's doctrine, and should 
receive the best, and refuse the worst. I would 
have you to teach them also the properties and 
manner of speakings of the Scripture, and how 
to expound proverbs and similitudes. And 
then, if they go abroad and walk by the fields 
and meadows of all manner [of] doctors and philosophers,
they could catch no harm. They 
should discern the poison from the honey, and 
bring home nothing but that which is wholesome. 

But now do ye clean contrary, ye drive them 
from God's word, and will let no man come thereto
until he have been two years master of art. 
First they nosel them in sophistry, and in benefundatum.
And there corrupt they their judgments
with apparent arguments, and with alleging
unto them texts of logic, of natural philautia,
of metaphysic, and moral philosophy, and of 
all manner [of] books of Aristotle, and of all manner
[of] doctors which they yet never saw. Moreover,
one holdeth this, another that; one is a real, 
another a nominal. What wonderful dreams have 
they of their predicaments, universals, second intentions,
qui dities, haec scities, and relatives. And 
whether specia fundata in chimera, be vera species.
And whether this proposition be true, non 
ens est aliquid, whether ens be aequivocum, or 
univocum. Ens is a voice only say some. Ens 
is univocum saith another, and descendeth into 
ens creatum, and into ens increatum, per modus 
intrinsecos. When they have thiswise brawled 
eight, ten, or twelve or more years, and after that 
their judgments are utterly corrupt: then they begin
their divinity; not at the Scripture, but every 
man taketh a sundry doctor, which doctors are as 
sundry and as divers, the one contrary unto the 
other, as there are divers fashions and monstrous 
shapes, none like another among our sects of religion.
Every religion, every university, and 
almost every man hath a sundry divinity. Now 
whatsoever opinions every man findeth with 
his doctor, that is his gospel, and that only 
is true with him, and that holdeth he all his 
life long, and every man to maintain his doctor 
withal, corrupteth the Scripture, and fashioneth 
it after his own imagination, as a potter doth his 
clay. Of what text thou provest hell, will 
another prove purgatory, another limbo patrum, 
and another the assumption of our lady, and 
another shall prove of the same text that an ape 
hath a tail. And of what text the grave friar 
proveth that our lady was without original sin, 
of the same shall the black friar prove that she 
was conceived in original sin; and all this do 
they with apparent reasons, with false similitudes 
and likenesses, and with arguments and persuasions
of man's wisdom. Now there is no other 
division or heresy in the world, save man's wisdom,
and when man's foolish wisdom interpreteth 
the Scripture. Man's wisdom scattereth, divideth 
and maketh sects; while the wisdom of one is 
that a white coat is best to serve God in, and 
another saith a black, and another a gray, another 
a blue; and while one saith God will hear your 
prayer in this place, another saith in that place;
and while one saith this place is holier, and 
another that place is holier; and this religion is 
holier than that; and this saint is greater with 
God than that; and an hundred thousand like 
things. Man's wisdom is plain idolatry, neither 
is there any other idolatry than to imagine of God 
after man's wisdom. God is not man's imagination, 
but that only which he saith of himself. God is 
nothing but his law and his promises, that is to say, 
that which he biddeth thee to do, and that which 
he biddeth thee believe and hope. God is but his 
word, as Christ saith, (John viii.) I am that I say 
unto you; that is to say, That which I preach am 
I, my words are spirit and life. God is that only 
which he testifieth of himself; and to imagine any 
other thing of God than that, is damnable idolatry. 
Therefore saith the cxviiith Psalm, Happy are 
they which search the testimonies of the Lord, 
that is to say, that which God testifieth and witnesseth
unto us. But how shall I that do, when ye 
will not let me have his testimonies or witnesses 
in a toungue which I understand? Will ye resist 
God? will ye forbid him to give his Spirit unto 
the lay as well unto you? Hath he not made the 
English tongue? Why forbid ye him to speak in 
the english tongue then as well as in the Latin?
Finally, that this threatening and forbidding the 
lay people to read the Scripture is not for love of 
your souls (which they care for, as the fox doth 
for the geese) is evident, and clearer than the 
sun, inasmuch as they permit and suffer you to read 
Robin Hood, Bevis of Hampton, Hercules, Hector, 
and Troilus, with a thousand histories and fables of 
love and wantonness, and of ribaldry, as filthy as 
heart can think; to corrupt the minds of youth 
withal, clean contrary to the doctrine of Christ 
and his apostles: for Paul (Eph. v.) saith, See 
that fornication and all uncleanness or covetousness
be not once named among you, as it 
becometh saints: neither filthiness, neither foolish 
talking nor jesting, which are not comely: for 
this ye know, that no whoremonger either unclean 
person or covetous person, (which is the worshipper
of images,) hath any inheritance in the 
kingdom of Christ and of God. And after saith 
he, Through such things cometh the wrath of God 
upon the children of unbelief. Now seeing they 
permit you freely to read those things which corrupt
your minds and rob you of the kingdom of 
God and Christ, and bring the wrath of God upon
you, how is this forbidding for love of your 
souls?

A thousand reasons more might be made, as 
thou mayest see in Paraclesis Erasmi, and in his 
preface to the Paraphrase of Matthew, unto 
which they should be compelled to hold their 
peace, or to give shameful answers. But I hope 
that these are sufficient unto them that trust the 
truth. God for his mercy and truth shall well open 
them more: yea, and other secrets of his godly 
wisdom, if they be diligent to cry unto him, 
which grace grant God. Amen. 


THE PROLOGUE 
UNTO THE BOOK. 

FORASMUCH as our holy prelates and our ghostly religious,
which ought to defend God's word, speak evil 
of it, and do all the shame they can to it; and rail on it and 
bear their captives in hand, that it causeth insurrection and 
teacheth the people to disobey their heads and governors, 
and moveth them to rise against their princes; and to make 
all common and to make havock of other men's goods:
therefore have I made this little treatise that followeth, containing
all obedience that is of God; in which, (whosoever 
readeth it) shall easily perceive, not the contrary only, and 
that they lie, but also the very cause of such blasphemy, 
and what stirreth them so furiously to rage and to belie the 
truth. 

Howbeit it is no new thing unto the word of God to be 
railed upon, neither is this the first time that hypocrites
have ascribed to God's word the vengeance whereof they 
themselves were ever [the] cause: for the hypocrites with
their false doctrine and idolatry have evermore led the wrath
and vengeance of God upon the people, so sore that God
could no longer forbear nor defer his punishment. Yet 
God, which is always merciful, before he would take vengeance,
hath ever sent his true prophets and true preachers, 
to warn the people that they might repent. But the people
for the most part, and namely the heads and rulers, through 
comfort and persuading of the hypocrites, have ever waxed 
more hard hearted than before, and have persecuted the 
word of God and his prophets. Then God, which is 
also righteous, hath always poured his plagues upon
them without delay, which plagues the hypocrites 
ascribe unto God's word, saying, See what mischief 
is come upon us since this new learning came up, and this 
new sect, and this new doctrine. This seest thou Jeremiah
xliv. where the people cried to go to their old idolatry
again, saying, Since we left it, we have been in all necessity
and have been consumed with war and hunger. But 
the prophet answered them that their idolatry went unto the 
heart of God, so that he could no longer suffer the maliciousness
of their own imaginations or inventions; and that 
the cause of all such mischiefs was, because they would not 
hear the voice of the Lord and walk in his law, ordinances, 
and testimonies. The scribes and the pharisees laid also 
to Christ's charge (Luke xxiii.) that he moved the 
people to sedition, and said to Pilate, we found this fellow
perverting the people and forbidding to pay tribute to 
Caesar, and saith that he is Christ, a king. And again in the
same chapter, he moveth the people, said they, teaching 
throughout Jewry, and began at Galilee even to this place. 
So likewise laid they to the apostles' charge, as thou mayest 
see in the Acts. St. Cyprian also, and St. Augustin, and 
many other more, made works in defence of the word of 
God against such blasphemies: so that thou mayest see 
how that it is no new thing, but an old and accustomed 
thing with the hypocrites, to wite God's word and the 
true preachers of all the mischief which their lying doctrine
in the very cause of.

Never the later in very deed, after the preaching of 
God's word, because it is not truly received, God sendeth 
great trouble into the world; partly to avenge himself of 
the tyrants and persecutors of his word, and partly to destroy
those wordly people which make of God's word nothing
but a cloak of their fleshly liberty. They are not 
all good that follow the gospel. Christ (Matt. xiii.) likeneth 
the kingdom of heaven unto a net cast into the sea that 
catcheth fishes both good and bad. The kingdom of 
heaven is the preaching of the gospel, unto which come 
both good and bad. But the good are few, Christ 
calleth them therefore a little flock. (Luke xii.) For they 
are ever few that come to the gospel of a true intent,
seeking therein nothing but the glory and praise of God,
and offering themselves freely and willingly to take adversity
with Christ for the gospel's sake, and for bearing 
record unto the truth, that all men may hear it. The 
greatest number come, and ever came, and followed even 
Christ himself for a worldly purpose: as thou mayest well 
see, (John vi.) how that almost five thousand followed 
Christ, and would also have made him a king, because he 
had well fed them: whom he rebuked, saying, Ye seek me 
not because ye saw the miracles, but because ye ate of the 
bread and were filled; and drove them away from him with 
hard preaching.

Even so now (as ever) the most part seek liberty; they 
be glad when they hear the unsatiable covetousness of the 
spiritualty rebuked; when they hear their falshood and 
wiles uttered; when tyranny and oppression is preached 
against; when they hear how kings and all officers should 
rule christianly and brotherly, and seek no other thing save 
the wealth of their subjects; and when they hear that they 
have no such authority of God so to pill and poll as they
do, and to raise up taxes and gatherings to maintain their 
phantasies and to make war they wot not for what cause:
and therefore because the heads will not so rule, will they 
also no longer obey, but resist and rise against their evil 
heads, and one wicked destroyeth another. Yet is not 
God's word the cause of this, neither yet the preachers, 
for though that Christ himself taught all obedience, how 
that it is not lawful to resist wrong, (but for the officer that 
is appointed thereunto:) and how a man must love his very 
enemy, and pray for them that persecute him, and bless 
them that curse him: and how that all vengeance must be 
remitted to God, and that a man must forgive if he will be 
forgiven of God: yet the people, for the most part received
it not: they were ever ready to rise and to fight.
Forever when the scribes and pharisees went about to take 
Christ, they were afraid of the people: Not on the holy 
day said they (Matt. xxvi.) lest any rumour arise among 
the people: and (Matt. xxi.) They would have taken him 
but they feared the people: and (Luke xx.) Christ asked 
the Pharisees a question unto which they durst not answer,
lest the people should have stoned them. 

Last of all: forasmuch as the very disciples and apostles 
of Christ, after so long hearing of Christ's doctrine, were 
yet ready to fight for Christ clean against Christ's teaching:
as Peter (Matt. xxvi.) drew his sword, but he was rebuked:
and (Luke ix.) James and John would have had fire to 
come from heaven to consume the Samaritans, and to 
avenge the injury of Christ, but were likewise rebuked. If 
Christ's disciples were so long carnal, what wonder is it if 
we be not all perfect the first day? Yea, inasmuch as we 
be taught even of very babes, to kill a Turk, to slay a Jew, 
to burn an heretic, to fight for the liberties and right of 
the church, as they call it; yea, and inasmuch as we are 
brought in belief, if we shed the blood of our even christian,
or if the son shed the blood of his father that begat him, 
for the defence, not of the Pope's godhead only, but also 
for whatsoever cause it be, yea, though it lie for no cause, 
but that his holiness commandeth it only, that we deserve 
as much as Christ deserved for us when he died on the 
cross; or if we be slain in the quarrel, that our souls go, 
nay even fly to heaven, and be there even before our blood 
be cold. Inasmuch I say as we have sucked in such bloody 
imaginations into the bottom of our hearts, even with our 
mother's milk, and have been so long hardened therein, 
what wonder were it, if, while we be yet young in Christ, 
we thought that it were lawful to fight for the true word of 
God? Yea, and though a man were thoroughly persuaded 
that it were not lawful to resist his king, though he would 
wrongfully take away life and goods; yet might be think 
that it were lawful to resist the hypocrites and to rise not 
against his king, but with his king to deliver his king, out 
of bondage and captivity, wherein the hypocrites hold him 
with wiles and falsehood, so that no man may be suffered 
to come at him, to tell him the truth. 

This seest thou, that it is the bloody doctrine of the pope, 
which caused disobedience, rebellion and insurrection, for 
he teacheth to fight and to defend his traditions, and whatsoever
he dreameth, with fire, water, and sword; and to disobey
father, mother, master, lord, king, and emperor: yea, 
and to invade whatsoever land or nation that will not receive 
and admit his godhead. Where the peaceable doctrine of 
Christ teacheth to obey and to suffer for the word of God, 
and to remit the vengeance and the defence of the word to 
God, which is mighty and able to defend it: which also as 
soon as the word is once openly preached, and testified or 
witnessed, unto the world, and when he hath given them a 
season to repent, is ready at once to take vengeance of his 
enemies, and shooteth arrows with heads dipt in deadly 
poison at them; and poureth his plague from heaven down 
upon them; and sendeth the murrain and pestilence among 
them; and sinketh the cities of them; and maketh the earth 
swallow them, and compasseth them in their wiles, and 
taketh them in their own traps and snares, and casteth 
them into the pits which they digged for other men; and
sendeth them a dazing in the head; and utterly destroyeth
them with their own subtle counsel.

Prepare thy mind therefore unto this little treatise, and 
read it discreetly, and judge it indifferently, and when I
allege any Scripture, look thou on the text whether I interpret
it right: which thou shalt easily perceive, by the circumstance
and process of them; if thou make Christ the 
foundation and the ground, and build all on him, and referest
all to him; and findest also that the exposition 
agreeth unto the common articles of the faith, and open 
Scriptures. And God the Father of mercy, which for his 
truth's sake raised our Saviour Christ up again to justify 
us, give thee his Spirit to judge what is righteous in his 
eyes, and give thee strength to abide by it, and to maintain
it with all patience and long-suffering, unto the 
example and edifying of his congregation, and glory of his 
name. Amen. 


THE 
OBEDIENCE OF ALL DEGREES 
PROVED BY GOD'S WORD: AND FIRST OF CHILDREN 
UNTO THEIR ELDERS.

GOD (which worketh all in all things,) for a secret 
judgment and purpose, and for his godly pleasure, 
provided an hour that thy father and mother should come 
together, to make thee through them. He was present 
with thee in thy mother's womb, and fashioned thee 
and breathed life into thee; and, for the great love he had 
unto thee, provided milk in thy mother's breasts for thee 
against thou were born; moved also thy father and 
mother, and all other, to love thee, to pity thee, and to 
care for thee. 

And as he made thee through them, so hath he cast thee 
under the power and authority of them, to obey and serve 
them in his stead; saying, Honour thy father and mother. 
(Exod. xx.) Which is not to be understood in bowing 
the knee, and putting off the cap only, but that thou love 
them with all thine heart; and fear and dread them, and 
wait on their commandments; and seek their worship,
pleasure, will and profit in all things; and give thy life for 
them, counting them worthy of all honour; remembering 
that thou art their good and possession, and that thou 
owest unto them thine own self, and all thou art able, yea, 
and more than thou art able to do.

Understand also, that whatsoever thou doest unto them 
(be it good or bad,) thou doest unto God. When thou 
pleasest them thou pleasest God; when thou displeasest 
them thou displeasest God; when they are angry with 
thee God is angry with thee: neither is it possible for 
thee to come to the favour of God again (no, though 
all the angels of heaven pray for thee,) until thou have 
submitted thyself unto thy father and mother again. 

If thou obey, (though it be but carnally, either for fear,
for vain glory, or profit,) thy blessing shall be long life 
upon the earth. For he saith, Honour thy father and 
mother, that thou mayest live long upon the earth. (Exod. 
xx.) Contrariwise, if thou disobey them, thy life shall 
be shortened upon the earth. For it followeth, (Exod. xxi.) 
He that smiteth his father or mother shall be put to death 
for it. And he that curseth, (that is to say, raileth or dishonoureth
his father or mother with opprobrious words,) 
shall be slain for it. And (Deut. xxi.) if any man have 
a son stubborn and disobedient, which heareth not the 
voice of his father and the voice of his mother, so that 
they have taught him nurture, and he regardeth them not, 
then let his father and mother take him, and bring him 
forth unto the seniors or elders of the city, and unto the 
gate of the same place. And let them say unto the 
seniors of that city, This our son is stubborn and disobedient:
he will not hearken unto our voice: he is a rioter 
and a drunkard. Then let the men of the city stone him 
with stones unto death: so shall ye put away wickedness
from among you, and all Israel shall hear and shall fear. 

And though that the temporal officers (to their own 
damnation,) be negligent in punishing such disobedience, 
(as the spiritual officers are to teach it,) and wink at it, 
or look on it through the fingers, yet shall they not escape 
unpunished. For the vengeance of God shall accompany 
them (as thou mayest see Deut. xxviii.) with all misfortune
and evil luck, and shall not depart from them until 
they be murdered, drowned, or hanged; either until by 
one mischance or another they be utterly brought to 
nought. Yea, and the world often times hangeth many a 
man for that they never deserved; but God hangeth them 
because they would not obey and hearken unto their 
elders, as the consciences of many will find when they 
come to the gallows. There can they preach and teach 
other that which they themselves would not learn in 
season. 

The marriage also of the children pertaineth unto their 
elders, as thou mayest see 1 Cor. vii. and throughout all 
the Scripture, by the authority of the said commandment
children obey father and mother. Which thing the 
heathen and Gentiles have ever kept, and to this day keep, 
to the great shame and rebuke of us Christians: inasmuch
as the weddings of our virgins (shame it is to speak 
it,) are more like to the salt of a bitch than the marrying
of a reasonable creature. See not we daily three or 
four challenging one woman before the commissary or 
official, of which not one hath the consent of her father and 
mother? And yet he that hath most money hath best 
right, and shall have her in the despite of all her friends, 
and in defiance of God's ordinances. 

Moreover, when she is given by the judge unto the 
one party, and also married, even then ofttimes shall the 
contrary party sue for an higher judge, or another that 
succeedeth the same, and for money divorce her again. So 
shamefully doth the covetousness and ambition of our 
prelates mock with the law of God. I pass over with 
silence how many years they will prolong the sentence 
with cavillations and subtlety, if they be well monied on
both parties; and if a damsel promise two, how shameful
counsel they will give the second, and also how the religious
of Satan do separate unseparable matrimony. For
after thou art lawfully married at the commandment of 
father and mother, and with the consent of all thy friends;
yet if thou wilt be disguised like unto one of them, and
swear obedience to their traditions, thou mayest disobey
father and mother, break the oath which thou hast sworn
to God before his holy congregation, and withdraw love
and charity, the highest of God's commandments, and 
that duty and service which thou owest unto thy wife;
whereof Christ cannot dispense with thee. For Christ is 
not against God, but with God, and came not to break 
God's ordinances, but to fulfil them. That is, he came 
to overcome thee with kindness, and to make thee to do of 
very love the thing which the law compelleth thee to do. 
For love only, and to do service unto thy neighbour, is the 
fulfilling of the law in the sight of God. To be a monk 
or a friar, thou mayest thus forsake thy wife before thou 
hast lain with her, but not to be a secular priest. And
yet, after thou art professed, the Pope for money will dispense
with thee, both for thy coat and all thy obedience,
and make a secular priest of thee: likewise, as it is 
simony to sell a benefice, (as they call it,) but to resign 
upon a pension, and then to redeem the same, is no simony 
at all. Oh, crafty jugglers and mockers with the word
of God! 


THE OBEDIENCE OF WIVES UNTO THEIR
HUSBANDS.

AFTER that Eve was deceived of the serpent, God said 
unto her, (Gen. iii.) Thy lust or appetite shall pertain 
unto thy husband, and he shall rule thee, or reign over
thee. God, which created the woman, knoweth what is
in that weak vessel, (as Peter calleth her,) and hath therefore
put her under the obedience of her husband, to rule her 
lusts and wanton appetites. Peter (1 Pet. iii.) exhorteth 
wives to be in subjection to their husbands; after the ensample
of the holy women which in old time trusted in 
God, and as Sarah obeyed Abraham and called him 
Lord. Which Sarah, before she was married, was
Abraham's sister, and equal with him; but as 
soon as she was married was in subjection, and became 
without comparison inferior. For so is the nature of 
wedlock by the ordinance of God. It were much better 
that our wives followed the ensample of the holy women of 
old time in obeying their husbands, than to worship them 
with a Paternoster, an Ave and a Credo, or to stick up 
candles before their images. Paul (Eph. v.) saith, Women, 
submit yourselves to your own husbands, as to the
Lord. For the husband is the wife's head, even as Christ is 
the head of the congregation. Therefore, as the congregation 
is in subjection to Christ, likewise let wives be in subjection
unto their husbands in all things. Let the woman, 
therefore, fear her husband, as Paul saith in the said place. 
For her husband is unto her in the stead of God, that she 
obey him, and wait on his commandments: and his commandments
are God's commandments. If she, therefore, 
grudge against him, or resist him, she grudgeth against 
God, and resisteth God.


THE OBEDIENCE OF SERVANTS UNTO THEIR 
MASTERS. 

SERVANTS, obey your carnal masters with fear and 
trembling, in singleness of your hearts as unto Christ;
not with service in the eye-sight as men pleasers, but as 
the servants of Christ, doing the will of God from the 
heart, with good will, even as though ye served the Lord 
and not men. (Eph. vi.) And (1 Pet. ii.) Servants obey 
your masters with all fear, not only if they be good and 
courteous, but also though they be froward. For it cometh 
of grace, if a man for conscience toward God endure 
grief, suffering wrongfully. For what praise is it, if when 
ye be buffetted for your faults, ye take it patiently? But, 
and if ye do well, ye suffer wrong and take it patiently, 
then is there thanks with God. Hereunto, verily, were ye 
called. For Christ also suffered for our sakes, leaving us 
an ensample to follow his steps. In whatsoever kind, therefore,
thou art a servant, during the time of thy covenants, 
thy master is unto thee in the stead and room of God;
and God, through him, feedeth thee, clotheth thee, ruleth 
thee, and learneth thee. His commandments are God's 
commandments, and thou oughtest to obey him as God, 
and in all things to seek his pleasure and profit. For 
thou art his good and possession, as his ox or his horse;
insomuch that whosoever doth but desire thee in his heart 
from him, without his love and license, is condemned of 
God, which saith, (Exod. xx.) See thou once covet not 
thy neighbour's servants. 

Paul the apostle sent home Onesimus unto his master 
(as thou readest in the Epistle of Paul to Philemon:) 
insomuch that though the said Philemon with his servant 
also was converted by Paul, and owed to Paul, and 
to the word that Paul preached, not his servant only, but 
also himself; yea, and though that Paul was in necessity, 
and lacked ministers to minister unto him in the bonds 
which he suffered for the gospel's sake; yet would he not 
retain the servant necessary unto the furtherance of the 
gospel, without the consent of the master.

O how sore differeth the doctrine of Christ and his 
apostles, from the doctrine of the Pope and of his apostles!
For if any man will obey neither father nor 
mother, neither lord nor master, neither king nor prince, 
the same needeth but only to take the mark of the beast, 
that is, to shave himself a monk, a friar or a priest, and is 
then immediately free and exempted from all service and 
obedience due to man. He that will obey no man (as 
they will not) is most acceptable unto them. The more 
disobedient that thou art unto God's ordinances, the more 
apt and meet art thou for their's. Neither is the professing, 
vowing and swearing obedience unto their ordinances, any 
other thing than the defying, denying, and foreswearing 
obedience to the ordinances of God.


THE OBEDIENCE OF SUBJECTS UNTO KINGS, 
PRINCES, AND KULERS. 

LET every soul submit himself unto the authority of 
the higher powers. There is no power but of 
God: the powers that be, are ordained of God, Whosoever
therefore resisteth the power, resisteth the ordinance
of God, They that resist shall receive to themselves
damnation. For rulers are not to be feared for 
good works, but for evil. Wilt thou be without fear of 
the power? Do well then, and so shalt thou be praised 
of the same: for he is the minister of God for thy
wealth. But, and if thou do evil, then fear; for he 
beareth not a sword for nought: for he is the minister 
of God, to take vengeance on them that do evil. Wherefore
ye must needs obey; not for fear of vengeance only, 
but also because of conscience. Even for this cause pay 
ye tribute: for they are God's ministers serving for the 
same purpose. 

Give to every man therefore his duty: tribute to whom 
tribute belongeth; custom to whom custom is due; fear 
to whom fear belongeth; honour to whom honour pertaineth. 
Owe nothing to any man; but to love one another: for 
he that loveth another fulfilleth the law. For these commandments,
Thou shalt not commit adultery, Thou shalt 
not kill, Thou shalt not steal, Thou shalt not bear false witness,
Thou shalt not desire, and so forth; if there be any other 
commandment, are all comprehended in this saying, Love 
thine neighbour as thyself. Love hurteth not his neighbour:
therefore is love the fulfilling of the law. 

As a father over his children is both Lord and judge, 
forbidding one brother to avenge himself on another, but
(if any cause of strife be between them) will have it 
brought unto himself, or his assigns, to be judged and 
correct; so God forbiddeth all men to avenge themselves,
and taketh the authority and office of avenging unto 
himself; saying, Vengeance is mine, and I will reward. 
(Deut. xxxii.) Which text Paul allegeth, (Rom. xii.) 
For it is imposible that a man should be a righteous, an 
egal, or an indifferent judge in his own cause — lusts and 
appetites so blind us. Moreover when thou avengest 
thyself, thou makest not peace, but stirrest up more 
debate. 

God therefore hath given laws unto all nations, and in 
all lands hath put kings, governors, and rulers in his own 
stead, to rule the world through them. And hath commanded
all causes to be brought before them, as thou readest 
(Exod. xxii.) In all causes (saith he) of injury or wrong, 
whether it be ox, ass, sheep or vesture, or any lost thing 
which another challengeth, let the cause of both parties be 
brought unto the gods, whom the gods condemn, the 
same shall pay double unto his neighbour. Mark, the 
judges are called gods in the Scriptures, because they are 
in God's room, and execute the commandments of God. 
And in another place of the said chapter, Moses chargeth 
saying; See that thou rail not on the gods, neither speak 
evil of the ruler of thy people. Whosoever therefore resisteth
them, resisteth God (for they are in the room of 
God) and they that resist shall receive the damnation. 

Such obedience unto father and mother, master, 
husband, emperor, king, lords and rulers, requireth God 
of all nations, yea of the very Turks and infidels. The 
blessing and reward of them that keep them, is the life of 
this world, as thou readest (Lev. xviii.) Keep my ordinances
and laws; which if a man keep, he shall live 
therein. Which text Paul rehearseth Rom x. proving 
thereby that the righteousness of the law is but worldly, 
and the reward thereof is the life of this world. And the 
curse of them that breaketh them, is the loss of this life, 
as thou seest by their punishment appointed for them.

And whosoever keepeth the law (whether it be for fear, 
for vain glory or profit) though no man reward him, yet 
shall God bless him abundantly, and send him worldly 
prosperity, as thou readest Deut xxviii. What good 
blessings accompany the keeping of the law, and as we 
see the Turks far exceed us Christian men in worldly prosperity
for their just keeping of their temporal laws. Likewise
though no man punish the breakers of the law, yet 
shall God send his curses upon them till they be utterly 
brought to nought, as thou readest most terribly even in 
the said place.

Neither may the inferior person avenge himself upon 
the superior, or violently resist him for whatsoever wrong 
it be. If he do, he is condemned in the deed doing: 
inasmuch as he taketh upon him that which belongeth to 
God only, which saith, Vengeance is mine, and I will reward.
(Deut. xxxii.) And Christ saith (Mat. xxvi.) All 
they that take the sword shall perish with the sword. 
Takest thou a sword to avenge thyself? so givest thou not 
room unto God to avenge thee, but robbest him of his 
most high honour, in that thou wilt not let him be judge 
over thee. 

If any man might have avenged himself upon his superior,
that might David most righteously have done upon
king Saul which so wrongfully persecuted David; even 
for no other cause, than that God had anointed him king, 
and promised him the kingdom. Yet when God had delivered
Saul into the hands of David, that he might have 
done what he would with him as thou seest in the first book 
of Kings the xxivth chapter, how Saul came into the camp 
where David was. And David came to him secretly, and 
cut off a piece of his garment. And as soon as he had 
done it his heart smote him, because he had done so much 
unto his lord. And when his men couraged him to slay 
him, he answered. The Lord forbid it me that I should 
lay mine hand on him. Neither suffered he his men to 
hurt him. When Saul was gone out, David followed and 
shewed him the piece of his garment, and said, Why believest
thou the words of men that say, David goeth about 
to do thee harm? perceive and see that there is neither 
evil nor wickedness in my hand, and that I have not trespassed
against thee, and yet thou layest await for my life, 
God judge between thee and me, and avenge me of thee, 
but mine hand be not upon thee. As the old proverb saith 
(said David) Out of the wicked shall wickedness proceed, 
but mine hand be not upon thee, meaning that God ever 
punisheth one wicked by another. And again (said David,) 
God be judge, and judge between thee and me, and 
behold and plead my cause, and give me judgment or right 
of thee. 

And in the xxvith chapter of the same book, when Saul 
persecuted David again, David came to Saul by night, 
as he slept and all his men, and took away his spear and a 
cup of water from his head. Then said Abishai, David's
servant, God hath delivered thee thine enemy into thine 
hand this day, let me now therefore nail him to the ground 
with my spear, and give him but even one stripe and no 
more. David forbad him saying, Kill him not, for who 
(said he) shall lay hands on the Lord's anointed and be 
not guilty? The Lord liveth or by the Lord's life (said he) 
he dieth not, except the Lord smite him, or that his day be 
come to die, or else go to battle and there perish.

Why did not David slay Saul, seeing he was so wicked,
not in persecuting David only, but in disobeying God's
commandments, and in that he had slain eighty-five of 
God's priests wrongfully? Verily for it was not lawful:
for if he had done it, he must have sinned against God:
for God hath made the king in every realm judge over 
all, and over him is there no judge. He that judgeth the 
king judgeth God, and he that layeth hands on the king, 
layeth hand on God, and he that resisteth the king resisteth
God, and damneth God's law and ordinance. If the 
subjects sin they must be brought to the king's judgment. 
If the king sin he must be reserved to the judgment, 
wrath, and vengeance of God. And as it is to resist the 
king, so is it to resist his officer, which is set or sent to 
execute the king's commandment. 

And in the first chapter of the second book of Kings, 
David commanded the young man to be slain which 
brought unto him the crown and bracelet of Saul, and said 
to please David withal, that he himself had slain Saul.
And in the fourth chapter of the same book, David commanded
those two to be slain which brought unto him the 
head of Ishbosheth, Saul's son, by whose means yet the
whole kingdom returned unto David, according unto the 
promise of God.

And Luke xiiith, when they showed Christ of the
Galileans, whose blood Pilate mingled with their own 
sacrifice, he answered, Suppose ye that these Galileans 
were sinners above all other Galileans, because they suffered
such punishment? I tell you nay, but except ye 
repent, ye shall likewise perish. This was told Christ, 
no doubt, of such an intent as they asked him, (Matt.
xxii.) Whether it were lawful to give tribute unto Caesar?
For they thought that it was no sin to resist an heathen 
prince: as few of us would think, if we were under the 
Turk, that it were sin to rise against him, and to rid ourselves
from under his dominions, so sore have our bishops 
robbed us of the true doctrine of Christ. But Christ 
condemned their deeds, and also the secret thoughts of 
all other, that consented there unto, saying: Except ye 
repent, ye shall likewise perish. As who should say, I 
know that ye are within in your hearts, such as they were 
outward in their deeds, and are under the same damnation:
except, therefore, ye repent, betimes, ye shall break out 
at the last into like deeds, and likewise perish, as it came 
afterward to pass. 

Hereby seest thou that the king is in this world without 
law, and may at his lust do right or wrong, and shall give 
accounts, but to God only. 

Another conclusion is this, that no person, neither any 
degree may be exempt from this ordinance of God. Nei- 
can the profession of monks and friars, or any thing that 
the pope or bishops can lay for themselves, except them
from the sword of the emperor or kings, if they break 
the laws. For it is written, let every soul submit himself 
unto the authority of the higher powers. Here is no man 
except, but all souls must obey. The higher powers are 
the temporal kings and princes, unto whom God hath 
given the sword to punish whosoever sinneth. God hath 
not given them swords to punish one, and to let another 
go free, and sin unpunished. Moreover, with what face 
durst the spiritualty, which ought to be the light and an 
ensample of good living unto all other, desire to sin unpunished,
or to be excepted from tribute, toll, or custom, 
that they would not bear pain with their brethren to 
the maintenance of kings and officers ordained of God to 
punish sin? There is no power but of God (by power 
understand the authority of kings and princes.) The 
powers that be are ordained of God. Whosoever, therefore,
resisteth power, resisteth God: yea, though he be 
pope, bishop, monk, or friar. They that resist shall 
receive unto themselves damnation. Why? For God's 
word is against them, which will have all men under the 
power of the temporal sword: for rulers are not to 
be feared for good works, but for evil. Hereby seest 
thou that they that resist the powers, or seek to be exempt 
from their authority, have evil consciences, and seek liberty 
to sin unpunished, and to be from bearing pain with 
their brethren. Wilt thou be without fear of the power?
So do well, and thou shall have laud of the same (that is 
to say of the ruler.) With good living ought the spiritualty 
to rid themselves from fear of the temporal sword, and 
not with craft and with blinding the kings, and bringing the 
vengeance of God upon them, and in purchasing license 
to sin unpunished. 

For he is the minister of God for thy wealth: to defend 
thee from a thousand inconveniences, from thieves, murderers,
and them that would defile thy wife, thy daughter, 
and take from thee all that thou hast: yea, life and all, 
if thou didst resist. Furthermore, though he be the 
greatest tyrant in the world, yet is he unto thee a great 
benefit of God, and a thing wherefore thou oughtest to 
thank God highly. For it is better to have somewhat 
than to be clean stript out of altogether: it is better 
to pay the tenth than to lose all: it is better to 
suffer one tyrant than many, and to suffer wrong of one 
than of every man. Yea, and it is better to have a tyrant 
unto thy king than a shadow, a passive king that doth 
nought himself, but suffer others to do with him what they 
will, and to lead him whither they list. For a tyrant 
though he do wrong unto the good, yet he punisheth the 
evil, and maketh all men obey, neither suffereth any man 
to poll but himself only. A king that is soft as silk and 
effeminate, that is to say, turned into the nature of a 
woman, what with his own lusts, which are as the longing 
of a woman with child, so that he cannot resist them, and 
what with the wily tyranny of them that ever rule him, shall 
be much more grievous unto the realm than a right tyrant. 
Read the Chronicles, and thou shalt find it ever so.

But and if thou do evil, then fear; for he beareth not 
a sword for nought: for he is the minister of God, to 
take vengeance on them that do evil. If the office of 
princes given them of God be to take vengeance of evil 
doers: then by this text and God's word, are all princes 
damned, even as many as give liberty or license unto the 
spiritualty to sin unpunished: and not only to sin unpunished
themselves; but also to open sanctuaries, privileged
places, churchyards, St. John's hold: yea, and if 
they come too short unto all these, yet to set forth a neck- 
verse to save all manner [of] trespassers from the fear of the 
sword of the vengeance of God put in the hands of princes 
to take vengeance on all such. 

God requireth the law to be kept of all men, let them
keep it for whatsoever purpose they will. Will they not 
keep the law? so vouchsafeth he not that they enjoy this
temporal life. Now are there three natures of men; one 
altogether beastly; which in no-wise receive the law in 
their hearts, but rise against princes and rulers whensoever 
they are able to make their party good. These are signified
by them that worshipped the golden calf. For 
Moses brake the tables of the law ere he came at 
them.

The second are not so beastly, but receive the law, and 
unto them the law cometh; but they look not Moses in 
the face. For his countenance is too bright for them;
that is, they understand not that the law is spiritual, and 
requireth the heart. They look on the pleasure, profit, 
and promotion that followeth the keeping of the law, and 
in respect of the reward keep they the law outwardly with 
works, but not in the heart. For if they might obtain 
like honour, glory, promotion and dignity, and also avoid 
all inconveniences, if they broke the law, so would they 
also break the law, and follow their lusts. 

The third are spiritual, and look Moses in the open face,
and are (as Paul saith Romans ii.) a law unto themselves,
and have the law written in their hearts by the spirit of 
God. These need neither of king nor officers to drive 
them, neither that any man proffer them any reward for 
to keep the law. For they do it naturally.

The first work for fear of the sword only. The second 
for reward. The third work for love freely. They look 
on the exceeding mercy, love and kindness which God 
hath showed them in Christ, and therefore love again and 
work freely. Heaven they take of the free-gift of God 
through Christ's deservings, and hope without all manner 
[of] doubting that God, according to his promise, will in this 
world also defend them, and do all things for them of his 
goodness, and for Christ's sake, and not for any goodness 
that is in them. They consent unto the law that it is 
holy and just, and that all men ought to do whatsoever 
God commandeth for no other cause, but because God 
commandeth it. And their great sorrow is, because that 
there is no strength in their members to do that which 
their heart lusteth to do, and is athirst to do. 

These of the last sort keep the law of their own accord,
and that in the heart, and have professed perpetual 
war against the lusts and appetites of the flesh, till they 
be utterly subdued; yet not through their own strength, 
but knowing and knowledging their weakness, cry ever 
for strength to God, which hath promised assistance 
unto all that call upon him. These follow God, and are 
led of his Spirit. The other two are led of lusts and appetites.

Lusts and appetites are divers and many, and that in 
one man: yea, and one lust contrary to another, and the 
greatest lust carrieth a man altogether away with him. 
We are also changed from one lust to another. Otherwise
are we disposed when we are children; otherwise 
when we are young men; and otherwise when we are old;
otherwise over even; and otherwise in the morning: yea, 
sometimes, altered six times in an hour. How fortuneth 
all this? Because that the will of man followeth the wit,
and is subject unto the wit, and as the wit erreth, so does 
the will, and as the wit is in captivity, so is the will; 
neither is it possible that the will should be free where the 
wit is in bondage. 

That thou mayest perceive and feel the thing in thine 
heart, and not be a vain sophister, disputing about words 
without perceiving; mark this. The root of all evil, the 
greatest damnation and most terrible wrath and vengeance 
of God that we are in, is natural blindness. We are all 
out of the right way, every man his ways: one judgeth 
this best, and another that to be best. Now is worldly
wit nothing else but craft and subtlety to obtain that which 
we judge, falsely, to be best. As I err in my wit, so err I
in my will. When I judge that to be evil which indeed
is good, then hate I that which is good. And when I 
suppose that good which is evil indeed, then love I evil. 
As if I be persuaded and borne in hand that my most 
friend is mine enemy, then hate I my best friend: and if 
I be brought in belief that my most enemy is my friend, 
then love I my most enemy. Now when we say, every 
man hath his free-will, to do what him lusteth, I say, 
verily, that men do what they lust. Notwithstanding, to 
follow lusts is not freedom, but captivity and bondage. 
If God open any man's wits to make him feel in his heart, 
that lusts and appetites are damnable, and give him power 
to hate and resist them, then is he free, even with the 
freedom wherewith Christ maketh free, and hath power 
to do the will of God. 

Thou mayest hereby perceive that all that is done in the
world (before the Spirit of God come, and giveth us light) 
is damnable sin, and the more glorious the more damnable;
so that, that which the world counteth most glorious, is 
more damnable in the sight of God, than that which the 
whore, the thief, and the murderer do. With blind reasons 
of worldly wisdom mayest thou change the minds of youths,
and make them give themselves to what thou wilt either for 
fear, for praise, or for profit, and yet doest but change 
them from one vice to another. As the persuasions of 
her friends made Lucrece chaste. Lucrece believed if 
she were a good housewife and chaste, that she should be 
most glorious, and that all the world would give her 
honour, and praise her. She sought her own glory in her 
chastity, and not God's. When she had lost her chastity, 
then counted she herself most abominable in the sight of 
all men, and for very pain and thought which she had, 
not that she had displeased God, but that she had lost her 
honour, slew herself. Look how great her pain and sorrow
was for the loss of her chastity, so great was her glory 
and rejoicing therein, and so much despised she them that 
were otherwise, and pitied them not; which pride God 
more abhorreth than the whoredom of any whore. Of 
like pride are all the moral virtues of Aristotle, Plato, 
and Socrates, and all the doctrine of the philosophers the 
very gods of our school-men. 

In like manner is it for the most part of our most holy 
religion. For they of like imagination do things which 
they of Bedlam may see, that they are but madness. 
They look on the miracles which God did by the saints to 
move the unbelieving unto the faith, and to confirm the 
truth of his promises in Christ, whereby all that believe 
are made saints: as thou seest in the last chapter of Mark. 
They preached (saith he) every where, the Lord working 
with them, and confirming their preaching with miracles 
that followed. And in the fourth of the Acts, the disciples
prayed that God would stretch forth his hands to 
do miracles and wonders in the name of Jesus. And 
Paul (1 Cor. xiii.) saith, that the miracle of speaking 
with divers tongues is but a sign for unbelievers, and not 
for them that believe. These miracles turn they to another
purpose, saying in their blind hearts, see what miracles
God hath showed for this saint, he must be verily 
great with God! And at once turn themselves from God's 
word, and put their trust and confidence in the saint and 
his merits, and make an advocate, or rather a God of the 
saint; and of their blind imagination make a testament or 
bond between the saint and them; the testament of Christ's 
blood clean forgotten. They look on the saints' garments 
and lives, or rather lies, which men lie on the saints:
and this wise imagine in their hearts, saying, The saint for
wearing such a garment, and for such deeds, is become 
so glorious in heaven. If I do likewise, so shall I be 
also. They see not the faith and trust which the saints had
in Christ, neither the word of God which the saints 
preached; neither the intent of the saints, how that the 
saints did such things, to tame their bodies, and to be an 
ensample to the world, and to teach that such things are 
to be despised which the world most wondereth at and magnified.
They see not also that some lands are so hot 
that a man can neither drink wine nor eat flesh therein:
neither consider they the complexion of the saints, and a 
thousand like things see they not. So when they have 
killed their bodies, and brought them in that case, that 
scarce with any restorative they can recover their health
again, yet had they lever die than to eat flesh. Why?
for they think, I have now this twenty, thirty, or forty 
years eaten no flesh, and have obtained, I doubt not, by 
this time as high a room as the best of them: should 
I now lose that? nay, I had lever die: and as Lucretia 
had lever have been slain, if he had not been too strong for 
her, than to have lost her glory, even so had these. They 
ascribe heaven unto their imaginations and mad inventions,
and receive it not of the liberality of God, by the merits 
and deservings of Christ. 

He now that is renewed in Christ, keepeth the law 
without any law written, or compulsion of any ruler or 
officer, save by the leading of the Spirit only: but the 
natural man is enticed and moved to keep the law carnally,
with carnal reasons and worldly persuasions, as for glory,
honour, riches and dignity. But the last remedy of all, 
when all other fail, is fear. Beat one, and the rest will 
abstain for fear: as Moses ever putteth in remembrance, 
saying, Kill, stone, burn. So shall thou put evil from 
thee, and all Israel shall hear and fear, and shall no more 
do so. If fear help not, then will God that they be taken 
out of this life. 

Kings were ordained then, as I before said, and the 
sword put in their hands, to take vengeance of evil doers,
that other might fear: and were not ordained to fight one 
against another, or to rise against the emperor to defend 
the false authority of the pope, that very antichrist. Bishops
they only can minister the temporal sword: their 
office, the preaching of God's word, laid apart, which they 
will neither do, nor suffer any man to do, but slay with the 
temporal sword (which they have gotten out of the hand 
of all princes) them that would. The preaching of God's 
word is hateful and contrary unto them: why? For it is 
impossible to preach Christ, except thou preach against 
antichrist; that is to say, them which, with their false doctrine
and violence of sword, enforce to quench the true 
doctrine of Christ. And as thou canst heal no disease,
except thou begin at the root: even so canst thou preach 
against no mischief, except thou begin at the bishops. 
Kings they are but shadows, vain names and things idle, 
having nothing to do in the world, but when our holy
father needeth their help. 

The Pope contrary unto all conscience and against all 
the doctrine of Christ, which saith, My kingdom is not of 
this world, (John xviii.) hath usurped the right of the Emperor.
And by policy of the bishops of Almany, and 
with corrupting the electors or choosers of the Emperor 
with money, bringeth to pass that such a one is ever chosen 
Emperor that is not able to make his party good with the 
Pope. To stop the Emperor that he come not at Rome,
he bringeth the French King up to Milan, and on the 
other side bringeth he the Venetians. If the Venetians 
come too nigh, the bishops of France must bring in the 
French King. And the Socheners are called and sent for 
to come and succour. And for their labour he giveth to 
some a rose, to another a cap of maintenance. One is 
called most Christian King, another Defender of the faith;
another the eldest son of the most holy seat. He blaseth 
also the arms of other, and putteth in the holy cross, the 
crown of thorn, or the nails, and so forth. If the French 
king go too high, and creep up either to Bononia or 
Naples: then must our English bishops bring in our king. 
The craft of the bishops is to entitle one king with another's 
realm. He is called king of Denmark and of England;
he king of England and of France. Then to blind the lords 
and the commons, the king must challenge his right. Then 
must the land be taxed and every man pay, and the treasure 
borne out of the realm and the land beggared. How many a 
thousand men's lives hath it cost? And how many an hundred
thousand pounds hath it carried out of the realm in our 
remembrance? Besides how abominable an example of 
gathering was there; such verily as never tyrant since the 
world began did, yea such as was never before heard or 
thought on, neither among Jews, Saracens, Turks, or 
heathen, since God created the sun to shine; that a beast 
should break up into the temple of God, that is to say, 
into the heart and consciences of men, and compel them 
to swear every man what he was worth, to lend that should 
never be paid again. How many thousands forswear 
themselves! How many thousands set themselves above 
their ability, partly for fear lest they should be forsworn 
and partly to save their credence! When the pope hath 
his purpose, then is peace made, no man wotteth how, 
and our most enemy is our most friend.

Now because the Emperor is able to obtain his right, 
French, English, Venetians and all must upon him. O 
great whore of Babylon, how abuseth she the princes of 
the world, how drunk hath she made them with her wine!
How shameful licenses doth she give them, to use necromancy,
to hold whores, to divorce themselves, to break the 
faith and promises that one maketh with another; that the 
confessors shall deliver unto the king the confession of 
whom he will, and dispenseth with them even of the very 
law of God, which Christ himself cannot do. 


AGAINST THE POPE'S FALSE POWER. 

CHRIST saith unto Peter, put up thy sword into his sheath. 
For all that lay hand upon the sword shall perish with 
the sword, (Matthew xxvi.) That is, whosoever without the 
commandment of the temporal officer, to whom God hath 
given the sword, layeth hand on the sword to take vengeance,
the same deserveth death in the deed doing. God did 
not put Peter only under the temporal sword, but also
Christ himself. As it appeareth in the fourth chapter to
the Galatians. And Christ saith (Mat. iii.) Thus becometh
it us to fulfil all righteousness, that is to say, all 
ordinances of God. If the head be then under the temporal
sword, how can the members be excepted? If 
Peter sinned in defending Christ against the temporal 
sword (whose authority and ministers the bishops then 
abused against Christ as ours do now) who can excuse our 
prelates of sin which will obey no man, neither king nor 
emperor? Yea, who can excuse from sin, either the kings 
that give, either the bishops that receive such exemptions 
contrary to God's ordinances, and Christ's doctrine? 

And Mat. xviith, both Christ and also Peter pay tribute, 
where the meaning of Christ's question unto Peter is, if 
princes take tribute of strangers only and not of their 
children, then verily ought I to be free which am the Son 
of God, whose servants and ministers they are, and of 
whom they have their authority. Yet because they neither 
knew that, neither Christ came to use that authority, but 
to be our servant, and to bear our burden, and to obey all 
ordinances, both in right and wrong for our sakes and to 
teach us; therefore said he to St. Peter, Pay for thee and 
me lest we offend them. Moreover though that Christ 
and Peter (because they were poor) might have escaped, 
yet would he not for fear of offending other and hurting 
their consciences. For he might well have given occasion 
unto the tribute gatherers to have judged amiss both of 
him and his doctrine; yea, and the Jews might happly 
have been offended thereby, and have thought that it had 
not been lawful for them to have paid tribute unto heathen 
princes and idolaters, seeing that he so great a prophet 
paid not. Yea, and what other thing causeth the lay 
so little to regard their princes, as that they see them both 
despised and disobeyed of the spiritualty? But our prelates
which care for none offending of consciences and less
for God's ordinances, will pay nought: but when princes
must fight in our most holy father's quarrel, and against
Christ. Then are they the first. There also is none so 
poor that then hath not somewhat to give.

Mark here, how past all shame our school doctors are, 
(as Rochester is in his sermon against Martin Luther) 
which, of this text of Matthew, dispute, that Peter, because
he paid tribute, is greater than the other apostles, and hath
more authority and power than they, and was head unto 
them all: contrary unto so many clear texts, where Christ 
rebuketh them: saying That is an heathenish thing that one 
should climb above another, or desire to be greater. To be 
great in the kingdom of heaven is to be a servant, and he 
that most humbleth himself, and becometh a servant to 
other (after the ensample of Christ, I mean, and his apostles, 
and not of the Pope and his apostles, our cardinals and 
bishops,) the same is greatest in that kingdom. If Peter 
in paying tribute became greatest, how cometh it, that they 
will pay none at all? But to pay tribute is a sign of subjection
verily; and the cause why Christ payed was because 
he had an household, and for the same cause payed Peter 
also. For he had an house, a ship and nets, as thou readest 
in the gospel. But let us go to Paul again. 

Wherefore ye must needs obey, not for fear of veangeance 
only, but also because of conscience. That is, though 
thou be so naughty (as now many years our Pope and 
prelates every where are) that thou needest not to obey the 
temporal sword for fear of vengeance: yet must thou obey 
because of conscience. First, because of thine own conscience.
For though thou be able to resist, yet shalt thou 
never have a good conscience, as long as God's word, law,
and ordinance, are against thee. Secondarily, for thy 
neighbour's conscience. For though through craft and 
violence thou mightest escape, and obtain liberty or privilege
to be free from all manner [of] duties: yet oughtest 
thou neither to sue or to seek for any such thing; neither 
yet admit or accept, if it were proffered, lest thy freedom 
make thy weak brother to grudge and rebel, in that he 
seeth thee go empty, and he himself more laden, thy part 
also laid on his shoulders. Seest thou not if a man favour 
one son more than another, or one servant more than 
another, how all the rest grudge, and how love, peace, and 
unity is broken? What Christianly love is in thee to thy 
neighbour ward, when thou canst find in thy heart to go 
up and down empty by him all day long, and see him over 
charged, yea, to fall under his burden, and yet wilt not 
once set to thine hand to help him? What good conscience 
can there be among our spiritualty, to gather so great 
treasure together, and with hypocrisy of their false learning 
to rob almost every man of house and lands, and yet not 
therewith content, but with all craft and wiliness to purchase 
so great liberties and exemptions from all manner [of] bearing 
with their brethren, seeking in Christ nothing but lucre?
I pass over with silence how they teach princes in every 
land, to lade new exactions and tyranny on their subjects 
more and more daily, neither for what purpose they do it 
say I. God I trust shall shortly disclose their juggling, 
and bring their falsehood to light; and lay a medicine to 
them, to make their scabs break out. Nevertheless this I 
say, that they have robbed all realms, not of God's word 
only; but also of all wealth and prosperity, and have driven 
peace out of all lands, and withdrawn themselves from all 
obedience to princes, and have separated themselves from 
the lay men, counting them viler than dogs; and have set 
up the whore of Babylon, antichrist of Rome, whom they 
call Pope, and have conspired against all common-wealths, 
and have made them a several kingdom, wherein it is 
lawful, unpunished, to work all abomination. In every 
parish have they spies, and in every great man's house, and 
in every tavern and alehouse. And through confessions 
know they all secrets, so that no man may open his mouth 
to rebuke whatsoever they do, but that he shall be shortly 
made an heretic. In all councils is one of them, yea the 
most part and chief rulers of the councils are of them:
but of their council is no man. 

Even for this cause pay ye tribute, that is to wit, for conscience'
sake to thy neighbour, and for the cause that followeth.
For they are God's ministers serving for the same 
purpose. Because God will so have it, we must obey. 
We do not look (if we have Christ's Spirit in us) what is 
good, profitable, glorious and honourable for us, neither 
on our own will, but on God's will only. Give to every 
man therefore his duty; tribute to whom tribute belongeth;
custom to whom custom is due; fear to whom fear belongeth;
honour to whom honour pertaineth. 

That thou mightest feel the working of the spirit of 
God in thee, and lest the beauty of the deed should deceive
thee, and make thee think that the law of God, 
which is spiritual, were content and fulfilled with the outward
and bodily deed it followeth. Owe nothing to any 
man, but to love one another: for he that loveth another
fulfilleth the law. For these commandments, Thou 
shalt not commit adultery, thou shalt not kill, thou shalt 
not steal, thou shalt not bear false witness, thou shalt not 
desire, and so forth, if there be any other commandment,
are all comprehended or contained in this saying, Love 
thy neighbour: therefore is love the fulfilling of the law. 
Here hast thou sufficient against all the sophisters work-
holy and justifiers in the world, which so magnify their 
deeds. The law is spiritual, and requireth the heart, and 
is never fulfilled with the deed in the sight of God. With 
the deed thou fulfillest the law before the world, and 
livest thereby, that is, thou enjoyest this present life, and 
avoidest the wrath and vengeance, the death and punishment
which the law threateneth to them that break it. 
But before God thou keepest the law if thou love only. 
Now what shall make us love? Verily that shall faith 
do. If thou behold how much God loveth thee in Christ, 
and from what vengeance he hath delivered thee for his 
sake, and of what kingdom he hath made thee here, then 
shalt thou see cause enough to love thy very enemy without
respect of reward, either in this life or in the life to 
come; but because that God will so have it, and Christ 
hath deserved it, yet thou shouldest feel in thine heart 
that all thy deeds to come are abundantly recompensed 
already in Christ. 

Thou wilt say haply, If love fulfil the law, then it justifieth.
I say that that wherewith a man fulfilleth the 
law declareth him justified; but that which giveth him 
wherewith to fulfil the law justifieth him. By justifying, 
understand the forgiveness of sins, and the favour of God.
Now saith the text, (Rom. x.) the end of the law, or the 
cause wherefore the law was made, is Christ, to justify 
all that believe. That is, the law is given to utter sin, to 
kill the consciences, to damn our deeds, to bring to repentance,
and to drive unto Christ: in whom God hath 
promised his favour and forgiveness of sin unto all that 
repent and consent to the law that it is good. If thou believe
the promises, then doth God's truth justify thee;
that is, forgiveth thee, and receiveth thee to favour for 
Christ's sake. In a surety whereof, and to certify thine 
heart, he sealeth thee with the Spirit. (Eph. i. and iv.) And, 
(2 Cor. V.) saith Paul; Which gave us his Spirit in earnest. 
Now the Spirit is given us through Christ. (Read the 
viiith chapter of the Epistle to the Romans, and Gal. iii. 
and 2 Cor. iii.) Nevertheless the Spirit, and his fruits, 
wherewith the heart is purified, as faith, hope, love, patience,
long-suffering and obedience, could never be 
seen without outward experience. For if thou were not 
brought sometime into cumbrance, whence God only 
could deliver thee, thou shouldest never see thy faith, yea, 
except thou foughtest sometime against desperation, hell, 
death, sin, and powers of this world, for thy faith's 
sake, thou shouldest never know true faith from a 
dream. Except thy brother now and then offended thee, 
thou couldest not know whether thy love were godly. 
For a Turk is not angry till he be hurt and offended;
but if thou love him that doth thee evil, then is thy 
love of God: likewise if thy rulers were alway kind, thou 
shouldest not know whether thine obedience were pure 
or no; but and if thou canst patiently obey evil rulers in 
all things that is not to the dishonour of God, and when 
thou hurtest not thy neighbours, then art thou sure that 
God's spirit worketh in thee, and that thy faith is no dream, 
nor any false imagination. 

Therefore counselleth Paul, (Rom. xii.) Recompense 
to no man evil. And on your part have peace with all 
men. Dearly beloved, avenge not yourselves, but give 
room unto the wrath of God. For it is written. Vengeance
is mine, and I will reward, saith the Lord. Therefore,
if thy enemy hunger, feed him: if he thirst, give 
him drink. For in so doing, thou shalt heap coals of fire 
on his head (that is, thou shalt kindle love in him.) Be 
not overcome of evil (that is, let not another man's wickedness
make thee wicked also.) But overcome evil with 
good, that is, with softness, kindness, and all patience 
win him; even as God with kindness won thee.

The law was given in thunder, lightning, fire, smoke, 
and the noise of a trumpet and terrible sight. (Exod. xx.)
So that the people quaked for fear, and stood afar off,
saying to Moses, Speak thou to us, and we will hear:
let not the Lord speak unto us, lest we die. No ear (if 
it be awaked and understandeth the meaning) is able to 
abide the voice of the law, except the promises of mercy 
be by. That thunder, except the reign of mercy be joined 
with it, destroyeth all, and buildeth not, The law is a 
witness against us, and testifieth that God abhorreth the 
sins that are in us, and us for our sins' sake.

In like manner, when God gave the people of Israel 
a king, it thundered and rained, that the people feared 
so sore, that they cried to Samuel for to pray for them 
that they should not die. (1 Kings xii.) As the law is a 
terrible thing; even so is the king. For he is ordained 
to take vengeance, and hath a sword in his hand, and not 
peacocks' feathers. Fear him, therefore, and look on
him as thou wouldest look on a sharp sword that hanged 
over thy head by an hair.

Heads and governors are ordained of God, and are 
even the gift of God, whether they be good or bad. And 
whatsoever is done to us by them, that doth God, be it 
good or bad. If they be evil, why are they evil? Verily, 
for our wickedness' sake are they evil. Because that when 
they were good, we would not receive that goodness of the 
hand of God, and be thankful: submitting ourselves unto 
his laws and ordinances, but abused the goodness of God 
unto our sensual and beastly lusts. Therefore doth God 
make his scourge of them, and turn them to wild beasts, 
contrary to the nature of their names and offices; even
unto lions, bears, foxes, and unclean swine, to avenge
himself of our unnatural and blind unkindness, and of 
our rebellious disobedience. 

In the cvith Psalm, thou readest, He destroyed the 
rivers, and dried up the springs of water, and turned the 
fruitful land into barrenness, for the wickedness of the inhabiters
therein. When the children of Israel had forgotten
God in Egypt, God moved the hearts of the 
Egyptians to hate them, and to subdue them with craft 
and wiliness. (Psalm civ. and Deut. iii.) Moses rehearseth,
saying, God was angry with me for your 
sakes. So that the wrath of God fell on Moses, for the 
wickedness of the people. And in the second chap. of 
the second book of Kings, God was angry with the people, 
and moved David to number them; when Joab and the 
other lords wondered why he would have them numbered;
and because they feared lest some evil should follow, dissuaded
the king; yet it holp not. God so hardened his 
heart in his purpose, to have an occasion to slay the wicked 
people. 

Evil rulers then are a sign that God is angry and wroth
with us. Is it not a great wrath and vengeance that the
father and mother should hate their children, even their
flesh and their blood? Or that an husband should be 
unkind unto his wife, or a master unto the servant that 
waiteth on his profit; or that lords and kings should be 
tyrants unto their subjects and tenants, which pay them 
tribute, toll, custom and rent, labouring and toiling to 
find them in honour, and to maintain them in their estate?
Is not this a fearful judgment of God, and a cruel wrath, 
that the very prelates and shepherds of our souls, which 
were wont to feed Christ's flock with Christ's doctrine, 
and to walk before them in living thereafter, and to give 
their lives for them, to their ensample and edifying; and 
to strengthen their weak faiths; are now so sore changed, 
that if they smell that one of their flock (as they now call 
them, and no longer Christ's) do but once long or desire 
for the true knowledge of Christ, they will slay him, 
burning him with fire most cruelly? What is the cause
of this; and that they also teach false doctrine, confirming
it with lies? Verily, it is the hand of God, to avenge the 
wickedness of them that have no love nor lust unto the 
truth of God, when it is preached, but rejoice in unrighteousness.
As thou mayest see in the second Epistle 
of Paul to the Thessalonians, where he speaketh of the 
coming of antichrist; Whose coming shall be (saith he)
by the working of Satan, with all miracles, signs and 
wonders, which are but lies, and in all deceivable unrighteousness
among them that perish, because they received 
not any love to the truth to have been saved. Therefore 
shall God send them strong delusion, to believe lies. Mark 
how God, to avenge his truth, sendeth to the unthankful 
false doctrine and false miracles, to confirm them, and to 
harden their hearts in the false way, that afterward it shall 
not be possible for them to admit the truth. As thou
seest in Exod. vii. and viii., how God suffered false miracles
to be showed in the sight of Pharaoh, to harden his 
heart, that he should not believe the truth, inasmuch as 
his sorcerers turned their rods into serpents, and turned 
water into blood, and made frogs by their enchantment:
so thought he that Moses did all his miracles by the same 
craft, and not by the power of God. And abode therefore
in unbelief, and perished in resisting God.

Let us receive all things of God, whether it be good or 
bad; let us humble ourselves under his mighty hand, and 
submit ourselves unto his nurture and chastising, and not 
withdraw ourselves from his correction. Read Heb. xii.
for thy comfort; and let us not take the staff by the end, 
or seek to avenge ourselves on his rod, which is the evil 
rulers. The child, as long as he seeketh to avenge himself
upon the rod, hath an evil heart. For he thinketh 
not that the correction is right, or that he hath deserved it, 
neither repenteth, but rejoiceth in his wickedness. And 
so long shall he never be without a rod: yea, so long 
shall the rod be made sharper and sharper. If he knowledge
his fault and take the correction meekly, and even 
kiss the rod, and amend himself with the learning and 
nurture of his father and mother, then is the rod taken
away and burnt. 

So, if we resist evil rulers, seeking to act ourselves at 
liberty, we shall, no doubt, bring ourselves into more 
evil bondage, and wrap ourselves in much more misery 
and wretchedness. For if the heads overcome, then 
lay they more weight on their backs, and make their 
yoke sorer and tie them shorter. If they overcome their 
evil rulers, then make they way for a more cruel nation, or 
for some tyrant of their own nation, which hath no right 
unto the crown. If we submit ourselves unto the chastising
of God, and meekly knowledge our sins for which 
we are scourged, and kiss the rod, and amend our living;
then will God take the rod away, that is, he will give the 
rulers a better heart. Or if they continue their malice 
and persecute you for well-doing, and because ye put your 
trust in God, then will God deliver you out of their 
tyranny for his truth's sake. It is the same God now that 
was in the old time, and delivered the fathers and the 
prophets, the apostles, and other holy saints. And whatsoever
he sware to them he hath sworn to us. And as he 
delivered them out of all temptation, cumbrance, and adversity,
because they consented and submitted themselves 
unto his will, and trusted in his goodness and truth: even 
so will he do to us if we do likewise. 

Whensoever the children of Israel fell from the way 
which God commanded them to walk in, he gave them up 
under one tyrant or another. As soon as they came to the 
knowledge of themselves, and repented, crying for mercy, 
and leaning unto the truth of his promises, he sent one to 
deliver them, as the histories of the Bible make mention.

A Christian man, in respect of God, is but a passive thing,
a thing that suffereth only and doth nought; as the sick, 
in respect of the surgeon or physician, doth but suffer 
only. The surgeon lanceth and cutteth out the dead flesh, 
searcheth the wounds, thrusteth in tents, seareth, burneth, 
seweth or sticheth, and lieth to caustics to draw out the
corruption; and, last of all, lieth to healing plaisters, and 
maketh it whole. The physician, likewise, giveth purgations
and drinks to drive out the disease, and then, with 
restoratives, bringeth health. Now if the sick resist the 
razor, the searching iron, and so forth, doth he not resist 
his own health, and is cause of his own death? So, likewise,
is it of us, if we resist evil rulers, which are the rod
and scourge wherewith God chastiseth us; the instruments 
wherewith God searcheth our wounds, and bitter drinks 
to drive out the sin and to make it appear, and caustics to 
draw out by the roots the core of the pox of the soul that 
fretteth inward. A Christian man, therefore, receiveth all 
things of the hand of God, both good and bad, both sweet 
and sour, both wealth and woe. If any person do me good, 
whether it be father, mother, and so forth, that receive I of 
God, and to God give thanks. For he gave wherewith, and 
gave a commandment, and moved his heart so to do. Adversity
also receive I of the hand of God as an wholesome 
medicine, though it be somewhat bitter. Temptation and 
adversity do both kill sin, and also utter it. For though a 
Christian man knoweth every thing how to live, yet is the 
flesh so weak, that he can never take up his cross himself 
to kill and mortify the flesh. He must have another to 
lay it on his back. In many, also, sin lieth hid within, 
and festereth and rotteth inward, and is not seen; so that 
they think how they are good and perfect, and keep the 
law. As the young man (Matt. xix.) said, he had observed
all of a child, and yet lied falsely in his heart, as 
the text following well declareth. When all is at peace, 
and no man troubleth us, we think that we are patient and 
love our neighbours as ourselves; but let our neighbour 
hurt us in word or deed, and then find we it otherwise.
Then fume we, and rage, and set up the bristles, and bend 
ourselves to take vengeance. If we loved with godly love 
for Christ's kindness' sake, we should desire no vengeance, 
but pity him, and desire God to forgive and amend him, 
knowing well that no flesh can do otherwise than sin;
except that God preserve him. Thou wilt say, What 
good doth such persecution and tyranny unto the righteous?
First, it maketh them feel the working of God's 
Spirit in them, and that their faith is unfeigned. Secondarily,
I say that no man is so great a sinner, if he repent 
and believe, but that he is righteous in Christ and in the 
promises: yet if thou look on the flesh, and unto the law, 
there is no man so perfect that is not found a sinner. Nor 
any man so pure that hath not somewhat to be yet purged.
This shall suffice at this time as concerning obedience. 

Because that God excludeth no degree from his 
mercy; but whosoever repenteth, and believeth his promises,
(of whatsoever degree he be of,) the same shall be 
partaker of his grace; therefore, as I have described the 
obedience of them that are under power and rule, even so 
will I, with God's help, (as my duty is,) declare how the 
rulers which God shall vouchsafe to call unto the knowledge
of the truth, ought to rule. 


THE OFFICE OF A FATHER, AND HOW HE 
SHOULD RULE. 

Fathers, move not your children unto wrath, but 
bring them up in the nurture and information of the 
Lord. (Eph. vi. and Col. iii.) Fathers, rate not your 
children, lest they be of desperate mind; that is, lest you 
discourage them. For where the fathers and mothers are 
wayward, hasty and churlish, ever brawling and chiding, 
there are the children anon discouraged and heartless, and 
apt for nothing; neither can they do any thing aright. 
Bring them up in the nurture and information of the 
Lord. Teach them to know Christ, and set God's ordinance
before them, saying, Son, or daughter, God hath 
created thee and made thee, through us thy father and 
mother, and at his commandment have we so long thus
kindly brought thee up, and kept thee from all perils; he 
hath commanded thee also to obey us, saying, Children,
obey thy father and mother. If thou meekly obey, so 
shalt thou grow both in the favour of God and man, and 
knowledge of our Lord Christ. If thou wilt not obey us 
at his commandment, then are we charged to correct thee;
yea, and if thou repent not, and amend thyself, God shall 
slay thee by his officers, or punish thee everlastingly. 
Nurture them not worldly, and with worldly wisdom, 
saying, Thou shalt come to honour, dignity, promotion, 
and riches; thou shalt be better than such and such; thou 
shalt have three or four benefices, and be a great doctor or 
a bishop, and have so many men waiting on thee, and do 
nothing but hawk and hunt, and live at pleasure; thou 
shalt not need to sweat, to labour, or to take any pain for 
thy living, and so forth; filling them full of pride, disdain,
and ambition, and corrupting their minds with worldly 
persuasions. Let the fathers and mothers mark how they 
themselves were disposed at all ages; and by experience 
of their own infirmities help their children, and keep 
them from occasions. Let them teach their children to 
ask marriages of their fathers and mothers. And let their 
elders provide marriages for them in season; teaching 
them also to know, that she is not his wife whom the son 
taketh, nor he her husband which the daughter taketh, 
without the consent and good will of their elders, or them 
that have authority over them. If their friends will not 
marry them, then are they not to blame if they marry 
themselves. Let not the fathers and mothers always take 
the utmost of their authority of their children; but at a 
time suffer with them, and bear their weaknesses, as Christ 
doth ours. Seek Christ in your children, in your wives, 
servants, and subjects. Father, mother, son, daughter, 
master, servant, king, and subject, be names in the worldly 
regiment. In Christ we are all one thing; none better 
than another, all brethren; and all must seek Christ, and 
our brother's profit in Christ. And he that hath the 
knowledge, whether he be the Lord or king, is bound to 
submit himself and serve his brethren, and to give himself 
for them to win them to Christ. 


THE OFFICE OF AN HUSBAND AND HOW HE 
OUGHT TO RULE. 

HUSBANDS, love your wives as Christ loved the congregation,
and gave himself for it, to sanctify it and 
cleanse it. Men ought to love their wives, as their own 
bodies. For this cause shall a man leave father and mother 
and shall continue with his wife, and shall be made both 
one flesh. See that every one of you love his wife even as 
his own body: All this saith Paul Eph. v. And Col. iii. 
he saith, Husbands, love your wives, and be not bitter unto 
them. And Peter in the iiird chapter of his first Epistle, 
saith, Men, dwell with your wives according to knowledge, 
(that is according to the doctrine of Christ) giving reverence
unto the wife, as unto the weaker vessel; (that is, 
help her to bear her infirmities) and as unto them that are 
heirs also of the grace of life, that your prayers be not let. 
In many things God hath made the men stronger than the 
women; not to rage upon them and to be tyrants unto them, 
but to help them to bear their weakness. Be courteous
therefore unto them, and win them unto Christ, and 
overcome them with kindness, that of love they may obey 
the ordinance that God hath made between man and wife. 


THE OFFICE OF A MASTER AND HOW HE 
OUGHT TO RULE.

PAUL (Eph. vith) saith: Ye masters do even the same 
things to them, (that is be master after the ensample 
and doctrine of Christ, as he before taught the servants to 
obey to their masters as to Christ) putting away threatenings
(that is, give them fair words, and exhort them 
kindly to do their duty; yea, nurture them as thy own 
sons with the Lord's nurture, that they may see in Christ 
a cause why they ought lovingly to obey) and remember 
(saith he) that your master also is in heaven. Neither is 
there any respect of persons with him; that is, he is indifferent
and not partial: as great in his sight is a servant 
as a master. And in the iiird chapter to the Colossians, 
saith he: Ye masters do unto your servants that which is 
just and equal, remembering that ye also have a master in 
heaven. Give your servants kind words, food, raiment 
and learning. Be not bitter unto them, rail not on them, 
give them no cruel countenance: but according to the ensample
and doctrine of Christ, deal with them. And 
when they labour sore, cherish them again. When ye 
correct them, let God's word be by, and do it with such 
good manner that they may see how that ye do it to amend 
them only, and to bring them to the way which God 
biddeth us walk in, and not to avenge yourselves, or to 
wreak your malice on them. If at a time through hastiness 
ye exceed measure in punishing, recompense it another 
way, and pardon them another time. 


THE DUTY OF LANDLORDS. 

Let Christian landlords be content with their rent and 
old customs; not raising the rent or fines, and bringing 
up new customs to oppress their tenants : neither letting 
two or three tenantries unto one man. Let them not take 
in their commons, neither make parks nor pastures of 
whole parishes. For God gave the earth to man to inhabit, 
and not unto sheep and wild deer. Be as fathers unto 
your tenants: yea be unto them as Christ was unto us, 
and shew unto them all love and kindness. Whatsoever 
business is among them, be not partial, favouring one more 
than another. The complaints, quarrels, and strife that 
are among them, count diseases of sick people, and as a 
merciful physician heal them with wisdom and good counsel.
Be pitiful and tender hearted unto them, and let not 
one of thy tenants tear out another's throat, but judge 
their causes indifferently, and compel them to make their 
ditches, hedges, gates and ways. For even for such causes 
were ye made landlords, and for such causes paid men rent 
at the beginning. For if such an order were not, one 
should slay another, and all should go to waste. If thy 
tenant shall labour and toil all the year to pay thee thy rent, 
and when he hath bestowed all his labour, his neighbours' 
cattle shall devour his fruits, how tedious and bitter should 
his life be! See therefore that ye do your duties again, and 
suffer no man to do them wrong, save the king only. If 
he do wrong, then must they abide God's judgment. 


THE DUTY OF KINGS, AND OF THE JUDGES 
AND OFFICERS. 

Let kings (if they had lever be Christian in deed, than 
so to be called) give themselves altogether to the wealth 
of their realms after the ensample of Christ; remembering 
that the people are God's, and not their's, yea are Christ's 
inheritance and possession bought with his blood. The 
most despised person in his realm is the king's brother, 
and fellow member with him, and equal with him in the 
kingdom of God and of Christ. Let him therefore not 
think himself too good to do them service, neither seek any 
other thing in them, than a father seeketh in his children, 
yea than Christ sought in us. Though that the king in the 
temporal regiment be in the room of God, and representeth 
God himself, and is without all comparison better than 
his subjects; yet let him put off that and become a brother, 
doing and leaving undone all things in respect of the commonwealth,
that all men may see that he seeketh nothing, 
but the profit of his subjects. When a cause that requireth 
execution is brought before him, then only let him take the 
person of God on him. Then let him know no creature 
but hear all indifferently; whether it be a stranger or one 
of his own realm, and the small as well as the great; and 
judge righteously, for the judgment is the Lord's. (Deut. 
i.) In time of judgment he is no minister in the kingdom 
of Christ; he preacheth no gospel but the sharp law 
of vengeance. Let him take the holy judges of the old 
Testament for an ensample, and namely Moses, which in 
executing the law was merciless otherwise; more than a 
mother unto them, never avenging his own wrongs, but 
suffering all things; bearing every man's weakness, teaching, 
warning, exhorting, and ever caring for them, and so 
tenderly loved them, that he desired God either to forgive 
them, or to damn him with them. 

Let the judges also privately, when they have put off 
the person of a judge, exhort with good counsel, and 
warn the people and help, that they come not at God's 
judgment: but the causes that are brought to them, 
when they sit in God's stead, let them judge and condemn 
the trespasser under lawful witnesses, and not break up 
into the consciences of men, after the example of antichrist's
disciples, and compel them either to forswear 
themselves by the almighty God, and by the holy gospel 
of his merciful promises, or to testify against themselves. 
Which abomination our prelates learned of Caiphas, 
(Matt. xxvi.) saying to Christ, I adjure or charge thee in 
the name of the living God, that thou tell us whether 
thou be Christ, the son of God: let that which is secret 
to God only, whereof no proof can be made, nor 
lawful witness brought, abide to the coming of the 
Lord, which shall open all secrets. If any malice break 
forth, let them judge only. For further authority hath 
God not given them. 

Moses (Deut. xvii.) warneth judges to keep them upright,
and to look on no man's person; that is, that they 
prefer not the high before the low, the great before the 
small, the rich before poor, his acquaintance, friend, 
kinsman, countryman, or one of his own nation before 
a stranger, a friend or an alien, yea, or one of their own 
faith before an infidel: but that they look on the cause 
only to judge indifferently. For the room that they are 
in, and the law that they execute, are God's; which, as 
he hath made all, and is God of all, and all are his sons: 
even so is he judge over all, and will have all judged by 
his law indifferently, and to have the right of his law, and 
will avenge the wrong done unto the Turk or Saracen. 
For though they be not under the everlasting testament of 
God in Christ, as few of us which are called Christian 
be, and even no more than to whom God hath sent his 
promises, and poured his Spirit into their hearts to believe
them, and through faith graven lust in their hearts, 
to fulfil the law of love; yet are they under the testament 
of the law natural, which is the laws of every land made 
for the common wealth there, and for peace and unity, 
that one may live by another. In which laws the infidels 
(if they keep them) have promises of worldly things. 
Whosoever, therefore, hindreth a very infidel from the 
right of that law, sinneth against God, and of him will 
God be avenged. Moreover, Moses warneth them that 
they receive no gifts, rewards or bribes. For those two 
points, favouring of one person more than another, and 
receiving rewards, pervert all right and equity, and is the 
only pestilence of all judges. 

And the kings warneth he that they have not too many 
wives, lest their hearts turn away: and that they read 
alway in the law of God, to learn to fear him, lest their 
hearts be lift up above their brethren. Which two points, 
women and pride, the despising of their subjects, which 
are in very deed their own brethren, are the common pestilence
of all princes. Read the stories, and see. 

The sheriffs, baily errants, constables, and such like officers, 
may let no man that hurteth his neighbour scape, but that 
they bring them before the judges, except they in the mean 
time agree with their neighbours, and make them amends. 

Let kings defend their subjects from the wrongs of other 
nations, but pick no quarrels for every trifle: no, let not 
our most holy father make them no more so drunk with 
vain names, with caps of maintenance, and like baubles, 
as it were puppetry for children, to beggar their realms, 
and to murder their people, for defending of our holy 
father's tyranny. If a lawful peace, that standeth with 
God's word, be made between prince and prince, and 
the name of God taken to record, and the body of our 
Saviour broken between them, upon the bond which they 
have made; that peace or bond can our holy father not 
dispense with, neither loose it with all the keys he hath:
no, verily, Christ cannot break it. For he came not to 
break the law, but to fulfil it. (Matt. v.) 

If any man have broken the law, or a good ordinance, 
and repent and come to the right way again, then hath 
Christ power to forgive him: but licence to break the 
law can he not give; much more his disciples and vicars 
(as they call themselves) cannot do it. The keys whereof 
they so greatly boast themselves, are no carnal things, but 
spiritual, and nothing else save knowledge of the law, 
and of the promises or gospel: if any man for lack of 
spiritual feeling desire authority of men, let him read the 
old doctors. If any man desire authority of Scripture, 
Christ saith, (Luke xi.) Woe be to you lawyers, for ye 
have taken away the key of knowledge: ye enter not in 
yourselves, and them that come in, ye forbid. That is, 
they had blinded the Scripture, whose knowledge (as it 
were a key) letteth into God, with glosses and traditions. 
Likewise findest thou, (Matt. xxiii.) as Peter answered 
in the name of all; so Christ promised him the keys in 
the person of all. (Matt. xvi.) And in the xxth of John, 
he paid them, saying, Receive the Holy Ghost; whosoever
sins ye remit, they are remitted or forgiven, and 
whosoever sins ye retain, they are retained or holden. 
With preaching the promises, loose they as many as repent
and believe. And for that John saith, Receive the 
Holy Ghost. Luke, in his last chapter, saith, Then 
opened he their wits, that they might understand the 
Scriptures, and said unto them, Thus it is written. And 
thus it behoved Christ to suffer, and to rise again the 
third day. And that repentance and remission of sins 
should be preached in his name among all nations. At 
preaching of the law, repent men; and at the preaching 
of the promises, do they believe, and are saved. Peter, 
in the second of the Acts, practised his keys, and by 
preaching the law, brought the people into the knowledge 
of themselves, and bound their consciences, so that they 
were pricked in their hearts, and said unto Peter and to 
the other apostles, What shall we do? Then brought 
they forth the key of the sweet promises, saying, Repent,
and be baptized every one of you, in the name of Jesus 
Christ, for the remission of sins, and ye shall receive the 
gift of the Holy Ghost. For the promise was made to 
you, and to your children, and to all that are afar, even 
as many as the Lord shall call. Of like ensamples is the 
Acts full, and Peter's Epistles, and Paul's Epistles, and 
all the Scripture; neither hath our holy father any other 
authority of Christ, or by the reason of his predecessor, 
Peter, than to preach God's word. As Christ compareth
the understanding of Scripture to a key, so 
compareth he it to a net, and to leaven, and to 
many other things for certain properties. I marvel, therefore,
that they boast not themselves of their net and 
leaven, as well as of their keys, for they are all one thing. 
But as Christ biddeth us beware of the leaven of the 
Pharisees, so beware of their counterfeited keys, and of 
their false net (which are their traditions and ceremonies, 
their hypocrisy and false doctrine, wherewith they catch, not 
souls unto Christ, but authority and riches unto themselves.) 

Let christian kings therefore keep their faith and truth, 
and all lawful promises and bonds, not one with another 
only, but even with the Turk or whatsoever infidel it be. 
For so it is right before God, as the Scriptures and ensamples
of the Bible testify. Whosoever voweth an unlawful
vow, promiseth an unlawful promise, sweareth an 
unlawful oath, sinneth against God, and ought therefore 
to break it. He needeth not sue to Rome for a licence, 
for he hath God's word, and not a licence only, but also a 
commandment to break it. They therefore that are sworn 
to be true to cardinals and bishops, that is to say, false 
unto God, the king, and the realm, may break their oaths 
lawfully without grudge of conscience, by the authority of 
God's word. In making them they sinned, but in repenting
and breaking them they please God highly, and receive
forgiveness in Christ, 

Let kings take their duty of their subjects, and that is 
necessary to the defence of the realm. Let them rule 
their realms themselves, with the help of lay men that are 
sage, wise, learned, and expert. Is it not a shame above 
all shames, and a monstrous thing, that no man should be 
found able to govern a worldly kingdom, save by bishops and 
prelates, that have forsaken the world, and are taken out of
the world, and appointed to preach the kingdom of God?
Christ saith that His kingdom is not of this world. (John 
xviii.) And (Luke xii.) unto the young man that desired
him to bid his brother to give him part of the inheritance, 
he answered, Who made made me a judge or a divider 
among you? No man that layeth his hand to the plough, 
and looketh back, is apt for the kingdom of heaven. (Luke 
ix.) No man can serve two masters, but he must despise 
the one. (Matt. vi.) 

To preach God's word is too much for half a man: and 
to minister a temporal kingdom is too much for half a man 
also: either other requireth an whole man; one therefore 
cannot well do both. He that avengeth himself on every 
trifle, is not meet to preach the patience of Christ, how 
that a man ought to forgive and to suffer all things. He
that is overwhelmed with all manner [of] riches, and doth 
but seek more daily, is not meet to preach poverty. He 
that will obey no man, is not meet to preach how we ought 
to obey all men. Peter saith (Acts vi.) It is not meet 
that we should leave the word of God and serve at the tables.
Paul saith in the ixth chapter of the first Corinthians,
Woe is me if I preach not. A terrible saying, verily,
for popes, cardinals, and bishops. If he had said, Woe 
be unto me if I fight not and move princes unto war, or if 
I increase not St. Peter's patrimony, (as they call it) it had 
been a more easy saying for them. 

Christ forbiddeth his disciples and that oft, (as thou mayest 
Matt. see xviii and also xx. Mark ix. and also x. Luke ix. and 
also xxii. even at his last supper) not only to clime above lords, 
kings, and emperors in worldly rule, but also to exalt themselves
one above another in the kingdom of God. But in 
vain, for the pope would not hear it, though he had commanded
it ten thousand times. God's word should rule
only; and not bishops' decrees, or the pope's pleasure: that
ought they to preach purely and spiritually, and to fashion 
their lives after, and with all ensample of godly living and 
long suffering to draw all to Christ: and not to expound 
the Scriptures carnally and worldly, saying, God spake this 
to Peter, and I am his successor, therefore this authority 
is mine only; and then bring in the tyranny of their fleshly 
wisdom, in prasentia majoris, cessat potestas minoris that is
in the presence of the greater the less hath hath no power.
There is no brotherhood where such philosophy is taught. 

Such philosophy, and so to abuse the Scriptures, and to
mock with God's word, is after the manner of the bishop 
of Rochester's divinity; for he in his Sermon of the condemnation
of Martin Luther, proveth by a shadow of the 
Old Testament, that is, by Moses and Aaron, that Satan 
and antichrist, our most holy father the pope, is Christ's 
vicar and head of Christ's congregation. 

Moses, saith he, signifieth Christ, and Aaron the pope; 
and yet the Epistle unto the Hebrews proveth, that the 
high priest of the old law signifieth Christ, and his offering
and his going in once in the year into the inner temple, 
signify the offering wherewith Christ offered himself, and 
Christ's going in unto the Father to be an everlasting Mediator
or Intercessor for us. Nevertheless, Rochester 
proveth the contrary by a shadow, by a shadow verily: for 
in shadows they walk without all shame, and the light will 
they not come at, but enforce to stop and quench it with 
all craft and falsehood, lest their abominable juggling 
should be seen. If any man look in the light of the New 
Testament he shall clearly see, that that shadow may not
be so understood.

Understand therefore that one thing in the Scripture representeth
divers things: a serpent figureth Christ in one 
place, and the devil in another, and a lion doth likewise. 
Christ by leaven signifieth God's word in one place, and in 
another signifieth thereby the traditions of the pharisees, 
which soured and altered God's word for their advantage. 
Now Moses verily in the said place representeth Christ, 
and Aaron, which was not yet high priest, represented not
Peter only or his successor, as my lord of Rochester would 
have it, (for Peter was too little to bear Christ's message 
unto all the world) but signifieth every disciple of Christ, and 
every true preacher of God's word. For Moses put in 
Aaron's mouth what he should say, and Aaron was Moses's
prophet, and spake not his own message (as the pope 
and bishops do) but that which Moses had received of God 
and delivered unto him. (Exod. iv. and also vii.) So ought 
every preacher to preach God's word purely, and neither 
to add nor minish. A true messenger must do his message 
truly, and say neither more nor less than he is commanded. 
Aaron when he is high priest, and offereth and purgeth the 
people of their worldly sin which they had fallen in, in 
touching uncleanly things, and in eating meats forbidden, 
(as we sin in handling the chalice and the altar stone, and 
are purged with the bishop's blessing) representeth Christ, 
which purgeth us from all sin in the sight of God: as the 
Epistle unto the Hebrews maketh mention: when Moses 
was gone up into the mount and Aaron left behind, and 
made the golden calf; there Aaron representeth all false 
preachers, and namely, our most holy father the pope, 
which in like manner maketh us believe in a bull, as the 
bishop of Rochester full well allegeth the place in his 
Sermon. 

If the pope be signified by Aaron, and Christ by Moses, 
why is not the pope as well content with Christ's law and 
doctrine, as Aaron was with Moses'? what is the cause 
that our bishops preach the pope and not Christ, seeing 
the apostles preached not Peter, but Christ? Paul (2 
Cor. iv.) saith of himself and his fellow apostles, We 
preach not ourselves, but Christ Jesus the Lord, and preach 
ourselves your servants for Jesus' sake: and (1 Cor. iii.) 
Let no man rejoice in men, for all things are yours, whether 
it be Paul, or Apollos, or Peter; whether it be the world, 
or life, or death; whether they be present things, or things 
to come; all are yours, and ye are Christ's, and 
Christ is God's. He leaveth out Ye are Peter's, or Ye are 
the pope's. And in the chapter following, he saith, Let men 
thuswise esteem us, even the ministers of Christ, \&c. And 
(2 Cor. xi.) Paul was jealous over his Corinthians, because 
they fell from Christ, to whom he had married them, and 
did cleave unto the authority of men, (for even then false 
prophets sought authority in the name of the high apostles,) 
I am, saith he, jealous over you with godly jealousy, for I 
coupled you to one man, to make you a chaste virgin to 
Christ; but I fear lest as the serpent deceived Eve 
through his subtlety, even so your wits should be corrupt 
from the singleness that is in Christ. And it followeth, If 
he that cometh to you preached another Jesus, or if ye receive
another spirit or another gospel, then might ye well 
have been content: that is, ye might have well suffered him 
to have authority above me: but I suppose, saith he, that 
I was not behind the high apostles; meaning in preaching 
Jesus and his gospel, and in ministering the Spirit. And 
in the said xith chapter, he proveth by the doctrine of 
Christ, that he is greater than the high apostles; for Christ 
saith to be great in the kingdom of God is to do service 
and to take pains for other. Upon which rule Paul disputeth,
saying, If they be the ministers of Christ, I am more. 
In labours more abundant, in stripes above measure, in 
prison more plenteously, in death oft, and so forth. If Paul 
preacheth Christ more than Peter, and suffered more for 
his congregation, then is he greater than Peter by the testimony
of Christ. And in the xiith he saith, In nothing was 
I inferior unto the high apostles: though I be nothing, yet 
the tokens of an apostle were wrought among you with all 
patience, with signs, and wonders, and mighty deeds. So 
proved he his authority, and not with a bull from Peter, 
sealed with cold lead, either with shadows of the Old Testament
falsley expounded. 

Moreover the apostles were sent immediately of Christ, 
and of Christ received they their authority, as Paul boasteth 
himself every where. Christ, saith he, sent me to preach 
the gospel. (1 Cor. i.) And I received of the Lord that 
which I delivered unto you. (1 Cor. xi. and Gal. i.) I certify
you, brethren, that the gospel which was preached of 
me was not after the manner of men, (that is to wit carnal
or fleshly) neither received I it of man, neither was 
it taught me, but I received it by the revelation of Jesus 
Christ. And (Gal ii.) He that was mighty in Peter in the 
apostleship over the circumcision, was mighty in me 
among the gentiles. And ist Tim. i. readest thou likewise.
And (John xx.) Christ sent them forth indifferently, 
and gave them like power: As my father sent me, saith he, 
so send I you; that is, to preach and to suffer as I have 
done; and not to conquer enemies and kingdoms, and to subdue
all temporal power under you with disguised hypocrisy. 
He gave them the Holy Ghost to bind and loose indifferently, 
as thou seest; and afterward he sent forth Paul with like 
authority, as thou seest in the Acts; and in the last of Matthew
saith he, All power is given me in heaven and in 
earth, go, therefore, and teach all nations, baptizing them in 
the name of the Father, and of the Son, and of the Holy 
Ghost, teaching them to observe whatsoever I commanded 
you. The authority that Christ gave them was to preach; 
yet not what they would imagine, but what he had commanded.
Lo, saith he, I am with you always, even unto 
the end of the world. He said not, I go my way, and lo here 
is Peter in my stead: but sent them every man to a sundry 
country, whithersoever the Spirit carried them, and went 
with them himself. And as he wrought with Peter where 
he went, so wrought he with the other where they went, 
as Paul boasteth of himself unto the Galatians. Seeing now 
that we have Christ's doctrine, and Christ's holy promises, 
and seeing that Christ is ever present with us his own self, 
how cometh it that Christ may not reign immediately over 
us as well as the pope which cometh never at us? seeing 
also that the office of an apostle is to preach only, how can 
the pope challenge with right any authority where he 
preacheth not? How cometh it also that Rochester will 
not let us be called one congregation, by the reason of one 
God, one Christ, one Spirit, one gospel, one faith, one 
hope, and one baptism, as well as because of one pope? 

If any natural beast with his wordly wisdom strive that 
one is greater than another, because that in congregations 
one is sent of another, as we see in the Acts; I answer 
that Peter sent no man, but was sent himself; and John 
was sent, and Paul, Silas, and Barnabas were sent.
Howbeit such manner [of] sendings are not worldly, as 
princes send ambassadors; no, nor as friars send their limiters
to gather their brotherhoods; which must obey 
whether they will or will not. Here all thing is free and willingly:
and the Holy Ghost bringeth them together, which 
maketh their wills free, and ready to bestow themselves upon
their neighbour's profit: and they that come offer themselves,
and all that they have or can do to serve the Lord 
and their brethren: and every man as he is found apt and 
meet to serve his neighbour, so is he sent or put in office. 
And of the Holy Ghost are they sent with the consent of 
their brethren and with their own consent also; and God's 
word ruleth in that congregation unto which word every 
man confirmeth his will: and Christ which is always present 
is the head. But as our bishops hear not Christ's voice 
so see they him not present, and therefore make them a 
God on the earth, of the kind I suppose of Aaron's calf, 
for he bringeth forth no other fruit but bulls. 

Forasmuch also as Christ is as great as Peter, why is 
not his seat as great as Peter's? Had the head of the empire
been at Jerusalem, there had been no mention made of 
Peter. It is verily, as Paul saith in the xith chapter of the 
iind Epistle to the Corinthians, The false apostles are deceitful
workers, and fashion themselves like unto the 
apostles of Christ; that is the shaven nation preach Christ 
falsely; yea, under the name of Christ preach themselves,
and reign in Christ's stead: have also taken away 
the key of knowledge and wrapped the people in ignorance, 
and have taught them to believe in themselves, in their traditions
and false ceremonies; so that Christ is but a vain
name, and after they had put Christ out of his room they
got themselves to the emperor and kings, and so long ministered
their business till they have also put them out of 
their rooms, and have got their authorities from them and
reign also in their stead; so that the emperor and kings are 
but vain names and shadows, as Christ is, having nothing to 
do in the world: thus reign they in the stead of God and 
man, and have all power under them, and do what they 
list.

Let us see another point of our great clerk; a little after
the beginning of his Sermon, intending to prove that which
is clearer than the sun, and serveth no more for his purpose
than Ite missa est serveth to prove that our lady was 
born without original sin; he allegeth a saying that Martin 
Luther saith, which is this, If we affirm that any one 
Epistle of Paul or any one place of his Epistles pertaineth 
not unto the universal church, (that is, to all the congregation
of them that believe in Christ,) we take away all St. 
Paul's authority. Whereupon saith Rochester, if it be thus 
of the words of St. Paul, much rather it is true of the gospels
of Christ and of every place of them. O malicious 
blindness! First, note his blindness. He understandeth 
by this word gospel, no more but the four Evangelists, 
Matthew, Mark, Luke, and John; and thinketh not that 
the Acts of Apostles, and the Epistles of Peter, of Paul,
and of John, and of other like, are also the gospel. Paul
calleth his preaching the gospel: (Rom. ii. and 1 Cor. iv.
and Gal. i. and 1 Tim. i.) The gospel is every where 
one, though it be preached of divers, and signifieth glad
tidings: that is to wit, an open preaching of Christ and
the holy Testament, and gracious promises that God hath 
made in Christ's blood, to all that repent and believe. 
Now, is there more gospel in one Epistle of Paul, that is to 
say, Christ is more clearly preached, and more promises 
rehearsed in one Epistle of Paul, than in ihe three first 
Evangelists, Matthew, Mark, and Luke. 

Consider also his maliciousness; how wickedly and how 
craftily he taketh away the authority of Paul! It is much 
rather true of the gospels, and of every place in them, than 
of Paul. If that which the four Evangelists wrote be 
truer than that which Paul wrote, then is it not one gospel 
that they preached, neither one Spirit that taught them.
If it be one gospel and one Spirit, how is one truer than 
the other? Paul proveth his authority to the Galatians 
and to the Corinthians, because that he received his gospel 
by revelation of Christ, and not of man: and because that 
when he communed with Peter and the high apostles of 
his gospel and preaching, they could improve nothing, 
neither teach him any thing: and because, also, that as 
many were converted, and as great miracles shewed by his 
preaching as at the preaching of the high apostles, and 
therefore will be of no less authority than Peter and other 
high apostles: nor have his gospel of less reputation 
than their's. 

Finally: that thou mayest know Rochester for ever, 
and all the remnant by him, what they are within the 
skin, mark how he playeth bo-peep with the Scripture. 
He allegeth the beginning of the tenth chapter to the 
Hebrews. Umbram habens lex futurorum bonorum, 
the law hath but a shadow of things to come. And immediately
expoundeth the figure clean contrary unto the 
chapter following, and to all the whole Epistle; making 
Aaron a figure of the Pope, whom the Epistle maketh a 
figure of Christ. 

He allegeth half a text of Paul, (1 Tim. iv.) In the 
latter days some shall depart from the faith, giving heed 
unto spirits of error and devilish doctrine: but it followeth 
in the text, Giving attendance, or heed, unto the devilish 
doctrine of them which speak false, through hypocrisy, 
and have their consciences marked with a hot iron, forbidding
to marry, and commanding to abstain from meats, 
which God hath created to be received with giving thanks. 
Which two things whoever did, save the pope, Rochester's
God? making sin in the creatures which God hath 
created for man's use, to be received with thanks. The 
kingdom of heaven is not meat and drink, saith Paul, but 
righteousness, peace, and joy in the Holy Ghost. For 
whosoever in these things serveth Christ, pleaseth God, 
and is allowed of men. (Rom. xiv.) Had Rochester, 
therefore, not a conscience marked with the hot iron of 
malice, so that he cannot consent unto the will of God 
and glory of Christ, he would not so have alleged the 
text, which is contrary to none save themselves. 

He allegeth another text of Paul, in the second chapter 
of his second Epistle to the Thessalonians. Erit dissessio 
primum; that is, saith Rochester, before the coming of 
Antichrist there shall be a notable departing from the 
faith. And Paul saith, The Lord cometh not, except 
there come a departing first. Paul's meaning is, that the 
last day cometh not so shortly, but that Antichrist shall 
come first and destroy the faith, and sit in the temple of 
God, and make all men worship him, and believe in him 
(as the Pope doth); and then shall God's word come to 
light again, (as it doth at this time,) and destroy him, and 
utter his juggling, and then cometh Christ unto judgment.
What say ye of this crafty conveyer? Would he 
spare, suppose ye, to allege and to wrest other doctors 
pestilently, which feareth not for to juggle with the holy 
Scripture of God, expounding that unto Antichrist which 
Paul speaketh of Christ? No, be ye sure. But even 
after this manner-wise pervert they the whole Scripture 
and all doctors, wresting them unto their abominable 
purpose, clean contrary to the meaning of the text, and to 
the circumstances that go before and after. Which devilish
falsehood, lest the laymen should perceive, is the 
very cause why that they will not suffer the Scripture to be
had in the English tongue; neither any work to be made
that should bring the people to knowledge of the 
truth. 

He allegeth, for the pope's authority, St. Cyprian, 
St. Augustine, Ambrose, Jerom, and Origen; of which 
never one knew of any authority that one bishop should 
have above another. And St. Gregory, allegeth he, which 
would receive no such authority above his brethren when it 
was proffered him. As the manner is to call Tully chief 
of orators, for his singular eloquence, and Aristotle chief 
of philosophers, and Virgil chief of poets, for their singular
learning; and not for any authority that they had 
over other: so was it the manner to call Peter chief of the 
apostles, for his singular activity and boldness; and not 
that he should be lord over his brethren, contrary to his 
own doctrine. Yet compare that chief apostle unto Paul, 
and he is found a great way inferior. This I say not that 
I would that any man should make a God of Paul, contrary
unto his own learning. Notwithstanding, yet this 
manner of speaking is left unto us of our elders; that 
when we say the apostle saith so, we understand Paul, for 
his excellency above other apostles. I would he would 
tell you how Jerome, Augustine, Bede, Origen, and other 
doctors, expound this text, Upon this rock I will build my 
congregation: and how they interpret the keys also. 
Thereto, Pasce, pasce, pasce, which Rochester leaveth 
without any English, signifieth not poll, sheer, and shave. 
Upon which text behold the faithful Exposition of Bede. 

Note also how craftily he would enfeoff the apostles of 
Christ with their wicked traditions and false ceremonies, 
which they themselves have feigned; alleging Paul, 
2 Thess. ii. I answer, that Paul taught by mouth such 
things as he wrote in his Epistles. And his traditions 
were the gospel of Christ, and honest manners and living, 
and such a good order as becometh the doctrine of Christ. 
As that a woman obey her husband, have her head covered, 
keep silence, and go womanly and Christianly apparelled; 
that children and servants be in subjection; and that the 
young obey their elders, that no man eat but he that laboureth,
and worketh; and that men make an earnest 
thing of God's word, and of his holy Sacraments; and 
to watch, fast, and pray, and such like as the Scripture 
commandeth: which things, he that would break, were 
no Christian man. But, we may well complain, and 
cry to God for help, that it is not lawful for the pope's 
tyranny, to teach the people what prayer is, what fasting 
is, and wherefore it serveth. There were also certain 
customs alway, which were not commanded in pain of 
hell, or everlasting damnation: as to watch all night, 
and to kiss one another: which as soon as the people 
abused, then they brake them. For which cause, the 
bishops might break many things now in like manner. 
Paul also, in many things which God had made free, gave 
pure and faithful counsel, without tangling of any man's 
conscience, and without all manner [of] commanding, under 
pain of cursing, pain of excommunication, pain of heresy, 
pain of burning, pain of deadly sin, pain of hell, and 
pain of damnation. As thou mayest see, 1 Cor. vii., 
where he counselleth the unmarried, the widows, and virgins,
that it is good so to abide, if they have the gift of 
chastity. Not to win heaven thereby, (for neither circumcision,
neither uncircumcision, is any thing at all; but the 
keeping of the commandments is altogether.) But that 
they might be without trouble, and might also the better 
wait on God's word, and freelier serve their brethren. And 
saith, (as a faithful servant) that he had none authority of 
the Lord, to give them any commandment. But, that the 
apostles gave us any blind ceremonies, whereof we should 
not know the reason, that I deny, and also defy, as a 
thing clean contrary unto the learning of Paul, everywhere. 

For Paul commandeth that no man once speak in the 
church (that is, in the congregation,) but in a tongue that all 
men understand, except that there be an interpreter by: 
he commandeth to labour for knowledge, understanding,
and feeling, and to beware of superstition, and 
persuasions of worldly wisdom, philosophy, and of hypocrisy
and ceremonies, and of all manner [of] disguising, and 
to walk in the plain and open truth. Ye were once darkness, 
(saith he,) but now are ye light in the Lord; walk, therefore,
as the children of light. (Eph. v.) How doth Paul, 
also, wish them increase of grace in every Epistle! How 
crieth he to God to augment their knowledge, that they 
should be no more children, wavering with every wind of 
doctrine, but would vouchsafe to make them full men in 
Christ, and in the understanding of the mysteries or 
secrets of Christ! So that it should not be possible for 
any man to deceive them with any enticing reasons of 
worldly wisdom, or to beguile them with blind ceremonies,
or to lead them out of the way with superstitiousness
of disguised hypocrisy. Unto which full knowledge 
are the spiritual officers ordained to bring them. (Eph. iv.) 
So far is it away that Christ's apostles should give them 
traditions of blind ceremonies, without signification, or of 
which no man should know the reason, as Rochester, 
which loveth shadows, and darkeneth, lieth on them: God 
stop his blasphemous mouth! 

Consider, also, how studiously Rochester allegeth 
Origen, both for his Pope, and also to stablish his blind 
ceremonies withal: which Origen, of all heretics, is condemned
to be the greatest. He is an ancient doctor, 
saith he; yea, and to whom, in this point great faith is to 
be given. Yea, verily, Aristotle and Plato, and even very 
Robinhood, is to be believed in such a point, that so 
greatly maintaineth our holy father's authority, and all his 
disguisings. 

Last of all: as once a crafty thief, when he was espied 
and followed, cried unto the people, Stop the thief! Stop 
the thief! And as many to begin withal cast first in another 
man's teeth that which he feared should be laid to his own 
charge; even so Rochester layeth to Martin Luther's 
charge the slaying and murdering of Christian men, 
because they will not believe in his doctrine; which thing 
Rochester and his brethren have not ceased to do now 
these certain hundred years, with such malice, that when 
they be dead, they rage, burning their bodies; of which 
some they themselves of likelihood killed before secretly. 
And because that all the world knoweth that Martin 
Luther slayeth no man, but killeth only with the spiritual 
sword, the word of God, such cankered consciences as 
Rochester hath; neither persecuteth, but suffereth persecution;
yet Rochester, with a goodly argument, proveth 
that he would do it if he could! And mark, I pray you, 
what an orator he is, and how vehemently he persuadeth 
it! Martin Luther hath burned the Pope's decretals; a 
manifest sign, saith he, that he would have burned the 
pope's holiness also, if he had had him. A like argument
(which I suppose to be rather true,) I make: Rochester
and his holy brethren, have burnt Christ's Testament;
an evident sign, verily, that they would have burnt 
Christ himself; also, if they had had him! 

I had almost, verily, left out the chiefest point of all. 
Rochester, both abominable and shameless, yea, and stark 
mad with pure malice, and so adased in the brains with 
spite, that he cannot overcome the truth that he seeth not, 
or rather careth not what he saith; in the end of his first 
destruction, I would say instruction, as he calleth it; intending
to prove that we are justified through holy works, 
allegeth half a text of Paul of the fifth to the Galatians, 
(as his manner is to juggle and convey craftily,) fides per 
dilectionem operans. Which text he thiswise Englisheth: 
Faith, which is wrought by love: and maketh a verb passive
of a verb deponent. Rochester will have love to go 
before, and faith to spring out of love. Thus Antichrist 
turneth the roots of the tree upward. I must first love a 
bitter medicine, (after Rochester's doctrine,) and then 
believe that it is wholesome. When, by natural reason, I 
first hate a bitter medicine, until I be brought in belief of 
the physician that it is wholesome, and that the bitterness 
shall heal me, and then afterward love it of that belief. 

Doth the child love the father first, and then believe that 
he is his son or heir? or rather, because be knoweth that 
he is his son or heir and beloved, therefore loveth again? 
John saith, in the third of his first Epistle, See what love 
the Father hath shewed upon us, that we should be called
his sons. Because we are sons, therefore love we. Now, 
by faith are we sons, as John saith in the first chapter of 
his gospel. He gave them power to be the sons of God, 
in that they believed on his name. And Paul saith, in 
the third chapter of his Epistle to the Galatians, We are 
all the sons of God, by the faith which is in Jesus Christ. 
And John, in the said chapter of his Epistle, saith, Hereby 
perceive we love, that he gave his life for us. We could 
see no love, nor cause to love again, except that we believed
that he died for us, and that we were saved through 
his death. And in the chapter following saith John. 
Herein is love: not that we loved God, but that he loved 
us, and sent his Son to make agreement for our sins. So 
God sent not his Son for any love that we had to him; 
but of the love that he had to us sent he his Son, that we 
might so love, and love again. Paul likewise, in the 
viiith chapter to the Romans, after that he hath declared 
the infinite love of God to usward, in that he spared not 
his own Son, but gave him for us, crieth out, saying, Who 
shall separate us from the love of God? Shall persecution,
shall a sword? \&c. No, saith he; I am sure that 
no creature shall separate us from the love of God 
that is in Christ Jesus our Lord: as who should say, 
We see so great love in God to usward in Christ's 
death, that though all misfortune should fall on us, we
cannot but love again. Now how know we that God 
loveth us? Verily by faith. So therefore, though Rochester
be a beast faithless, yet ought natural reason to
have taught him, that love springeth out of faith and 
knowledge; and not faith and knowledge out of love. 
But let us see the text. Paul saith thus: In Christ Jesus 
neither circumcision is any thing worth, nor uncircumcision, 
but faith which worketh through love; or which through 
love is strong or mighty in working, and not which is 
wrought by love, as the juggler saith. Faith that loveth 
God's commandments justifieth a man. If thou believe 
God's promises in Christ, and love his commandments, 
then art thou safe. If thou love the commandment, then 
art thou sure that thy faith is unfeigned, and that God's 
Spirit is in thee. 

How faith justifieth before God in the heart, and how 
love springeth of faith, and compelleth us to work, and 
how the works justify before the world, and testify what 
we are, and certify us that our faith is unfeigned, and that 
the right Spirit of God is in us, see in my book of the Justifying
of Faith, and there shalt thou see all thing abundantly.
Also of the controversy between Paul and 
James see there. Neverthelater, when Rochester saith 
if faith only justified, then both the devils and also 
sinners that lie still in sin should be saved, his argument is 
not worth a straw. For neither the devils, nor yet sinners 
that continue in sin of purpose and delectation, have any 
such faith as Paul speaketh of. For Paul's faith is to 
believe God's promises. Faith (saith he Rom. x.) cometh 
by hearing, and hearing cometh by the word of God. 
And how shall they hear without a preacher, and how shall 
they preach except they be sent? As it is written (saith 
he) how beautiful are the feet that bring glad tidings of 
peace, and bring tidings of good things. Now when 
sent God any messengers unto the devils to preach them 
peace, or any good thing? The devil hath no promise; 
he is therefore excluded from Paul's faith. The devil 
believeth that Christ died, but not that he died for his sins. 
Neither doth any that consenteth in the heart to continue 
in sin, believe that Christ died for him. For to believe 
that Christ died for us, is to see our horrible damnation, 
and how we were appointed unto eternal pains, and to feel, 
and to be sure that we are delivered therefrom through 
Christ: in that we have power to hate our sins, and to 
love God's commandments. All such repent and have 
their hearts loosed out of captivity and bondage of sin, 
and are therefore justified through faith in Christ. Wicked 
sinners have no faith, but imaginations and opinions about 
Christ; as our schoolmen have in their principles, about 
which they brawl so fast one with another. It is another 
thing to believe that the king is rich, and that he is rich 
unto me, and that my part is therein: and that he will not 
spare a penny of his riches at my need. When I believe that 
the king is rich, I am not moved. But when I believe 
that he is rich for me, and that he will never fail me at my 
need, then love I, and of love am ready to work unto the 
uttermost of my power. But let us return at the last 
unto our purpose again. 

What is the cause that lay men cannot now rule, as 
well as in times past, and as the Turks yet do? Verily 
because that antichrist with the mist of his juggling hath 
beguiled our eyes, and hath cast a superstitious fear upon 
the world of Christian men, and hath taught them to 
dread not God and his word, but himself and his word; 
not God's law and ordinances, princes and officers which 
God hath set to rule the world, but his own law and ordinances,
traditions and ceremonies, and disguised disciples, 
which he hath set every where to deceive the world, and 
to expel the light of God's word that his darkness may 
have room. For we see by daily experience of certain 
hundred years long, that he which feareth neither God nor 
his word, neither regardeth father, mother, master, or 
Christ himself; which rebelleth against God's ordinances, 
riseth against the kings, and resisteth his officers, dare not 
once lay hands on one of the pope's anointed: no, though 
he slay his father before his face, or do violence unto his 
brother, or defile his sister, wife or mother. Like honour 
give we unto his traditions and ceremonies. What devotion
have we when we are blessed (as they call it) with the 
chalice, or when the bishop lifteth up his holy hand over 
us? Who dare handle the chalice, touch the altar stone, 
or put his hand in the font, or his finger into the holy 
oil? What reverence give we unto holy water, holy fire, 
holy bread, holy salt, hallowed bells, holy wax, holy 
boughs, holy candles, and holy ashes! And last of all unto 
the holy candle commit we our souls at our last departing. 
Yea, and of the very clout which the bishop or his chaplain
that standeth by, knitteth about children's necks at 
confirmation, what lay person dare be so bold as to unloose
the knot? Thou wilt say, do not such things bring 
the Holy Ghost and put away sin and drive away spirits? 
I say that a steadfast faith, or belief in Christ and in the 
promises that God hath sworn to give us for his sake, 
bringeth the Holy Ghost as all the Scriptures make mention,
and as Paul saith (Acts xix.) Have ye received the 
Holy Ghost through faith, or believing? Faith is the 
rock whereon Christ buildeth his congregation, against 
which saith Christ, (Matt. xvi.) hell gates shall not prevail. 
As soon as thou believest in Christ, the Holy Ghost 
cometh, sin falleth away, and devils fly. When we cast 
holy water at the devil, or ring the bells, he fleeth, as men 
do from young children, and mocketh with us, to bring 
us from the true faith that is in God's word, unto a superstitious,
and a false belief of our own imagination. If 
thou hadst faith and threwest an unhallowed stone at 
his head, he would earnestly flee, and without mocking; 
yea though thou threwest nothing at all, he would not yet 
abide. 

Though that at the beginning miracles were shewed 
through such ceremonies, to move the infidels to believe 
the word of God; as thou readest how the apostles 
anointed the sick with oil, and healed them, and Paul sent 
his pertelet or gyrkyn to the sick and healed them also; 
yet was it not the ceremony that did the miracle, but 
faith of the preacher and the truth of God, which had 
promised to confirm, and stablish his gospel with such 
miracles. Therefore as soon as the gift of miracles 
ceased, ought the ceremony to have ceased also; or else if 
they needs will have a ceremony to signify some promise 
or benefit of God (which I praise not, but would have 
God's Word preached every Sunday, for which intent Sundays
and holy days were ordained) then let them tell the 
people what it meaneth; and not set up a bald and a 
naked ceremony without signification, to make the people 
believe therein, and to quench the faith that ought to be 
given unto the word of God. 

What helpeth it also that the priest when he goeth to 
mass disguiseth himself with a great part of the passion 
of Christ, and playeth out the rest under silence, with 
signs and proffers, with nodding, becking and mowing, 
as it were jackanapes, when neither he himself, neither any 
man else wotteth what he meaneth? not at all verily, but 
hurteth, and that exceedingly. Forasmuch as it not only 
destroyeth the faith, and quencheth the love that should be 
given unto the commandments, and maketh the people 
unthankful, in that it bringeth them into such superstition, 
that they think that they have done abundantly enough for 
God, yea, and deserved above measure if they be present 
once in a day at such mumming: but also maketh the 
infidels to mock us and abhor us, in that they see nothing 
but such apes' play among us, whereof no man can give 
a reason. 

All this cometh to pass to fulfil the prophesy which 
Christ prophesied. (Mark xiii. and Luke xxi.) That 
there shall come in his name which shall say that they 
themselves are Christ. That do verily the Pope and our 
holy orders of religion. For they under the name of Christ, 
preach themselves, their own word and their own traditions,
and teach the people to believe in them. The pope 
giveth pardons of his full power, of the treasure of the 
church, and of the merits of saints. The friars likewise 
make their benefactors (which only they call their brethren 
and sisters,) partakers of their masses, fasting, watchings, 
prayings, and woolward goings. Yea, and when a novice 
of the observants is professed, the father asketh him, 
Will ye keep the rules of holy St. Francis? and he saith 
Yea. Will ye so in deed? saith he; the other answereth, 
Yea, forsooth, father. Then saith the father, And I 
promise you again everlasting life. O, blasphemy! If 
eternal life be due unto the pilled traditions of lousy friars, 
where is the Testament become that God made unto us 
in Christ's blood? Christ saith, (Matt. xxiv. and Mark 
xiii.) That there shall come pseudo-Christi; which, 
though I for a consideration have translated false Christs, 
keeping the Greek word, yet signifieth it in the English, 
false anointed, and ought so to be translated. There 
shall come (saith Christ,) false anointed, and false prophets,
and shall do miracles and wonders; so greatly, that 
if it were possible, the very elect, or chosen, should be 
brought out of the way. Compare the pope's doctrine 
unto the word of God, and thou shalt find that there hath 
been, and yet is, a great going out of the way; and that 
evil men and deceivers (as Paul prophesied 2 Tim. iii.) 
have prevailed, and waxed worse and worse, beguiling 
other as they are beguiled themselves. Thou tremblest 
and quakest, saying, Shall God let us go so sore out of 
the right way? I answer, It is Christ that warneth us; 
which, as he knew all that should follow, so prophesied 
he before, and is a true prophet, and his prophecies must 
needs be fulfilled. 

God anointed his Son Jesus with the Holy Ghost, and 
therefore called him Christ; which is as much to say as 
anointed. Outwardly he disguised him not, but made 
him like other men, and sent him into the world to bless 
us, and to offer himself for us a sacrifice of a sweet savour, 
to kill the stench of our sins, that God henceforth should 
smell them no more, nor think on them any more; and to 
make full and sufficient satisfaction, or amends, for all 
them that repent, believing the truth of God, and submitting
themselves unto his ordinances, both for the sins 
that they do, have done, and shall do. For sin we through 
fragility never so oft, yet as soon as we repent and come 
into the right way again, and unto the Testament which 
God hath made in Christ's blood, our sins vanish away 
as smoke in the wind, and as darkness at the coming of 
light; or as thou castest a little blood or milk into the main 
sea. Insomuch, that whosoever goeth about to make 
satisfaction for his sins to Godward, saying in his heart, 
This much have I sinned, this much will I do again; or 
this wise will I live to make amends withal; or this will I 
do to get heaven withal; the same is an infidel, faithless, 
and damned in his deed-doing, and hath lost his part in 
Christ's blood: because he is disobedient unto God's 
Testament, and setteth up another of his own imagination,
unto which he will compel God to obey. If we 
love God, we have a commandment to love our neighbour 
also, as saith John in his Epistle. And if we have offended 
him, to make him amends; or if we have not wherewith, 
to ask him forgiveness, and to do and suffer all things for 
his sake; to win him to God, and to nourish peace and 
unity: but to Godward, Christ is an everlasting satisfaction,
and ever sufficient. 

Christ when he had fulfilled his course, anointed his 
apostles and disciples with the same spirit, and sent them 
forth, without all manner [of] disguising, like other men 
also, to preach the atonement and peace which Christ had 
made betwen God and man. The apostles likewise disguised
no man, but chose men anointed with the same 
spirit: one to preach the word of God, whom we call 
after the Greek tongue, a bishop or a priest; that is, in 
English, an overseer and an elder. How he was anointed, 
thou readest, (1 Tim. iii.) A bishop or an overseer must 
be faultless, the husband of one wife. (Many Jews, and 
also Gentiles, that were converted unto the faith, had at 
that time divers wives, yet were not compelled to put any 
of them away; which Paul, because of ensample, would 
not have preachers, forasmuch as in Christ we return 
again unto the first ordinance of God, that one man and 
one woman should go together) he must be sober, of 
honest behaviour, honestly apparelled, harborous; that is, 
ready to lodge strangers; apt to teach, no drunkard, 
no fighter, not given to filthy lucre; but gentle, abhorring
fighting, abhorring covetousness, and one that 
ruleth his own household honestly, having children under 
obedience, with all honesty. For if a man cannot rule his 
own house, how can he care for the congregation of God? 
He may not be young in the faith: or as a man would 
say, a novice, lest he swell and fall into the judgment of 
the evil speaker; that is, he may not be unlearned in the 
secrets of the faith. For such are at once stubborn, and 
headstrong, and set not a little by themselves. But alas, 
we have above twenty thousand that know no more Scripture
than is written in their portesses, and among them is 
he exceeding well learned that can turn to his service. 
He must be well reported of them that are without, lest 
he fall into rebuke, and into the snare of the evil speaker; 
that is, lest the iniidels which yet believe not, should be 
hurt by him, and driven from the faith, if a man that were 
defamed were made head or overseer of the congregation. 

He must have a wife for two causes; one, that it may 
thereby be known who is meet for the room. He is unapt 
for so chargeable an office, which had never household to 
rule. Another cause is, that chastity is an exceeding 
seldom gift, and unchastity exceeding perilous for that 
degree. Inasmuch as the people look as well unto the 
living as unto the preaching, and are hurt at once, if the 
living disagree, and fall from the faith, and believe not 
the word. 

This overseer, because he was taken from his own 
business and labour, to preach God's word unto the 
parish, hath right by the authority of his office, to challenge
an honest living of the parish, as thou mayest see 
in the Evangelists, and also in Paul. For who will have 
a servant, and will not give him meat, drink, and raiment, 
and all things necessary? How they would pay him, 
whether in money, or assign him so much rent, or in 
tithes, as the guise is now in many countries, was at their 
liberty. 

Likewise in every congregation chose they another after 
the same ensample, and even so anointed, as it is to see 
in the said chapter of Paul, and Acts vi. Whom after 
the Greek word we call deacon; that is to say in English, 
a servant or a minister, whose office was to help and assist
the priest, and to gather up his duty, and to gather for 
the poor of the parish, which were destitute of friends, 
and could not work: common beggars to run from door 
to door, were not then suffered. On the saints' days, 
namely, such as had suffered death for the word sake, 
came men together into the church, and the priest preached 
unto them, and exhorted them to cleave fast unto the word, 
and to be strong in the faith, and to fight against the 
powers of the world, with suffering for their faith's sake, 
after the ensample of the saints. And taught them not to 
believe in the saints, and to trust in their merits, and to 
make gods of them: but took the saints for an ensample 
only, and prayed God to give them like faith and trust 
in his word, and like strength and power to suffer therefore,
and to give them so sure hope of the life to come, as 
thou mayest see in the collects of St. Lawrence and of St. 
Stephen, in our lady matins. And in such days, as we 
now offer, so gave they every man his portion according 
to his ability; and as God put in his heart, to the maintenance
of the priest, deacon, and other common ministers,
and of the poor, and to find learned men to teach, 
and so forth. And all was put in the hands of the deacon, 
as thou mayest see in the life of St. Lawrence, and in the 
histories. And for such purposes gave men lands afterwards,
to ease the parishes; and made hospitals, and also 
places to teach their children, and to bring them up, and 
to nurture them in God's word; which lands our monks 
now devour. 


ANTICHRIST.

ANTICHRIST of another manner hath sent forth his 
disciples, those false anointed of which Christ warneth 
us before, that they should come and show miracles and 
wonders, even to bring the very elect out of the way, if it 
were possible. He anointeth them after the manner of 
the Jews, and shaveth them and sheareth them after the 
manner of the heathen priests, which serve the idols. He 
sendeth them forth not with false oil only, but with false 
names also. For compare their names unto their deeds, 
and thou shalt find them false. He sendeth them forth, 
as Paul prophesied of them, (2 Thess. ii.) with lying signs 
and wonders. What sign is the anointing? that they 
be full of the Holy Ghost. Compare them to the signs 
of the Holy Ghost which Paul reckoneth, and thou shalt 
find it a false sign. A bishop must he faultless, the husband
of one wife. Nay, saith the pope, the husband of 
no wife, but the holder of as many whores as he listeth. 
God commandeth all degrees, if they burn, and cannot live 
chaste, to marry. The pope saith, If thou burn, take a 
dispensation for a concubine, and put her away when thou 
art old, or else, as our lawyers say, Si non caste, tamen 
caute; that is, If ye live not chaste, see ye carry clean, and 
play the knave secretly. Harbourous, yea to whores and 
bauds; for a poor man shall as soon break his neck as his 
fast with them, but of the scraps and with the dogs, when 
dinner is done. Apt to teach, and as Peter saith, (1 Pet. ii.) 
Ready always to give an answer to every man that asketh 
you a reason of the hope that ye have, and that with meekness.
Which thing is signified by the boots which doctors 
of divinity are created in, because they should be ready 
always to go through thick and thin, to preach God's word, 
and by the bishop's two-horned mitre, which betokeneth 
the absolute and perfect knowledge that they ought to 
have in the new Testament and the old. Be not these 
false signs? For they beat only, and teach not. Yea 
saith the pope, If they will not be ruled, cite them to 
appear, and pose them sharply, what they hold of the 
pope's power, of his pardons, of his bulls, of purgatory, 
of ceremonies, of confession, and such like creatures of 
our most holy father's. If they miss in any point, make 
heretics of them, and burn them. If they be of mine anointed,
and bear my mark, disgrace them, I would say disgraduate
them, and (after the ensample of noble Antiochus, 
2 Mach. vii.) pare the crowns and the fingers of them, 
and torment them craftily, and for very pain make them 
deny the truth. But now say our bishops, because the 
truth is come too far abroad, and the lay people begin to 
smell our wills, it is best to oppress them with craft secretly,
and to tame them in prison. Yea, let us find the 
means to have them in the king's prison, and to make 
treason of such doctrine: yea, we must stir up some war 
one where or another, to bring the people into another 
imagination. If they be gentlemen, abjure them secretly. 

Curse them four times in the year. Make them afraid 
of every thing; and namely, to touch mine anointed; and 
make them to fear the sentence of the church, suspensions, 
excommunications and curses. Be they right or wrong, 
bear them in hand that they are to be feared yet. Preach 
me and mine authority, and how terrible a thing my curse 
is, and how black it maketh their souls. On the holidays, 
which were ordained to preach God's word, set up long 
ceremonies, long matins, long masses, and long evensongs,
and all in Latin, that they understand not: and 
roll them in darkness, that ye may lead them whither 
ye will. And lest such things should be too tedious, sing 
some, say some, pipe some, ring the bells, and lull them 
and rock them asleep. And yet Paul (2 Cor. xiv.) forbiddeth
to speak in the church or congregation, save in 
the tongue that all understand. For the layman thereby 
is not edified or taught. How shall the layman say Amen, 
(saith Paul) to thy blessing or thanksgiving, when he 
wotteth not what thou sayest? He wotteth not whether 
thou bless or curse. 

What then saith the pope, What care I for Paul? I 
command by the virtue of obedience, to read the gospel 
in Latin, let them not pray but in Latin, no not their 
Pater noster. If any be sick, go also and say them 
a gospel, and all in Latin: yea, to the very corn and 
fruits of the field, in the procession week, preach the 
gospel in Latin, Make the people believe, that it shall 
grow the better. It is verily as good to preach it to swine 
as to men, if thou preach it in a tongue they understand 
not. How shall I prepare myself to God's commandments?
How shall I be thankful to Christ for his kindness?
How shall I believe the truth and promises which 
God hath sworn, while thou tellest them unto me in a 
tongue which I understand not? 

What then saith my lord of Canterbury, to a priest 
that would have had the New Testament gone forth in 
English: What (saith he) wouldest thou that the lay- 
people should wete what we do? 

No fighter, which I suppose is signified by the cross 
that is borne before the high prelates, and borne before 
them in procession: Is that also not a false sign? What 
realm can be in peace for such turmoilers? What so 
little a parish is it, but they will pick one quarrel or 
another with them, either for some surplus, chrism or 
mortuary, either for one trifle or other, and cite them to 
the arches? Traitors they are to all creatures, and have 
a secret conspiration between themselves. One craft they 
have, to make many kingdoms, and small, and to nourish 
old titles or quarrels, that they may ever move them to 
war at their pleasure. And if much lands, by any chance, 
fall to one man, ever to cast a bone in the way, that he 
shall never be able to obtain it, as we now see in the 
emperor. Why? For as long as the kings be small, if 
God would open the eyes of any to set a reformation in 
his realm, then should the Pope interdict his land, and 
send in other princes to conquer it. 

Not given to filthy lucre, but abhorring covetousness. 
And as Peter saith, (1 Pet. v.) Taking the oversight of 
them, not as though ye were compelled thereunto; but 
willingly. Not for desire of filthy lucre, but of a good 
mind: not as though ye were lords over the parishes, (over 
the parishes, quoth he) O Peter, Peter, thou wast too 
long a fisher, thou wast never brought up at the arches, 
neither wast master of the Rolls, nor yet Chancellor of 
England. They are not content to reign over king and 
emperor, and the whole earth: but challenge authority 
also in heaven and in hell. It is not enough for them to 
reign over all that are quick, but have created them a 
a purgatory, to reign also over the dead, and to have one 
kingdom more than God himself hath. But that ye be 
an ensample to the flock; (saith Peter) and when the 
chief Shepherd shall appear, ye shall receive an incorruptible
crown of glory. This abhorring of covetousness 
is signified, as I suppose, by shaving and shearing of the 
hair, that they have no superfluity. But is not this also 
a false sign? Yea, verily, it is to them a remembrance 
to shear and shave, to heap benefice upon benefice, promotion
upon promotion, dignity upon dignity, bishoprick 
upon bishoprick, with pluralities, unions and tot quots. 

First, by the authority of the gospel, they that preach 
the word of God in every parish, and other necessary 
ministers, have right to challenge an honest living like 
unto one of the brethren, and therewith ought to be content.
Bishops and priests that preach not, or that preach 
ought save God's word, are none of Christ's, nor of his 
anointing: but servants of the beast, whose mark they 
bear, whose word they preach, whose law they maintain 
clean against God's law, and with their false sophistry give 
him greater power than God ever gave to his son Christ. 

But they, as unsatiable beasts, not unmindful why they 
were shaven and shorn, because they will stand at no man's 
grace, or be in any man's danger, have gotten into their 
own hands, first the tithe or tenth of all the realm. Then 
I suppose within a little, or altogether the third foot of 
all the temporal lands. 

Mark well how many parsonages or vicarages are there 
in the realm, which at the least have a plow-land a-piece. 
Then note the lands of bishops, abbots, priors, nuns, 
knights of St. John's, cathedral churches, colleges, 
chauntries and free-chapels. For though the house fall 
in decay, and the ordinance of the founder be lost, yet 
will not they loose the lands. What cometh once in, may 
never more out. They make a free-chapel of it, so that 
he which enjoyeth it shall do nought therefore. Besides 
all this, how many chaplains do gentlemen find at their 
own cost in their houses? How many sing for souls by 
testaments? Then the proving of testaments, the prizing 
of goods, the bishop of Canterbury's prerogative. Is that 
not much through the realm in a year? Four offering 
days and privy tithes. There is no servant, but that he 
shall pay somewhat of his wages. None shall receive 
the body of Christ at Easter, be he never so poor a 
beggar, or never so young a lad or maid, but they must 
pay somewhat for it. Then mortuaries for forgotten tithes 
(as they say). And yet what parson or vicar is there that 
will forget to have a pigeon-house, to peck up somewhat
both at sowing-time and harvest, when corn is ripe? 
They will forget nothing. No man shall die in their 
debt; or if any man do, he shall pay it when he is dead. 
They will lose nothing. Why? It is God's; it is not 
theirs. It is St. Hubert's rents, St. Alban's lands, St. 
Edmond's right, St. Peter's patrimony, say they, and none 
of ours. Item if a man die in another man's parish, besides
that he must pay at home a mortuary for forgotten 
tithes, he must there pay also the best that he there hath. 
Whether it be an horse of twenty pound, or how good 
soever he be: either a chain of gold of an hundred 
marks, or five hundred pounds, if it so chance. It is 
much, verily, for so little pains taken in confession, and 
in ministering the sacraments. Then bead-rolls. Item 
christenings, churchings, banns, weddings, offering at weddings,
offering at buryings, offering to images, offering of wax 
and lights, which come to their damage; besides the superstitious
waste of wax in torches and tapers throughout 
the land. Then brotherhoods and pardoners. What get 
they also by confessions? Yea, and many enjoin penance, 
to give a certain [sum] for to have so many masses said, and 
desire to provide a chaplain themselves. Soul-masses, 
dirges, month-minds, peace-minds, All-souls-duy and 
trentals. The mother church and the high altar must 
have somewhat in every testament. Offerings at priests' 
first masses. Item no man is professed, of whatsoever 
religion it be; but he must bring somewhat. The hallowing,
or rather conjuring of churches, chapels, altars, 
super-altars, chalice, vestments and bells. Then book, 
bell, candlestick, organs, chalice, vestments, copes, altar- 
cloths, surplices; towels, basins, ewers, sheep, censer, 
and all manner [of] ornaments, must be found them freely, 
they will not give a mite thereunto. Last of all, what 
swarms of begging friars are there! The parson sheareth, 
the vicar shaveth, the parish priest polleth, the friar 
scrapeth, and the pardoner pareth; we lack but a butcher 
to pull off the skin. 

What get they in their spiritual law as (they call it) in a 
year, at the arches and in every diocese? what get the commissaries,
and officials with their somners and apparitors 
by bawdery in a year? shall ye not find curates enough 
which to flatter the commissaries and officials withal, that 
they may go quit themselves, shall open unto them the 
confessions of the richest of their parishes: whom they 
cite privily, and lay to their charges secretly. If they desire
to know their accusers, Nay, say they, the matter is 
known well enough, and to more than ye are ware of. 

Come lay your hand on the book, if ye forswear yourself,
we shall bnng proofs, we will handle you, we will make
an ensample of you. Oh how terrible are they! Come, and 
swear, say they that you will be obedient unto our injunctions.
And by that craft wring they their purses, and 
make them drop as long as there is a penny in them. In 
three or four years shall they in those offices get enough to 
pay for a bishop's bull. What other thing are these in a 
realm save horse-leeches, and even very maggots, cankers, 
and caterpillars, which devour no more but all that is green; 
and those wolves which Paul prophesied should come and 
should not spare the flock. (Acts xxth chapter.) And 
which Christ said should come in lamb's skins, and bade us 
beware of them and judge them by their works. 

Though as I have before sufficiently proved, a christian
man must suffer all things, be it never so great unright,
as long as it is not against God's commandment, neither is
it lawful for him to cast any burden off his back by his
own authority, till God pull it off, which laid it on for our
deservings, yet ought the kings everywhere to defend their 
realms from such oppression, if they were christians, which 
is seldom seen: and is an hard thing, verily, though not impossible.
For alas! they be captives or ever they be kings, 
yea almost ere they be born. No man may be suffered 
about him but flatterers, and such as are first sworn true 
unto our most holy fathers the bishops, that is to say, false 
to God and man. 

If any of the nobles of the realm be true to the king, 
and so bold that he dare counsel him that which should be 
to his honour and for the wealth of the realm; they will 
wait a season for him as men say, they will provide a 
ghostly father for him. God bring their wickedness to 
light. There is no mischief whereof they are not the root, 
nor bloodshed but through their cause either by their counsel,
or in that they preach not true obedience, and teach 
not the people to fear God. If any faithful servant be in 
all the court, he shall have twenty spies waiting upon him, 
he shall be cast out of the court, or, as the saying is, conveyed
to Calais, and made a captain or an ambassador, he 
shall be kept far enough from the king's presence. 

The kings ought, I say, to remember that they are in 
God's stead, and ordained of God, not for themselves, but 
for the wealth of their subjects. Let them remember that 
their subjects are their brethren, their flesh and blood, 
members of their own body, and even their ownselves in 
Christ. Therefore ought they to pity them and to rid 
them from such vile tyranny which increaseth more and 
more daily. And though that the kings by the falsehood
of the bishops and abbots be sworn to defend such liberties:
yet ought they not to keep their oaths, but to 
break them. Forasmuch as they are unright and clean 
against God's ordinance, and even but cruel oppression, 
contrary unto brotherly love and charity. Moreover the 
spiritual officer ought to punish no sin, but and if any sin 
break out, the king is ordained to punish it, and they not: 
but to preach and exhort them to fear God, and that they 
sin not. 

And let the kings put down some of their tyranny, and 
turn some unto a commonwealth. If the tenth part of 
such tyranny were given the king yearly, and laid up in the 
shire towns against the realm had need, what would it grow 
to in certain years? Moreover one king one law, is God's 
ordinance in every realm. Therefore ought not the king to 
suffer them to have a several law by themselves, and to 
draw his subjects thither. It is not mete, will they say 
that a spiritual man should be judged by a worldly or temporal
man. O abomination, see how they divide and separate
themselves: if the lay men be of the world, so is he 
not of God! If he believe in Christ, then is he a member 
of Christ, Christ's brother, Christ's flesh, Christ's blood, 
Christ's spouse, coheir with Christ, and hath his Spirit 
in earnest, and is also spiritual. If they would rob us of 
the Spirit of God, why should they fear to rob us of 
worldly goods? because thou art put in office to preach 
God's word, art thou therefore no more one of the 
brethren? is the major of London no more one of the city 
because he is the chief officer? is the king no more of the 
realm because he is head thereof? The king is in the room 
of God, and his law is God's law, and nothing but the law 
of nature and natural equity, which God graved in the 
hearts of men. Yet antichrist is too good to be judged 
by the law of God, he must have a new of his own making. 
It were meet verily that they went to no law at all. No 
more needed they, if they would study to preach God's word 
truly, and be contented with sufficient, and to be like one 
of their brethren. 

If any question arose about the faith of the Scripture, that 
let them judge by the manifest and open Scriptures, not 
excluding the lay men. For there are many found among 
the lay men, which are as wise as the officers. Or else 
when the officer dieth, how could we put another in his 
room? wilt thou so teach twenty, thirty, forty, or fifty 
years, that no man shall have knowledge or judgment in 
God's word save thou only? is it not a shame that we 
Christians come so oft to church in vain, when he of four- 
score years old knoweth no more than he that was born 
yesterday? 

Moreover, when the spiritual officers have excommunicated
any man, or have condemned any opinion for 
heresy: let not the king nor temporal officers punish and 
slay by and by at their commandment. But let them look 
on God's word and compare then judgment unto the Scripture,
and see whether it be right or no, and not believe 
them at the first chop, whatsoever they say, namely, in 
things that pertain unto their own authorities and power. 
For no man is a right judge in his own cause. Why doth 
Christ command the Scripture to be preached unto all 
creatures, but that it pertaineth unto all men to know 
them? Christ referreth himself unto the Scriptures; (John 
v.) and in the xith chapter of Matthew, unto the question 
of John Baptist's disciples, he answered, The blind see, the 
lepers are cleansed, the dead arise again, \&c. meaning that 
if I do the works which are prophesied that Christ should 
do when he cometh, why doubt ye whether I be he or no? 
as who should say, Ask the Scripture whether I be Christ or 
no, and not myself. How happeneth it then that our prelates
will not come to the light also, that we may see whether
their works be wrought in God or no? why fear they 
to let the lay men see what they do? why make they all their 
examinations in darkness? why examine they not their 
causes of heresy openly, as the lay men do their felons, and 
murderers? wherefore did Christ and his apostles also 
warn us so diligently of Antichrist and of false prophets 
that should come? because that we should slumber or 
sleep careless, or rather that we should look in the light of 
the Scripture with all diligence to spy them when they came, 
and not to suffer ourselves to be deceived and led out of the 
way? John biddeth judge the spirits. Whereby shall we 
judge them but by the Scriptures? how shalt thou know 
whether the prophet be true or false, or whether he speak 
God's word of his own head, if thou wilt not see the Scriptures?
why said David in the second Psalm, Be learned ye 
that judge the earth, lest the Lord be angry with you, and 
ye perish from the right way?

A terrible warning verily: yea, and look on the stories 
well, and thou shalt find very few kings since the beginning 
of the world that have not perished from the right way, and 
that because they would not be learned. 

The emperor and kings are nothing now a days but even 
hangmen unto the pope and bishops, to kill whosoever 
they condemn without any more ado, as Pilate was unto the 
Scribes and Pharisees and the high bishops to hang Christ. 
For as those prelates answered Pilate, (when be asked 
what he had done) if he were not an evil doer, we would 
not have brought him unto thee. As who should say, we 
are too holy to do any thing amiss, thou mayest believe us 
well enough: yea, and his blood on our heads said they, kill 
him hardly, we will bear the charge, our souls for thine: 
We have also a law by which he ought to die, for he calleth 
himself God's son. Even so say our prelates, he ought to 
die by our laws, he speaketh against the church. And 
your grace is sworn to defend the liberties and ordinances 
of the church and to maintain our most holy father's authority;
our souls for yours, ye shall do a meritorious deed 
therein. Nevertheless as Pilate escaped not the judgment of 
God, even so is it to be feared lest our temporal powers shall 
not. Wherefore be learned ye that judge the earth, lest the 
Lord be angry with you and ye perish from the right way. 

Who slew the prophets? who slew Christ? who slew his 
apostles? who the martyrs and all the righteous that ever 
were slain? The kings and the temporal sword at the request
of the false prophets. They deserved such murder 
to do, and to have their part with the hypocrites, because 
they would not be learned, and see the truth themselves. 
Wherefore suffered the prophets? because they rebuked 
the hypocrites which beguiled the world, and namely princes
and rulers, and taught them to put their trust in things 
of vanity, and not in God's word. And taught them to do 
such deeds of mercy as were profitable unto no man but 
unto the false prophets themselves only, making merchandize
of God's word. Wherefore slew they Christ? even 
for rebuking the hypocrites: because he said Woe be to 
you Scribes and Pharisees, hypocrites, for ye shut up the 
kingdom of heaven before men: (Matt. xxiii.) that is, as it 
is written Luke xi. Ye have taken away the key of knowledge.
The law of God which is the key wherewith men 
bind, and the promises which are the keys wherewith 
men loose, have our hypocrites also taken away. They 
will suffer no man to know God's word, but burn it and 
make heresy of it: yea, and because the people began to 
smell their falsehood, they make it treason to the king, and 
breaking of the king's peace to have so much as their paternoster
in English. And instead of God's law, they bind 
with their own law. And instead of God's promises, they 
loose and justify with pardons aud ceremonies, which 
they themselves have imagined for their own profit. They 
preach, It were better for thee to eat flesh on good Friday, 
than to hate thy neighbour; but let any man eat flesh but 
on a Saturday, or break any other tradition of theirs, and 
he shall be bound, and not loosed, till he have paid the 
uttermost farthing, either with shame most vile, or death 
most cruel: but hate thy neighbour as much as thou wilt, 
and thou shalt have no rebuke of them; yea, rob him, 
murder him, and then come to them and welcome. They 
have a sanctuary for thee; to save thee; yea, and a neck- 
verse, if thou canst but read a little Latin, though 
it be never so sorrily, so that you be ready to receive the 
beast's mark. They care for no understanding; it is 
enough if thou canst roll up a pair of matins, or an evensong,
and mumble a few ceremonies. And because they 
be rebuked, this they rage. Be learned, therefore, ye 
that judge the world, lest God be angry with you, and ye 
perish from the right way. 

Woe be to you, Scribes and Pharisees, hypocrites! saith 
Christ, (Matt. xxi.) for ye devour widows' houses under a 
colour of long prayer. Our hypocrites rob not the widows 
only, but knight, squire, lord, duke, king, and emperor, and 
even the whole world under the same colour; teaching 
the people to trust in their prayers and not in Christ, for 
whose sake God hath forgiven all the sin of the whole 
world unto as many as repent and believe. They fear, 
then, with purgatory, and promise to pray perpetually, lest 
the lands should ever return home again unto the right 
heirs. What hast thou bought with robbing thy heirs, 
or with giving the hypocrites that which thou robbest of 
other men? Perpetual prayer? Yea, perpetual pain. 
For they appoint thee no time of deliverance, their 
prayers are so mighty. The pope, for money, can empty 
purgatory when he will. It is, verily, purgatory; for it 
purgeth and maketh clean riddance: yea, it is hell, for it 
devoureth all things. His Fatherhood sendeth them to 
heaven with scala cali; that is, with a ladder to scale the 
walls. For by the door, Christ, will they not let them 
come in. That door have they stopped up, and that 
because ye should buy ladders of them. For some they 
pray daily, which gave them perpetuities, and yet make 
saints of them, receiving offerings in their names, and 
teaching other to pray to them. None of them, also, 
which taketh upon them to save other with their prayers, 
trusteth to be saved thereby themselves, but hire other to 
pray for them. 

Moses taketh record of God, that he took not of any 
of the people no much as an ass, neither vexed any of 
them. (Numb. xvi.) Samuel, in the first book of Kings 
the xiith chapter, asked all Israel, Whether he had taken 
any man's ox, or ass; or had vexed any man, or had taken 
any gift or reward of any man? And all the people testified
Nay; yet these two both taught the people,and also 
prayed for them, as much as our prelates do. (Pet. i.) 
Peter (vth.) exhorteth the elders to take the oversight of 
Christ's flock, not for filthy lucre; but of a good will, 
even for love, Paul (Acts xx.) taketh the priests, or 
elders, to record, that he had taught repentance and faith, 
and all the counsel of God. And yet had desired no 
man's gold, silver, or vesture; but fed himself with the 
labour of his hands. And yet these two taught, and 
prayed for the people, as much as our prelates do, with 
whom it goeth after the common saying, No penny, no 
paternoster. Which prelates, yet as they teach not, but 
beat only, so wot they not what prayer meaneth. 

Moreover, the law of love, which Christ left among us, 
is to give, and not to receive. What prayer is it, then, 
that thus robbeth all the world, contrary to that great 
commandment, which is the end of all commandments, 
and in which all others are contained? If men should 
continue to buy prayer four or five hundred years more, 
as they have done, there would not be a foot of ground in 
Christendom, neither any worldly thing, which they, that 
will be called spiritual only, should not thus possess. 
And thus all should be called spiritual. 

Woe be to you lawyers! for ye lade men with burdens 
which they are not able to bear, and ye yourselves touch 
not the packs with one of your fingers, saith Christ. 
(Luke xi.) Our lawyers, verily, have laden us a thousand
times more. What spiritual kindred have they 
made in baptism, to let matrimony? besides that, they 
have added certain degrees unto the law, natural for the 
same purpose. What an unbearable burthen of chastity 
do they violently thrust on other men's backs, and how 
easily bear they it themselves! How sore a burthen! 
How cruel a hangman! How grievous a torment! Yea, 
and how painful an hell is this ear-confession unto men's 
consciences! For the people are brought in belief, that 
without that they cannot be saved. Insomuch, that some 
fast certain days in the year, and pray certain superstitious 
prayers all their lives long, that they may not die without 
confession. In peril of death, if the priest be not by, the 
shipmen shrive themselves unto the mast. If any be 
present, they run then every man into his ear; but to 
God's promises fly they not, for they know them not. If 
any man have a death's wound, he crieth immediately for 
a priest. If a man die without shrift, many take it for a 
sign of damnation. Many, by reason of that false belief, 
die in desperation. Many, for shame, keep back of their 
confession twenty, thirty years, and think all the while 
that they be damned. I knew a poor woman with child 
which longed, and being overcome of her passion, ate 
flesh on a Friday, which thing she durst not confess in 
the space of eighteen years, and thought all that while 
that she had been damned, and yet sinned she not at all. 
Is not this a sore burden, that so weigheth down the soul 
unto the bottom of hell? What should I say? A great 
book were not sufficient to rehearse the snares which they 
have laid to rob men both of their goods, and also of the 
trust which they should have in God's word. 

The Scribes and Pharisees do all their works to be 
seen of men. They set abroad their phylacteries, and 
make long borders on their garments, and love to sit 
uppermost at feasts, and to have the chief seats in the 
synagogues; that is, in the congregations or councils, 
and to be called Rabbi; that is to say, Masters, saith 
Christ. (Matt. xxiii.) Behold the deeds of our spiritualty
and how many thousand fashions are among them 
to be known by? which, as none is like another, so loveth 
none another. For every one of them supposeth that all 
other poll too fast and make too many captives: yet to resist
Christ are they all agreed, lest they should be all compelled
to deliver up their prisoners to him. Behold the 
monsters, how they are disguised with mitres, crosiers, 
and hats: with crosses, pillars, and poleaxes; and with 
three crowns! What names have they? My Lord 
Prior, my Lord Abbot, my Lord Bishop, my Lord Arch- 
bishop, Cardinal, and Legate; If it please your Fatherhood;
If it please your Lordship; If it please your 
Grace; If it please your Holiness; and innumerable 
such like. Behold how they are esteemed, and how high 
they be crept up above all; not into worldly seats only, but 
into the seat of God, the hearts of men, where they sit 
above God himself. For both they, and whatsoever they 
make of their own heads, is more feared and dread than 
God and his commandments. In them and their deservings
put we more trust than in Christ, and his merits. To 
their promises give we more faith, than to the promises 
which God hath sworn in Christ's blood. 

The hypocrites say unto the kings and lords, These 
heretics would have us down first, and then you, to make 
of all common. Nay, ye hypocrites and right heretics, 
approved by open Scripture, the kings and lords are down 
already, and that so low, that they cannot go lower. Ye 
tread them under your feet, and lead them captive, and 
have made them your bond-servants to wait on your 
filthy lusts; and to avenge your malice on every man, 
contrary unto the right of God's word. Ye have not only 
robbed them of their land, authority, honour, and due 
obedience which ye owe unto them; but also of their 
wits, so that they are not without understanding in God's 
word only; but even in worldly matters, that pertain unto 
their offices, they are more than children. Ye bear them 
in hand what ye will, and have brought them even in case 
like unto them which, when they dance naked in nets, believe
they are invisible. We would have them up again, 
and restored unto the room and authority which God hath 
given them, and whereof ye have robbed them. And your 
inward falsehood we do but utter only with the light of 
God's word, that your hypocrisy might be seen. Be 
learned, therefore, ye that judge the world, lest God be 
angry with you, and ye peirish from the right way. 

Woe be to you, Scribes and Pharisees, hypocrites! 
For ye make clean the utterside of the cup and of the 
platter, but within they are full of bribery and excess, 
saith Christ. (Matt. xxiii.) Is that which our hypocrites 
eat and drink, and all their riotous excess, any other thing 
save robbery, and that which they have falsely gotten with 
their lying doctrine? Be learned, therefore, ye that judge 
the world, and compel them to make restitution again. 

Ye blind guides, saith Christ, ye strain out a gnat and 
swallow a camel. (Matt. xxiii.) Do not our blind 
guides also stumble at a straw, and leap over a block, 
making narrow consciences at trifles, and at matters of 
weight none at all? If any of them happen to swallow 
his spittle, or any of the water wherewith he washed his 
mouth, ere he go to mass; or touch the Sacrament with 
his nose; or if the ass forget to breathe on him, or happen 
to handle it with any of his fingers which are not anointed; 
or say Alleuia instead of Laus tibi Domine; or Ite 
missa est instead of Benedicamus Domino; or pour too 
much wine in the chalice; or read the gospel without 
light; or make not his crosses aright, how trembleth he! 
How feareth he! What an horrible sin is committed! 

I cry God mercy, saith he, and you, my ghostly father. 
But to hold an whore, or another man's wife, to buy a 
benefice, to set one realm at variance with another, and to 
cause twenty thousand men to die on a day, is but a 
trifle and a pastime with them! 

The Jews boasted themselves of Abraham. And 
Christ said unto them, (John viii.) If ye were Abraham's
children ye would do the deeds of Abraham. Our 
hypocrites boast themselves of the authority of Peter, and of 
Paul, and the other apostles, clean contrary unto the deeds 
and doctrine of Peter, Paul, and of all the other apostles; 
which both obeyed all worldly authority and power, 
usurping none to themselves, and taught all other to fear 
the kings and rulers, and to obey them in all things, not 
contrary to the commandment of God, and not to resist 
them, though they took away life and goods wrongfully; 
but patiently to abide God's vengeance. This did our 
spiritualty never yet, nor taught it. They taught not to 
fear God in his commandments, but to fear them in their 
traditions. Insomuch, that the evil people, which fear 
not to resist a good king, and to rise against him, dare 
not lay hands on one of them, neither for defiling of wife, 
daughter, or very mother. When all men lose life and 
lands, they remain always sure and in safety, and ever 
win somewhat. For whosoever conquereth other men's 
lands unrightfully, ever giveth them part with them. To 
them is all thing lawful. In all councils and parliaments 
are they the chief. Without them may no king be 
crowned, neither until he be sworn to their liberties. 
All secrets know they, even the very thoughts of men's 
hearts. By them all things are ministered No king nor 
realm may, through their falsehood, live in peace. To 
believe they teach not in Christ, but in them and their 
disguised hypocrisy. And of them compel they all men 
to buy redemption and forgiveness of sins. The people's 
sin they eat, and thereof wax fat. The more wicked the 
people are, the more prosperous is their commonwealth. 

If kings and great men do amiss, they must build abbeys 
and colleges; mean men build chantreys, poor find 
trentals and brotherhoods and begging friars. Their own 
heirs do men disinherit to endote them. All kings are 
compelled to submit themselves to them. Read the story 
of king John, and of other kings. They will have their 
causes avenged, though whole realms should therefore 
perish. Take from them their disguising, so are they not 
spiritual. Compare that they have taught us unto the 
Scripture, so are we without faith. 

Christ saith, (John, vth chapter,) How can ye believe 
which receive glory one of another? If they that seek to 
be glorious can have no faith, then are our prelates faithless,
verily. And (John viith.) he saith, He that speaketh 
of himself, seeketh his own glory. If to seek glory and 
honour be a sure token that a man speaketh of his ownself,
and doth his own message, and not his master's; 
then is the doctrine of our prelates of themselves, and not 
of God. Be learned, therefore, ye that judge the earth, 
lest God be angry with you, and ye perish from the right 
way. 

Be learned, lest the hypocrites bring the wrath of God 
upon your heads, and compel you to shed innocent blood; 
as they have compelled your predecessors to slay the 
prophets, to kill Christ and his apostles, and all the 
righteous that since were slain. God's word pertaineth 
unto all men; as it pertaineth unto all servants to know 
their master's will and pleasure; and to all subjects to 
know the laws of their prince. Let not the hypocrites do 
all things secretly. What reason is it that mine enemy 
should put me in prison at his pleasure, and their diet 
me, and handle me as he lusteth; and judge me himself,
and that secretly; and condemn me by a law of his 
own making, and then deliver me to Pilate to murder me? 
Let God's word try every man's doctrine, and whomsoever 
God's word proveth unclean, let him be taken for a leper. 
One Scripture will help to declare another. And the circumstances,
that is to say, the places that go before and 
after, will give light unto the middle text. And the open 
and manifest Scriptures will ever improve the false and 
wrong exposition of the darker sentences. Let the temporal
power, to whom God hath given the sword, to take 
vengeance, look or ever that they leap, and see what they 
do. Let the causes be disputed before them, and let him 
that is accused have room to answer for himself. The 
powers to whom God hath committed the sword shall give 
account for every drop of blood that is shed on the 
earth. Then shall their ignorance not excuse them, nor 
the saying of the hypocrites help them, — My soul for 
your's, your grace shall do a meritorious deed; your 
grace ought not to hear them, it is an old heresy 
condemned by the church. The king ought to look in 
the Scripture, and see whether it were truly condemned 
or no, if he will punish it. If the king, or his officer for 
him, will slay me, so ought the king, or his officer, to 
judge me. The king cannot, but unto his damnation, 
lend his sword to kill whom he judgeth not by his own 
laws. Let him that is accused stand on the one side, and 
the accuser on the other side; and let the king's judge sit 
and judge the cause, if the king will kill, and not be a 
murderer before God. 

Hereof may ye see, not only that our persecution is for 
the same cause that Christ's was, and that we say nothing 
that Christ said not; but also that all persecution is only 
for rebuking of hypocrisy; that is to say, of man's righteousness,
and of holy deeds which man hath imagined to 
please God, and to be saved by, without God's word, and 
beside the Testament that God hath made in Christ. 
If Christ had not rebuked the Pharisees because they 
taught the people to believe in their traditions and holiness,
and in offerings that came to their advantage, and 
that they taught the widows, and them that had their 
friends dead, to believe in their prayers; and that through 
their prayers the dead should be saved; and through that 
means, robbed them both of their goods, and also of 
the Testament and promises that God had made, to all 
that repented in Christ to come, he might have been uncrucified
unto this day. 

If St. Paul also had not preached against circumcision, 
that it justified not; and that vows, offerings, and ceremonies
justified not; and that righteousness, and forgiveness
of sins, came not by any deserving of our deeds, 
but by faith, or believing the promises of God, and by the 
deserving and merits of Christ only, he might have lived 
unto this hour. Likewise, if we preached not against 
pride, covetousness, lechery, extortion, usury, simony, 
and against the evil living both of the spiritualty, as well 
as of the temporalty, and against inclosings of parks, 
raising of rents and fines, and of the carrying out of wool 
out of the realm, we might endure long enough. But 
touch the scab of hypocrisy, or pope-holiness, and go 
about to utter their false doctrine wherewith they reign as 
gods in the heart and consciences of men, and rob them 
not of lands, goods, and authority only, but also of the 
Testament of God, and salvation that is in Christ; then 
helpeth thee neither God's word, nor yet if thou didst 
miracles; but that thou art not an heretic only, and hast 
the devil within thee, but also a breaker of the king's 
peace, and a traitor. But let us return unto our lying 
signs again. 

What signifieth that the prelates are so bloody, and 
clothed in red? that they be ready every hour to suffer 
martyrdom for the testimony of God's word. Is that also 
a false sign? When no man dare for them once open 
his mouth to ask a question of God's word, because they 
are ready to burn him. 

What signifieth the poleaxes that are borne before 
high legates, a latere? Whatsoever false sign they make of 
them I care not; but of this I am sure, that as the old 
hypocrites when they had slain Christ, set pole-axes to 
keep him in his sepulchre, that he should not rise again, 
even so have our hypocrites buried the Testament that 
God made unto us in Christ's blood, and to keep it 
down, that it rise not again, is all their study; whereof 
these poleaxes are the very sign. 

Is not that shepherd's hook, the bishop's cross, a false 
sign? Is not that white rochet that the bishops and 
canons wear, so like a nun, and so effeminately, a false 
sign? What other things are their sandals, gloves, mitres, 
and all the whole pomp of their disguising, than false signs 
in which Paul prophesied that they should come? And 
as Christ warned us to beware of wolves in lamb's skins,
and bade us look rather unto their fruits and deeds, than
to wonder at their disguisings. Run throughout all our
holy religions, and thou shalt find them likewise all 
clothed in falsehood. 


OF THE SACRAMENTS. 

FORASMUCH as we be come to signs, we will speak 
a word or two of the signs which God hath ordained, 
that is to say, of the Sacraments which Christ left among 
us for our comfort, that we may walk in light and in truth 
and in feeling of the power of God. For he that walketh 
in the day stumbleth not; when contrariwise he that 
walketh in the night stumbleth. (John xi.) And they that 
walk in darkness wot not whither they go. (John xii.) 

This word sacrament is as much to say as an holy sign, 
and representeth alway some promise of God. As in the 
Old Testament God ordained that the rainbow should represent
and signify unto all men an oath that God sware 
to Noah, and to all men after him, that he would no more 
drown the world through water. 


THE SACRAMENT OF THE BODY AND BLOOD 
OF CHRIST. 

SO the Sacrament of the body aod blood of Christ, hath 
a promise annexed, which the priest should declare in 
the English tongue. This is my body that is broken for 
you. This is my blood that is shed for many unto the 
forgiveness of sins. This do in remembrance of me, saith 
Christ. (Luke xxii. and 1 Cor. ii.) If when thou seest 
the Sacrament, or eatest his body, or drinkest his blood, 
thou have this promise fast in thine heart (that his body 
was slain, and his blood shed for thy sins) and believest it, 
so art thou saved and justified thereby. If not, so helpeth 
it thee not, though thou hearest a thousand masses in a 
day, or though thou doest nothing else all thy life long than 
eat his body, or drink his blood: no more than it should 
help thee in a dead thirst, to behold a bush at a tavern 
door, if thou knewest not thereby that there were wine 
within to be sold. 


BAPTISM. 

BAPTISM hath also his word and promise, which the 
priest ought to teach the people, and christen them in 
the English tongue, and not to play the popinjay with 
Credo say ye, Volo say ye, and Baptismum say ye, 
for there ought to be no mumming in such a matter. The 
priest before he baptiseth, asketh, saying: Believest thou 
in God the Father Almighty, and in his Son Jesus Christ, 
and in the Holy Ghost, and that the congregation of 
Christ is holy? And they say, Yea, Then the priest 
upon this faith baptizeth the child in the name of the 
Father, and of the Son, and of the Holy Ghost, for the 
forgiveness, of sins, as Peter saith, (Acts ii.) 

The washing without the word helpeth not: but through 
the word it purifieth and cleanseth us. Aa thou readest 
(Eph. v.) How Christ cleanseth the congregation in the 
fountain of water through the word. The word is the 
promise that God hath made. Now as a preacher in 
preaching the word of God saveth the hearers that believe, 
so doth the washing, in that it preacheth and representeth 
unto us the promise that God hath made unto us in Christ. 
The washing preacheth unto us, that we are cleansed 
with Christ's bloodshedding, which was an offering and a 
satisfaction for thy sin of all that repent and believe, consenting
and submitting themselves unto the will of God. 
The plunging into the water signifieth that we die, and 
are buried with Christ, as concerning the old life of sin 
which is in Adam. And the pulling out again, signifieth 
that we rise again with Christ in a new life, full of the 
Holy Ghost, which shall teach us and guide us, and work 
the will of God in us, as thou seest Rom. vi. 


OF WEDLOCK. 

MATRIMONY or wedlock is a state or a degree ordained
of God, and an office wherein the husband 
serveth the wife, and the wife the husband. It was ordained
for a remedy, and to encrease the world, and for 
the man to help the woman, and the woman the man with 
all love and kindness; and not to signify any promise that 
ever I heard or read of in the Scripture. Therefore 
ought it not to be called a Sacrament. It hath a promise 
that we sin not in that state, if a man receive his wife as 
a gift given to him of God, and the wife her husband 
likewise; as all manner [of] meats and drinks have a promise
that we sin not, if we use them measurably with 
thanksgiving. If they call matrimony a sacrament because 
the Scripture useth the similitude of matrimony to express 
the marriage, or wedlock, that is between us and Christ; 
(for as a woman though she be never so poor, yet when 
she is married, is as rich as her husband: even so we when 
we repent and believe the promises of God in Christ, 
though we be never so poor sinners, yet are as rich as 
Christ; all his merits are ours with all that he hath;) if 
for that cause they call it a sacrament, so will I mustard 
seed, leaven, a net, keys, bread, water, and a thousand 
other things which Christ and the prophets, and all the 
Scripture use, to express the kingdom of heaven and God's 
word withal. They praise wedlock with their mouth, and 
say it is an holy thing, as it is verily, but had lever be sanctified
with an whore, than to come within the sanctuary. 


OF ORDER. 

SUBDEACON, Deacon, Priest, Bishop, Cardinal, 
Patriarch and Pope, be names of offices and service, or 
should be, and not Sacraments. There is no promise 
coupled therewith. If they minister their offices truly, it 
is a sign that Christ's Spirit is in them, if not, that the 
devil is in them. Are these all Sacraments, or which one 
of them? Or what thing in them is that holy sign or Sacrament?
The shaving, or the anointing? What also is 
the promise that is signified thereby? But what word 
printeth in them that character, that spiritual seal? O 
dreamers and natural beasts without the seal of the Spirit 
of God; but sealed with the mark of the beast and with 
cankered consciences. 

There is a word called in Latin Sacerdos, in Greek 
Hiereus, in Hebrew Cohan, that is, a minister, an officer, 
a sacrificer or a priest; as Aaron was a priest and sacrificed 
for the people, and was a mediator between God and them. 
And in the English should it have had some other name 
than priest? But antichrist hath deceived us with unknown
and strange terms, to bring us into confusion and 
superstitious blindness. Of that manner is Christ a 
priest for ever, and all we priests through him, and need no 
more of any such priest on earth to be a mean for us 
unto God. For Christ hath brought us all into the inner 
temple, within the veil or forehanging, and unto the mercy- 
stool of God. And hath coupled us unto God, where 
we offer every man for himself the desires and petitions of 
his heart, and sacrifice and kill the lusts and appetites of 
his flesh, with prayer, fasting, and all manner [of] godly 
living. 

Another word is there in Greek, called presbiter, in 
Latin senior, in English an elder, and is nothing but an 
officer to teach, and not to be a mediator between God 
and us. This needeth no anointing of man. They of 
the Old Testament were anointed with oil, to signify the 
anointing of Christ, and of us through Christ with the 
Holy Ghost. Thiswise is no man priest but he that is 
chosen, save as in time of necessity every person christeneth,
so may every man teach his wife and household, and 
the wife her children. So in time of need if I see my 
brother sin, I may between him and me rebuke him, and 
damn his deed by the law of God. And may also comfort
them that are in despair with the promises of God, 
and save them if they believe. 

By a priest then in the New Testament, understand nothing
but an elder to teach the younger, and to bring 
them unto the full knowledge and understanding of Christ; 
and to minister the Sacraments which Christ ordained, 
which is also nothing but to preach Christ's promises. 
And by them that give all their study to quench the light 
of truth, and to hold the people in darkness, understand 
the disciples af Satan and messengers of antichrist, whatsoever
names they have, or whatsoever they call themselves. 
And as concerning that our spiritualty (as they will be
called) make themselves holier than the lay people, and 
take so great lands and goods to pray for them, and promise
them pardons and forgiveness of sins, or absolution; 
without preaching of Christ's promises, is falsehood, and 
the working of antichrist: and (as I have said) the ravening
of those wolves which Paul (Acts xx.) prophesied 
should come after his departing not sparing the flock. 
Their doctrine is that merchandise whereof Peter speaketh,
saying: Through covetousness shall they with feigned words 
make merchandise of you. (2 Pet. ii.) And their reasons,
wherewith they prove their doctrine, are, (as saith Paul 
1 Tim. vi.) Superfluous disputings, arguings or brawlings of 
men with corrupt minds, and destitute of truth, which 
think that lucre is godliness. But Christ saith (Matt. vii.) 
By their fruits shalt thou know them; that is, by their 
filthy covetousness, and shameless ambition, and drunken 
desire of honour, contrary unto the ensample and doctrine 
of Christ and of his apostles. Christ said to Peter, (the 
last chapter of John :) Feed my sheep: and not shear thy 
flock. And Peter saith, (1 Pet. v.) Not being lords over 
the parishes: but these shear, and are become lords. Paul 
saith (2 Cor. ii.) Not that we be lords over your faith: 
but these will be lords, and compel us to believe whatsoever
they lust, without any witness of Scripture, yea, clean 
contrary to the Scripture: when the open text rebuketh it, 
Paul saith, It is better to give, than to receive, (Acts. xx.) 
But these do nothing in the world but lay snares to catch 
and receive whatsoever cometh, as it were the gaping 
mouth of hell. And (2 Cor. xii.) I seek not yours, but you: 
but these seek not you to Christ, but your's to themselves, 
and therefore lest their deeds should be rebuked will not 
come at the light. 

Nevertheless the truth is, that we are all equally beloved
in Christ, and God hath sworn to all indifferently. 
According, therefore, as every man believeth God's promises,
longeth for them, and is diligent to pray unto God 
to fulfil them, so is his prayer heard, and as good is the 
prayer of a cobbler, as of a cardinal; and of a butcher, 
as of a bishop; and the blessing of a baker that knoweth 
the truth, is as good as the blessing of our most holy 
father the pope. And by blessing, understand not the
wagging of the pope's or bishop's hand over thine head;
but prayer, as when we say, God make thee a good man,
Christ put his Spirit in thee, or give thee grace and power 
to walk in the truth, and to follow his commandments, 
\&c. As Rebecca's friends blessed her when she departed, 
(Gen. xxiv.) saying, Thou art our sister: grow unto 
thousand thousands, and thy seed possess the gates of their 
enemies. And as Isaac blessed Jacob, (Gen. xxvii.) saying,
God give thee of the dew of heaven, and of the fatness 
of the earth, abundance of corn, wine and oil, \&c. And 
(Gen. xxviii.) Almighty God bless thee, and make thee 
grow, and multiply thee, that thou mayest be a great 
multitude of people, and give to thee and to thy seed 
after thee, the blessings of Abraham, that thou mayest 
possess the land wherein thou art a stranger, which he 
promised to thy grandfather, and such like. 

Last of all, one singular doubt they have; what maketh 
the priest; the anointing or putting on of the hands, or 
what other ceremony, or what words? About which they 
brawl and scold, one ready to tear out another's throat. 
One saith this, and another that, but they cannot agree. 
Neither can any of them make so strong a reason which 
another cannot improve. For they are all out of the way, 
and without the Spirit of God to judge spiritual things. 
Howbeit to this I answer, that when Christ called twelve up 
into the mountain, and chose them, then immediately without
any anointing or ceremony were they his apostles; 
that is to wit, ministers chosen to be sent to preach his 
Testament unto all the whole world. And after the resurrection,
when he had opened their wits, and given 
them knowledge to understand the secrets of his Testament, 
and how to bind and loose, and what he would have them 
to do in all things, then he sent them forth with a commandment
to preach, and bind the unbelieving that continue
in sin, and to loose the believing that repent. And 
that commandment or charge made them bishops, priests, 
popes, and all thing. If they say that Christ made them 
priests at his maundy, or last supper, when he said, Do 
this in the remembrance of me: I answer, Though the apostles
wist not then what he meant, yet I will not strive nor 
say thereagainst. Neverthelater the commandment and the 
charge which he gave them made them priests. 

And (Acts the first,) when Matthias was chosen by lot, 
it is not to be doubted but that the apostles, after their 
common manner, prayed for him that God would give him 
grace to minister his office truly; and put their hands on 
him, and exhorted him, and gave him charge to be diligent
and faithful, and then was he as great as the best. 
And (Acts vi.) when the disciples that believed had chosen 
six deacons to minister to the widows, the apostles prayed 
and put their hands on them, and admitted them without 
more ado. Their putting on of hands was not after the 
manner of the dumb blessing of our holy bishops, with 
two fingers; but they spake unto them, and told them their 
duty, and gave them a charge, and warned them to be 
faithful in the Lord's business: as we choose temporal 
officers, and read their duty to them, and they promise to 
be faithful ministers, and then are admitted. Neither is 
there any other manner or ceremony at all required in 
making of our spiritual officers, than to choose an able 
person, and then to rehearse him his duty, and give him 
his charge, and so to put him in his room. And as for 
that other solemn doubt, as they call it, Whether Judas 
was a priest or no? I care not what he then was; but of
this I am sure, that he is now not only priest, but also
bishop, cardinal, and pope. 


PENANCE. 

PENANCE is a word of their own forging, to deceive 
us withal, as many others are. In the Scripture 
we find, Penitentia, repentance. Agite penitentiam, do 
repent; Peniteat vos, let it repent you. Metanoyte, in 
Greek, forthink ye, or let it forthink you. Of repentance
they have made penance, to blind the people,
and to make them think that they must take pains, 
and do some holy deeds to make satisfaction for their 
sins; namely, such as they enjoin them. As thou mayest 
see in the chronicles, when great kings and tyrants 
(which with violence of sword conquered other kings' 
lands, and slew all that came to hand) came to themselves,
and had conscience of their wicked deeds, then the 
bishops coupled them not to Christ; but unto the pope, 
and preached the pope unto them, and made them to 
submit themselves, and also their realms, unto the holy 
father the pope, and to take penance, as they call it; 
that is to say, such injunctions as the pope and bishops 
would command them to do, to build abbeys, to endote 
them with livelihood, to be prayed for for ever: and to 
give them exemptions, and privilege, and license, to do 
what they lust, unpunished. 

Repentance goeth before faith, and prepareth the way
to Christ, and to the promises. For Christ cometh no
but unto them that see their sins in the law, and repent. 
Repentance, that is to say, this mourning and sorrow of 
the heart, lasteth all our lives long. For we find ourselves 
all our lives long too weak for God's law, and therefore 
sorrow and mourn, longing for strength. Repentance is 
no sacrament: as faith, hope, love, and knowledge of a 
man's sins are not to be called sacraments. For they are 
spiritual and invisible. Now must a sacrament be an outward
sign that may be seen, to signify, to represent, and 
to put a man in remembrance of some spiritual promise, 
which cannot be seen but by faith only. Repentance, 
and all the good deeds which accompany repentance, to 
slay the lusts of the flesh, are signified by baptism. For 
Paul saith, (Rom vi.) (as it is above rehearsed) Remember 
ye not (saith he) that all we which are baptized in the 
name of Christ Jesus, are baptized to die with him? We 
are buried with him in baptism for to die; that is, to kill
the lusts and the rebellion which remaineth in the flesh.
And after that he saith, Ye are dead, as concerning sin, 
but live unto God, through Jesus Christ our Lord. If 
thou look on the profession of our hearts, and on the 
spirit and forgiveness which we have received through 
Christ's merits, we are full dead; but if thou look on the 
rebellion of the flesh, we do but begin to die, and to be 
baptized; that is, to drown and quench the lusts, and are 
full baptized at the last minute of death. And as concerning
the working of the Spirit, we begin to live, and 
grow every day more and more both in knowledge, and 
also in godly living, according as the lusts abate. As a 
child receiveth the full soul at the first day, yet groweth 
daily in the operations and works thereof. 


OF CONFESSION. 

CONFESSION is divers: one followeth true faith inseparably,
and is the confessing and knowledging 
with the mouth, wherein we put our trust and confidence.
As when we say our Credo, confessing that we 
trust in God the Father Almighty, and in his truth and 
promises; and in his son Jesus, our Lord, and in his 
merits and deservings; and in the Holy Ghost, and in his 
power, assistance and guiding. This confession is necessary
unto all men that will be saved. For Christ saith, 
(Matt. x.) He that denieth me before men, him will I 
deny before my Father that is in heaven. And of this 
confession, saith the holy apostle Paul, in the xth chapter, 
The belief of the heart justifieth; and to knowledge with 
the mouth maketh a man safe. This is a wonderful text 
for our philosophers, or rather sophisters, our worldly 
wise enemies to the wisdom of God, our deep and profound
wells without water, our clouds without moisture of 
rain; that is to say, natural souls without the Spirit of 
God, and feeling of godly things. To justify and to 
make safe are both one thing. And to confess with the 
mouth is a good work, and the fruit of a true faith, as all 
other works are. 

If thou repent and believe the promises, then God's 
truth justifieth thee; that is, forgiveth thee thy sins, and 
sealeth thee with his Holy Spirit, and maketh thee heir of 
everlasting life, through Christ's deservings. Now if thou 
have true faith, so seest thou the exceeding and infinite 
love and mercy which God hath showed thee freely in 
Christ: then must thou needs love again: and love 
cannot but compel thee to work, and boldly to confess and 
knowledge thy Lord Christ, and the trust which thou 
hast in his word. And this knowledge maketh thee safe; 
that is, declareth that thou art safe already, certifieth thine 
heart, and maketh thee feel that thy faith is right, and 
that God's Spirit is in thee, as all other good works do. 
For if when it cometh unto the point, thou hast no lust 
to work, nor power to confess, how couldest thou presume
to think that God's Spirit were in thee? 

Another confession is there which goeth before faith, 
and accompanieth repentance. For whosoever repenteth, 
doth knowledge his sins in his heart. And whosoever 
doth knowledge his sins, receiveth forgiveness (as saith 
John, in the first of his first Epistle.) If we knowledge 
our sins, he is faithful and just to forgive us our sins, and
to cleanse us from all unrighteousness; that is, because 
he hath promised, he must for his truth's sake do it. This 
confession is necessary all our lives long, as in repentance. 
And as thou understandest of repentance, so understand 
of this confession, for it is likewise included in the sacrament
of baptism. For we always repent, and always 
knowledge or confess our sins unto God, and yet despair 
not; but remember that we are washed in Christ's blood;
which thing our baptism doth represent and signify 
unto us. 

Shrift in the ear is verily a work of Satan, and that the
falsest that ever was wrought, and that most hath devoured
the faith. It began among the Greeks, and was
not as it if now, to reckon all a man's sins in the priest's
ear; but to ask counsel of such doubts as men had, as thou 
mayest see in St. Jerome, and in other authors. Neither 
went they to priests only, which were very few at that 
time, no more than preached the word of God, for this 
so great vantage in so many masses saying was not yet 
found; but went indifferently, where they saw a good and 
a learned man. And for because of a little knavery 
which a deacon at Constantinople played through confession
with one of the chief wives of the city, it was laid 
down again. But we, antichrist's possession, the more 
knavery we see grow thereby daily, the more we stablish 
it. A Christian man is a spiritual thing, and hath God's 
word in his heart, and God's Spirit to certify him of all 
things. He is not bound to come to any ear. And as for 
the reasons which they make, are but persuasions of man's 
wisdom. First, as pertaining unto the keys and manner 
of binding and loosing, is enough above rehearsed, and in 
other places. Thou mayest also see how the apostles used 
them in the Acts, and in Paul's Epistles how at the
preaching of faith the Spirit came, and certified their
hearts that they were justified through believing the promises.

When a man feeleth that his heart consenteth unto the 
law of God, and feeleth himself meek, patient, courteous, 
and merciful to his neighbour, altered and fashioned like 
unto Christ, why should he doubt but that God hath forgiven 
him, and chosen him, and put his Spirit in him, though 
he never cram his sin into the priest's ear? 

One blind reason have they, saying, How shall the 
priest unbind, loose, and forgive the sin which he knoweth 
not? How did the apostles? The Scripture forsake 
they, and run unto their blind reasons, and draw the 
Scripture unto a carnal purpose. When I have told thee 
in thine ear all that I have done my life long, in order and 
with all circumstances after the shamefullest manner, what 
canst thou do more than preach me the promises, saying, 
If thou repent and believe, God's truth shall save thee 
for Christ's sake? Thou seest not mine heart, thou 
knowest not whether I repent or no, neither whether I 
consent to the law, that it is holy, righteous, and good. 
Moreover, whether I believe the promises or no, is also 
unknown to thee. If thou preach the law and the promises
(as the apostles did) so should they that God hath 
chosen, repent, and believe and be saved: even now as 
well as then. Howbeit antichrist must know all secrets 
to stablish his kingdom, and to work his mysteries withal. 

They bring also for them the stories of the ten lepers, 
which is written in the xviith chapter of Luke. Here 
mark their falsehood, and learn to know them for ever. 
The fourteenth Sunday after the feast of the Trinity, the 
beginning of the seventh lesson, is the said gospel, and 
the eighth and the ninth lessons, with the rest of the 
seventh, is the exposition of Bede upon the said gospel. 
Where, saith Bede, of all that Christ healed, of whatsoever
disease it were, he sent none unto the priests, but 
the lepers. And by the lepers interpreteth followers of 
false doctrine only, which the spiritual officers, and the 
learned men of the congregation ought to examine, and 
rebuke their learning with God's word, and to warn the
congregation to beware of them. Which, if they were 
afterward healed by the grace of Christ, ought to come 
before the congregation, and there openly confess their 
true faith. 

But all other vices (saith he) doth God heal within in 
the conscience. 

Though they thiswise read at matins, yet at high mass, if 
they have any sermon at all, they lie clean contrary unto this 
open truth. Neither are they ashamed at all. For why? 
they walk altogether in darkness. 


OF CONTRITION. 

CONTRITION and repentance are both one, and nothing
else but a sorrowful and a mourning heart. And 
because that God hath promised mercy unto a contrite heart, 
that is, to a sorrowful and repenting heart, they, to beguile 
God's word and to stablish their wicked tradition, have 
feigned that new word attrition, sayings Thou canst not 
know whether thy sorrow or repentance be contrition or attrition,
except thou be shriven. When thou art shriven then 
it is true contrition. Oh! sorry Pharisee, that is thy leaven, 
of which Christ so diligently bade us beware, (Matt. vi.) 
And the very prophesy of Peter, Through covetousness 
with feigned words shall they make merchandise of you. 
(2 Peter ii.) With such glosses corrupt they God's word, to 
sit in the consciences of the people, to lead them captive, 
and to make a prey of them: buying and selling their sins 
to satisfy their unsatiable covetousness. Nevertheless the 
truth is, when any man hath trespassed against God, if he 
repent and knowledge his trespass, God promiseth him forgiveness
without ear shrift. 

If he that hath offended his neighbour repent and 
knowledge his fault, asking forgiveness, if his neighbour 
forgive him, God forgiveth him also by his holy promise. 


(Matt. xviii.) Likewise, if he that sinneth openly when he is 
openly rebuked, repent and turn, then if the congregation 
forgive him, God forgiveth him; and so forth whosoever 
repenteth, and when he is rebuked knowledgeth his fault, is 
forgiven.

He also that doubteth, or hath his conscience tangled, 
ought to open his mind unto some faithful brother that is 
learned, and he shall give him faithful counsel to help him 
withal.

To whom a man trespaaseth, unto him he ought to confess.
But to confess myself unto thee O antichrist, whom I 
have not offended, am I not bound. 

They of the old law had no confession in the ear. Neither
the apostles, nor they that followed many hundred years 
after, knew of any such whispering. Whereby then was their 
attrition turned unto contrition? yea, why are we, which 
Christ came to loose, more bound than the Jews? Yea, 
and why are we more bound without Scripture? for Christ 
came not to make us more bound, but to loose us and to 
make a thousand things no sin which before were sin, and 
are now become sin again. He left none other law with us 
but the law of love. He loosed us not from Moses to bind 
us unto antichrist's ear. God had not tied Christ unto 
antichrist's ear, neither hath poured all his mercy in thither, 
for it hath no record in the Old Testament, that antichrist's 
ear should be Propiciatorium, that is to wit, God's mercy- 
stool, and that God should creep into so narrow a hole, so 
that he could no where else be found. Neither did 
God write his laws, neither yet his holy promises in 
antichrist's ear; but hath graved them with his holy Spirit 
in the hearts of them that believe, that they might have 
them always ready at hand to be saved thereby. 


SATISFACTION. 

AS pertaining unto satisfaction, thiswise understand, 
that he that loveth God hath a commandment, (as 
St. John saith in the fourth chapter of his first Epistle) 
to love his neighbour also: whom if thou have offended, 
thou must make him amends or satisfaction, or at the least- 
way, if thou be not able, ask him forgiveness, and if he will 
have mercy of God, he is bound to forgive thee. If he will 
not, yet God forgiveth thee, if thou thus submit thyself. 
But unto Godward, Christ is a perpetual and an everlasting 
satisfaction for evermore. 

As oft as thou fallest through frailty, repent and come 
again, and thou art safe and welcome, as thou mayest 
see by the similitude of the riotous son, (Luke xv.) 
If thou be lopen out of sanctuary, come in again. If 
thou be fallen from the way of truth, come thereto again, 
and thou art safe: if thou be gone astray, come to the fold 
again, and the Shepherd, Christ, shall save thee, yea, and the 
angels of heaven shall rejoice at thy coming, so far it is off 
that any man shall beat thee or chide thee. If any Pharisee
envy thee, grudge at thee, or rail upon thee, thy Father 
shall make answer for thee, as thou seest in the fore-re- 
hearsed likeness or parable. Whosoever therefore is gone 
out of the way, by whatsoever chance it be, let him come 
to his baptism again, and unto the profession thereof, and 
he shall be safe. 

For though that the washing of baptism be past, yet the 
power thereof, that is to say, the word of God, which 
baptism preacheth, lasteth ever and saveth for ever. As 
Paul is past and gone, nevertheless the word that Paul 
preached lasteth ever, and saveth ever as many as come 
thereto with a repenting heart and a steadfast faith. 

Hereby seest thou that when they make penance of repentance
and call it a sacrament, and divide it into contrition,
confession, and satisfaction, they speak of their own 
heads, and lie falsely. 


ABSOLUTION. 

THEIR absolution also justifieth no man from sin. For 
with the heart do men believe to be justified withal, 
saith Paul; (Rom. x.) that is, through faith and believing 
the promises are we justified, as I have sufficiently proved 
in other places with the Scripture. Faith (saith Paul in 
the same place) cometh by hearing, that is to say, by 
hearing the preacher that is sent from God, and preacheth 
God's promises. Now when thou absolvest in Latin, the 
unlearned heareth not. For how, saith Paul, (1 Cor. xiv.) 
When thou blessest in an unknown tongue, shall the unlearned
say Amen unto thy thanksgiving? for he wotteth not 
what thou sayest. So likewise the lay wotteth not whether 
thou loose or bind, or whether thou bless or curse. In 
like manner is it if the lay understand Latin, or though the 
priest absolve in English. For in his absolution he rehearseth
no promise of God: but speaketh his own words, 
saying, I, by the authority of Peter and Paul, absolve or 
loose thee from all thy sins. Thou sayest so, which art 
but a lying man, and never more than now verily. 

Thou sayest, I forgive thee thy sins, and the Scripture, 
(John the first) That Christ only forgiveth and taketh away 
the sins of the world. And Paul and Peter, and all the 
apostles, preach that all is forgiven in Christ and for Christ's 
sake. God's word only looseth, and thou in preaching 
that mightest loose also, and else not. 

Whosoever hath ears let him hear, and let him that hath 
eyes see. If any man love to be blind, his blindness on 
his own head and not on mine. 


They allege for themselves the saying of Christ to 
Peter, (Matt. xvi.) Whatsoever thou bindest on earth, it 
shall be bound; and whatsoever thou looseth and so forth. 
Lo, say they, whatsoever we bind and whatsoever we loose; 
here is nothing excepted. And another text lay they of 
Christ in the last of Matthew; All power is given to me, 
saith Christ, in heaven and in earth: go therefore and 
preach, \&c. Preaching leaveth the pope out, and saith 
Lo, all power is given me in heaven and in earth. And 
thereupon taketh upon him temporal power above king and 
emperor, and maketh laws and bindeth them. And like 
power taketh he over God's laws, and dispenseth with them 
at his lust, making no sin of that which God maketh sin, 
and maketh sin where God maketh none: yea, and wipeth 
out God's laws clean, and maketh at his pleasure, and 
with him is lawful what he lusteth. He bindeth where 
God looseth, and looseth where God bindeth. He blesseth
where God curseth, and curseth where God blesseth. 
He taketh authority also to bind and loose in purgatory.
That permit I unto him, for it is a creature of his 
own making. He also bindeth the angels. For we read 
of popes that have commanded the angels to set divers 
out of purgatory. Howbeit I am not yet certified whether 
they obeyed or no. 

Understand therefore that to bind and to loose, is to 
preach the law of God and the gospel or promises, as thou 
mayest see in the iiid chapter of the second Epistle to tbe 
Corinthians, where Paul called the preaching of the law the 
ministration of death and damnation, and the preaching 
of the promises, the ministering of the Spirit and of 
righteousness. For when the law is preached, all men are 
found sinners, and therefore damned: and when the gospel
and glad tidings are preached, then are all that repent 
and believe, found righteous in Christ. And so expound 
it all the old doctors. Saint Jerome saith upon this text, 
Whatsoever thou bindest, the bishops and priests, saith he, 
for lack of understanding, take a little presumption of the 
Pharisees upon them; and think that they have authority to 
bind innocents and to loose the wicked, which thing our 
pope and bishops do. For they say the curse is to be
feared, be it right or wrong. Though thou have not deserved,
yet if the pope curse thee, thou art in peril of thy 
soul as they lie: yea, and though he be never so wrongfully
cursed, he must be fain to buy absolution. But
Saint Jerome saith, As the priest of the old law made the
lepers clean or unclean, so bindeth and unbindeth the 
priest of the new law. 

The priest there made no man a leper, neither cleansed 
any man, but God: and the priest judged only by Moses' 
law who was clean and who was unclean, when they were 
brought unto him. 

So here we have the law of God to judge what is sin and 
what is not, and who is bound and who is not. Moreover 
if any man have sinned, yet if he repent and believe the 
promise, we are sure by God's word that he is loosed 
and forgiven in Christ. Other authority than this- 
wise to preach have the priests not. Christ's apostles 
had no other themselves; as it appeareth throughout all 
the New Testament. Therefore it is manifest that they 
have not. 

Saint Paul saith, (1 Cor. xv.) When we say all things
are under Christ, he is to be excepted that put all under
him. God the Father is not under Christ, but above
Christ, and Christ's head. (1 Cor. vi.)

Christ saith (John xii.) I have not spoken of mine own
head, but my Father which sent me gave a commandment
what I should say and what I should speak. Whatsoever
I speak therefore, even as my Father bade me so I speak.
If Christ had a law what he should do, how happeneth it 
that the pope so runneth, at large lawless? though that all 
power were given unto Christ in heaven and in earth: yet 
had he no power over his Father, nor yet to reign temporally
over temporal princes, but a commandment to obey 
them. How hath the pope then such temporal authority 
over king and emperor? How hath he authority above 
God's laws, and to command the angels, the saints, and 
God himself? 

Christ's authority which he gave to his disciples, was to 
preach the law and to bring sinners to repentance; and 
then to preach unto them the promises which the Father 
had made unto all men for his sake. And the same to 
preach only sent he his apostles. As a king sendeth forth 
his judges, and giveth them his authority, saying, What ye 
do that do I. I give you my full power. Yet meaneth 
he not by that full power, that they should destroy any 
town or city, or oppress any man, or do what they list, or 
should reign over the lords and dukes of his realm and 
over his own self. But giveth them a law with them, and 
authority to bind and loose, as farforth as the law 
stretcheth and maketh mention: that is, to punish the 
evil that do wrong, and to avenge the poor that suffer
wrong. And so far as the law stretcheth, will 
the king defend his judge against all men. And as the 
temporal judges bind and loose temporally, so do the 
priests spiritually, and no other ways. Howbeit by falsehood
and subtlety the pope reigneth under Christ, as cardinals
and bishops do under kings, lawless. 

THE pope (say they,) absolveth or looseth, a paena et 
culpa; that is, from the fault or trespass, and from the 
pain due unto the trespass. God, if a man repent, forgiveth
the offence only, and not the pain also, say they, 
save turneth the everlasting pain unto a temporal pain. 
And appointeth seven years in purgatory for every deadly 
sin. But the pope for money forgiveth both, and hath 
more power than God, and is more merciful than God. 
This do I, saith the pope, of my full power, and of the 
treasure of the church, —- of deservings of martyrs, confessors,
and merits of Christ. 

First, the merits of the saints did not save themselves, 
but were saved by Christ's merits only. 

Secondarily, God hath promised Christ's merits unto 
all that repent; so that whosoever repenteth, is immediately
heir of all Christ's merits, and beloved of God as 
Christ is. How then came this foul monster to be lord 
over Christ's merits, so that he hath power to sell that 
which God giveth freely? O dreamers! yea, O devils, 
and O venemous scorpions, what poison have ye in your 
tails! O pestilent leaven, that so turneth the sweet bread 
of Christ's doctrine into the bitterness of gall! 

The friars run in the same spirit, and teach, saying, 
Do good deeds, and redeem the pains that abide you in 
purgatory; yea, give us somewhat to do good works for 
you. And thus is sin become the profitablest merchandise
in the world. Oh! the cruel wrath of God upon us, 
because we love not the truth. 

For this is the damnation and judgment of God, to 
send a false prophet unto him that will not hear the truth. 
I know you, saith Christ, (John v.) that ye have not the 
love of God in you. I am come in my Father's name, 
and ye receive me not; if another shall come in his own 
name, him shall ye receive. This doth God avenge himself
on the malicious hearts which have no love to his 
truth. 

All the promises of God have they either whipped clean 
out, or thus leavened them with open lies, to stablish 
their confession withal. And to keep us from knowledge 
of the truth they do all thing in Latin. 

They pray in Latin, they christen in Latin, they bless 
in Latin, they give absolution in Latin, only curse they 
in the English tongue. Wherein they take upon them 
greater authority than ever God gave them. For in their 
curses, as they call them, with book, bell, and candle, 
they command God and Christ, and the angels, and all 
saints, to curse them: Curse them God (say they,) Father, 
Son, and Holy Ghost; curse them Virgin Mary, \&c. O 
ye abominable! who gave you authority to command God 
to curse? God commandeth you to bless, and ye command
him to curse! Bless them that persecute you: 
bless but curse not, saith St. Paul, (Rom. xii.) What 
tyranny will these not use over men, which presume and 
take upon them to be lords over God, and to command 
him? If God shall curse any man, who shall bless and 
make him better? No man can amend himself, except 
God pour his Spirit unto him. Have we not a commandment
to love our neighbour as ourselves? How can 
I love him and curse him also? James saith, It is not 
possible that blessing and cursing should come both out of 
one mouth. Christ commandeth, (Matt. v.) saying, Love 
your enemies. Bless them that curse you. Do good to 
them that hate you. Pray for them that do you wrong 
and persecute you, that ye may be the children of your 
heavenly Father. 

In the marches of Wales it is the manner if any man 
have an ox or a cow stolen, he cometh to the curate, 
and desireth him to curse the stealer. And he commandeth
the parish to give him every man God's curse and 
his. God's curse and mine have he, saith every man in 
the parish. O merciful God! what is blasphemy, if 
this be not blasphemy and shaming of the doctrine of 
Christ? 

Understand, therefore, the power of excommunication 
is this: if any man sin openly, and amendeth not when 
he is warned, then ought he to be rebuked openly before 
all the parish. And the priest ought to prove, by the 
Scripture, that all such have no part with Christ. For 
Christ serveth not but for them that love the law of God, 
and consent that it is good, holy, and righteous: and 
repent, sorrowing and mourning for power and strength 
to fulfil it. And all the parish ought to be warned to 
avoid the company of all such, and to take them as 
heathen people. This is not done that he should perish, 
but to save him, to make him ashamed, and to kill the 
lusts of the flesh, that the Spirit might come unto the 
knowledge of truth. And we ought to pity him, and to 
have compassion on him, and with all diligence to pray 
unto God for him, to give him grace to repent, and to 
come to the right way again; and not to use such tyranny 
over God and man, commanding God to curse. And if 
he repent, we ought with all mercy to receive him in 
again. This mayest thou see Matt. xviii. and 1 Cor. v. 
and 2 Cor. ii. 


CONFIRMATION. 

IF confirmation have a promise, then it justifieth as far 
as the promise extendeth. If it have no promise, then 
is it not of God, as the bishops be not. The apostles and 
ministers of God preach God's word; and God's signs or 
sacraments signify God's word also, and put us in remembrance
of the promises which God hath made unto us in 
Christ. Contrariwise, antichrist's bishops preach not, 
and their sacraments speak not, but as the disguised 
bishop's mum; so are their superstitious sacraments 
dumb. After that the bishops had left preaching, then 
feigned they this dumb ceremony of confirmation, to have 
somewhat at the leastway, whereby they might reign over 
their dioceses. They reserved unto themselves also the 
christening of bells, and conjuring or hallowing of 
churches and church-yards, and of altars and super-altars, 
and hallowing of chalices, and so forth; whatsoever is of 
honour or profit. Which confirmation, and the other conjurations,
also, they have now committed to their suffragans;
because they themselves have no leisure to 
minister such things, for their lusts and pleasures, and 
abundance of all things; and for the cumbrance that they 
have in the king's matters and business of the realm. One 
keepeth the privy seal; another the great seal; the third is 
confessor, (that is to say, a privy traitor and a secret Judas,) 
he is president of the prince's council; he is an ambassador;
another sort of the king's secret council. Woe 
is unto the realms where they are of the council. As profitable
are they, verily, unto the realms with their counsel, 
as the wolves unto the sheep, or the foxes unto the geese. 

They will say that the Holy Ghost is given through 
such ceremonies. If God had so promised, so should it 
be; but Paul saith, (Gal. iii.) that the Spirit is received 
through preaching of the faith. And (Acts x.) while 
Peter preached the faith, the Holy Ghost fell on Cornelius
and on his household. How shall we say then to that 
which they will lay against us, in the eighth chapter of 
the Acts of the Apostles, where Peter and John put their 
hands on the Samaritans, and the Holy Ghost came? I 
say, that by putting, or with putting, or as they put their 
hands on them, the Holy Ghost came. Nevertheless, the 
putting on of the hands did neither help nor hinder. For 
the text saith, They prayed for them that they might receive 
the Holy Ghost. 

God had made the apostles a promise, that he would 
with such miracles confirm their preaching, and move 
other to the faith. (Mark, the last.) The apostles, therefore,
believed and prayed God to fulfil his promise; and 
God, for his truth's sake, even so did. So was it the 
prayer of faith that brought the Holy Ghost, as thou 
mayest see also in the last of James. If any man be sick, 
saith James, call the elders of the congregation, and let 
them pray over him, anointing him with oil in the name 
of the Lord, and the prayer of faith shall heal the sick. 
Where a promise is, there is faith bold to pray, and God 
true to give her her petition. Putting on of the hands is 
an indifferent thing. For the Holy Ghost came by 
preaching of the faith, and miracles were done at the 
prayer of faith, as well without putting on of the hands 
as with, as thou seest in many places. Putting on of the 
hands was the manner of that nation, as it was to rend 
their clothes, and to put on sackcloth, and to sprinkle 
themselves with ashes and earth, when they heard of or 
saw any sorrowful thing, as it was Paul's manner to 
stretch out his hand when he preached. And as it 
is our manner to hold up our hands when we pray, and 
as some kiss their thumb nail, and put it to their eyes, 
and as we put our hands on children's heads when we 
bless them, saying, Christ bless thee, my son, and God 
make thee a good man: which gestures neither help nor 
hinder. This mayest thou well see by the xiith of the 
Acts, where the Holy Ghost commanded to separate 
Paul and Barnabas, to go and preach. Then the other 
fasted and prayed, and put their hands on their heads and 
sent them forth. They received not the Holy Ghost 
then by putting on of hands, but the other as they put 
their hands on their heads prayed for them, that God 
would go with them, and strength them, and couraged 
them also, bidding them to be strong in God, and 
warned them to be faithful and diligent in the work of 
God, and so forth. 


ANOILING. 

LAST of all cometh the anoiling, without promise, and 
therefore without the Spirit, and without profit, but altogether
unfruitful and superstitious. The sacraments, 
which they have imagined are all without promise, and 
therefore help not. For whatsoever is not of faith is sin. 
(Rom. xv.) Now without a promise can there be no
faith. The sacraments which Christ himself ordained,
which have also promises, and would save us if we knew
them, and believed them, them minister they in the Latin
tongue. So are they also become as unfruitful as the
other. Yea they make us believe that the work itself 
without the promise saveth us, which doctrine they learned 
of Aristotle. And thus are we become an hundred times 
worse than the wicked Jews, which believed that the very 
work of their sacrifice justified them. Against which 
Paul fighteth in every Epistle, proving that nothing helpeth 
save the promises which God hath sworn in Christ. 
Ask the people what they understand by their baptism or 
washing? And thou shalt see, that they believe, how that 
the very plunging into the water saveth them: of the promises
they know not, nor what is signified thereby. Baptism
is called volowing in many places of England, because 
the priest saith, Volo say ye. The child was well volowed 
(say they) yea, and our vicar is as fair a volower as ever a 
priest within this twenty miles. 

Behold how narrowly the people look on the ceremony. 
If ought be left out, or if the child be not altogether dipt 
in the water, or if, because the child is sick, the priest 
dare not plunge him into the water, but pour water on his 
head, how tremble they! how quake they! how say ye, 
Sir John, say they, is this child christened enough? hath it 
his full Christendom? They believe verily that the child 
is not christened; yea I have known priests, that have 
gone unto the orders again, supposing that they were not 
priests, because that the bishop left one of his ceremonies 
undone. That they call confirmation, the people call bishoping.
They think that if the bishop butter the child 
in the forehead, that it is safe. They think that the work 
maketh safe, and likewise suppose they of anoiling. Now 
is this false doctrine, verily. For James saith in the first 
chapter of his Epistle: Of his good will begat he us with 
the word of life; that is, with the word of promise. In 
which we are made God's sons, and heirs of the goodness 
of God, before any good works. For we cannot work 
God's will till we be his sons, and know his will and 
have his Spirit to teach us. And St. Paul saith in the 
fifth chapter of his Epistle to the Ephesians: Christ 
cleansed the congregation in the fountain of water, through 
the word. And Peter saith in the first of his first Epistle: 
Ye are born anew, not of mortal seed, but of immortal 
seed, by the word of God, which liveth and lasteth ever. 
Paul in every epistle warneth us, that we put no trust in 
works, and to beware of persuasions or arguments of man's 
wisdom, of superstitiousness, of ceremonies, of pope-holiness,
and of all manner [of] disguising. And exhorteth 
us to cleave fast unto the naked and pure word of God. 
The promise of God is the anchor that saveth us in all 
temptations. If all the world be against us, God's word 
is stronger than the world. If the world kill us that shall 
make us alive again. If it be possible for the world to 
cast us into hell from thence, yet shall God's word bring 
us again. Hereby seest thou that it is not the work, but 
the promise that justifieth us through faith. Now where 
no promise is, there can no faith be, and therefore no justifying,
though there be never so glorious works. The 
sacrament of Christ's body after thiswise preach they. 
Thou must believe that it is no more bread, but the very 
body of Christ, flesh, blood, and bone; even as he went 
here on earth, save his coat. For that is here yet, I wot 
not in how many places. I pray thee what helpeth all 
this? Here is no promise. The devils know that Christ 
died on a Friday, and the Jews also. What are they holp 
thereby? We have a promise that Christ, and his body, 
and his blood, and all that he did, and suffered, is a sacrifice,
a ransom, and a full satisfaction for our sins; that 
God for his sake will think no more on them, if they have 
power to repent and believe. 

Holy workmen think that God rejoiceth in the deed 
self, without any farther respect. They think also that 
God, as a cruel tyrant, rejoiceth, and hath delectation in 
our pain taking without any farther respect. And therefore
many of them martyr themselves without cause, after 
the ensample of Baal's priests which (2 Kings xviii.) cut 
themselves to please their god withal, and as the old heathen
pagans sacrificed their children in the fire unto their 
gods. The monks of the Charterhouse think that the very 
eating of fish in itself pleaseth God, and refer not the 
eating to the chastening of the body. For when they have 
slain their bodies with cold phlegm of fisheating; yet then 
will they eat no flesh, and slay themselves before their 
days. We also, when we offer our sons or daughters, and 
compel or persuade them to vow and profess chastity, 
think that the very pain, and that rage and burning which 
they suffer in abstaining from a make, pleaseth God; and 
so refer not our chastity to our neighbour's profit. For 
when we see thousands fall to innumerable diseases thereby,
and to die before their days; yea, though we see them 
break the commandments of God daily, and also of very 
impatiency work abominations against nature, too shameful
to be spoken of: yet will we not let them marry, but 
compel them to continue still with violence. And thus 
teach our divines, as it appeareth by their arguments. He 
that taketh most pain, say they, is greatest; and so forth. 

The people are thoroughly brought in belief that the 
deed in itself without any farther respect saveth them; 
if they be so long at church; or say so many paternosters; 
and read so much in a tongue which they understand not; 
or go so much a pilgrimage; and take so much pain; or 
fast such a superstitious fast; or observe such a superstitious
observance, neither profitable to himself nor to his 
neighbour: but done of a good intent only say they, to 
please God withal. Yea, to kiss the pax they think it a meritorious
deed; when to love their neighbour, and to forgive
him, (which thing is signified thereby,) they study not 
to do, nor have power to do, nor think that they are 
bound to do it, if they be offended by him. So sore 
have our false prophets brought the people out of their 
wits, and have wrapped them in darkness, and have rocked 
them asleep in blindness and ignorance. Now is all such 
doctrine false doctrine, and all such faith, false faith. For 
the deed pleaseth not, but as far forth as it is applied to 
our neighbour's profit, or the taming of our bodies to keep 
the commandment. 

Now must the body be tamed only, and that with the 
remedies that God hath ordained, and not killed. Thou 
must not forswear the natural remedy which God hath ordained;
and bring thyself into such case that thou shouldest 
either break God's commandment, or kill thyself, or 
burn night and day without rest, so that thou canst not 
once think a godly thought. Neither is it lawful to forsake 
thy neighbour; and to withdraw thyself from serving him, 
and to get thee into a den, and live idly, profitable to no 
man; but robbing all men, first of faith, and then of goods 
and land, and of all he hath; with making him believe in 
the hypocrisy of thy superstitious prayers, and pope-holy 
deeds. The prayer of faith, and the deeds thereof that
spring of love, are accepted before God. The prayer is
good, according to the proportion of faith; and the deed,
according to the measure of love. Now he that bideth
in the world, as monks call it, hath more faith than the
cloisterer. For he hangeth on God in all things. He
must trust God to send him good speed, good luck,
favour, help, a good master, a good neighbour, a good 
servant, a good wife, a good chapman-merchant, to send 
his merchandise safe to land, and a thousand like. He 
loveth also more, which appeareth in that he doth service always
unto his neighbour. To pray one for another are we 
equally bound; and to pray is a thing that we may always 
do whatsoever we have in hand; and that to do may no man 
hire another: Christ's blood hath hired us already. Thus 
in the deed delighteth God, as far forth as we do it, either 
to serve our neighbour withal, (as I have said) or to tame 
the flesh that we may fulfil the commandment from the 
bottom of the heart. 

And as for our pain taking, God rejoiceth not therein 
as a tyrant; but pitieth us, and as it were mourneth with 
us, and is alway ready and at hand to help us, if we call, 
as a merciful father and a kind mother. Neverthelater 
he suffereth us to fall into many temptations, and much 
adversity: yea, himself layeth the cross of tribulation on 
our backs, not that he rejoiceth in our sorrow, but to 
drive sin out of the flesh, which can none otherwise be 
cured: as the physician and surgeon do many things which 
are painful to the sick, not that they rejoice in the pains 
of the poor wretches, but to persecute and to drive out 
the diseases which can no otherwise be healed. 

When the people believe, therefore, if they do so 
much work, or suffer so much pain, or go so much a 
pilgrimage, that they are safe, [it] is a false faith. For a 
Christian man is not saved by works, but by faith in the 
promises before all good works; though that the works 
(when we work God's commandment with a good will, 
and not works of our own imagination) declare that we 
are safe, and that the Spirit of Him that hath made us 
safe is in us: yea, and as God throngh preaching of faith 
doth purge and justify the heart, even so through working 
of deeds, doth he purge and justify the members, making 
us perfect both in body and soul, after the likeness of 
Christ. 

Neither needeth a Christian man to come hither or 
thither, to Rome, to Jerusalem, or St. James; or any 
other pilgrimage far or near, to be saved thereby, or to 
purchase forgiveness of his sins. For a Christian man's 
health and salvation is within him: even in his mouth. 
(Rom. x.) The word is nigh thee, even in thy mouth, 
and in thine heart; that is, the word of faith which we 
preach, saith Paul. If we believe the promises with our 
hearts, and confess them with our mouths, we are safe. 
This is our health within us. But how shall they believe 
that they hear not? And how shall they hear without a 
preacher? saith Paul. (Rom. x.) For look on the promises
of God, and so are all our preachers dumb. Or 
if they preach them, they so sauce them and leaven them, 
that no stomach can brook them, nor find any savour in 
them. For they paint us such an ear confession, as is 
impossible to be kept, and more impossible that it should 
stand with the promises and testament of God. And 
they join them penance, as they call it, to fast, to go 
pilgrimages, and give so much to make satisfaction withal. 
They preach their masses, their merits, their pardons, 
their ceremonies, and put the promise clean out of possession.
The word of health and salvation is nigh thee, 
in thy mouth and thine heart, saith Paul. Nay, say they, 
thy salvation is in our faithful care. This is their 
hold, thereby know they all secrets, thereby mock 
they all men, and all mens' wives; and beguile knight 
and squire, lord and king, and betray all realms. The 
bishops with the pope have a certain conspiration and 
secret treason against the whole world. And by confession,
know they what kings and emperors think. If 
ought be against them, do they never so evil, then move 
they their captives to war and to fight, and give them 
pardons to slay whom they will have taken out of the way. 
They have with falsehood taken from all kings and emperors 
their right and duties, which now they call their freedoms, 
liberties, and privileges, and have perverted the ordinances 
that God left in the world, and have made every king swear 
to defend their falsehood against their ownselves. So that 
now, if any man preach God's word truly, and show 
the freedom and liberty of the soul which we have 
in Christ, or intend to restore the kings again to their 
duties and right, and to the room and authority which 
they have of God, and of shadows to make them kings 
in deed, and to put the world in his order again: then the 
kings deliver their swords and authority unto the hypocrites, 
to slay him. So drunken are they with the wine of 
the whore. 

The text that followeth, in Paul, will they haply lay 
to my charge and others. How shall they preach, except 
they be sent, saith Paul in the said xth to the Romans. We 
(will they say,) are the pope, cardinals and bishops: all 
authority is ours. The Scripture pertaineth unto us, and 
is our possession. And we have a law, that whosoever 
presume to preach without the authority of the bishops, 
is excommunicate in the deed-doing. Whence, therefore,
hast thou thine authority? will they say. The old 
pharisees had the Scripture in captivity likewise, and 
asked Christ, By what authority doest thou these things? 
As who should say, we are pharisees, and thou art none of our 
order, nor hast authority of us. Christ asked them another 
question, and so will I do our hypocrites. Who sent you? 
God? Nay, he that is sent of God speaketh God's 
word. (John iii.) Now speak ye not God's word, nor 
any thing save your own laws, made clean contrary unto 
God's word. Christ's apostles preached Christ, and not 
themselves. He that is of the truth preacheth the truth. 
Now ye preach nothing but lies, and therefore are of 
the devil, the father of all lies, and of him are ye sent. 
And as for mine authority, or who sent me, I report me 
unto my works as Christ. (John v. and x.) If God's 
word bear record that I say truth, why should any man 
doubt, but that God, the father of truth and of light, hath 
sent me; as the father of lies and of darkness hath sent 
you, and that the Spirit of truth and of light is with me, 
as the spirit of lies and of darkness is with you? By this 
means thou wilt that every man be a preacher, will they say. 
Nay, verily. For God will that not, and therefore will 
I it not; no more than I would that every man of London 
were mayor of London, or every man of the realm king 
thereof. God is not the author of dissention and strife, 
but of unity and peace, and of good order. I will, therefore,
that where a congregation is gathered together in 
Christ, one be chosen after the rule of Paul, and that he 
only preach, and else no man openly; but that every man 
teach his household after the same doctrine. But if the 
preacher preach false: then whosoever's heart God moveth, 
to the same it shall be lawful to rebuke and improve the 
false teacher, with the clear and manifest Scripture, and 
that same is no doubt a true prophet sent of God. For 
the Scripture is God's, and theirs that believe, and not 
the false prophet's. 

Sacrament is then as much to say as an holy sign. 
And the sacraments, which Christ ordained, preach God's 
word unto us; and therefore justify and minister the 
Spirit to them that believe, as Paul through preaching the 
gospel was a minister of righteousness, and of the Spirit, 
unto all that believed his preaching. Dumb ceremonies 
are no sacraments, but superstitiousness. Christ's sacraments
preach the faith of Christ, as his apostles did, and 
thereby justify. Antichrist's dumb ceremonies preach not 
the faith that is in Christ, as his apostles, our bishops and 
cardinals, do not. But as antichrist's bishops are ordained 
to kill whosoever preach the true faith of Christ; so are 
his ceremonies ordained to quench the faith which Christ's 
sacraments preach. And hereby mayest thou know the
difference between Christ's signs or sacraments, and antichrist's
signs or ceremonies — that Christ's signs speak, and
antichrist's be dumb.

Hereby seest thou what is to be thought of all other 
ceremonies — as hallowed water, bread, salt, boughs, bells, 
wax, ashes, and so forth; and all other disguisings and apes'- 
play; and of all manner [of] conjurations, as the conjuring 
of church and churchyards, and of altar-stones, and such like. 
Where no promise of God is, there can be no faith, nor justifying,
nor forgiveness of sins. For it is more than madness
to look for any thing of God, save that he hath promised:
how far he hath promised, so far is he bound to them 
that believe, and further not. To have a faith, therefore,
or a trust in any thing, where God hath not promised, is
plain idolatry, and a worshipping of thine own imagination
instead of God. Let us see the pith of a ceremony
or two, to judge the rest by. In conjuring of holy 
water, they pray, that whosoever be sprinkled therewith 
may receive health as well of body as of soul: and likewise
in making holy bread, and so forth, in the conjurations
of other ceremonies. Now we see by daily experience,
that half their prayer is unheard. For no man 
receiveth health of body thereby. No more, of likelihood, 
do they of soul. Yea, we see also by experience, that 
no man receiveth health of soul thereby. For no man by 
sprinkling himself with holy water, and with eating holy bread, 
is more merciful than before, or forgiveth wrong, or becometh
at one with his enemy, or is more patient and less 
covetous, and so forth. Which are the sure tokens of the 
soul's health. 

They preach also that the wagging of the bishop's hand 
over us blesseth us, and putteth away our sins. Are these 
works not against Christ? How can they do more shame 
unto Christ's blood? For if the wagging of the bishop's 
hand over me be so precious a thing in the sight of God 
that I am thereby blessed, how then am I full blessed with 
all spiritual blessing in Christ? as Paul saith: (Eph. i.) 
Or if my sins be full done away in Christ, how remaineth 
there any to be done away by such phantasies? The 
apostles knew no ways to put away sin, or to bless us, but by 
preaching Christ. Paul saith (Gal. ii.) If righteousness 
come by the law, then Christ died in vain. So dispute I 
here: If blessing come by the wagging of the bishop's 
hand, then died Christ in vain, and his death blesseth us 
not. And a little afore saith Paul, If while we seek to be 
justified by Christ, we be yet found sinners, (so that we 
must be justified by the law or ceremonies) is not Christ 
then a minister of sin? So dispute I here: If while we 
seek to be blessed in Christ we are yet unblessed, and 
must be blessed by the wagging of the bishop's hand, what 
have we then of Christ but curse? Thou wilt say: when we 
come first to the faith, then Christ forgiveth us and 
blesseth us. But the sins which we afterward commit are 
forgiven us through such things. I answer, if any man 
repent truly, and come to the faith, and put his trust in 
Christ, then as oft as he sinneth of frailty, at the sigh of the 
heart is his sin put away in Christ's blood. For Christ's 
blood purgeth ever and blesseth ever. For John saith in 
the second of his First Epistle, This I write unto you that 
ye sin not. And though any man sin (meaning of frailty, 
and so repent) yet have we an Advocate with the Father, 
Jesus Christ which is righteous, and he it is that obtaineth 
grace for our sins. And (Heb. vii.) it is written, But this 
man (meaning Christ) because he lasteth or abideth ever, 
hath an everlasting priesthood. Therefore is he able also 
ever to save them that come to God through him, seeing 
he ever liveth to make intercession for us. The bisbops 
therefore ought to bless us in preaching Christ, and not 
to deceive us and to bring the curse of God upon us with 
wagging their hands over us. To preach is their duty 
only, and not to offer their feet to be kissed, or testicles or 
stones to be groped. We feel also by experience that 
after the pope's, bishop's, or cardinal's blessing, we are 
no otherwise disposed in our souls than before. 

Let this be sufficient as concerning the sacraments and
ceremonies, with this protestation: that if any can say 
better or improve this with God's word, no man shall be 
better content therewith than I. For I seek nothing but 
the truth, and to walk in the light. I submit therefore this 
work and all other that I have made or shall make, (if 
God will that I shall more make) unto the judgments, 
not of them that furiously burn all truth, but of them 
which are ready with God's word to correct, if any thing 
be said amiss, and to further God's word. 

I will talk a word or two after the worldly wisdom with 
them, and make an end of this matter. If the sacraments 
justify, as they say, (I understand by justifying, forgiveness 
of sins,) then do they wrong unto the sacraments, inasmuch
as they rob the most part of them through confession 
of their effect, and of the cause wherefore they were ordained.
For no man may receive the body of Christ, no 
man may marry, no man may be oiled or anoiled as they call 
it; no man may receive orders, except he be first shriven. 
Now when the sins be forgiven by shrift aforehand, there 
is nought left for the sacraments to do. They will answer 
that at the leastway they increase grace, and not the sacraments
only, but also hearing of mass, matins and evensong;
and receiving of holy water, holy bread, and of the 
bishop's blessing, and so forth by all ceremonies. By 
grace I understand the favour of God, and also the gifts 
and working of his Spirit in us; as love, kindness, 
patience, obedience, mercifulness, despising of worldly 
things, peace, concord, and such like. If after thou hast 
heard so many masses, matins, and evensongs; and after 
thou hast received holy bread, holy water, and the bishop's 
blessing, or a cardinal's or the pope's; if thou wilt be 
more kind to thy neighbour, and love him better than 
before; if thou be more obedient unto thy superiors; more 
merciful, more ready to forgive wrong done unto thee, 
more despisest the world, and more athirst after spiritual 
things; if after that a priest hath taken orders he be less 
covetous than before; if a wife after so many and oft pilgrimages
be more chaste, more obedient unto her husband,
more kind to her maids and other servants; if gentlemen,
knights, lords, and kings and emperors, after they 
have said so often daily service with their chaplains, know 
more of Christ than before, and can better skill to rule their 
tenants, subjects, and realms christianly than before, and 
be content with their duties; then do such things increase 
grace. If not, it is a lie. Whether it be so or no, I report 
me to experience. If they have any other interpretations 
of justifying or grace, I pray them to teach it me. For I 
would gladly learn it. Now let us go to our purpose 
again. 


OF MIRACLES AND WORSHIPPING OF SAINTS. 

ANTICHRIST shall not only come with lying signs,
and disguised with falsehood, but also with lying miracles
and wonders, saith Paul in the said place, (2 Thess. 
ii.) All the true miracles which are of God, are shewed (as 
I above rehearsed) to move us to hear God's word, and to 
stablish our faith therein: and to confirm the truth of 
God's promises, that we might without all doubting believe 
them. For God's word, through faith, bringeth the Spirit 
into our hearts and also life, as Christ saith, (John vi.) 
The words which I speak are spirit and life. The word 
also purgeth us and cleanseth us, as Christ saith, (John 
xv.) Ye are clean by the means of the word. Paul saith, 
(1 Tim. ii.) One God, one Mediator (that is to say, advocate,
intercessor, or an at-one-maker) between God and 
man: the man Christ Jesus which gave himself a ransom 
for all men. Peter saith of Christ (Acts iv.) Neither is there 
health in any other: neither yet also any other name 
given unto men wherein we must be saved. So now 
Christ is our peace, our redemption or ransom for our sins, 
our righteousness, satisfaction, and all the promises of God 
are yea and Amen in him; (2 Cor. i.) And we, for the 
infinite love which God hath to us in Christ, love him 
again, love also his laws, and love one another. And the
deeds which we henceforth do, do we not to make satisfaction
or to obtain heaven: but to succour our neighbour, to
tame the flesh, that we may wax perfect and strong men 
in Christ, and to be thankful to God again for his mercy, 
and to glorify his name. 

Contrariwise the miracles of antichrist are done to
pull thee from the word of God, and from believing his promises,
and from Christ, and to put thy trust in a man, or a
ceremony wherein God's word is not. As soon as God's word 
is believed, the faith spread abroad, then cease the miracles 
of God. But the miracles of antichrist, because they are 
wrought by the devil, to quench the faith, grow daily more 
and more: neither shall cease until the world's end among 
them that believe not God's word and promises. Seest 
thou not how God loosed and sent forth all the devils in 
the old world among the heathen or Gentiles? and how 
the devils wrought miracles and spake to them in every 
image? even so shall the devil work falsehood by one craft 
or another, until the world's end among them that believe 
not God's word. For the judgment and damnation of him 
that hath no lust to hear the truth, is to hear lies, and to 
be stablished and grounded therein through false miracles; 
and he that will not see, is worthy to be blind, and he that
biddeth the Spirit of God go from him, is worthy to be 
without him. 

Paul, Peter, and all true apostles preached Christ only.
And the miracles did but confirm and stablish their
preaching, and those everlasting promises and eternal testament
that God had made between man and him in
Christ's blood; and the miracles did testify also that they 
were true servants of Christ. Paul preached not himself,
he taught not any man to trust in him or his holiness, or in
Peter or in any ceremony, but in the promises which God 
hath sworn only; yea, he mightily resisteth all such false
doctrine both to the Corinthians, Galatians, Ephesians,
and everywhere. If this be true (as it is true and nothing 
more true,) that if Paul had preached himself, or taught 
any man to believe in his holiness or prayer, or in any thing,
save in the promises that God hath made and sworn to
give us for Christ's sake, he had been a false prophet:
why am not I also a false prophet, if I teach thee to trust 
in Paul or in his holiness or prayer, or in any thing save 
in God's word, as Paul did? 

If Paul were here and loved me, (as he loved them 
his time of whom he was sent, and to whom he was a servant 
to preach Christ,) what good could he do for me or wish 
me, but preach Christ and pray to God for me, to open 
mine heart, to give me his Spirit, and to bring me unto the 
full knowledge of Christ: unto which port or haven, when 
I am once come, I am as safe as Paul, fellow with Paul, 
joint heir with Paul of all the promises of God, and God's 
truth heareth my prayer as well as Paul's. I also now could 
not but love Paul and wish him good, and pray for him, 
that God would strengthen him in all his temptations and 
give him victory, as he would do for me. Nevertheless 
there are many weak and young consciences always in 
the congregation, which they that have the office to preach, 
ought to teach, and not to deceive them. 

What prayers pray our clergy for us, which stop us and 
exclude us from Christ, and seek all the means possible to 
keep us from knowledge of Christ! They compel us to 
hire friars, monks, nuns, canons, and priests, and to 
buy their abominable merits, and to hire the saints that 
are dead to pray for us, (for the very saints have they made 
hirelings also:) because that their offerings come to their 
profit. What pray all those? that we might come to the 
knowledge of Christ, as the apostles did? Nay verily, For 
it is a plain case, that all they which enforce to keep us 
from Christ, pray not that we might come to the knowledge
of Christ. And as for the saints, (whose prayer was, 
when they were alive, that we might be grounded, established
and strengthened in Christ only) if it were of God 
that we should thiswise worship them contrary unto their 
own doctrine, I dare be bold to affirm, that by the means 
of their prayers we should have been brought long ago 
unto the knowledge of God and Christ again, though that 
these beasts had done their worst to let it. Let us therefore
set our hearts at rest in Christ and in God'a promises, 
for so I think it best; and let us take the saints for an ensample
only, and let us do as they both taught and did. 

Let us set God's promises before our eyes, and desire 
him for his mercy and for Christ's sake to fulfil them. 
And he is as true as ever he was, and will do it, as well 
as ever he did, for to us are the promises made as well 
to them. 

Moreover, the end of God's miracles is good; the end to 
these miracles are evil. For the oferings which are the 
cause of the miracles do but minister and maintain vice, 
sin, and all abomination, and are given to them that have 
too much; so that for very abundance they foam out their 
own shame, and corrupt the whole world with the stench 
of their filthiness. 

Thereto whatsoever is not of faith is sin. (Rom. xiv. 
Faith cometh by hearing God's word. (Rom. x.) When 
now thou fastest or doest any thing in the worship of any
saint, believing to come to the favour of God, or to be saved 
thereby if thou have God's word, then is it true faith and 
shall save thee. If thou have not God's word, then is it a
false faith, superstitiousness and idolatry, and damnable 
sin. 

Also in the collects of the saints, with which we pray
God to save us through the merits or deservings of the 
saints, (which saints yet were not saved by their own deservings
themselves) we say, per Christum dominum nostrum;
that is, for Christ our Lord's sake. We say, Save us 
good Lord, through the saints' merits for Christ's sake.
How can he save us through the saints' merits for Christ's 
sake and for his deserving merits and love? Take an ensample.
A gentleman saith unto me, I will do the uttermost
of my power for thee, for the love which I owe unto 
thy father. Though thou hast never done me pleasure, 
yet I love thy father well; thy father is my friend, and hath 
deserved that I do all that I can for thee. \&c. Here is a 
testament and a promise made unto me in the love of my 
father only. If I come to the said gentleman in the name 
of one of his servants which I never saw, never spake with, 
neither have any acquaintance at all with, and say: Sir, 
I pray you be good master unto me in such a cause. I
have not deserved that he should so do. Nevertheless I
pray you do it for such a servant's sake: yea, I pray you
for the love that you owe to my father, do that for me for
such a servant's sake. If I thiswise made my petition,
would not men think that I came late out of St. Patrick's
purgatory, and had left my wits behind me? This do we.
For the Testament and promises are all made unto us in
Christ. And we desire God to fulfil his promises for
the saints' sake: yea, that he will for Christ's sake, do it
for the saints' sake.

They have also martyrs which never preached God's
word, neither died therefore: but for privileges and liberties
which they falsely purchased contrary unto God's ordinances.
Yea, and such saints, though they be dead,
yet rob now as fast as ever they did, neither are less covetous
now than when they were alive. I doubt not but that
they will make a saint of my lord cardinal, after the death
of us that be alive, and know his juggling and crafty conveyance,
and will shrine him gloriously, for his mightily
defending of the right of the holy church, except we
diligent to leave a commemoration of that Nimrod behind
us. 

The reasons wherewith they prove their doctrine are
but fleshly: and as Paul calleth them, Enticing words of
man's wisdom; that is to wit, sophistry and brawling arguments
of men with corrupt minds and destitute of the truth,
whose God is their belly, unto which idol whosoever offereth
not, the same is an heretic, and worthy to be burnt. 

The saint was great with God when he was alive, as it 
appeareth by the miracles which God shewed for him; he 
must therefore be great now, say they. This reason appeareth
wisdom, but it is very foolishness with God. For 
the miracle was not shewed that thou should put thy trust
in the saint, but in the word which the saint preached; 
which word, if thou believest, would save thee, as God 
hath promised and sworn, and would make thee also great 
with God, as it did the saint. 

If a man have a matter with a great man, or a king, he 
must go first unto one of his mean servants, and then higher 
and higher till he come at the king. This enticing argument 
is but blind reason of man's wit. It is not like in the
kingdom of the world, and in the kingdom of God and 
Christ. 

With kings, for the most part, we have none acquaintance,
neither promise. They be also most commonly 
merciless. Moreover if they promise, they are yet men, 
as unconstant as are other people, and as untrue. But 
with God, if we have belief, we are accounted, and have 
an open way in unto him by the door Christ, which is 
never shut, but through unbelief; neither is there any porter
to keep any man out. By him, saith Paul, (Eph. ii.)
that is to say, by Christ, we have an open way in unto the 
Father. So are ye now no more strangers and foreigners,
(saith he) but citizens with the saints, and of the household
of God. God hath also made us promises, and hath 
sworn: yea, hath made a testament or a covenant, and hath 
bound himself, and hath sealed his obligation with Christ's 
blood, and continued it with miracles. He is also merciful
and kind, and complaineth that we will not come unto 
him. He is mighty and able to perform that he promiseth.
He is true, and cannot be but true, as he cannot be but
God. Therefore is it not like with the king and God. 

We be sinners, say they, God will not hear us. Behold 
how they flee from God as from a tyrant merciless. Whom 
a man counteth most merciful, unto him he soonest flieth. 
But these teachers dare not come at God. Why? for 
they are the children of Cain. If the saints love whom 
God hateth, then God and his saints are divided. When 
thou prayest to the saints, how do they know, except that
God whom thou countest merciless, tell them? If God be
so cruel and so hateth thee, it is not likely that he will tell
the saints that thou prayest unto them. 

When they say we be sinners: I answer, that Christ is 
no sinner, save a satisfaction, and an offering for sin. 
Take Christ from the saints and what are they? what is 
Paul without Christ? Is he any thing save a blasphemer 
a persecutor, a murderer, and a shedder of Christian blood? 
But as soon as he came to Christ he was no more a sinner, 
but a minister of righteousncas: he went not to Rome to 
take penance upon him, but went and preached unto his 
brethren the same mercy, which he had received free, 
without doing penance or hiring of saints, or of monks, or 
friars. Moreover, if it be God's word that thou should 
put thy trust in the saints' merits or prayers, then be bold. 
For God's word shall defend thee and save thee. If it be 
but thine own reason, then fear. For God commandeth 
by Moses, (Deut. xii.) saying: What I command you, 
that observe and do, and put nothing to, nor take ought 
therefrom: yea, and Moses warneth straightly in an hundred 
places, that we do that only which God commandeth, and 
which seemeth good and righteous in his sight, and not in
our own sight. For nothing bringeth the wrath of God 
so soon and so sore on a man, as the idolatry of his own 
imagination. 

Last of all, these arguments are contrary to the arguments
of Christ and of his apostles. Christ disputeth 
(Luke xi.) saying, If the son ask the father bread, will he 
give him a stone? or if he ask him fish, will he give him a 
serpent? and so forth. If ye then (saith he,) which are 
evil can give good gifts to your children, how much rather 
shall your heavenly Father give a good spirit unto them 
that ask him? And a little before in the same chapter he 
saith, If a man came never so out of season to his neighbour
to borrow bread, even when he is in his chamber, 
and the door shut, and all his servants with him; nevertheless,
yet if he continue knocking and praying, he will 
rise and give him as much as he needeth, though not for 
love, yet to be rid of him, that he may have rest. As who 
should say, What will God do if a man pray him, seeing 
that prayer overcometh an evil man? Ask, therefore, 
(saith he,) and it shall be given you; seek, and ye shall 
find; knock, and it shall be opened unto you. And 
(Luke xviii.) he putteth forth the parable, or similitude, 
of the wicked judge, which was overcome with the importunate
prayer of the widow. And concludeth, saying, 
Hear what the wicked judge did. And shall not God 
avenge his elect, which cry unto him night and day? 
Whether, therefore, we complain of the intolerable oppression
and persecution that we suffer, or of the flesh that 
cumbreth and resisteth the Spirit, God is merciful to hear 
us, and to help us. Seest thou not also how Christ cureth 
many, and casteth out devils out of many, unspoken to, 
how shall he not help, if he be desired and spoken to? 

When the old Pharisees (whose nature is to drive 
sinners from Christ,) asked Christ why he did eat with 
publicans and sinners? Christ answered, That the whole 
needed not the physician, but the sick; that is, he came to 
have conversation with sinners to heal them. He was a 
gift given unto sinners, and a treasure to pay their debts. 
And Christ sent the complaining and disdaining Pharisees 
to the prophet Hosea, saying, Go and learn what this 
meaneth; I desire, or require mercy, and not sacrifice. As 
who should say, Ye Pharisees love sacrifice and offering 
for to feed that god your bellies withal, but God commandeth
to be merciful. Sinners are ever captives, and 
a prey to the Pharisees and hypocrites for to offer unto 
their bellies, and to buy merits, pardons, and forgiveness 
of sins of them. And therefore fear they them away from 
Christ, with arguments of their belly-wisdom. For he 
that receiveth forgiveness free of Christ, will buy no forgiveness
of them. I came (saith Christ,) to call, not the 
righteous, but the sinners unto repentance. The Pharisees
are righteous, and therefore have no part with Christ, 
neither need they; for they are gods themselves and 
saviours. But sinners that repent pertain to Christ. 
If we repent, Christ hath made satisfaction for us 
already. 

God so loved the world that he gave his only Son, that 
none that believe on him should perish, but should have 
everlasting life. For God sent not his Son into the world 
to condemn the world, but that the world through him 
might be saved. He that believeth on him shall not be 
damned; but be that believeth not is damned already. 
(John iii.) 

Paul (Rom. v.) saith, Because we are justified through 
faith, we are at peace with God through our Lord Jesus 
Christ; that is, because that God, which cannot lie, hath 
promised and sworn to be merciful unto us, and to forgive
us for Christ's sake, we believe, and are at peace in 
our consciences; we run not hither and thither for pardon; 
we trust not in this friar nor that monk, neither in anything,
save in the word of God only. As a child, when 
his father threateneth him for his fault, hath never rest till 
he hear the word of mercy and forgiveness of his father's 
mouth again; but as soon as he heareth his father say, 
Go thy way, do me no more so, I forgive thee this fault: 
then is his heart at rest — then is he at peace, — then 
runneth he to no man to make intercession for him. 
Neither, though there come any false merchant, saying, 
What wilt thou give me, and I will obtain pardon of thy 
father for thee? will he suffer himself to be beguiled. 
No, he will not buy of a wily fox for that which his father 
hath given him freely. 

It followeth, God setteth out his love that he hath to us; 
that is, he maketh it appear, that men may perceive love if 
they be not more than stock blind. Inasmuch (saith 
Paul,) as while we were yet sinners, Christ died for us. 
Much more now, (saith he,) seeing we are justified by his 
blood, shall we be preserved from wrath through him: 
for if when we were enemies we were reconciled to God, 
by the death of his Son; much more, seeing we are reconciled,
we shall be preserved by his life. As who should 
say, If God loved us when we knew him not, much more 
loveth he us now we know him. If he were merciful to 
us while we hated his law, how much more merciful will 
he be now, seeing we love it, and desire strength to fulfil 
it. And in the viiith he argueth, if God spared not his 
own Son, but gave him for us all, how shall he not with 
him give us all things also? 

Christ prayeth (John xvii.) not for the apostles only, but 
also for as many as should believe through their preaching, 
and was heard; whatsoever we ask in his name the Father 
giveth us. (John xvi.) Christ is also as merciful as the 
saints. Why go we not straightway unto him? Verily, 
because we feel not the mercy of God, neither believe his 
truth. God will at the leastway (say they,) hear us the 
sooner for the saints' sake. Then loveth he the saints 
better than Christ, and his own truth. Heareth he us for 
the saints' sake? so heareth he us not for his mercy: for 
merits and mercy cannot stand together. 

Finally: If thou put any trust in thine own deeds, or 
in the deeds of any other man, of any saint, then minishest 
thou the truth, mercy, and goodness of God. For if God 
look unto thy works, or unto the works of any other man, 
or goodness of the saint, then doth he not all things of 
pure mercy and of his goodness, and for the truth's sake, 
which he hath sworn in Christ. Now saith Paul, (Tit. iii.) 
not of the righteous deeds which we did, but of his 
mercy saved he us. 

Our blind disputers will say, If our good deeds justify 
us not; if God look not on our good deeds, neither regard 
them, nor love us the better for them, what need we to do 
good deeds? I answer, God looketh on our good deeds, 
and loveth them; yet loveth us not for their sakes. God 
loveth us first in Christ of his goodness and mercy, and 
poureth his Spirit into us, and giveth us power to do good 
deeds. And because he loveth us, he loveth our good 
deeds; yea, because he loveth us, he forgiveth us our evil 
deeds, which we do of frailty, and not of purpose or for 
the nonce. Our good deeds do but testify only that we 
are justified and beloved. For except we were beloved, 
and had God's Spirit, we could neither do, nor yet consent
unto any good deed. Antichrist turneth the roots of 
the trees upward. He maketh the goodness of God the 
branches, and our goodness the roots. We must be first 
good after antichrist's doctrine, and move God, and compel
him to be good again for our goodness' sake: so must 
God's goodness spring out of our goodness. Nay, verily, 
God's goodness is the root of all goodness; and our goodness,
if we have any, springeth out of his goodness. 


OF PRAYER. 

OF prayer and good deeds, and of the order of love, or 
charity, I have abundantly written in my book of the 
Justifying of Faith. Neverthelater, that thou mayest see 
what the prayers and good works of our monks and friars, 
and of other ghostly people, are worth, I will speak a 
word or two, and make an end. Paul saith, (Gal. iii.) 
All ye are the sons of God through faith in Jesus Christ; 
for all ye that are baptized have put Christ on you; that 
is, ye are become Christ himself. There is no Jew, (saith 
he,) neither Greek, neither bond nor free, neither man nor 
woman, but ye are all one thing in Christ Jesus. In Christ 
there is neither French nor English; but the Frenchman 
is the Englishman's ownself, and the English the Frenchman's
ownself. In Christ there is neither father nor son, 
neither master nor servant, neither husband nor wife, 
neither king nor subject; but the father is the son's self, 
and the son the father's ownself; and the king is the subject's
ownself, and the subject is the king's ownself; and 
so forth. I am thou thyself, and thou art I myself, and can 
be no nearer of kin. We are all the sons of God, all 
Christ's servants bought with his blood; and every man 
to other, Christ his ownself. And (Col. iii.) Ye have put 
on the new man, which is renewed in the knowledge of 
God, after the image of him that made him (that is to say, 
Christ;) where is (saith he,) neither Greek nor Jew, circumcision
nor uncircumcision, barbarous or Scythian, bond 
or free; but Christ is all in all things. I love thee not 
now because thou art my father, and hast done so much 
for me; or my mother, and hast borne me, and given me
suck of thy breasts, (for so do Jews and Saracens,) but 
because of the great love that Christ hath shewed me. I 
serve thee, not because thou art my master, or my king, 
for hope of reward, or fear of pain, but for the love of 
Christ; for the children of faith are under no law (as 
thou seest in the Epistles to the Romans, to the Galatians, 
in the first to Timothy,) but are free. The Spirit of Christ 
hath written the lively law of love in their hearts, which 
driveth them to work of their own accord freely and willingly,
for the great love's sake only which they see in 
Christ, and therefore need they no law to compel them. 
Christ is all in all things to them that believe, and the cause 
of all love. Paul saith (Eph. vi.) Servants, obey unto 
your carnal or fleshly masters, with fear and trembling, in 
singleness of your hearts as unto Christ: not with eye- 
service, as men-pleasers, but as the servants of Christ, 
doing the will of God from the heart, even as though ye 
served the Lord, and not men. And remember, that 
whatsoever good thing any man doth, that shall he receive 
again of the Lord, whether he be bond or free. Christ 
thus is all in all things, and cause of all to a Christian man. 
And Christ saith, (Matt. xxv.) Inasmuch as ye have done 
it to any of the least of these my brethren, ye have done 
it to me. And inasmuch as ye have not done it unto one 
of the least of these, ye have not done it to me. Here 
seest thou that we are Christ's brethren, and even Christ 
himself; and whatsoever we do one to another, that do we 
to Christ. If we be in Christ, we work for no worldly 
purpose, but of love. As Paul saith (2 Cor. v.) The love 
of Christ compelleth us: (as who should say, we work 
not of a fleshly purpose:) for (saith he,) we know hence- 
forth no man fleshly; no, though we once knew Christ 
fleshly, we do so now no more. We are otherwise minded 
than when Peter drew his sword to fight with Christ. We 
are now ready to suffer with Christ, and to lose life and 
all for our very enemies to bring them unto Christ. If we 
be in Christ, we are minded like unto Christ, which knew 
nothing fleshly, or after the will of the flesh, as thou seest 
Matt. xii. when one said to him, Lo, thy mother and thy 
brethren stand without, desiring to speak with thee. He 
answered, Who is my mother, and who are my brethren? 
And stretched his hand over his disciples, saying, See, my 
mother and my brethren: for whosoever doth the will of 
my Father which is in heaven, the same is my brother, my 
sister, and my mother. He knew not his mother in that 
she bare him, but in that she did the will of his Father in 
heaven. So now, as God the Father's will and commandment
is all to Christ, even so Christ is all to a 
Christian man. 

Christ is the cause why I love thee, why I am ready to 
do the uttermost of my power for thee, and why I pray for 
thee. And as long as the cause abideth, so long lasteth 
the effect: even as it is always day, so long as the sun 
shineth. Do therefore the worst thou canst unto me, take 
away my goods, take away my good name: yet as long 
as Christ remaineth in my heart, so long I love thee not a 
whit the less, and so long art thou as dear unto me as 
mine own soul, and so long am I ready to do thee good 
for thine evil, and so long I pray for thee with all my 
heart: for Christ desireth it of me, and hath deserved it 
of me. Thine unkindness compared unto his kindness is 
nothing at all; yea, it is swallowed up as a little smoke of a 
mighty wind, and is no more seen or thought upon. 
Moreover that evil which thou didst to me, I receive not 
of thy hand, but of the hand of God, and as God's 
scourge to teach me patience, and to nurture me. And 
therefore have no cause to be angry with thee, more than 
the child hath to be angry with his father's rod: or a sick 
man with a sour or bitter medicine that healeth him, or a 
prisoner with his fetters, or he that is punished lawfully 
with the officer that punisheth him, Thus is Christ all 
and the whole cause why I love thee. And to all can 
nought he added. Therefore cannot a little money make 
me love thee better, or more bound to pray for thee, nor 
make God's commandment greater. Last of all, if I be 
in Christ, then the love of Christ compelleth me. And 
therefore I am ready to give thee mine, and not to take 
thine from thee. If I be able I will do thee service freely: 
if not, then if thou minister to me again, that receive I of 
the hand of God, which ministereth it to me by thee: for 
God careth for his and ministereth all things unto them, 
and moveth Turks, and Saracens, and all manner [of] infidels
to do them good; as thou seest in Abraham, Isaac, 
and Jacob, and how God went with Joseph into Egypt, 
and gat him favour in the prison, and in every place, which 
favour Joseph received of the hand of God, and to God 
gave the thanks. This is God and Christ all in all, good 
and bad receive I of God. Them that are good I love, 
because they are in Christ, and the evil, to bring them to 
Christ. When any man doth well I rejoice that God is 
honoured, and when any man doth evil, I sorrow because 
that God is dishonoured. Finally, inasmuch as God 
hath created all, and Christ bought all with his blood, 
therefore ought all to seek God and Christ in all, and else 
nothing. 

But contrariwise unto monks, friars, and to the other of 
our holy spiritualty, the belly is all in all, and cause of all 
love. Offer thereto, so art thou father, mother, sister, and 
brother unto them. Offerest thou not, so know they thee 
not; thou art neither father, mother, sister, brother, nor 
any kin at all to them. She is a sister of ours, he is a 
brother of ours, say they; he is verily a good man, for he 
doth much for our religion. She is a mother to our convent:
we be greatly bound to pray for them. And as for 
such and such, (say they) we know not whether they be 
good or bad, or whether they be fish or flesh, for they do 
nought for us: we be more bound to pray for our benefactors
(say they) and for them that give us, than for them 
that give us not. For them that give little, are they little 
bound, and them they love little: and for them that give 
much, they are much bound, and them they love much. 
And for them that give nought, are they nought bound, 
and them they love not at all. And as they love thee 
when thou givest, so hate they thee when thou takest 
away from them, and run all under a stool, and curse thee 
as black as pitch. So is cloister love belly-love, cloister 
prayer belly-prayer, and cloister brotherhood belly-brotherhood.
Moreover, love that springeth of Christ seeketh 
not her ownself, (1 Cor. xiii.) but forgetteth herself, and 
bestoweth her upon her neighbour's profit, as Christ sought 
our profit, and not his own. He sought not the favour of 
God for himself, but for us; yea, he took the wrath and 
vengeance of God from us unto himself, and bare it on his 
own back, to bring us unto favour. Likewise doth a 
Christian man give to his brethren, and robbeth them not 
as friars and monks do: but as Paul commandeth, 
(Eph. iv.) laboureth with his hands some good work to have 
wherewith to help the needy. They give not, but receive 
only. They labour not, but live idly of the sweat of the 
poor. There is none so poor a widow, though she have 
not to find herself and her children, nor any money to give, 
yet shall the friar snatch a cheese, or somewhat. They 
preach, sayest thou, and labour in the word. First, I say, 
they are not called, and therefore ought not: for it is the 
curate's office. The curate cannot sayest thou. What 
doth the thief there then? Secondarily, a true preacher 
preacheth Christ's Testament only, and maketh Christ the 
cause and reward of all our deeds, and teacheth every man 
to bear his cross willingly for Christ's sake. But these 
are enemies unto the cross of Christ, and preach their 
belly, which is their God: (Phil. iii.) and they think that 
lucre is the serving of God. (1 Tim. vi.) That is, they 
think them Christian only which offer unto their bellies, 
which when thou hast filled, then spue they out prayers 
for thee, to be thy reward, and yet wot not what prayer
meaneth. Prayer is the longing for God's promises, 
which promises, as they preach them not, so long they not 
for them, nor wish them unto any man. Their longing is 
to fill their paunch whom they serve, and not Christ: and 
through sweet preaching and flattering words deceive the 
hearts of the simple and unlearned. (Rom. xvi.) 

Finally, as Christ is the whole cause why we do all 
thing for our neighbour, even so is he the cause why God 
doth all thing for us, why he receiveth us into his holy 
testament, and maketh us heirs of all his promises, and 
poureth his Spirit into us, and maketh us his sons, and 
fashioneth us like unto Christ, and maketh us such as he 
would have us to be. The assurance that we are the sons, 
beloved, and heirs with Christ, and have God's Spirit in 
us, is the consent of our hearts unto the law of God. 
Which law is all perfection, and the mark whereat all we 
ought to shoot. And he that hitteth that mark, so that 
he fulfilleth the law with all his heart, soul, and might, and 
with full love and lust, without all let or resistance, is pure 
gold, and needeth not to be put any more in the fire; he is 
straight and right, and needeth to be no more shaven: he 
is full fashioned like Christ, and can have no more added 
unto him. Nevertheless there is none so perfect in this 
life, that findeth not let and resistance by the reason of original
sin, or birth poison that remaineth in him, as thou 
mayest see in the lives of all the saints throughout all the 
Scripture, and in Paul, (Rom. vii.) The will is present, 
(saith he) but I find no means to perform that which is 
good. I do not that good thing which I would: but that 
evil do I which I would not. I find by the law that when 
I would do good, evil is present with me. I delight in the 
law, as concerning the inner man, but I find another law 
in my members rebelling against the law of my mind, and 
subduing me unto the law of sin. Which law of sin is 
nothing but a corrupt and a poisoned nature which breaketh
into evil lusts, and from evil lusts into wicked deeds, and 
must be purged with the true purgatory of the cross of 
Christ: that is, thou must hate it with all thine heart, and 
desire God to take it from thee. And then whatsoever 
cross God putteth on thy back, bear it patiently, whether 
it be poverty, sickness, or persecution, or whatsoever it be, 
and take it for the right purgatory, and think that God 
hath nailed thee fast to it, to purge thee thereby. For he 
that loveth not the law and hateth his sin, and hath not 
professed in his heart to fight against it, and mourneth not 
to God to take it away and to purge him of it, the same 
hath no part with Christ. If thou love the law and findest
that thou hast yet sin hanging on thee, whereof thou 
sorrowest to be delivered and purged: as for an ensample, 
thou hast a covetous mind, and mistrustest God, and 
therefore art moved to beguile thy neighbour, and art unto 
him merciless, not caring whether he sink or swim, so thou 
mayest win by him or get from him that he hath: then get 
thee to the observant which is so purged from that sin, 
that he will not once handle a penny, and with that wile 
doth the subtle fox make the goose come flying into his 
hole, ready prepared for his mouth without his labour or 
sweat; and buy of his merits, which he hath in store, and 
give thy money not into his holy hands, but to offer him 
that he hath hired either with part of his prayers or part 
of his prey, to take the sin upon him and to handle his 
money for him. In like manner, if any person that is under
obedience unto God's ordinance (whether it be son, or 
daughter, servant, wife or subject) consent unto the ordinance,
and yet find contrary motions: let him go also to 
them that have professed an obedience of their own making, 
and buy part of their merits. If thy wife give thee nine 
words for three, go to the charterhouse and buy of their 
silence: and so if the abstaining of the observant from 
handling money, heal thine heart from desiring money, 
and the obedience of them that will obey nothing but their 
own ordinance, heal thy disobedience to God's ordinance, 
and the silence of the charterhouse monk tame thy wife's 
tongue, then believe that their prayers shall deliver thy 
soul from the pains of that terrible and fearful purgatory 
which they have feigned to purge thy purse withal. 

The spiritualty increaseth daily. More prelates, more 
priests, more monks, friars, canons, nuns, and more heretics; 
I would say heremites, with like draff. Set before thee the increase
of St. Francis's disciples in so few years. Reckon how 
many thousands, yea, how many twenty thousands, not disciples
only; but whose cloisters are sprung out of hell of 
them in so little space. Pattering of prayers encreaseth 
daily. Their service, as they call it, waxeth longer and 
longer, and the labour of their lips greater; new saints, 
new service, new feasts, and new holidays. What take 
all these away? Sin? Nay. For we see the contrary 
by experience, and that sin groweth as they grow. But 
they take away first God's word with faith, hope, peace, 
unity, love and concord; then house and land, rent and 
fee, tower and town, goods and cattle, and the very meat 
out of men's mouths. All these live by purgatory, When 
other weep for their friends, they sing merrily; when 
other loose their friends, they get friends. The pope with 
all his pardons is grounded on purgatory. Priests, monks, 
canons, friars, with all other swarms of hypocrites, do but 
empty purgatory, and fill hell. Every mass, say they, delivereth
one soul out of purgatory. If that were true, 
yea, if ten masses were enough for one soul, yet were the 
parish priests and curates of every parish sufficient to 
scour purgatory. All the other costly work of men might 
be well spared. 


THE FOUR SENSES OF THE SCRIPTURE. 

THEY divide the Scripture into four senses, the literal, 
tropological, allegorical, and anagogical. The literal 
sense is become nothing at all. For the pope hath taken 
it clean away, and hath made it his possession. He hath 
partly locked it up with the false and counterfeited keys 
of his traditions, ceremonies, and feigned lies. And partly 
driveth men from it with violence of sword. For no man 
dare abide by the literal sense of the text, but under a 
protestation, if it shall please the pope. The chopological
sense pertaineth to good manners (say they) and 
teacheth what we ought to do. The allegory is appropriate
to faith; and the anagogical to hope and things 
above. Tropological and anagogical are terms of their 
own feigning, and altogether unnecessary. For they are 
but allegories both two of them, and this word allegory 
comprehendeth them both, and is enough. For tropological
is but an allegory of manners, and anagogical, an 
allegory of hope. And allegory is as much to say as
strange speaking, or borrowed speech. As when we say
of a wanton child, this sheep hath magots in his tail, he 
must be anointed with birchen salve; which speech I 
borrow of the shepherds. 

Thou shalt understand, therefore, that the Scripture hath
but one sense, which is the literal sense. And that literal
sense is the root and ground of all, and the anchor that never
faileth, whereunto if thou cleave thou canst never err, or 
go out of the way. And if thou leave the literal sense, 
thou canst not but go out of the way. Neverthelater, 
the Scripture useth proverbs, similitudes, riddles, or allegories,
as all other speeches do; but that which the proverb,
similitude, riddle, or allegory signifieth is ever the 
literal sense which thou must seek out diligently. As in 
the English, we borrow words and sentences of one thing, 
and apply them unto another, and give them new significations.
We say, Let the sea rise as high as he will, yet 
hath God appointed how far he shall go: meaning that 
the tyrants shall not do what they would, but that only 
which God hath appointed them to do. Look ere thou leap: 
whose literal sense is, Do nothing suddenly, or without 
avisement. Cut not the bough that thou standest upon: 
whose literal sense is, Oppress not the commons; and is 
borrowed of hewers. When a thing speedeth not well, we 
borrow speech, and say, The bishop hath blessed it, because 
that nothing speedeth well that they meddle with withal. 
If the porridge be burned too, or the meat over roasted, 
we say, the bishop hath put his foot in the pot, or the 
bishop hath played the cook, because the bishops burn 
whom they lust, and whomsoever displeaseth them. He is a 
pontifical fellow, that is, proud and stately. He is popish, 
that is, superstitious and faithless. It is a pastime for a 
prelate. It is a pleasure for a pope. He would be free, 
and yet will not have head shaven. He would that no 
man should smite him, and yet hath not the pope's mark. 
And of him is betrayed, and wotteth not how, we say, he 
hath been at shrift. She is master parson's sister's daughter; 
he is the bishop's sister's son; he hath a cardinal to his 
uncle; she is a spiritual whore; it is the gentlewomen of 
the parsonage; he gave me a Kyrie eleyson. And of her 
that answereth her husband six words for one, we say, 
She is a sister of the charter-house: as who should say, 
She thinketh that she is not bound to keep silence, their 
silence shall be a satisfaction for her. And of him that 
will not be saved by Christ's merits, but by the works of 
his own imagination, we say it is a holy workman. Thus 
borrow we, and feign new speech in every tongue. All 
fables, prophecies, and riddles, are allegories; as Esop's 
fables, and Merlin's prophecies, and the interpretation of 
them are the literal sense. 

So in like manner the Scripture borroweth words and 
sentences of all manner [of] things, and maketh proverbs 
and similitudes or allegories. As Christ saith, (Luke iv.) 
Physician, heal thyself: whose interpretation is, do that at 
home, which thou dost in strange places; and that is the 
literal sense. So when I say, Christ is a lamb; I mean 
not a lamb that beareth wool, but a meek and a patient 
Lamb which is beaten for other men's faults. Christ is a 
vine, not that beareth grapes; but out of whose root the 
branches that believe suck the Spirit of life, and mercy, 
and grace, and power to be the sons of God, and to do 
his will. The similitudes of the gospel are allegories
borrowed of worldly matters to express spiritual things.
The Apocalypse or Revelations of John are allegories 
whose literal sense is hard to find in many places. 

Beyond all this, when we have found out the literal sense 
of the Scripture, by the process of the text, or by a like 
text of another place, then go we, and as the Scripture 
borroweth similitudes of worldly things, even so we again 
borrow similitudes or allegories of the Scripture, and apply 
them to our purposes; which allegories are no sense of
the Scripture, but free things besides the Scripture, and
altogether in the liberty of the Spirit. Which allegories
I may not make at all the wild adventures; but must 
keep me within the compass of the faith, and ever apply 
mine allegory to Christ, and unto the faith. Take an ensample:
thou hast the story of Peter, how he smote off 
Malchus's ear, and how Christ healed it again. There 
hast thou in the plain text great learning, great fruit, and 
great edifying, which I pass over because of tediousness. 
Then come I, when I preach of the law and the gospel, 
and borrow this ensample, to express the nature of the 
law, and of the gospel, and to paint it unto thee before 
thine eyes. And of Peter and his sword make I the law, 
and of Christ the gospel; saying, As Peter's sword cutteth 
off the ear, so doth the law. The law damneth, the law 
killeth, and mangleth the conscience. 

There is no ear so righteous that can abide the hearing 
of the law. There is no deed so good but that the law 
damneth it. But Christ, that is to say the gospel, the promises
and testament that God hath made in Christ, healeth 
the ear and conscience which the law hath hurt. The 
gospel is life, mercy and forgiveness freely, and altogether 
an healing plaister. And as Peter doth but hurt and 
make a wound where was none before, even so doth the 
law. For when we think that we are holy and righteous, and 
full of good deeds; if the law be preached aright, our righteousness
and good deeds vanish away, as smoke in the 
wind, and we am left damnable sinners only. And as 
thou seest how that Christ healeth not till Peter had 
wounded, and as an healing plaister helpeth not, till the 
corrosive hath troubled the wound; even so the gospel 
helpeth not, but when the law hath wounded the conscience,
and brought the sinner into the knowledge of his 
sin. This allegory proveth nothing, neither can do. For 
it is not the Scripture, but an ensample or a similitude 
borrowed of the Scripture to declare a text, or a conclusion
of the Scripture more expressly, and to root it and 
grave it in the heart. For a similitude, or an ensample, 
doth print a thing much deeper in the wits of a man, than 
doth a plain speaking, and leaving behind him as it were a 
sting to prick him forward, and to awake him withal. 
Moreover if I could not prove with an open text that 
which the allegory doth express, then were the allegory a 
thing to be jested at, and of no greater value than a tale 
of Robinhood. This allegory as touching his first part is 
proved by Paul in the iiird chapter of his Epistle to the 
Romans, where he saith, The law causeth wrath. And 
in the viith chapter to the Romans, When the law or 
commandment came, sin revived, and I became dead. 
And in the iind Epistle to the Corinthians, in the third 
chapter, the law is called the minister of death and damnation,
\&c. And as concerning the second part, Paul 
saith to the Romans in the vth chapter, In that we are 
justified by faith we are at peace with God. And in the 
iid Epistle to the Corinthians in the third, The gospel is 
called the ministration of justifying and of the Spirit. 
And (Gal. iv.) The Spirit cometh by preaching of the faith 
\&c. Thus doth the literal sense prove the allegory, and 
bear it, as the foundation beareth the house. And because 
that allegories prove nothing, therefore are they to be used 
soberly and seldom, and only where the text offereth thee 
an allegory. 

And of this manner (as I above have done) doth Paul 
borrow a similitude, a figure or allegory of Genesis to 
express the nature of the law, and of the gospel: and by 
Hagar and her son, declareth the property of the law, and
of her bond children, which will be justified by deeds;
and by Sarah and her son declareth the property of the 
gospel, and of her free children which are justified by faith; 
and how the children of the law which believe in their 
works persecute the children of the gospel which believe 
in the mercy and truth of God, and in the Testament of 
his son Jesus our Lord. And likewise do we borrow 
likenesses or allegories of the Scripture, as of Pharaoh 
and Herod, and of the scribes and pharisees, to express our 
miserable captivity and persecution under antichrist the pope. 

The greatest cause of which captivity and the decay of 
the faith, and this blindness wherein we now are, sprang
first of allegories. For Origen, and the doctors of his
time, drew all the Scripture unto allegories. Whose ensample
they that came after followed so long, till they at 
last forgot the order and process of the text, supposing 
that the Scripture served but to feign allegories upon. Insomuch
that twenty doctors expound one text twenty ways, 
as children make descant upon plain song. Then came 
our sophisters with their anagogical and chopological sense, 
and with an antitheme of half an inch, out of which some 
of them draw a thread of nine days long. Yea, thou shalt 
find enough that will preach Christ, and prove whatsoever 
point of the faith that thou wilt, as well out of a fable of 
Ovid, or any other poet, as out of St. John's Gospel or 
Paul's Epistles. Yea they are come unto such blindness, 
that they not only say the literal sense profiteth not, but 
also that it is hurtful and noisome and killeth the soul. 
Which damnable doctrine they prove by a text of Paul, 
(2 Cor iii.) where he saith, The letter killeth, but the 
spirit giveth life. Lo, say they, the literal sense killeth 
and the spiritual sense giveth life. We must therefore, 
say they, seek out some chopological sense. 

Here learn what sophistry is, and how blind they are, 
that thou mayest abhor them, and spue them out of thy 
stomach for ever. Paul by the letter meaneth Moses's 
law, which the process of the text following declareth 
more bright than the sun. But it is not their guise to 
look on the order of any text; but as they find it in their 
doctors, so allege they it, and so understand it. Paul 
maketh a comparison between the law and the gospel, 
and calleth the law the letter, because it was but letters 
graven in two tables of cold stone. For the law doth but 
kill and damn the conscience, as long as there is no lust 
in the heart to do that which the law commandeth. Contrariwise,
he calleth the gospel the administration of the 
Spirit, and of righteousness, or justifying. For when 
Christ is preached, and the promises which God hath 
made in Christ are believed, the Spirit entereth the heart, 
and looseth the heart, and giveth lust to do the law, and 
maketh the law a lively thing in the heart. Now as soon 
as the heart lusteth to do the law, then are we righteous 
before God, and our sins forgiven. Nevertheless the law 
of the letter graved in stone, and not in their hearts, was 
so glorious, and Moses's face shone so bright, that the 
children of Israel could not behold his face for brightness. 
It was also given in thunder and lightning and terrible 
signs; so that they, for fear, came to Moses, and desired 
him that he would speak to them, and let God speak no 
more; Lest we die (said they) if we hear him any more: 
as thou mayest see Exod. xx. Whereupon Paul maketh 
his comparison, saying: If the ministration of death 
through the letters figured in stones was glorious, so that 
the children of Israel could not behold the face of Moses 
for the glory of his countenance; why shall not the administration
of the Spirit be glorious? And again: If 
the administration of damnation be glorious, much more 
shall the administration of righteousness exceed in glory: 
that is, if the law that killeth sinners, and helpeth them 
not, be glorious; then the gospel which pardoneth 
sinners, and giveth them power to be the sons of God, 
and to overcome sin, is much more glorious. And the 
text that goeth before is as clear. 

For the holy apostle Paul saith, Ye Corinthians are 
our epistle, which is understood and read of all men, in 
that ye are known how that ye are the epistle of Christ 
ministered by us, and written, not with ink, (as Moses's law) 
but with the Spirit of the living God; not in tables of 
stone, (as the ten commandments) but in the fleshy tables 
of the heart: as who should say, We write not a dead law 
with ink, and in parchment, nor grave that which damned 
you in tables of stone; but preach you that which bringeth 
the Spirit of life unto your breasts, which Spirit writeth 
and graveth the law of love in your hearts, and giveth you 
lust to do the will of God. And furthermore, saith he, 
Our ableness cometh of God, which hath made us able to 
minister the New Testament, not of the letter, (that is to 
say, not of the law) but of the Spirit: for the letter (that 
is to say, the law) killeth; but the Spirit giveth life; (that 
is to say, the Spirit of God) which entereth your hearts 
when ye believe the glad tidings that are preached you in 
Christ; quickeneth your hearts, and giveth you life and 
lust, and maketh you to do of love and of your own accord 
without compulsion, that which the law compelled you to 
do, and damned you, because ye could not do with love 
and lust, and naturally. Thus seest thou that the letter signifieth
not the literal sense, and the spirit the spiritual sense. 
And Rom. ii. Paul useth this term Litera, for the law. 
And Rom. vii. where he setteth it so plain, that if the 
great wrath of God had not blinded them, they could 
never have stumbled at it. 

God is a Spirit, and all his words are spiritual. His 
literal sense is spiritural, and all his words are spiritual. 
When thou readest (Matt. i.) She shall bear a son, and 
thou shalt call his name Jesus; for he shall save his 
people from their sins: this literal sense is spiritual and 
everlasting life unto as many as believe it. And the literal 
sense of these words, (Matt. v.) Blessed are the merciful, 
for they shall have mercy; are spiritual and life. Whereby 
they that are merciful may of right, by the truth and promise
of God, challenge mercy. And like is it of these 
words, (Matt. vi.) If you forgive other men their sins, your 
heavenly Father shall forgive you yours. And so is it of 
all the promises of God. Finally, all God's words are spiritual,
if thou have eyes of God to see the right meaning of 
the text, and whereunto the Scripture pertaineth, and the 
final end and cause thereof. 

All the Scripture is either the promises and testament of 
God in Christ, and stories pertaining thereunto to strength 
thy faith; either the law, and stories pertaining thereto, to 
fear thee from evil doing. There is no story nor gest, seem it 
never so simple or so vile unto the world, but that thou 
shalt find therein spirit and life and edifying in the literal 
sense. For it is God's Scripture, written for thy learning 
and comfort. There is no clout or rag there that hath not 
precious relics wrapt therein of faith, hope, patience and 
long suffering, and of the truth of God, and also of his 
righteousness. Set before thee the story of Reuben which 
defiled his father's bed. Mark what a cross God suffered 
to fall on the neck of his elect Jacob. Consider first the 
shame among the heathen, when as yet there was no more 
of the whole world within the testament of God, but he and 
his household. I report me to our prelates which swear 
by their honour, whether it were a cross or no. Seest 
thou not how our wicked builders rage, because they see 
their buildings burn, now they are tried by the fire of God's 
word; and how they stir up the whole world, to quench the 
word of God, for fear of losing their honour? then what 
business had he to pacify his children? look what ado he 
had at the defiling of his daughter Dinah. And be thou 
sure that the brethren there were no more furious for the 
defiling of their sister, than the sons here for defiling of 
their mother. Mark what followed Reuben, to fear other, 
that they shame not their fathers and mothers. He was 
cursed, and lost the kingdom, and also the priestdom, and 
his tribe or generation was ever few in number as it appeareth
in the stories of the Bible. 

The adultery of David with Bathsheba is an ensample 
not to move us to evil: but if (while we follow the way of 
righteousness) any chance drive us aside, that we despair 
not. For if we saw not such infirmities in God's elect, we, 
which are so weak and fall so oft, should utterly despair, 
and think that God had clean forsaken us. It is therefore 
a sure, and an undoubted conclusion, whether we be holy 
or unholy, we are all sinners. But the difference is, that 
God's sinners consent not to their sin. They consent unto
the law that is both holy and righteous, and mourn to 
have their sin taken away. But the devil's sinners consent
unto their sin, and would have the law and hell taken 
away, and are enemies unto the righteousness of God. 

Likewise in the homely gest of Noah, when he was 
drunk, and lay in his tent with his privy members open, 
hast thou great edifying in the literal sense! Thou seest 
what became of the cursed children of wicked Ham, which 
saw his father's privy members and jested thereof unto his 
brethren. Thou seest also what blessing fell on Shem and 
Japhet, which went backward and covered their father's 
members and saw them not. And thirdly thou seest what 
infirmity accompanieth God's elect, be they never so holy, 
which yet is not imputed unto them. For the faith and 
trust they have in God swalloweth up all their sins. 

Notwithstanding, this text offers us an apt and an handsome
allegory or similitude to describe our wicked Ham, 
Antichrist, the pope, which many hundred years hath done 
all the shame that heart can think unto the privy member 
of God; which is the word of promise, or the word of 
faith as Paul calleth it Rom. x.; and the gospel and testament
of Christ, wherewith we are begotten; as thou 
seest 1 Pet. i. and James i. And as the cursed children 
of Ham grew into giants, so mighty and great that the 
children of Israel seemed but grasshoppers in respect of 
them; so the cursed sons of our Ham, the pope, his cardinals,
bishops, abbots, monks, and friars, are become 
mighty giants above all power and authority; so that the 
children of faith, in respect of them, are much less than 
grasshoppers. They heap mountain upon mountain, and 
will to heaven by their own strength, by a way of their own 
making and not by the way Christ. Neverthelater, 
those giants, for the wickedness and abominations which 
they had wrought, did God utterly destroy, part of them 
by the children of Lot, and part by the children of Esau, 
and seven nations of them by the children of Israel. So 
no doubt shall he destroy these for like abominations, and 
that shortly. For their kingdom is but the kingdom of 
lies and falsehood, which must needs perish at the coming
of the truth of God's word, as the night vanisheth away 
at the presence of day. The children of Israel slew not 
those giants, but the power of God; God's truth and promises,
as thou mayest see in Deuteronomy. So it is not we 
that shall destroy those giants, as thou mayest see by Paul, 
(2 Thes. ii.) speaking of our Ham Antichrist: Whom the 
Lord shall destroy (saith he) with the spirit of his mouth; 
that is, by the words of truth: and by the brightness of his 
coming; that is, by the preaching of his gospel. 

And as I have said of allegories, even so it is of worldly 
similitudes, which we make either when we preach, 
either when we expound the Scripture. The similitudes 
prove nothing, but are made to express more plainly that 
which is contained in the Scripture, and to lead thee into 
the spiritual understanding of the text. As the similitude 
of matrimony is taken to express the marriage that is between
Christ and our souls, and what exceeding mercy we 
have there, whereof all the Scriptures make mention. 
And the similitude of the members, how every one of them 
careth for other, is taken to make thee feel what it is to 
love thy neighbour as thyself. That preacher therefore
that bringeth a naked similitude to prove that which is
contained in no text of Scripture, nor followeth of a text,
count a deceiver, a leader out of the way, and a false prophet,
and beware of his philosophy and persuasions of
man's wisdom, as Paul (i. Cor. ii.) saith: My words and
my preaching were not with enticing words, and persuasions
of man's wisdom, but in showing of the Spirit and
power. That is, he preached not dreams, confirming them 
with similitudes; but God's word confirming it with miracles
and with working of the Spirit, the which made 
them feel every thing in their hearts. That your faith, 
saith he, should not stand in the wisdom of man; but in 
the power of God. For the reasons and similitudes of
man's wisdom make no faith, but wavering and uncertain
opinions only: one draweth me this way with his argument,
another that way, and of what principle thou provest
black, another proveth white: and so am I ever uncertain.
As if thou tell me of a thing done in a far land, and another
tell me the contrary, I wot not what to believe. 
But faith is wrought by the power of God, that is, when
God's word is preached, the Spirit entereth thine heart,
and maketh thy soul feel it, and maketh thee so sure of it,
that neither adversity, nor persecution, nor death, neither 
hell, nor the powers of hell, neither yet all the pains of hell 
could once prevail against thee, or move thee from the 
sure rock of God's word, that thou shouldst not believe
that which God hath sworn.

And Peter (2 Pet. i.) saith, We followed not deceivable
fables, when we opened unto you the power and
coming of our Lord Jesus Christ; but with our eyes, we
saw his majesty. And again, we have (saith he) a more 
sure word of prophecy, whereunto if ye take heed, as 
unto a light shining in a dark place, ye do well. The 
word of prophecy was the old Testament which beareth 
record unto Christ in every place, without which record 
the Apostles made neither similitudes nor arguments of 
worldly wit. Hereof seest thou, that all the allegories, 
similitudes, persuasions and arguments, which they bring 
without Scripture, to prove praying to saints, purgatory, 
ear confession; and that God will hear thy prayer more 
in one place than in another; and that it is more meritorious 
to eat fish, than flesh; and that to disguise thyself and put 
on this or that manner [of] coat is more acceptable than to go 
as God hath made thee; and that widowhood is better than 
matrimony, and virginity than widowhood; and to prove 
the assumption of our lady, and that she was born without 
original sin, yea, and with a kiss say some, are but false 
doctrine. 

Take an ensample how they prove that widowhood and 
virginity exceed matrimony. They bring this worldly similitude:
he that taketh most pain for a man deserveth 
most, and to him a man is most bound; so likewise must 
it be with God, and so forth. Now the widow and virgin 
take more pain in resisting their lusts than the married wife, 
therefore is their state holier. First, I say, that in their 
own sophistry, a similitude is the worst aod feeblest argument
that can be, and proveth least, and soonest deceiveth. 
Though that one son do more service for his father than 
another, yet is the father free, and may with right reward 
them all alike. For though I had a thousand brethren, 
and did more than they all, yet do I not my duty. The 
fathers and mothers also care most for the least and weakest,
and them that can do least: yea, for the worst care 
they most, and would spend, not their goods only, but also 
their blood, to bring them to the right way. And even so 
is it of the kingdom of Christ, as thou mayest well see in 
the similitude of the riotous son. (Luke xv.) More- 
over Paul saith, (1 Cor. vii.) It is better to marry than 
to burn. For the person that burneth cannot quietly serve 
God, inasmuch as his mind is drawn away, and the thoughts 
of his heart occupied with wonderful and monstrous imaginations.
He can neither see, nor hear, nor read, but 
that his wits are rapt, and he clean from himself. And 
again, saith he, circumcision is nothing, uncircumcision is 
nothing: but the keeping of the commandments is altogether. 
Look wherein thou canst best keep the commandments; 
thither get thyself and therein abide; whether thou be 
widow, wife, or maid, and then hast thou all with God.
If we have infirmities that draw us from the laws of God,
let us cure them with the remedies that God hath made.
If thou burn, marry. For God hath promised thee no
chastity, as long as thou mayest use the remedy that he
hath ordained: no more than he hath promised to slake
thine hunger without meat.

How to ask of God more than he hath promised, 
cometh of a false faith, and is plain idolatry: and to 
desire a miracle where there is natural remedy, is tempting 
of God. And of pain-taking thiswise understand. He 
that taketh pains, to keep the commandments of God, is 
sure thereby that he loveth God, and that he hath God's 
spirit in him. And the more pain a man taketh (I mean
patiently and without grudging) the more he loveth God,
and the perfecter he is, and nearer unto that health which 
the souls of all Christian men long for, and the more purged 
from the infirmity and sin that remaineth in the flesh. But 
to look for any other reward or promotion in heaven or in 
the life to come, than that which God hath promised for 
Christ's sake; and which Christ hath deserved for us with 
his pain taking; is abominable in the sight of God. For 
Christ only hath purchased the reward; and our pain taking, 
to keep the commandments, doth but purge the sin that 
remaineth in the flesh, and certify us that we are chosen 
and sealed with God's spirit unto the reward that Christ 
hath purchased for us. 

I was once at the creating of Doctors of Divinity, where 
the opponent brought the same reason to prove that the 
widow had more merit than the virgin, because she had 
greater pain, for as much as she had once proved the pleasures
of matrimony. Ego nego, Domine Doctor, said the 
respondent. For though the virgin have not proved, yet 
she imagined that the pleasure is greater than it is indeed, 
and therefore is more moved, and hath greater temptation 
and greater pain. Are not these disputers they that Paul 
speaketh of in the sixth chapter of the first Epistle to 
Timothy? that they are not content with the wholesome 
words of our Lord Jesus Christ, and doctrine of godliness; 
and therefore know nothing: but waste their brains 
about questions and strife of words, whereof spring envy, 
strife and railing of men with corrupt minds, destitute of 
the truth. 

As pertaining to our lady's body, where it is, or where 
the body of Elias; of John the Evangelist, and of many 
other be, pertaineth not to us to know. One thing are 
we sure of, that they are where God hath laid them. If 
they be in heaven, we have never the more in Christ: if 
they be not there, we have never the less. Our duty is to 
prepare ourselves unto the commandments and to be 
thankful for that which is opened unto us, and not to search 
the unsearchable secrets of God. Of God's secrets can 
we know no more than he openeth unto us. If God shut, 
who shall open? How then can natural reason come by 
the knowledge of that which God hath hid unto himself? 

Yet let us see one of their reasons wherewith they prove 
it. The chief reason is this, every man doth more for his 
mother, say they, than for other; in like manner must 
Christ do for his mother, therefore hath she this pre-eminence,
that her body is in heaven. And yet Christ, in the 
xiith chap. of Matt. knoweth her not for his mother, but 
as farforth as she kept his Father's commandments. And 
Paul, in the iind Epistle to the Corinthians chap. v. knoweth
not Christ himself fleshly, or after a worldly purpose. 

Last of all, God is free aud no further bouad than he 
bindeth bimself: if he have made her any promise he is 
bound; if not, then is he not. Finally, if thou set this 
above rehearsed chapter of Matthew before thee where Christ 
would not know his mother, and the iind of John where 
he rebuked her, and the iind of Luke where she lost him, 
and how negligent she was to leave him behind her at 
Jerusalem unawares, and to go a day's journey ere she 
sought for him, thou mightest resolve many of their reasons 
which they make of this matter, and that she was not without
original sin: read also Erasmus's Annotations in the 
said places. And as for me, I commit all such matters 
unto those idle bellies which have nought else to do than 
to move such questions, and give them free liberty to hold 
what they list, as long as it hurteth not the faith, whether 
it be so or no: exhorting yet with Paul all that will please 
God, and obtain that salvation that is in Christ, that they 
give no heed unto unnecessary and brawling disputations, 
and that they labour for the knowledge of those things 
without winch they cannot be saved. And remember that 
the sun was given us to guide us in our way and works 
bodily. Now if thou leave the natural use of the sun, and 
will look directly on him to see how bright he is, and suchlike 
curiosity, then will the sun blind thee. So was the Scripture 
given us to guide us in our way and works ghostly. The 
way is Christ, and the promises in him are our salvation, if we 
long for them. Now if we shall leave that right use and turn 
ourselves unto vain questions, and to search the unsearchable 
secrets of God, then no doubt shall the Scripture blind us 
as it hath done our schoolmen and our subtle disputers. 

And as they are false prophets which prove with allegories,
similitudes, and worldly reasons, that which is no 
where made mention of in the Scripture; even so count 
them for false prophets which expound the Scriptures 
drawing them unto a worldly purpose clean contrary unto 
the ensample, living, and practising of Christ and of his 
apostles, and of all the holy prophets. For, saith Peter, 
(2 Pet. i.) No prophecy iu the Scripture hath any private
interpretation. For the Scripture came not by the will 
of man; but the holy men of God spake as they were 
moved by the Holy Ghost. No place of the Scripture may 
have a private exposition, that is, it may not be expounded 
after the will of man, or after the will of the flesh, or drawn 
unto a worldly purpose contrary unto the open texts, and 
the general articles of the faith, and the whole course of the 
Scripture, and contrary to the living and practising of 
Christ and the apostles and holy prophets. For as they 
came not by the will of man, so may they not be drawn or 
expounded after the will of man: but as they came by the 
Holy Ghost, so must they be expounded and understood by 
the Holy Ghost. The Scripture is that wherewith God 
draweth us unto him, aud not wherewith we should be led 
from him. The Scriptures spring out of God, and flow 
unto Christ, and were given to lead us to Christ. Thou 
must therefore go along by the Scripture as by a line, until 
thou come at Christ, which is the way's end and resting- 
place. If any man, therefore, use the Scripture to draw 
thee from Christ, and to nosel thee in any thing save in 
Christ, the same is a false prophet. And that thou mayest 
perceive what Peter meaneth, it followeth in the text, 
There were false prophets among the people (whose prophesies
were belly wisdom) as there shall be false teachers 
among you, which shall privily bring in damnable sects, 
(as thou seest how we are divided into monstrous sects or 
orders of religion) even denying the Lord that hath 
bought them; (for every one of them taketh on him to 
sell thee, for money, that which God in Christ promiseth 
thee freely,) and many shall follow their damnable ways, 
by whom the way of truth shall be evil spoken of (as thou 
seest how the way of truth is become heresy, seditious, or 
cause of insurrection, and breaking of the king's peace, 
and treason unto his highness). And through covetousness 
with feigned words shall they make merchandise of you. 

Covetousneas is the conclusion; for covetousness and 
ambition, that is to say, lucre and desire of honour, is the 
final end of all false prophets and of all false teachers. 
Look upon the pope's false doctrine, what is the end 
thereof, and what seek they thereby? wherefore serveth 
purgatory, but to purge thy purse, and to poll thee, and 
rob both thee and thy heirs of house and lands, and of all 
thou hast, that they may be in honour? Serve not pardons 
for the same purpose? whereto pertaineth praying to 
saints, but to offer unto their bellies? wherefore serveth 
confession, but to sit in thy conscience and to make thee 
fear and tremble at whatsoever they dream, and that thou 
worship them as gods? and so forth, in all their traditions, 
ceremonies, and conjurations, they serve not the Lord, 
but their bellies. And of their false expounding the 
Scripture, and drawing it contrary unto the ensample of 
Christ? and the apostles and holy prophets unto their damnable
covetousness and filthy ambition, take an ensample: 

When Peter saith to Christ, (Matt. xvi.) Thou art the 
Son of the living God; and Christ answered, Thou art 
Peter, and upon this rock I will build my congregation. 
By the rock interpret they Peter. And then cometh the 
pope, and will be Peter's successor, whether Peter will 
or will not; yea, whether God will, or will not; and though 
all the Scripture say Nay to any such succession, and faith, 
Lo I am the rock, the foundation, and head of Christ's 
church. Now saith all the Scripture, that the rock is 
Christ, the faith, and God's word. As Christ saith, 
(Matt. vii.) He that heareth my words and doth thereafter, 
is like a man that buildeth on a rock. For the house that 
is built on God's word will stand, though heaven should 
fall. And (John xv.) Christ is the vine, and we the 
branches; so is Christ the rock, the stock and foundation 
whereon we be built. And Paul (1 Cor. iii.) calleth 
Christ our foundation; and all other, whether it be Peter 
or Paul, he calleth them servants to preach Christ, and to 
build us on him. If therefore the pope be Peter's successor,
his duty is to preach Christ only, and other authority 
hath he none. And (2d xi.) Paul marrieth us unto 
Christ, and driveth us from all trust and confidence in man. 
And, (Eph. ii.) saith Paul, Ye are built on the foundation
of the apostles and prophets; that is on the word 
which they preached; Christ being, saith he, the head 
corner stone, in whom every building, coupled together, 
groweth up into an holy temple in the Lord, in whom also 
ye are built together and made an habitation for God in 
the Spirit. And Peter, in the iid of his first Epistle, buildeth 
us on Christ, contrary to the pope, vrhich buildeth on 
himself. Hell gates shall not prevail against it; that is to 
say, against the congregation that is build upon Christ's 
faith, and upon God's word. Now were the pope the 
rock, hell gates could not prevail against him. For the 
house could not stand if the rock and foundation whereon 
it is built did perish: but the contrary see we in our popes. 
For hell gates have prevailed against them many hundred 
years, and have swallowed them up, if God's word be 
true, and the stories that are written of them; yea, or if 
it be true that we see with our eyes. I will give thee the 
keys of heaven, saith Christ, and not, I give. And (John 
xxth.) after the resurrection paid it, and gave the keys to 
them all indifferently. Whatsoever thou bindest on earth, 
it shall be bound in heaven; and whatsoever thou loosest 
on earth, it shall be loosed in heaven. Of this text maketh 
the pope what he will, and expoundeth it contrary to all 
the Scripture, contrary to Christ's practising, and the 
apostles, and all the prophets. Now the Scripture giveth 
record to himself, and ever expoundeth itself by another 
open text. If the pope then cannot bring for his exposition 
the practising of Christ or of the apostles and prophets, or 
an open text, then is his exposition false doctrine. Christ 
expoundeth himself, (Matt. xviii.) saying. If the brother 
sin against thee, rebuke him betwixt him and thee alone. 
If he hear thee, thou hast won thy brother: but if he hear 
thee not, then take with thee one or two, and so forth, as 
it standeth in the text. He concludeth, saying to them all, 
Whatsoever ye bind in earth, it shall be bound in heaven; 
and whatsoever ye loose on earth, it shall be loosed in 
heaven. Where binding is but to rebuke them that sin, 
and loosing to forgive them that repent. And (John xx.) 
Whose sins ye forgive, they are forgiven; and whose sins 
ye hold, they are holden. And Paul (1 Cor. v.) bindeth, 
and (2 Cor. ii.) looseth after the same manner. 

Also this binding and loosing is one power; and as he 
bindeth, so looseth he: yea, and bindeth first ere he can 
loose. For who can loose that is not bound? Now whatsoever
Peter bindeth, or his successor, (as he will be called 
and is not, but indeed the very successor of Satan) is not 
so to be understood, that Peter, or the pope hath power 
to command a man to be in deadly sin, or to be damned, 
or to go into hell, saying, Be thou in deadly sin, be thou 
damned, go thou to hell, go thou to purgatory. For that 
exposition is contrary to the everlasting Testament that 
God hath made unto us in Christ. He sent his Son Christ 
to loose us from sin, and damnation, and hell, and that to 
testify unto the world, sent he his disciples. (Acts i.) Paul 
also hath no power to destroy, but to edify. (2 Cor. x. 
and xiii.) How can Christ give his disciples power against 
himself, and against his everlasting Testament? Can he 
send them to preach salvation, and give them power to 
damn whom they lust? What mercy and profit have we in 
Christ's death, and in his gospel, if the pope which passeth 
all men in wickedness, hath power to send whom he will 
to hell, and to damn whom he lusteth? we had then no 
cause, to call him Jesus, that is to say, Saviour: but 
might of right call him destroyer. Wherefore, then this 
binding is to be understood as Christ interpreteth it in the
places above rehearsed, and as the apostles practised it,
and is nothing but to rebuke men of their sins, by preaching 
the law. A man must first sin against God's law, ere the 
pope can bind him: yea, and a man must first sin against 
God's law, ere he need to fear the pope's curse. For 
cursing and binding are both one, and nothing saving to 
rebuke a man of his sins by God's law. It followeth 
also, then, that the loosing is of like manner, and is nothing 
but forgiving of sin to them that repent, through preaching 
of the promises which God hath made in Christ, in whom 
only we have all forgiveness of sins, as Christ interpreteth 
it, and as the apostles and prophets practised it. So is it 
a false power that the pope taketh on him, to loose God's 
laws, as to give a man license to put away his wife to 
whom God hath bound him, and to bind them to chastity 
which God commandeth to marry; that is to wit, them 
that burn and cannot life chaste. It is also a false power 
to bind that which God's word maketh free, making sin 
in the creatures which God hath made for man's use. 

The pope which so fast looseth and purgeth in purgatory,
cannot with all the loosings and purgations that he 
hath, either loose or purge our appetites, and lust and rebellion
that is in us against the law of God. And yet 
the purging of them is the right purgatory. If he cannot 
purge them that are alive, wherewith purgeth he them 
that are dead? The apostles knew no other ways to 
purge, but through preaching God's word, which word 
only is that that purgeth the heart, as thou mayest see 
John xv. Ye are pure saith Christ, through the word. 
Now the pope preacheth not to them whom they feign to 
lie in purgatory, no more than he doth to us that are alive. 
How then purgeth he them? The pope is kin to Robin 
Goodfellow which sweepeth the house, washeth the dishes, 
and purgeth all by night. But when day cometh there 
is nothing found clean. 

Some man will say the pope bindeth them not, they 
bind themselves. I answer, he that bindeth himself to the 
pope, and had lever have his life and soul ruled by the 
pope's will than by the will of God, and by the pope's 
word, than by the word of God, is a fool. And he that 
had lever be bond than free is not wise. And he that will 
not abide in the freedom wherein Christ hath set us, is 
also mad. And he that maketh deadly sin where none is, 
and seeketh causes of hatred between him and God, is not 
in his right wits. Furthermore, no man can bind himself
further than he hath power over himself. He that 
is under the power of another man, cannot bind himself 
without license, as son, daughter, wife, servant, and 
subject. Neither canst thou give God that which is not 
in thy power. Chastity canst thou not give further than 
God lendeth it thee: if thou cannot live chaste, thou art 
bound to marry or to be damned. Last of all, for what 
purpose thou bindest thyself must be seen. If thou do 
it to obtain thereby that which Christ hath purchased for 
thee freely, so art thou an infidel, and hast no part with 
Christ, and so forth. If thou wilt see more of this matter 
look in Deuteronomy, and there shalt thou find it more 
largely entreated. 

Take another ensample of their false expounding the 
Scripture. Christ saith, (Matt. xxiii.) The scribes and 
the pharisees sit on Moses' seat; whatsoever they bid you 
observe, that observe and do; but after their works do 
not. Lo, say our sophisters or hypocrites, live we never 
So abominably, yet is our authority never the less. Do 
as we teach, therefore, (say they) and not as we do. And 
yet Christ saith they sit on Moses' seat; that is as long as 
they teach Moses, do as they teach. For the law of 
Moses is the law of God. But for their own traditions 
and false doctrine Christ rebuked them, and disobeyed 
them, and taught other to beware of their leaven. So if 
our pharisees sit on Christ's seat and preach him, we 
ought to hear them; but when they sit on their own seat, 
then ought we to beware as well of their pestilent doctrine 
as of their abominable living. 

Likewise where they find mention made of a sword, they 
turn it uiilo the pope's power. The disciples said unto 
Christ, (Luke xxii.) Lo, here be two swords. And Christ 
answered two is enough. Lo, say they, the pope 
hath two swords, the spiritual sword and the temporal 
sword. And therefore is it lawful for him to fight and 
make war. 

Christ a little before he went to his passion, asked his 
disciples, saying, When I sent you out without all provision,
lacked ye any thing? and they said Nay. And he 
answered, But now let him that hath a wallet take it with 
him, and he that hath a scrip likewise, and let him that 
hath never a sword, sell his coat and buy one: as who 
should say, it shall go otherwise now than then. Then ye 
went forth in faith of my word, and my Father's promises, 
and it fed you and made provision for you, and was your 
sword and shield, and defender; but now it shall go as 
thou readest Zechariah xiii. I will smite the shepherd, and 
the sheep of the flock shall be scattered. Now shall my 
Father leave me in the hands of the wicked; and ye also 
shall be forsaken and destitute of faith, and shall trust in 
yourselves, and in your own provision, and in your own 
defence. Christ gave no commandment, but prophesied 
what should happen. And they, because they understood 
him not, answered, Here are two swords. And Christ (to 
make an end of such babbling) answered, Two is enough. 
For if he had commanded every man to buy a sword, how 
had two been enough? also if two were enough, and pertained
to the pope only, why are they all commanded to 
buy every man a sword? By the sword, therefore, Christ 
prophesied that they should be left unto their own defence. 
And two swords were enough, yea, never-a-one had been 
enough. For if every one of them had had ten swords they 
would have fled ere midnight. 

In the same chapter of Luke, not twelve lines from the 
foresaid text, the disciples, even at the last supper, 
asked who should be the greatest. And Christ rebuked 
them, and said it was an heathenish thing, and there 
should be no such thing among them, but that the 
greatest should be as the smallest, and that to be great 
was to do service as Christ did. But this text because 
it is brighter than the sun, that they can make no sophistry
of it, therefore will they not hear it, nor let other 
know it. 

Forasmuch now as thou partly seest the falsehood of 
our prelates, how all their study is to deceive us and to 
keep us in darkness, to sit as gods in our consciences, and 
handle us at their pleasure, and to lead us whether they 
lust; therefore I read thee, get thee to God's word, and 
thereby try all doctrine, and against that receive nothing. 
Neither any exposition contrary unto the open texts, neither 
contrary to the general articles of the faith, neither contrary
to the living and practising of Christ and his apostles. 
And when they cry, Fathers, fathers, remember that it were
the fathers that blinded and robbed the whole world, and 
brought us into this captivity, wherein these enforce to 
keep us still. Furthermore, as they of the old time are 
fathers to us, so shall these foul monsters be fathers to 
them that come after us; and the hypocrites that follow us 
will cry of these and of their doings, Fathers, fathers, as 
these cry Fathers, fathers, of them that are past. And as we 
feel our Fathers, so did they that are past feel their fathers: 
neither were there in the world any other fathers than such 
as we both see and feel this many hundred years; as their 
decrees bear record, and the stories and chronicles well testify.
If God's word appeared any where, they agreed all 
against it. When they had brought that asleep, then 
strove they one with another about their own traditions, 
and one pope condemned anothers' decrees, and were 
sometimes two, yea, three popes at once. And one bishop 
went to law with another, and one cursed another for their 
own fantasies, and such things as they had falsely gotten. 
And the greatest saints are they that most defended the liberties
of the church; (as they call it) which they falsely got 
with blinding kings; neither bad the world any rest this 
many hundred years, for reforming of friars and monks 
and ceasing of schisms that were among our clergy. And 
as for the holy doctors, as Augustine, Jerome, Cyprian, 
Chrisostomus, and Bede, will they not hear. If they 
wrote any thing negligently, (as they were men) that draw 
they clean contrary to their meaning, and thereof triumph 
they. Those doctors knew of none authority that one 
bishop should have above another, neither thought or once 
dreamed that ever any such should be, or of any such 
whispering or of pardons, or scouring of purgatory, as 
they have feigned. 

And when they cry, Miracles, miracles, remember that 
God hath made an everlasting testament with us in Christ's 
blood, against which we may receive no miracles; no, neither 
the preaching of Paul himself if he came again, by his 
own teaching to the Galatians, neither yet the preaching of 
the angels of heaven. Wherefore either they are no miracles
but they have feigned them, (as is the miracle that 
St. Peter hallowed Westminster) or else if there be miracles 
that confirm doctrine contrary to God's word, then are they 
done of the devil, (as the maid of Ipswich and of Kent) to 
prove us whether we will cleave fast to God's word, and 
to deceive them that have no love to the truth of God's 
word, nor lust to walk in his laws. 

And forasmuch as they to deceive withal arm themselves 
against them with arguments and persuasions of fleshly 
wisdom; with worldly similitudes; with shadows; with false 
allegories; with false expositions of the Scripture, contrary 
unto the living and practising of Christ and the apostles; 
with lies and false miracles; with false names; dumb ceremonies;
with disguising of hypocrisy; with the authorities 
of the fathers; and last of all with the violence of the temporal
sword: therefore do thou contrariwise arm thyself to 
defend thee withal, as Paul teacheth in the last chapter of 
the Ephesians, Gird on thee the sword of the Spirit, which 
is God's word, and take to thee the shield of faith: which is 
not to believe a tale of Robin Hood, or Gestus Romanorum,
or of the Chronicles, but to believe God's word that 
lasteth ever. 

And when the pope with his falsehood challengeth temporal
authority above king and emperor, set before thee 
the xxvth. chap, of St. Matt. where Christ commandeth
Peter to put up his sword. And set before thee 
Paul iind. Cor. xth. where he saith the weapons of our 
war are not carnal things, but mighty in God to bring all 
understanding in captivity under the obedience of Christ; 
that is, the weapons are God's word and doctrine, and not 
swords of iron and steel, and set before thee the doctrine 
of Christ and of his apostles and their practice. 

And when the pope challengeth authority over his fellow 
bishops, and over all the congregation of Christ by succession
of Peter, set before thee the first of the Acts where 
Peter for all his authority put no man in the room of 
Judas; but all the apostles chose two indifferently, and 
cast lots, desiring God to temper them that the lot might 
fall on the most able. And (Acts viii.) the apostles sent 
Peter, and in the xith. call him to reckoning, and to give 
accounts of that he hath done. 

And when the pope's law commandeth, saying, though 
that the pope live never so wickedly and draw with him 
through his evil ensample innumerable thousands into hell, 
yet see that no man presume to rebuke him, for he is head 
over all, and no man over him, set before thee Galatians
iid. where Paul rebuketh Peter openly. And see 
how both to the Corinthians, and also to the Galatians, he 
will have no superior but God's word, and he that could
teach better by God's word. And because when he rehearsed
his preaching and his doings unto the high apostles,
they could improve nothing, therefore will he be equal 
with the best. 

And when the friars say, they do more than their duty,
when they preach, and more than they are bound to: To
say our service are we bound (say they) and that is our
duty, and to preach is more than we are bound to. Set 
thou before thee how that Christ's blood-shedding hath 
bound us to love one another with all our might, and to do 
the uttermost of our power one to another. And Paul 
saith, (1 Cor. ix.) Woe be unto me, if I preach not: yea, 
woe is unto him that hath wherewith to help his neighbour 
and to make him better and do it not. If they think it 
more than their duty to preach Christ unto you, then they 
think it more than their duty to pray that he should come 
to the knowledge of Christ. And therefore it is no marvel 
though they take so great labour: yea, and so great wages 
also to keep you still in darkness. 

And when they cry furiously, Hold the heretics unto 
the wall, and if they will not revoke, burn them without 
any more ado; reason not with them, it is an article condemned
by the fathers; set thou before thee the saying 
of Peter: (1 Pet. iii.) To all that ask you, be ready to 
give an answer of the hope that is in you, and that with 
meekness. The fathers of the Jews and the bishops, 
which had as great authority over them as ours have over 
us, condemned Christ and his doctrine. If it be enough 
to say the fathers have condemned it, then are the Jews 
to be holden excused; yea, they are yet in the right way, 
and we in the false; but and if the Jews be bound to 
look in the Scripture and to see whether their fathers 
have done right or wrong, then are we likewise bound to 
look in the Scripture whether our fathers have done right 
or wrong, and ought to believe nothing without a reason 
of the Scripture, and authority of God's word. 

And of this manner defend thyself against all manner [of] 
wickedness of our spirits, armed always with God's word 
and with a strong and a steadfast faith thereunto. Without
God's word do nothing. And to his word add nothing, 
neither pull any thing therefrom, as Moses every where 
teacheth thee. Serve God in the Spirit, and thy 
neighbour with all outward service. Serve God as 
he hath appointed thee, and not with thy good intent 
and good zeal. Remember Saul was cast away of 
God for ever, for his good intent. God requireth 
obedience unto his word, and abhorreth all good intents 
and good zeals which are without God's word. For they 
are nothing else than plain idolatry and worshipping of 
false Gods. 

And remember that Christ is the end of all things. In Christ 
He only is our resting place, and he is our peace.
(Eph. chap. ii.) For as there is no salvation in any other
name, so is there no peace in any other name. Thou 
shalt never have rest in thy soul, neither shall the worm of 
conscience ever cease to gnaw thine heart, till thou come 
at Christ: till thou hear the glad tidings, how that God 
for his sake hath forgiven thee all freely. If thou trust in 
thy works there is no rest. Thou shalt think, I have not 
done enough. Have I done it with so great love as I 
should do? was I so glad in doing as I would be to receive
help at my need? I have left this or that undone, and 
such like. If thou trust in confession, then shalt thou 
think, Have I told all? have I told all the circumstances?
Did I repent enough? had I as great sorrow in 
my repentance for my sins as I had pleasure in doing them? 
Likewise in our holy pardons and pilgrimages, gettest thou 
no rest. For thou seest that the very gods themselves 
which sell their pardon so good, cheap, or somewhiles 
give them freely for glory sake, trust not therein themselves.
They build colleges, and make perpetuities to be 
prayed for for ever, and lade the lips of their beadmen, 
or chaplains with so many masses, and dirges, and so long 
service, that I have known some that have bid the devil 
take their founders' souls, for very impatience and weariness
of so painful labour. 

As pertaining to good deeds therefore, do the best
thou canst, and desire God to give strength to do better
daily; but in Christ put thy trust, and in the pardon
and promises that God hath made thee for his sake, and 
on that rock build thine house and there dwell. For there 
only shalt thou be sure from all storms and tempests, and 
from all wily assaults of our wicked spirits which study with 
all falsehood to undermine us. And the God of all mercy 
give thee grace so to do, unto whom be glory for ever. Amen. 


A COMPENDIOUS REHEARSAL OF THAT WHICH 
GOETH BEFORE. 

I HAVE described unto you the obedience of children, 
servants, wives, and subjects. These four orders are of 
God's making, and the rules thereof are God's word. 
He that keepeth them shall be blessed: yea, is blessed 
already, and he that breaketh them shall be cursed. If 
any person of impatience or of a stubborn and rebellious 
mind, withdraw himself from any of these, and get him to 
any other order, let him not think thereby to avoid the 
vengeance of God in obeying rules and traditions of man's 
imagination. If thou pollest thine head in the worship of 
thy father, and breakest his commandments, shouldest thou 
so escape? or if thou paintest thy Master's image on a wall 
and stickest up a candle before it, shouldest thou therewith 
make satisfaction for the breaking of his commandments? 
Or if thou wearest a blue coat in the worship of the king 
and breakest his laws shouldest thou so go quit? Let a man's 
wife make herself a sister of the charterhouse, and answer 
her husband when he biddeth her hold her peace, My 
brethren keep silence for me, and see whether she shall so 
escape. And be thou sure God is more jealous over his 
commandments than man is over his, or than any man is 
over his wife. 

Because we be blind, God hath appointed in the Scripture
how we should serve him and please him. As pertaining
unto his own person he is abundantly pleased when 
we believe his promises and Holy Testament which he 
hath made unto us in Christ, and for the mercy which he 
there shewed us, love his commandments. All bodily 
service must be done to man in God's stead. We must 
give obedience, honour, toll, tribute, custom, and rent 
unto whom they belong. Then if thou have ought more 
to bestow, give unto the poor which are left here in Christ's 
stead, that we shew mercy on them. If we keep the commandments
of love, then are we sure that we fulfil the law 
in the sight of God, and that our blessing shall be everlasting
life. Now when we obey patiently, and without 
grudging, evil princes that oppress us, and persecute us, 
and be kind and merciful to them that are merciless to us, 
and do the worst they can to us, and so take all fortune 
patiently, and kiss whatsoever cross God layeth on our 
backs, then are we sure that we keep the commandments 
of love. 

I declared that God hath taken all vengeance into his 
own hands, and will avenge all unright himself: either by 
the powers or offices which are appointed thereto, or else, 
if they be negligent, he will send his curses upon the 
transgressors, and destroy them with his secret judgments. 
I shewed also that whosoever avengeth himself is damned 
in the deed doing, and falleth into the hands of the temporal
sword, because he taketh the office of God upon him 
and robbeth God of his most high honour, in that he will 
not patiently abide his judgment. I shewed you of the 
authority of princes, how they are in God's stead, and how 
they may not be resisted, do they never so evil, they must 
be reserved unto the wrath of God. Neverthelater, if 
they command to do evil we must then disobey and say, 
We are otherwise commanded of God: but not to rise 
against them. They will kill us then, sayest thou. Therefore,
I say, is a christian called, to suffer even the bitter 
death for his hope's sake, and because he will do no evil. 
I shewed also that the kings and ruler's (be they never so 
evil) are yet a great gift of the goodness of God, and defend
us from a thousand things that we see not. 

I proved also that all men without exception are under 
the temporal sword, whatsoever names they give themselves. 
Because the priest is chosen out of the laymen, to teach 
this obedience, is that a lawful cause for him to disobey? 
Because he preacheth that the layman should not steal, is 
it therefore lawful for him to steal unpunished? Because 
thou teachest me that I may not kill, or if I do, the king 
must kill me again, is it therefore lawful for thee to kill 
and go free? either whether is it rather mete that thou 
which art my guide to teach me the right way shouldst 
walk therein before me? The priests of the old law with 
their high bishop Aaron, and all his successors, though 
they were anointed by God's commandment, and appointed 
to serve God in his temple, and exempt from all offices, 
and ministering of worldly matters, were yet nevertheless 
under the temporal sword, if they brake the laws. Christ 
saith to Peter, All that take the sword shall perish by the 
sword. Here is none exception. Paul saith, All souls 
must obey. Here is none exception. Paul himself is 
here not exempt. God saith, (Gen. ix.) Whosoever 
sheddeth man's blood, by man shall his blood be shed 
again. Here is none exception. 

Moreover Christ became poor to make other men rich, 
and bound to make other free. He left also with his 
disciples the law of love. Now love seeketh not her own 
profit, but her neighbour's: love seeketh not her own 
freedom, but becometh surety and bond to make her 
neighbour free. Damned, therefore, are the spiritualty 
by all the laws of God, which through falsehood and disguised
hypocrisy have sought so great profit, so great 
riches, so great authority, and so great liberties, and have 
so beggared the lay, and so brought them in subjection 
and bondage, and so despised them, that they have set up 
franchises in all towns and viilages, for whosoever robbeth, 
murdereth or slayeth them, and even for traitors unto the 
king's person also. 

I proved also that no king hath power to grant them 
such liberty; but are as well damned for their giving, as 
they for their false purchasing. For as God giveth the 
father power over his children, even so giveth he him a 
commandment to execute it, and not to suffer them to do 
wickedly unpunished, but unto his damnation, as thou mayest 
see by Eli, the high priest, \&c. And as the master hath 
authority over his servants, even so hath he a commandment 
to govern them. And as the husband is head over his wife, 
even so hath he commandment to rule her appetites, and is 
damned, if he suffer her to be an whore and a mis-liver, 
or submit himself to her, and make her his head. And
even in like manner as God maketh the king head over his
realm, even so giveth he him commandment to execute
the laws upon all men indifferently. For the law is God's,
and not the king's. The king is but a servant, to execute
the law of God, and not to rule after his own imagination. 

I showed also that the law and the king are to be 
feared, as things that were given in fire, and in thunder, 
and lightning, and terrible signs. I showed the cause why 
rulers are evil, and by what means we might obtain better. 
I showed also how wholesome those bitter medicines evil 
princes are to right Christian men. 

I declared how they which God hath made governors in 
the world ought to rule, if they be Christian. They 
ought to remember that they are heads and arms, to defend
the body, to minister peace, health, and wealth, and 
even to save the body; and that they have received their 
offices of God, to minister and to do service unto their 
brethren: king, subject, master, servant, are names in the 
world; but not in Christ. In Christ we are all one, and 
even brethren. No man is his own, but we are all 
Christ's servants, bought with Christ's blood. Therefore 
ought no man to seek himself or his own profit; but 
Christ and his will. In Christ no man ruleth as a king 
his subjects, or a master his servants; but serveth as one 
hand doth to another, and as the hands do unto the feet, 
and the feet to the hands, as thou seest 1 Cor. xii. We 
also serve not as servants unto masters; but as they which 
are bought with Christ's blood serve Christ himself. We 
be here all servants unto Christ. For whatsoever we do 
one to another in Christ's name, that do we unto Christ, 
and the reward of that shall we receive of Christ. The 
king counteth his commons Christ himself, and therefore 
doth them service willingly; seeking no more of them than 
is sufficient to maintain peace and unity, and to defend the 
realm. And they obey again willingly and lovingly, as 
unto Christ. And of Christ every man seeketh his reward. 

I warned the judges that they take not an ensample 
how to minister their offices of our spiritualty, which are 
bought and sold to do the will of Satan; but of the 
Scripture, whence they have their authority. Let that 
which is secret abide secret till God open it, which is the 
Judge of secrets. For it is more than a cruel thing to 
break up into a man's heart, and to compel him to put 
either soul or body in jeopardy, or to shame himself. If 
Peter, that great pillar, for fear of death, forsook his 
Master, ought we not to spare weak consciences? 

I declared how the king ought to rid his realm from the 
wily tyranny of the hypocrites, and to bring the hypocrites 
under his laws: yea, and how he ought to be learned, and 
to hear, and to look upon the causes himself, which he 
will punish; and not to believe the hypocrites, and to 
give them his sword to kill whom they will. 

The king ought to count what he hath spent in the 
pope's quarrel since he was king. The first voyage 
cost upon fourteen hundred thousand pounds. Reckon 
since what hath been spent by sea and land between us 
and Frenchmen, and Scots, and then in triumphs, and in 
embassies, and what hath been sent out of the realm secretly,
and all to maintain our holy father, and I doubt 
not but that will surmount the sum of forty or fifty hundred 
thousand pounds. For we had no cause to spend one 
penny but for our holy father. The king therefore ought 
to make them pay this money every farthing, and set it out 
of their mitres, crosses, shrines, and all manner [of] treasure
of the church, and pay it to his commons again: 
not that only which the cardinal and his bishops compelled
the commons to lend, and made them swear with 
such an ensample of tyranny as was never before thought 
on; but also all that he hath gathered of them. Or else 
by the consent of the commons, to keep it in store for the 
defence of the realm. Yea, the king ought to look in 
the chronicles, what the popes have done to kings in time 
past, and make them restore it also; and ought to take 
away from them their lands which they have gotten with 
their false prayers, and restore it unto the right heirs again; 
or with consent and advisement turn them unto the maintaining
of the poor, and bringing up of youth virtuously, 
and to maintain necessary officers and ministers for to 
defend the commonwealth. 

If he will not do it, then ought the commons to take 
patience, and to take it for God's scourge, and to think 
that God hath blinded the king for their sins' sake, and 
commit their cause to God: and then shall God make a 
a scourge for them, and drive them out of his temple, 
after his wonderful judgment. 

On the other side, I have also uttered the wickedness 
of the spiritualty, the falsehood of the bishops, and 
juggling of the pope, and how they have disguised themselves,
borrowing some of their pomp of the Jews, and 
some of the Gentiles, and have with subtle wiles turned 
the obedience that should be given to God's ordinance unto 
themselves. And how they have put out God's Testament 
and God's truth, and set up their own traditions and lies, 
in which they have taught the people to believe, and thereby 
sit in their consciences as God; and have by that means 
robbed the world of lands and goods, of peace and unity, 
and of all temporal authority, and have brought the people 
into the ignorance of God, and have heaped the wrath of 
God upon all realms; and namely, upon the kings: 
whom they have robbed, (I speak not of worldly things 
only) but even of their very natural wits. They make 
them believe that they are most Christian, when they live 
most abominably; and will suffer no man in their realms 
that believeth on Christ; and that they are defenders of 
the faith, when they burn the gospel and promises of God, 
out of which all faith springeth. 

I showed how they have ministered Christ, king and 
emperor out of their rooms; and how they have made 
them a several kingdom, which they got at the first in 
deceiving of princes, and now pervert the whole Scripture,
to prove that they have such authority of God. 
And lest the laymen should see how falsely they allege 
the places of the Scripture, is the greatest cause of this 
persecution. 

They have feigned confession for the same purpose to 
stablish their kingdom withal. All secrets know they 
thereby. The bishop knoweth the confession of whom 
he lusteth throughout all his diocese. Yea, and his chancellor
commandeth the ghostly father to deliver it written. 
The pope, his cardinals and bishops, know the confession 
of the emperor, kings, and of all lords: and by confession
they know all their captives. If any believe in Christ, 
by confession they know him. Shrive thyself where thou 
wilt, whether at Sion, Charterhouse, or at the observant's, 
thy confession is known well enough. And thou, if thou 
believe in Christ, art waited upon. Wonderful are the 
things that thereby are wrought. The wife is feared, and 
compelled to utter not her own only, but also the secrets 
of her husband, and the servant the secrets of his master. 
Besides that through confession they quench the faith of 
all the promises of God, and take away the effect and 
virtue of all the sacraments of Christ. 

They have also corrupted the saints' lives with lies and 
feigned miracles, and have put many things out of the 
sentence or great curse, as raising of rent and fines, and 
hiring men out of their houses, and whatsoever wickedness
they themselves do; and have put a great part of the 
stories and chronicles out of the way lest their falsehood 
should be seen. For there is no mischiefs or disorder, 
whether it be in the temporal regiment, or else in the 
spiritual, whereof they are not the chief causes, and even 
the very fountain and springs, and as we say, the well- 
head; so that it is impossible to preach against mischief 
except thou begin at them, or to set any reformation in 
the world except thou reform them first. Now are they 
indurate and tough as Pharaoh, and will not bow unto 
any right way or order. And therefore persecute they 
God's word and the preachers thereof; and on the other 
side lie await unto all princes, and stir up all mischief in 
the world, and send them to war, and occupy their minds 
therewith, or with other voluptuousness, lest they should 
have leisure to hear the word of God, and to set an order 
in their realms. 

By them is all things ministered, and by them are all 
kings ruled: yea, in every king's conscience sit they ere 
he be king, and persuade every king what they lust, and 
make them both to believe what they will, and to do what 
they will. Neither can any king or any realm have rest for 
their businesses. Behold king Henry the Vth, whom they 
sent out for such a purpose as they sent out our king that 
now is. See how the realm is inhabited. Ask where the 
goodly towns and their walls, and the people that was 
wont to be in them are become, and where the blood 
royal of the realm is become also? Turn thine eyes 
whither thou wilt, and thou shalt see nothing prosperous 
but their subtle polling. With that it is flowing water: 
yea, and I trust it will be shortly a full sea. 

In all their doings, though they pretend outwardly the 
honour of God or a commonwealth, their intent and 
secret counsel is only to bring all under their power, and 
to take out of the way whosoever letteth them, or is too 
mighty for them. As when they send the princes to 
Jerusalem, to conquer the holy land, and to fight against 
the Turks, whatsoever they pretend outwardly, their 
secret intent is, while the princes there conquer them more 
bishoprics, to conquer their lands in the mean season with 
their false hypocrisy, and to bring all under them; which 
thou mayest easily perceive by that they will not let us 
know the faith of Christ. And when they are once on 
high, then are they tyrants above all tyrants, whether they 
be Turks or Saracens. How minister they proving of 
testaments? How causes of wedlock? or if any man 
die intestate? If a poor man die, and leave his wife 
and half a dozen young children, and but one cow to find 
them, that will they have for a mortuary mercy-lease: let 
come of wife and children what will. Yea, let any thing 
be done against their pleasure, and they will interdict the 
whole realm, sparing no person. 

Read the chronicles of England (out of which yet they 
have put a great part of their wickedness,) and thou shalt 
find them always both rebellious and disobedient to the 
kings, and also churlish and unthankful, so that when all 
the realm gave the king somewhat to maintain him in his 
right, they would not give a mite. Consider the story 
of king John, where I doubt not but they have put the 
best and fairest for themselves, and the worst of king 
John. For I suppose they make the chronicles themselves.
Compare the doings of their holy church (as they 
ever call it) unto the learning of Christ and of his apostles. 
Did not the legate of Rome assoil all the lords of the 
realm of their due obedience which they ought to the 
king by the ordinance of God? Would he not have 
cursed the king with his solemn pomp, because he would 
have done that office which God commandeth every king 
to do, and wherefore God hath put the sword in every 
king's hand? that is to wit, because king John would 
have punished a wicked clerk that had coined false 
money. The laymen that had not done half so great 
faults must die, but the clerk must go escape free! Sent 
not the pope also unto the king of France remission of 
his sins, to go and conquer king John's realm? So now 
remission of sins cometh not by faith in the testament that 
God hath made in Christ's blood; but by fighting and 
murdering for the pope's pleasure. Last of all, was not 
king John fain to deliver his crown unto the legate, and 
to yield up his realm unto the pope, wherefore we pay 
Peter-pence. They might be called the polling-pence of 
false prophets well enough. They care not by what mischief
they come by their purpose. War and conquering 
of lands is their harvest. The wickeder the people are, 
the more they have the hypocrites in reverence, the more 
they fear them, and the more they believe in them. And 
they that conquer other men's lands, when they die, make 
them their heirs, to be prayed for for ever. Let there come
one conquest more in the realm, and thou shalt see them
get yet as much more as they have (if they can keep down
God's word, that their juggling come not to light) yea,
thou shalt see them take the realm whole into their hands,
and crown one of themselves king thereof. And verily, 
I see no other likelihood, but that the land shall be shortly 
conquered. The stars of the Scripture promise us none 
other fortune, inasmuch as we deny Christ with the wicked 
Jews, and will not have him reign over us; but will be 
still children of darkness under antichrist, and antichrist's 
possession, burning the gospel of Christ, and defending 
a faith that may not stand with his holy Testament. 

If any man shed blood in the church, it shall be interdicted
till he have paid for the hallowing. If he be not 
able, the parish must pay, or else shall it stand always interdicted.
They will be avenged on them that never 
offended. Full well prophesied of them Paul, in the 
2d Epistle to Tim. chap. iii. Some man will say, Wouldest 
thou that men should fight in the church unpunished? Nay, 
but let the king ordain a punishment for them, as he doth 
for them that fight in his palace, and let not all the parish 
be troubled for one's fault. And as for their hallowing, it 
is the juggling of antichrist. A Christian man is the 
temple of God and of the Holy Ghost, and hallowed in 
Christ's blood. A Christian man is holy in himself, by 
reason of the Spirit that dwelleth in him; and the place 
wherein he is is holy by reason of him, whether he be in 
the field or town. A Christian husband sanctifieth an 
unchristian wife, and a Christian wife an unchristian husband,
(as concerning the use of matrimony) saith Paul to 
the Corinthians. If now while we seek to be hallowed in 
Christ, we are found unholy, and must be hallowed by the 
ground, or place, or walls, then died Christ in vain. How- 
beit, antichrist must have wherewith to sit in men's consciences,
and to make them fear where is no fear, and 
to rob them of their faith, and to make them trust in 
that cannot help them, and to seek holiness of that which 
is not holy in itself. 

After that the old king of France was brought down out 
of Italy, mark what pageants have been played, and what 
are yet a playing to separate us from the emperor, (lest by 
the help or aid of us he should be able to recover his right 
of the pope) and to couple us to the Frenchmen, whose 
might the pope ever abuseth to keep the emperor from 
Italy. What prevaileth it for any king to marry his 
daughter or his son, or to make any peace or good ordinance
for the wealth of his realm? For it shall no longer 
last than it is profitable to them. Their reason is so secret 
that the world cannot perceive it. They dissimil those 
things which they are only cause of, and simil discord 
among themselves when they are most agreed. One shall 
hold this, and another shall dispute the contrary: but 
the conclusion shall be that most maintaineth their false- 
hood, though God's word be never so contrary. What 
have they wrought in our days; yea, and what work 
they yet, to the perpetual dishonour of the king, and 
rebuke of the realm, and shame of all the nation, in whatsoever
realms they go! 

I uttered unto you partly the malicious blindness of the 
bishop of Rochester, his juggling, his conveying, his foxey- 
wiliness, his bo-peep, his wresting, renting, and shameful 
abusing of the Scripture; his oratory and alleging of heretics,
and how he would make the apostles authors of blind 
ceremonies, without signification, contrary to their own 
doctrine, and have set him for an ensample to judge all 
other by. Whatsoever thou art that readest this, I exhort 
thee in Christ, to compare his sermon and that which I 
I have written, and the Scripture together, and judge. 
There shalt thou find of our holy father's authority, and 
what it in to be great, and how to know the greatest. 

Then followeth the cause why laymen cannot rule 
temporal offices, which is the falsehood of the bishops. 
There shalt thou find of miracles and ceremonies without 
signification; of false anointing, and lying signs, and false 
names; and how the spiritualty are disguised in false- 
hood, and how they rule the people in darkness, and do 
all thing in the Latin tongue, and of their petty pillage. 
Their polling is like a soaking consumption, wherein a man 
complaineth of feebleness and of faintness, and wotteth not 
whence his disease cometh; it is like a pock that fretteth 
inward, and consumeth the very marrow of the bones. 

There seest thou the cause why it is impossible for 
kings to come to the knowledge of the truth. For the 
spirits lay await for them, and serve their appetites at all 
points; and through confession, buy and sell and betray 
both them and all their true friends, and lay baits for
them, and never leave them till they have blinded them 
with their sophistry, and have brought them into their 
nets. And then when the king is captive, they compel 
all the rest with violence of his sword. For if any man 
will not obey them, be it right or wrong, they cite him, 
suspend him, and curse or excommunicate him. If he 
then obey not, they deliver him to Pilate, that is to say, 
unto the temporal officers, to destroy them. Last of all, 
there findest thou the very cause of all persecution, which 
is the preaching against hypocrisy. 

Then come we to the sacraments, where thou seest 
that the work of the sacrament saveth not, but the faith 
in the promise, which the sacrament signifieth, justifieth 
us only. There hast thou that a priest is but a servant to 
teach only, and whatsoever he taketh upon him more than 
to preach and to minister the sacraments of Christ, (which 
is also preaching) is falsehood. 

Then cometh how they juggle through dumb ceremonies,
and how they make merchandise with feigned words; 
penance, a paena et a culpa, satisfaction, attrition, character,
purgatory, pick-purse; and how through confession 
they make the sacraments and all the promise of none 
effect or value. There seest thou that absolving is but 
preaching the promises; and cursing or excommunicating, 
preaching the law; and of their power, and of their keys, 
of false miracles, and praying to saints. There seest thou 
that ceremonies did not the miracles, but faith: even as it 
was not Moses' rod that did the miracles, but Moses' 
faith in the promises of God. Thou seest also that to 
have a faith where God hath not a promise, is idolatry. 
And there also seest thou how the pope exalteth himself 
above God, and commandeth him to obey his tyranny. 
Last of all, thou hast there that no man ought to preach 
but he that is called. 

Then followeth the belly-brotherhood of monks and 
friars. For Christ hath deserved nought with them. For 
his sake gettest thou no favour. Thou must offer unto 
their bellies, and then they pray bitterly for thee. There 
seest thou that Christ is the only cause; yea, and all the 
cause why God doth ought for us, and heareth our complaint.
And there hast thou doctrine how to know and 
to be sure that thou art elect and hast God's spirit in thee. 
And hast there learning to try the doctrine of our spirits. 

Then follow the four senses of the Scripture, of which 
three are no senses; and the fourth, that is to wit the literal 
sense, which is the very sense, hath the pope taken to himself.
It may have no other meaning than as it pleaseth 
his fatherhood. We must abide his interpretation. And 
as his belly thinks, so must we think, though it be impossible
to gather any such meaning of the Scripture. Then 
hast thou the very use of allegories, and how they are 
nothing but ensamples borrowed of the Scripture to express
a text or an open conclusion of the Scripture, and as 
it were to paint it before thine eyes, that thou mayest feel 
the meaning and the power of the Scripture in thine heart. 
Then cometh the use of worldly similitudes, and how
they are false prophets which bring a worldly similitude 
for any other purpose, save to express more plainly that 
which is contained in an open text. And so are they also 
which draw the Scripture contrary to the open places, and 
contrary to the ensample, living, and practising of Christ, 
the apostles, and of the holy prophets. And then, finally, 
hast thou of our holy father's power, and of his keys, and 
of his binding and excommunicating, and of his cursing 
and blessing, with ensamples of every thing. 
